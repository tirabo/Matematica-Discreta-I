% PDFLaTeX
\documentclass[a4paper,11pt,twoside,spanish]{amsbook}
%\documentclass[a4paper,11pt,twoside]{book}
%\documentclass[a4paper,11pt,twoside,spanish]{amsbook}

%%%---------------------------------------------------
%\usepackage[math]{kurier} %++
%\usepackage{cmbright} %++

\usepackage{etex}
\tolerance=10000
\renewcommand{\baselinestretch}{1.3}

\renewcommand{\familydefault}{\sfdefault} % la font por default es sans serif

% Para hacer el  indice en linea de comando hacer
% makeindex main
%% En http://www.tug.org/pracjourn/2006-1/hartke/hartke.pdf hay ejemplos de packages de fonts libres, como los siguientes:
%\usepackage{cmbright}
%\usepackage{pxfonts}
%\usepackage[varg]{txfonts}
%\usepackage{ccfonts}
%\usepackage[math]{iwona}
%\usepackage[math]{kurier}


\usepackage{t1enc}
%\usepackage[spanish]{babel}
\usepackage{latexsym}
\usepackage[utf8]{inputenc}
\usepackage{verbatim}
\usepackage{multicol}
\usepackage{amsgen,amsmath,amstext,amsbsy,amsopn,amsfonts,amssymb}
\usepackage{verbatim}
\usepackage{amsthm}
\usepackage{calc}         % From LaTeX distribution
\usepackage{graphicx}     % From LaTeX distribution
\usepackage{ifthen}
\input{random.tex}        % From CTAN/macros/generic
\usepackage{subfigure}
\usepackage{tikz}
\usetikzlibrary{arrows}
\usetikzlibrary{matrix}
%\usetikzlibrary{graphs}
%\usepackage{tikz-3dplot} %for tikz-3dplot functionality
%\usepackage{pgfplots}
\usepackage{mathtools}
\usepackage{stackrel}
\usepackage{enumerate}
\usepackage{tkz-graph}
\usepackage{array}
\makeindex

%%%----------------------------------------------------------------------------
\usepackage[a4paper, top=3cm, left=3cm, right=2cm, bottom=2.5cm]{geometry}
%% CONTROLADORES DE.
% Tamaño de la hoja de impresión.
% Tamaños de los laterales del documento.
%%%%%%%%%%%%%%%%%%%%%%%%%%%%%%%%%%%%%%%%%%%%%%%%%%%%%%%%%%%%%%%%%%%%%%%%%%%%%%%%%
%%% \theoremstyle{plain} %% This is the default
%\oddsidemargin 0.0in \evensidemargin -1.0cm \topmargin 0in
%\headheight .3in \headsep .2in \footskip .2in
%\setlength{\textwidth}{16cm} %ancho para apunte
%\setlength{\textheight}{21cm} %largo para apunte
%%%%\leftmargin 2.5cm
%%%%\rightmargin 2.5cm
%\topmargin 0.5 cm
%%%%%%%%%%%%%%%%%%%%%%%%%%%%%%%%%%%%%%%%%%%%%%%%%%%%%%%%%%%%%%%%%%%%%%%%%%%%%%%%%%%

\usepackage{hyperref}
\hypersetup{
    colorlinks=true,
    linkcolor=blue,
    filecolor=magenta,
    urlcolor=cyan,
}
\usepackage{hypcap}
\usepackage{fancyhdr}

\renewcommand{\thesection}{\thechapter.\arabic{section}}
\renewcommand{\thesubsection}{\thesection.\arabic{subsection}}

\newtheorem{teorema}{Teorema}[section]
\newtheorem{proposicion}[teorema]{Proposici\'on}
\newtheorem{corolario}[teorema]{Corolario}
\newtheorem{lema}[teorema]{Lema}
\newtheorem{propiedad}[teorema]{Propiedad}

\theoremstyle{definition}

\newtheorem{definicion}{Definici\'on}[section]
\newtheorem{ejemplo}{Ejemplo}[section]
\newtheorem{problema}{Problema}[section]
\newtheorem{ejercicio}{Ejercicio}[section]
\newtheorem{ejerciciof}{}[section]

\theoremstyle{remark}
\newtheorem{observacion}{Observaci\'on}[section]
\newtheorem{nota}{Nota}[section]

\renewcommand{\partname }{Parte }
\renewcommand{\indexname}{Indice }
\renewcommand{\figurename }{Figura }
\renewcommand{\tablename }{Tabla }
\renewcommand{\proofname}{Demostraci\'on}
\renewcommand{\appendixname }{}
\renewcommand{\contentsname }{Contenidos }
\renewcommand{\chaptername }{}
\renewcommand{\bibname }{Bibliograf\'\i a }

\newcommand{\img}{\operatorname{Im}}
\newcommand{\nuc}{\operatorname{Nu}}
\newcommand\im{\operatorname{Im}}
\renewcommand\nu{\operatorname{Nu}}
\newcommand{\la}{\langle}
\newcommand{\ra}{\rangle}
\renewcommand{\t}{{\operatorname{t}}}
\renewcommand{\sin}{{\,\operatorname{sen}}}
\newcommand{\Q}{\mathbb Q}
\newcommand{\R}{\mathbb R}
\newcommand{\C}{\mathbb C}
\newcommand{\K}{\mathbb K}
\newcommand{\F}{\mathbb F}
\newcommand{\Z}{\mathbb Z}
\newcommand{\cB}{{\mathcal{B}}}
\newcommand{\cC}{{\mathcal{C}}}
\newcommand{\Id}{\operatorname{Id}}

\begin{document}
\pagenumbering{gobble}

    \pagestyle{fancy}

\lhead{Matemática Discreta I}
\rfoot{\color{gray}\hrule \vskip .1cm Página \thepage}


\
\centerline{\textbf{\large{Matemática Discreta I}}}
\

\centerline{\textbf{\large{Parcial 1: Abril 20, 2023}}}

\centerline{\textbf{\large{Turno Tarde - Tema 1}}}

\vskip .5cm


\noindent{\bf Nombre y  apellido: }
\vskip 0.4cm
\noindent{\bf Correo UNC: }
\vskip 0.4cm
\noindent{\bf COMISIÓN (tal como figura en Guaraní):} 



        \medbreak
\begin{enumerate}
        \item 
         (10 \%) Dada la siguiente definición recursiva:
            $$
            c_1 = -1, \qquad c_2 =1, \qquad c_n = 3c_{n-1} - 2c_{n-2}, \text{ para $n\geq 3$},
            $$
            calcular el valor numérico de  los términos $c_{3}$, $c_{4}$, $c_{5}$ y $c_{6}$.
    
    \medbreak
        \item
        \begin{enumerate}
       \item  (15 \%) Demostrar por inducción 
            $$
            \sum_{i = 2}^n  \frac{1}{i(i-1)} = \frac{n-1}{n} \,.
            $$
        \item  ({25 \%})
         Sea $\{a_n\}_{n\in\mathbb N_0}$ la sucesi\'on definida recursivamente por
            $$\begin{cases}
            a_1=3, \\a_2=-9, \\a_{n} = 7a_{n-1}-10a_{n-2}, \text{ para $n\geq 3$}.
            \end{cases}$$
            \vspace{0.3cm}
            Probar que $a_n= 4 \cdot 2^n -5^n$ para todo $n\in \mathbb N$.
        \end{enumerate}
\medbreak

        \item (15\%) ¿Cuántas palabras distintas pueden formarse con las letras de la palabra\newline  ELECTROENCEFALOGRAMA?
\medbreak
        \item En un campo hay $32$ vacas, $15$ perros, $12$ gatos,  y $14$ gallinas. Queremos elegir un total de $5$ animales para fotografiar. ¿De cuántas formas podemos hacerlo?
    
    \medbreak
    \begin{enumerate}
        \item (5\%) Si no hay restricciones. 
        \item (10\%) Si debe haber $5$ vacas y $3$ perros.
        \item  (10\%) Si debe haber al menos $3$ gallinas.
        \item (10\%) Si debe haber al menos un animal vaca, un perro, un gato y  una gallina.
    \end{enumerate}
\end{enumerate}
\vskip .8cm
\begin{center}
    \renewcommand{\arraystretch}{1.3}
    \begin{tabular}{|p{1cm}|p{1cm}|p{1cm}|p{1cm}|b{1.5cm}|p{1.5cm}|}\hline
        1 & 2 & 3 & 4 &  Total & Nota\\ \hline
        &   &   &   &   &      \\ \hline
    \end{tabular}    
\end{center}

\end{document}