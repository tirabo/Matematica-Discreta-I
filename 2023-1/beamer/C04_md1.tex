%\documentclass{beamer} 
\documentclass[handout]{beamer} % sin pausas
\usetheme{CambridgeUS}
%\setbeamertemplate{background}[grid][step=8 ] % cuadriculado

\usepackage{etex}
\usepackage{t1enc}
\usepackage[spanish,es-nodecimaldot]{babel}
\usepackage{latexsym}
\usepackage[utf8]{inputenc}
\usepackage{verbatim}
\usepackage{multicol}
\usepackage{amsgen,amsmath,amstext,amsbsy,amsopn,amsfonts,amssymb}
\usepackage{amsthm}
\usepackage{calc}         % From LaTeX distribution
\usepackage{graphicx}     % From LaTeX distribution
\usepackage{ifthen}
%\usepackage{makeidx}
\input{random.tex}        % From CTAN/macros/generic
\usepackage{subfigure} 
\usepackage{tikz}
\usepackage[customcolors]{hf-tikz}
\usepackage[most]{tcolorbox}
\usetikzlibrary{arrows}
\usetikzlibrary{matrix}
\tikzset{
    every picture/.append style={
        execute at begin picture={\deactivatequoting},
        execute at end picture={\activatequoting}
    }
}
\usetikzlibrary{decorations.pathreplacing,angles,quotes}
\usetikzlibrary{shapes.geometric}
\usepackage{mathtools}
\usepackage{stackrel}
%\usepackage{enumerate}
\usepackage{enumitem}
\usepackage{tkz-graph}
\usepackage{polynom}
\polyset{%
    style=B,
    delims={(}{)},
    div=:
}
\renewcommand\labelitemi{$\circ$}
\setlist[enumerate]{label={(\arabic*)}}
%\setbeamertemplate{background}[grid][step=8 ] % cuadriculado
\setbeamertemplate{itemize item}{$\circ$}
\setbeamertemplate{enumerate items}[default]
\definecolor{links}{HTML}{2A1B81}
\hypersetup{colorlinks,linkcolor=,urlcolor=links}


\newcommand{\Id}{\operatorname{Id}}
\newcommand{\img}{\operatorname{Im}}
\newcommand{\nuc}{\operatorname{Nu}}
\newcommand{\im}{\operatorname{Im}}
\renewcommand\nu{\operatorname{Nu}}
\newcommand{\la}{\langle}
\newcommand{\ra}{\rangle}
\renewcommand{\t}{{\operatorname{t}}}
\renewcommand{\sin}{{\,\operatorname{sen}}}
\newcommand{\Q}{\mathbb Q}
\newcommand{\R}{\mathbb R}
\newcommand{\C}{\mathbb C}
\newcommand{\K}{\mathbb K}
\newcommand{\F}{\mathbb F}
\newcommand{\Z}{\mathbb Z}
\newcommand{\N}{\mathbb N}
\newcommand\sgn{\operatorname{sgn}}
\renewcommand{\t}{{\operatorname{t}}}
\renewcommand{\figurename }{Figura}

\include{definiciones}

\newcommand{\nc}{\newcommand}

%%%%%%%%%%%%%%%%%%%%%%%%%LETRAS

\nc{\FF}{{\mathbb F}} \nc{\NN}{{\mathbb N}} \nc{\QQ}{{\mathbb Q}}
\nc{\PP}{{\mathbb P}} \nc{\DD}{{\mathbb D}} \nc{\Sn}{{\mathbb S}}
\nc{\uno}{\mathbb{1}} \nc{\BB}{{\mathbb B}} \nc{\An}{{\mathbb A}}

\nc{\ba}{\mathbf{a}} \nc{\bb}{\mathbf{b}} \nc{\bt}{\mathbf{t}}
\nc{\bB}{\mathbf{B}}

\nc{\cP}{\mathcal{P}} \nc{\cU}{\mathcal{U}} \nc{\cX}{\mathcal{X}}
\nc{\cE}{\mathcal{E}} \nc{\cS}{\mathcal{S}} \nc{\cA}{\mathcal{A}}
\nc{\cC}{\mathcal{C}} \nc{\cO}{\mathcal{O}} \nc{\cQ}{\mathcal{Q}}
\nc{\cB}{\mathcal{B}} \nc{\cJ}{\mathcal{J}} \nc{\cI}{\mathcal{I}}
\nc{\cM}{\mathcal{M}} \nc{\cK}{\mathcal{K}}

\nc{\fD}{\mathfrak{D}} \nc{\fI}{\mathfrak{I}} \nc{\fJ}{\mathfrak{J}}
\nc{\fS}{\mathfrak{S}} \nc{\gA}{\mathfrak{A}}
%%%%%%%%%%%%%%%%%%%%%%%%%LETRAS

\title[Clase 4 - Inducción]{Matemática Discreta I \\ Clase 4 - Inducción}
%\author[C. Olmos / A. Tiraboschi]{Carlos Olmos / Alejandro Tiraboschi}
\institute[]{\normalsize FAMAF / UNC
    \\[\baselineskip] ${}^{}$
    \\[\baselineskip]
}
\date[28/03/2023]{28 de marzo   de 2023}




\begin{document}
%\title{El centro geográfico de Argentina}   
%\author{} 
%\date{Villa Huidobro \\ 4/12/2018} 



\frame{\titlepage} 

%\frame{\frametitle{Índice}\tableofcontents} 




\begin{frame}\frametitle{El principio de inducción} 

    Queremos  analizar la suma de los primeros $n$ números impares, es decir
    $$
    1+3+5+\cdots+(2n-1).
    $$
    \pause
    Por la definición recursiva de la sumatoria,  tenemos que 
    \begin{equation*}
    a_1 = 1, \;\text{  y } \; a_n = a_{n-1} + 2n-1,
    \end{equation*}\pause
    
    
    Analicemos los primero valores
    
    \begin{itemize}
        \item $a_1 = 1$,
        \item $a_2 = 1 + 3 = 4$, 
        \item $a_3 = 1 + 3 + 5 = 9$, 
        \item $a_4 = 1 + 3 + 5 + 7= 16 $, 
        \item $a_5 = a_4 + 9= 25   $, 
    \end{itemize}
    
\end{frame}


\begin{frame}    
    Entonces, podemos conjeturar que\pause
    $$
    a_n = 1+3+5+\cdots+(2n-1) = n^2.
    $$
    \pause
    Para convencernos de que la fórmula es ciertamente correcta procedemos de la siguiente manera
    \vskip .4cm
    \textbf{Caso n=1.} La fórmula es verdadera cuando $n=1$ puesto que     $1=1^2$.\pause
    
    
    \vskip .4cm
    \textbf{Paso recursivo.}Supongamos que es correcta para un valor específico
    de $n$, digamos para $n=k$, de modo que
    \begin{equation*}
    a_k = 1+3+5+\cdots+(2k-1) = k^2. \tag{a}
    \end{equation*}



\end{frame}

\begin{frame}
        Podemos utilizar (a) para simplificar la expresión definida
    recursivamente a la izquierda cuando $n$ es igual a $k+1$,\pause
    $$
    \begin{aligned}
    1+3+5+\cdots+(2k+1) &= 1+3+5+\cdots+(2k-1) +(2k+1) \\
    &\stackrel{(a)}{=}k^2 +(2k+1) \\
    &=(k+1)^2.
    \end{aligned}
    $$\pause
    Por lo tanto si el resultado es correcto cuando $n=k$, entonces lo es cuando $n=k+1$. \pause 
    
    \vskip .4cm
    Se comienza observando que si es correcto cuando $n=1$, debe ser por lo tanto correcto cuando $n=2$. 
    \pause 
    
    \vskip .4cm
    
    Con el mismo argumento como es correcto cuando $n=2$ debe serlo cuando $n=3$. Continuando de esta forma veremos que es correcto para todos los enteros positivos $n$.
\end{frame}

\begin{frame}
        
    La esencia de este argumento es comúnmente llamada {\it principio de inducción}.
    \pause
    \medskip 
    
     Con $S$ denotemos al subconjunto de $\mathbb N$ para el cual el resultado es correcto: por supuesto, nuestra intención es probar que $S$ es todo $\mathbb N$. 
     \pause
     
     \vskip .8cm
     
    %\begin{teorema}\label{t1.4} 
    {\color{blue} Teorema }
    \vskip .2cm 
    {\it
        Supongamos que $S$ es un subconjunto de $\mathbb N$ que satisface las condiciones \index{principio de inducción}
        \begin{enumerate}
            \item[a)] $1 \in S$,
            \item[b)] para cada $k \in \mathbb N$, si $ k \in S$ entonces $k+1\in S$.
        \end{enumerate}
        Entonces se sigue que $S=\mathbb N$.
}
    %\end{teorema}
\end{frame}

\begin{frame}


    %\begin{proof}[Demostración]
    
    {\color{blue} Demostración.}
    \vskip .2cm 
        Si la conclusión es falsa, $S \not= \mathbb N$ y
        el conjunto complementario $S^{\text{c}}$ definido por
        $$
        S^{\text{c}}= \{ r \in \mathbb N | r\not\in S\}
        $$
        es no vacío. \pause 
        
        \medskip 
        
        Por el axioma del buen orden, $S^{\text{c}}$ tiene un menor
        elemento (mínimo) $m$. 
        
        \pause\medskip 
        
        Como $1$ pertenece a $S$, $m\not=1$. Se sigue que
        $m-1$ pertenece a $\mathbb N$ y como $m$ es el mínimo de
        $S^{\text{c}}$, $m-1$ debe pertenecer a $S$.
        
        \pause\medskip 
        
         Poniendo $k=m-1$ en
        la condición (b), concluimos que $m$ esta en $S$, lo cual
        contradice el hecho de que $m$ pertenece a $S^{\text{c}}$.
        
        \pause\medskip 
        
         De este
        modo, la suposición $S \not= \mathbb N$ nos lleva a un absurdo, y
        por lo tanto tenemos $S= \mathbb N$. \qed
    %\end{proof}
\end{frame}

\begin{frame}    

    
    El principio de inducción es útil para probar la veracidad de propiedades relativas a los
    números naturales. Por ejemplo, consideremos la siguiente propieda de $P(n)$:
    \medskip 
    \begin{itemize}
        \item $P(n)$ es la propiedad: \quad $\displaystyle\sum_{i=1}^n 2i-1 = n^2$,
    \end{itemize}
\medskip 
    Hemos notado que $P(n)$ es verdadera para cualquier $n$ natural y lo podríamos probar usando el siguiente teorema: 
    \medskip 



\end{frame}

\begin{frame}    
    
    
    
    \begin{teorema}[(Principio de inducción)] 
    \vskip .2cm 
    {\it Sea $P(n)$ una propiedad para $n \in \mathbb N$ tal que:
        \begin{enumerate}
            \item[a)] $P(1)$ es verdadera.
            \item[b)] Para todo $k \in \mathbb N$, $P(k)$ verdadera implica $P(k + 1)$ verdadera.
        \end{enumerate}
        Entonces $P(n)$ es verdadera para todo $n \in \mathbb N$. }
    \end{teorema}
    \vskip .2cm \pause
%\begin{proof}[Demostración] 
    {\color{blue} Demostración.}
    \vskip .2cm 

    \pause
    Basta tomar
    $$S = \{n \in \mathbb N| P(n) \text{ es verdadera} \}.$$ \pause
    Entonces $S$ es un subconjunto de $\mathbb N$ y las condiciones (a) y (b) nos dicen que $1 \in S$ y  si $ k \in S$ entonces $k+1\in S$. 
    \vskip .2cm 
    \pause Por el teorema anterior se sigue que $S= \mathbb N$, es decir que $P(n)$ es verdadera para todo $n$.
    natural.\qed
    
    \vskip 1cm
%\end{proof}
\end{frame}

\begin{frame}    

    
        En la práctica, generalmente presentamos una ``demostración por
    inducción'' en términos más descriptivos. 
    
    \medskip
    
    En la notación del teorema anterior, 
    \medskip
    \begin{itemize}
        \item (a) es llamado  el {\em caso base},
        \item  (b) es llamado el  {\em paso inductivo} y
        \item  $P(k)$ es llamada la {\em hipótesis inductiva}.
    \end{itemize}
    \medskip
    El paso inductivo  consiste en probar que $P(k) \Rightarrow P(k + 1)$ o, equivalentemente, podemos suponer $P(k)$ verdadera y a partir de ella probar $P(k + 1)$. 
    
\end{frame}

\begin{frame}
        
    \begin{ejemplo} 
    El entero $x_n$ esta definido recursivamente por
        $$
        x_1=2, \qquad x_n=x_{n-1} +2n, \qquad n\ge 2.
        $$
        Demostremos que  \qquad 
        $
        x_n = n(n+1)  \text{ para todo } n\in \mathbb N.
        $
    \end{ejemplo}
    \vskip .2cm\pause
\begin{proof}
    \noindent({\it Caso  base}) El resultado es verdadero
        cuando $n=1$ pues $ 2 = 1 \cdot 2$.
        \vskip .6cm\pause
        \setbeamercolor{postit}{fg=black,bg=example text.fg!75!black!10!bg}
    \hskip 2cm\begin{beamercolorbox}[wd=0.7\textwidth,rounded=true,shadow=true]{postit}
        \textit{\quad Caso base $n=1$:\quad $x_1 = 2 = 1\cdot 2 = n(n+1)$}
\end{beamercolorbox}
        \medskip\pause
\end{proof}

\end{frame}

\begin{frame}

\begin{proof}
        \noindent ({\it Paso  inductivo})
        Supongamos que el resultado verdadero cuando $n=k$, o
        sea, que 
        \vskip .2cm
        \setbeamercolor{postit}{fg=black,bg=example text.fg!75!black!10!bg}
        \hskip 1.8cm \begin{beamercolorbox}[wd=0.55\textwidth,rounded=true,shadow=true]{postit}
            \; $x_k = k(k+1)$ hipótesis inductiva (HI).
        \end{beamercolorbox}
        \pause
        \vskip .2cm
        En el paso inductivo queremos probar que 
        \vskip .2cm
        \setbeamercolor{postit}{fg=black,bg=example text.fg!75!black!10!bg}
        \hskip 1.8cm \begin{beamercolorbox}[wd=0.65\textwidth,rounded=true,shadow=true]{postit}
            \; $x_k = k(k+1)$ \; $\Rightarrow$ \; $x_{k+1} = (k+1)(k+2)$.
        \end{beamercolorbox}
        \pause
        \vskip .2cm
    Entonces

        \medskip
        
        
        \begin{tabular}{lllll}
        $x_{k+1}$ &$=$& $x_k + 2(k+1)$ &\qquad &$\text{(por la definición recursiva)}$ \\
        &$=$& $k(k+1)+2(k+1)$ &\qquad &$\text{(por hipótesis inductiva)}$ \\
        &$=$& $(k+1)(k+2).$&\qquad &$\text{($(k+1)$ factor común)}$
        \end{tabular}
        \medskip
        \pause
        
        Luego el resultado es verdadero cuando $n=k+1$ y por el principio de inducción, es verdadero para todos los enteros positivos $n$.\qed
    \end{proof}
    
\end{frame}


\begin{frame}\frametitle{} 

    \begin{ejercicio}[(Serie aritmética)]\label{serie-aritmetica}
        Probar que 
        $$
        1 + 2+ 3 + \cdots + n = \frac{n(n+1)}{2},
        $$
        para $n \in \N$.
    \end{ejercicio}

    \vskip .2cm\pause
    \begin{proof}
        \vskip .4cm \pause
        \noindent({\it Caso  base}) El resultado es verdadero
            cuando $n=1$ pues $ 1 = \frac{1 \cdot (1+1)}{2}$.
        
        \vskip .4cm 

    \end{proof}
    
\end{frame}


\begin{frame}
    \frametitle{}

        \noindent ({\it Paso  inductivo}) 
        \pause \vskip .4cm
        
        Supongamos que el resultado verdadero cuando $n=k$, o
        sea, que 
        $$
        \sum_{i=1}^{k} i = \frac{k(k+1)}{2}\quad \text{hipótesis inductiva (HI).}
        $$
        \vskip .4cm\pause
        En el paso inductivo queremos probar que 
        $$
        \sum_{i=1}^{k} i = \frac{k(k+1)}{2} \quad \Rightarrow \quad    \sum_{i=1}^{k+1} i =\frac{(k+1)(k+2)}{2}.
        $$
\vskip 1cm

\end{frame}

\begin{frame}
    \frametitle{}

    Entonces

    \medskip
    
    
    \begin{tabular}{lllll}
        $\displaystyle\sum_{i=1}^{k+1} i$ &$=$& $\displaystyle\sum_{i=0}^{k} i + (k+1)$ &\qquad &$\text{(por la definición recursiva de $\sum$)}$ \\[0.6cm]
    &$=$& $\displaystyle\frac{k(k+1)}{2}+(k+1)$ &\qquad &$\text{(por hipótesis inductiva)}$ \\[0.6cm]
    &$=$& $\displaystyle(k+1)(\frac{k}{2}+1))$ &\qquad &$\text{($(k+1)$ factor común)}$ \\[0.6cm]
    &$=$& $\displaystyle(k+1)\frac{(k+2)}{2}.$&\qquad &\\[0.6cm]
    &$=$& $\displaystyle\frac{(k+1)(k+2)}{2}.$&\qquad &
    \end{tabular}
    \medskip
    \pause
    
    Luego el resultado es verdadero cuando $n=k+1$ y por el principio de inducción, es verdadero para todos los enteros positivos $n$.\qed

\end{frame}


\begin{frame}
    \begin{observacion} En los teoremas del principio de inducción podemos reemplazar $\N$ por $\N_0$ y el teorema queda:

        \begin{teorema}[(Principio de inducción)] 
            \vskip .2cm 
            {\it Sea $P(n)$ una propiedad para $n \in \mathbb N_0$ tal que:
                \begin{enumerate}
                    \item[a)] $P(0)$ es verdadera.
                    \item[b)] Para todo $k \in \mathbb N_0$, $P(k)$ verdadera implica $P(k + 1)$ verdadera.
                \end{enumerate}
                Entonces $P(n)$ es verdadera para todo $n \in \mathbb N_0$. }
            \end{teorema}

    \end{observacion}
\end{frame}


\begin{frame}\frametitle{} 

    \begin{ejercicio}[(Serie geométrica)]\label{serie-geometrica}
        Probar que 
        $$
        1 + q+ q^2 + \cdots + q^n = \sum_{i=0}^{n} q^i = \frac{q^{n+1}-1}{q -1},
        $$
        para $n \in \N_0$, $q >0$ y $q \ne 1$.
    \end{ejercicio}

    \vskip .2cm\pause
    \begin{proof} \pause
        \vskip .2cm 
        \noindent({\it Caso  base}) El resultado es verdadero
            cuando $n=0$ pues 
            $$q^0 = 1 =  \frac{q^{0+1}-1}{q -1}.$$
        
        \vskip .4cm 

    \end{proof}
    
\end{frame}

\begin{frame}
    \frametitle{}

        \noindent ({\it Paso  inductivo})
        Supongamos que el resultado verdadero cuando $n=k$, o
        sea, que 
        $$
        \sum_{i=0}^{k} q^i = \frac{q^{k+1}-1}{q -1},\quad \text{hipótesis inductiva (HI).}
        $$
        \vskip .4cm\pause
        En el paso inductivo queremos probar que 
        $$
        \sum_{i=0}^{k} q^i = \frac{q^{k+1}-1}{q -1} \quad \Rightarrow \quad   \sum_{i=0}^{k+1} q^i = \frac{q^{k+2}-1}{q -1}.
        $$
\vskip 1cm

\end{frame}

\begin{frame}
    \frametitle{}

    Entonces

    \medskip
    
    
    \begin{tabular}{lllll}
        $\displaystyle\sum_{i=0}^{k+1} q^i $ &$=$& $\displaystyle\sum_{i=0}^{k} q^i  + q^{k+1}$ &\qquad &$\text{(por la definición recursiva)}$ \\[0.6cm]
    &$=$& $\displaystyle\ \frac{q^{k+1}-1}{q -1}+ q^{k+1}$ &\qquad &$\text{(por hipótesis inductiva)}$ \\[0.6cm]
    &$=$& $\displaystyle\ \frac{q^{k+1}-1+ (q-1) q^{k+1}}{q -1}$ &\qquad & \\[0.6cm]
    &$=$& $\displaystyle\ \frac{q^{k+1}-1+ q\cdot q^{k+1}- q^{k+1}}{q -1}$ &\qquad &\\[0.6cm]
    &$=$& $\displaystyle\ \frac{q^{k+2}-1}{q -1}$ &\qquad &
    \end{tabular}
    \medskip
    \pause
    
    Luego el resultado es verdadero cuando $n=k+1$ y por el principio de inducción, es verdadero para todos los enteros positivos $n$.\qed

\end{frame}


\end{document}

