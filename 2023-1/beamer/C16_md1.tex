\documentclass{beamer} 
%\documentclass[handout]{beamer} % sin pausas
\usetheme{CambridgeUS}
\setbeamertemplate{background}[grid][step=8 ] % cuadriculado

\usepackage{etex}
\usepackage{t1enc}
\usepackage[spanish,es-nodecimaldot]{babel}
\usepackage{latexsym}
\usepackage[utf8]{inputenc}
\usepackage{verbatim}
\usepackage{multicol}
\usepackage{amsgen,amsmath,amstext,amsbsy,amsopn,amsfonts,amssymb}
\usepackage{amsthm}
\usepackage{calc}         % From LaTeX distribution
\usepackage{graphicx}     % From LaTeX distribution
\usepackage{ifthen}
%\usepackage{makeidx}
\input{random.tex}        % From CTAN/macros/generic
\usepackage{subfigure} 
\usepackage{tikz}
\usepackage[customcolors]{hf-tikz}
\usetikzlibrary{arrows}
\usetikzlibrary{matrix}
\tikzset{
    every picture/.append style={
        execute at begin picture={\deactivatequoting},
        execute at end picture={\activatequoting}
    }
}
\usetikzlibrary{decorations.pathreplacing,angles,quotes}
\usetikzlibrary{shapes.geometric}
\usepackage{mathtools}
\usepackage{stackrel}
%\usepackage{enumerate}
\usepackage{enumitem}
\usepackage{tkz-graph}
\usepackage{polynom}
\polyset{%
    style=B,
    delims={(}{)},
    div=:
}
\renewcommand\labelitemi{$\circ$}
\setlist[enumerate]{label={(\arabic*)}}

\setbeamertemplate{itemize item}{$\circ$}
\setbeamertemplate{enumerate items}[default]
\definecolor{links}{HTML}{2A1B81}
\hypersetup{colorlinks,linkcolor=,urlcolor=links}


\newcommand{\Id}{\operatorname{Id}}
\newcommand{\img}{\operatorname{Im}}
\newcommand{\nuc}{\operatorname{Nu}}
\newcommand{\im}{\operatorname{Im}}
\renewcommand\nu{\operatorname{Nu}}
\newcommand{\la}{\langle}
\newcommand{\ra}{\rangle}
\renewcommand{\t}{{\operatorname{t}}}
\renewcommand{\sin}{{\,\operatorname{sen}}}
\newcommand{\Q}{\mathbb Q}
\newcommand{\R}{\mathbb R}
\newcommand{\C}{\mathbb C}
\newcommand{\K}{\mathbb K}
\newcommand{\F}{\mathbb F}
\newcommand{\Z}{\mathbb Z}
\newcommand{\N}{\mathbb N}
\newcommand\sgn{\operatorname{sgn}}
\renewcommand{\t}{{\operatorname{t}}}
\renewcommand{\figurename }{Figura}

\include{definiciones}

\newcommand{\nc}{\newcommand}

%%%%%%%%%%%%%%%%%%%%%%%%%LETRAS

\nc{\FF}{{\mathbb F}} \nc{\NN}{{\mathbb N}} \nc{\QQ}{{\mathbb Q}}
\nc{\PP}{{\mathbb P}} \nc{\DD}{{\mathbb D}} \nc{\Sn}{{\mathbb S}}
\nc{\uno}{\mathbb{1}} \nc{\BB}{{\mathbb B}} \nc{\An}{{\mathbb A}}

\nc{\ba}{\mathbf{a}} \nc{\bb}{\mathbf{b}} \nc{\bt}{\mathbf{t}}
\nc{\bB}{\mathbf{B}}

\nc{\cP}{\mathcal{P}} \nc{\cU}{\mathcal{U}} \nc{\cX}{\mathcal{X}}
\nc{\cE}{\mathcal{E}} \nc{\cS}{\mathcal{S}} \nc{\cA}{\mathcal{A}}
\nc{\cC}{\mathcal{C}} \nc{\cO}{\mathcal{O}} \nc{\cQ}{\mathcal{Q}}
\nc{\cB}{\mathcal{B}} \nc{\cJ}{\mathcal{J}} \nc{\cI}{\mathcal{I}}
\nc{\cM}{\mathcal{M}} \nc{\cK}{\mathcal{K}}

\nc{\fD}{\mathfrak{D}} \nc{\fI}{\mathfrak{I}} \nc{\fJ}{\mathfrak{J}}
\nc{\fS}{\mathfrak{S}} \nc{\gA}{\mathfrak{A}}
%%%%%%%%%%%%%%%%%%%%%%%%%LETRAS



\title[Clase 16 - Teorema de Fermat / RSA]{Matemática Discreta I \\ Clase 16 - Teorema de Fermat / RSA}
%\author[A. Tiraboschi]{Alejandro Tiraboschi}
\institute[]{\normalsize FAMAF / UNC
    \\[\baselineskip] ${}^{}$
    \\[\baselineskip]
}
\date[16/05/2023]{16 de mayo de 2023}




\begin{document}
    
    \frame{\titlepage} 
    
    \begin{frame}\frametitle{El Teorema (pequeño) de Fermat}
        El siguiente lema nos sirve de preparación para la demostración
        del Teorema (o fórmula) de Fermat.
        
        \vskip .4cm \pause 
        \begin{lema} \label{l4.3} Sea $p$ un número primo, entonces
            \begin{enumerate}
                \item[(a)] $p|\binom{p}{r}$, con $0< r <p$, \vskip .4cm \pause 
                \item[(b)] $(a+b)^p \equiv a^p+b^p \pmod{p}$. 
            \end{enumerate}
        \end{lema}
        
    \vskip 3cm
    \end{frame}
    
    
    \begin{frame}
        {}

        {\color{blue}Demostración}\pause 
        \vskip .2cm

        \noindent({a})  Escribamos el número binomial de otra forma: 
            $$
            \binom{p}{r}=\frac{p!}{r!(p-r)!}=p\cdot\,\frac{(p-1)!}{r!(p-r)!},
            $$ 
            luego 
            \begin{equation*}\label{binp}
                \binom{p}{r} \cdot r!(p-r)! = p \cdot(p-1)!.
            \end{equation*}\pause 
            Por lo tanto, 
            \vskip .2cm 
            \begin{enumerate}
                \item[(1)] $p | \binom{p}{r} \cdot r!(p-r)!$. Además, \pause 
                \vskip .3cm 
                
                \item[(2)] $r < p$ $\Rightarrow$ $p \not| r!$. \vskip .3cm \pause 
                
                \item[(3)] $r >0$ $\Rightarrow$ $p-r < p$ $\Rightarrow$ $p \not| (p-r)!$.
            \end{enumerate}
            
            
    \end{frame}
    
    \begin{frame}
        
        De (1), (2) y (3),  
        
        \begin{equation*}
            p | \binom{p}{r} \cdot r!(p-r)!\qquad \wedge \qquad p \not| r!(p-r)!
        \end{equation*}\pause 
        
        por lo tanto ($p$ es primo)
        \begin{equation*}
            p | \binom{p}{r}.
        \end{equation*}
        \vskip .3cm \pause 
        \noindent({b}) Por el teorema del binomio sabemos que
        $$
        (a+b)^p =\sum_{i=0}^{p} \binom{p}{i} a^ib^{p-i}.
        $$\pause 
        Por ({a}) es claro que $ \binom{p}{i} a^ib^{p-i}\equiv 0 \pmod{p}$,
        si $0<i<p$.
        \vskip .3cm \pause 
        Luego se deduce el resultado.
        \qed
    \end{frame}
    
    \begin{frame}
        El siguiente es el llamado teorema de Fermat.
        
        \begin{teorema}\label{t4.3} Sea $p$ un número primo y $a$ número
            entero. Entonces
            $$
            a^p\equiv a\pmod{p}.
            $$
        \end{teorema}\pause 
        \begin{proof} \pause 
            
            Dividiremos la demostración en 2 casos (1) $a\ge 0$, (2)  $a < 0$.
            \vskip .3cm\pause 
            {(1) $a\ge 0$.}     Por inducción sobre $a$. 
            \vskip .3cm
            \textit{Caso base $a=0$.} $0^p\equiv 0\pmod{p}$,  es trivial.
            \vskip .3cm
            
            
        \end{proof}
    \end{frame}
    
    
    \begin{frame}
        \textit{Paso  inductivo.} Si $k \ge 0$, la hipótesis inductiva es:
        \begin{equation}
            k^p \equiv k \pmod{p}. \tag{HI}
        \end{equation}
        
        Debemos probar,
        \begin{equation}
            (k+1)^p \equiv k+1 \pmod{p}. \tag{T}
        \end{equation}
        \pause 
        Ahora bien,
        \begin{align*}
            (k+1)^p &\equiv   k^p + 1^p \pmod{p}&\quad&(\text{por (b) del  lema}) \\
            &\equiv k +1 \pmod{p}&\quad&(\text{por HI}).  \\
        \end{align*}
        Es  decir $(k+1)^p \equiv  k+1 \pmod{p}$,  que es lo que queríamos probar.
    \end{frame}
    
    
    
    
    
    \begin{frame}
        
        
        {(2) $a<0$.} \pause    Como $a<0$, entonces $-a>0$, luego por (1): 
        
        $(-a)^p \equiv -a \pmod{p}$ o,  equivalentemente
        \begin{equation}\label{eq:fermat-menos-1}
            (-1)^pa^p \equiv (-1)a \pmod{p}
        \end{equation}
        
        \pause 
        Ahora bien,
        \vskip .3cm 
        
        $p>2$, entonces $(-1)^p=-1$,  en particular  $(-1)^p \equiv -1 \pmod{p}$.
        
        \vskip .3cm 
        $p=2$, entonces $(-1)^p=1$, pero como $1\equiv -1 \pmod{2}$, $(-1)^p \equiv -1 \pmod{p}$.
        \vskip .3cm 
        \pause 
        Luego  $(-1)^p \equiv -1 \pmod{p}$ para todo $p$ primo y la ecuación (\ref{eq:fermat-menos-1}) es equivalente a:
        \begin{equation*}\label{eq:fermat-menos}
            (-1)a^p \equiv (-1)a \pmod{p}
        \end{equation*}
        \vskip .3cm \pause 
        Multiplicando por $-1$ la ecuación  obtenemos $a^p \equiv a \pmod{p}$. \qed
        
        
    \end{frame}
    
    
    \begin{frame}
        Supongamos que $a$ y $p$ son coprimos, por Fermat 
        $$
        p|(a^p     -a)=a(a^{(p-1)} -1).
        $$
        
        Como $p$ no divide a $a$, tenemos que $p|(a^{(p-1)} -1)$, \pause es decir 
        \vskip .4cm 
        \begin{teorema}
            Si  $a$ y $p$ coprimos y $p$ es primo, entonces  
            $$a^{(p-1)}\equiv 1\pmod{p}.
            $$
        \end{teorema}
        
        \vskip .3cm \pause 
        Este último enunciado es también conocido como teorema de Fermat.
    \end{frame}
    
    \begin{frame}
        
        
        \begin{definicion} Sea $n \ge 1$, La \textit{función de Euler} se define
            \begin{equation*}
                \phi(n) := |\{x \in \mathbb N: \operatorname{mcd}(x,n) =1  \wedge  x < n \}|.
            \end{equation*}
            donde $|\cdot|$ significa la cardinalidad del conjunto.
        \end{definicion}
        \vskip .3cm 
        El     teorema de Fermat, 2° versión, admite la siguiente generalización, llamada
        teorema de Euler: 
        \vskip .3cm 
        \begin{teorema}
            Si $n$ un entero
            positivo y $a$ un número entero coprimo con $n$, entonces
            $$
            a^{\phi(n)} \equiv 1\pmod{n}.
            $$
        \end{teorema}
        
        
    \end{frame}
    
    \begin{frame}
        \begin{ejemplo}
            Usar el teorema de Fermat, 2° versión, para calcular el resto de dividir $3^{332}$ por $23$.
        \end{ejemplo}
        
        \begin{solucion}
            \vskip .3cm 
            Como $23$ es un  número  primo (y  es coprimo con 3), por el teorema de Fermat (2º versión):
            $$
            3^{22} \equiv 1 \pmod{23}.
            $$
            Ahora bien: $332 = 22 \cdot 15 + 2$, 
            Luego
            \begin{equation*}
                3^{332} \equiv  3^{22\cdot 15 + 2} \equiv  3^{22\cdot 15} 3^2\equiv  ( 3^{22})^{15} 3^2\equiv 1^{15} 3^2\equiv    3^2\equiv   9 \pmod{23}
            \end{equation*}
            Luego  el resto de dividir $3^{332}$ por $23$ es $9$. \qed
            
        \end{solucion}
    \end{frame}
    
    \begin{frame}\frametitle{Algoritmo RSA}
        Dados primos distintos $p$ y $q$ suficientemente grandes tomamos $n = pq$. 
        \begin{itemize}
            \item[$\circ$]  Sea $e$ con $1 < e < (p-1)(q-1)$ tal que $$\operatorname{mcd}(e, (p-1)(q-1)) = 1.$$
            \item[$\circ$]  Sea $d$ tal $0 \le d <(p-1)(q-1)$ y que $$ed \equiv 1 \pmod{(p-1)(q-1)}.$$ 
        \end{itemize}
        
        \begin{proposicion} Si $0 \le m <n$, entonces
            $$m\equiv m^{ed} \pmod{n}.$$
        \end{proposicion} 
        
        
    \end{frame}
    
    
    \begin{frame}\frametitle{Algoritmo RSA - procedimiento}
        
        Decimos que:
        \begin{itemize}
            \item[$\circ$]  $(e,n)$ es la \textit{clave pública.}
            \item[$\circ$]  $d$ es la \textit{clave privada.}
        \end{itemize}
        
        \vskip .2cm    
        $A$ le quiere enviar  un mensaje encriptado a $B$.
        \vskip .2cm             
        \textbf{Preliminares}
        
        \begin{enumerate}
            \item[$\circ$] $A$ conoce la clave pública $(e,n)$. 
            \item[$\circ$] $B$ conoce la clave pública y una clave privada $d$. 
        \end{enumerate}
        \vskip .2cm    
        \textbf{Protocolo }
        
        \begin{enumerate}
            \item[$\circ$] $A$  le quiere enviar el mensaje $m$  a $B$. 
            \item[$\circ$] $A$ calcula $c \equiv m^e \pmod{n}$ y  le envía $c$ a $B$
            \item[$\circ$] $B$ descifra el mensaje:  $c^d \equiv (m^e)^d \equiv m \pmod{n}$.
        \end{enumerate}
        
    \end{frame}
    
    
    \begin{frame}
        
        \begin{observacion}
            
            
            \begin{itemize}
                \item[$\circ$]\vskip .4cm    
                Los dos primos $p$ y $q$ deberían tener alrededor de $100$ dígitos cada uno (longitud considerada segura en este momento).
                \item[$\circ$]\vskip .4cm    
                El número $e$ puede elegirse pequeño y se selecciona haciendo prueba y error con el algoritmo de Euclides, es decir probando hasta encontrar un $e$ tal que $\operatorname{mcd}(e, (p-1)(q-1)) = 1$.
                \item[$\circ$]\vskip .4cm    
                La existencia de $d$ está garantizada por  la ecuación lineal de congruencia), pues $e$ y $(p-1)(q-1)$ son coprimos.
            \end{itemize}
            
        \end{observacion}
    \end{frame}
    
    \begin{frame}\frametitle{Consideraciones finales sobre el algoritmo RSA}

        Hay dos ``obstrucciones'' para una implementación del algoritmo RSA:
        
        \vskip .4cm 
            \begin{enumerate}
                \item ¿Cómo calcular un número elevado a una potencia de más de 200 dígitos? Hay dos problemas 
                \begin{enumerate}
                    \item[(a)] La cantidad de multiplicaciones necesarias va más allá de $2^{200}$, imposibles de realizar.
                    \item[(b)] Los números se tornan tan grandes que no entrarían en ninguna memoria.
                \end{enumerate}
                
                \item   
                ¿Cómo saber si un número de más de 100 dígitos es primo o no?\\
                No  es posible conocer los divisores  de un número de ese tamaño.
            \end{enumerate}

    \end{frame}

    \begin{frame}\frametitle{Consideraciones finales sobre el algoritmo RSA}

        El problema (1) es fácil de resolver, se usa la técnica llamada \textit{exponenciación modular},  explicada en el apunte, capítulo 4, sección 5.
        \begin{itemize}
            \item La idea para (1)(a) es usar la propiedad siguiente: si $m = 2q +r$, entonces 
            $$
            a^m = (a^q)^2 a^r
            $$
            y definir recursivamente la potencia.
            \item La idea para (1)(b) es usar la idea de (1)(a) y en cada paso reducir módulo $n$. Es decir,  utilizar la propiedad
            $$
            a^m \equiv (a^q)^2 a^r \equiv s\pmod{n}
            $$
            con $0 \le s < n$.
        \end{itemize}

        \vskip .4 cm

        
    \end{frame}
    
    \begin{frame}\frametitle{Consideraciones finales sobre el algoritmo RSA}

        El  problema (2) es más complicado y se usa el llamado el test de primalidad de Miller-Rabin probabilístico. 
        
        \vskip .4 cm
        
        En  el apunte, capítulo 4, sección 6 se explica en que consiste este test.

        \vskip 4 cm
    \end{frame}
    
\end{document}

