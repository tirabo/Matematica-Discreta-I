%\documentclass{beamer} 
\documentclass[handout]{beamer} % sin pausas
\usetheme{CambridgeUS}
%\setbeamertemplate{background}[grid][step=8 ] % cuadriculado


\usepackage{etex}
\usepackage{t1enc}
\usepackage[spanish,es-nodecimaldot]{babel}
\usepackage{latexsym}
\usepackage[utf8]{inputenc}
\usepackage{verbatim}
\usepackage{multicol}
\usepackage{amsgen,amsmath,amstext,amsbsy,amsopn,amsfonts,amssymb}
\usepackage{amsthm}
\usepackage{calc}         % From LaTeX distribution
\usepackage{graphicx}     % From LaTeX distribution
\usepackage{ifthen}
%\usepackage{makeidx}
\input{random.tex}        % From CTAN/macros/generic
\usepackage{subfigure} 
\usepackage{tikz}
\usepackage[customcolors]{hf-tikz}
\usetikzlibrary{arrows}
\usetikzlibrary{matrix}
\tikzset{
    every picture/.append style={
        execute at begin picture={\deactivatequoting},
        execute at end picture={\activatequoting}
    }
}
\usetikzlibrary{decorations.pathreplacing,angles,quotes}
\usetikzlibrary{shapes.geometric}
\usepackage{mathtools}
\usepackage{stackrel}
%\usepackage{enumerate}
\usepackage{enumitem}
\usepackage{tkz-graph}
\usepackage{polynom}
\polyset{%
    style=B,
    delims={(}{)},
    div=:
}
\renewcommand\labelitemi{$\circ$}
\setlist[enumerate]{label={(\arabic*)}}
\setbeamertemplate{itemize item}{$\circ$}
\setbeamertemplate{enumerate items}[default]
\definecolor{links}{HTML}{2A1B81}
\hypersetup{colorlinks,linkcolor=,urlcolor=links}


\newcommand{\Id}{\operatorname{Id}}
\newcommand{\img}{\operatorname{Im}}
\newcommand{\nuc}{\operatorname{Nu}}
\newcommand{\im}{\operatorname{Im}}
\renewcommand\nu{\operatorname{Nu}}
\newcommand{\la}{\langle}
\newcommand{\ra}{\rangle}
\renewcommand{\t}{{\operatorname{t}}}
\renewcommand{\sin}{{\,\operatorname{sen}}}
\newcommand{\Q}{\mathbb Q}
\newcommand{\R}{\mathbb R}
\newcommand{\C}{\mathbb C}
\newcommand{\K}{\mathbb K}
\newcommand{\F}{\mathbb F}
\newcommand{\Z}{\mathbb Z}
\newcommand{\N}{\mathbb N}
\newcommand\sgn{\operatorname{sgn}}
\renewcommand{\t}{{\operatorname{t}}}
\renewcommand{\figurename }{Figura}

\include{definiciones}

\newcommand{\nc}{\newcommand}

%%%%%%%%%%%%%%%%%%%%%%%%%LETRAS

\nc{\FF}{{\mathbb F}} \nc{\NN}{{\mathbb N}} \nc{\QQ}{{\mathbb Q}}
\nc{\PP}{{\mathbb P}} \nc{\DD}{{\mathbb D}} \nc{\Sn}{{\mathbb S}}
\nc{\uno}{\mathbb{1}} \nc{\BB}{{\mathbb B}} \nc{\An}{{\mathbb A}}

\nc{\ba}{\mathbf{a}} \nc{\bb}{\mathbf{b}} \nc{\bt}{\mathbf{t}}
\nc{\bB}{\mathbf{B}}

\nc{\cP}{\mathcal{P}} \nc{\cU}{\mathcal{U}} \nc{\cX}{\mathcal{X}}
\nc{\cE}{\mathcal{E}} \nc{\cS}{\mathcal{S}} \nc{\cA}{\mathcal{A}}
\nc{\cC}{\mathcal{C}} \nc{\cO}{\mathcal{O}} \nc{\cQ}{\mathcal{Q}}
\nc{\cB}{\mathcal{B}} \nc{\cJ}{\mathcal{J}} \nc{\cI}{\mathcal{I}}
\nc{\cM}{\mathcal{M}} \nc{\cK}{\mathcal{K}}

\nc{\fD}{\mathfrak{D}} \nc{\fI}{\mathfrak{I}} \nc{\fJ}{\mathfrak{J}}
\nc{\fS}{\mathfrak{S}} \nc{\gA}{\mathfrak{A}}
%%%%%%%%%%%%%%%%%%%%%%%%%LETRAS



\title[Clase 6 - Conteo]{Matemática Discreta I \\ Clase 6 - Conteo}
%\author[C. Olmos / A. Tiraboschi]{Carlos Olmos / Alejandro Tiraboschi}
\institute[]{\normalsize FAMAF / UNC
    \\[\baselineskip] ${}^{}$
    \\[\baselineskip]
}
\date[30/03/2023]{39 de marzo de 2023}




\begin{document}
%\title{El centro geográfico de Argentina}   
%\author{} 
%\date{Villa Huidobro \\ 4/12/2018} 



\frame{\titlepage} 

%\frame{\frametitle{Índice}\tableofcontents} 



\begin{frame}\frametitle{Problemas de conteo}     

    Contar,  a veces, no es una tarea sencilla. \pause

    \vskip .4cm

    Por ejemplo: \pause


    \begin{itemize}
        \item ¿Cuántas chapas patente  es posible hacer con el actual esquema de numeración? Nos referimos a las patentes de automóviles en Argentina.  \pause
        \item ¿De cuantas formas se pueden elegir 7 personas entre 20? (no importa el orden de la elección)
    \end{itemize}
    \pause
    \vskip .4cm

    En las próximas clases podremos resolver estos dos problemas utilizando \textit{técnicas de conteo.}
    \pause
    \vskip .4cm

    Las técnicas de conteo son estrategias matemáticas que permiten determinar el número total de resultados que pueden haber a partir de hacer combinaciones dentro de un conjunto o conjuntos de objetos.

\end{frame}

\begin{frame}\frametitle{Cardinal de un conjunto }     

    Un conjunto $A$ es finito si podemos contar la cantidad de elementos que tiene. En ese caso denotaremos $|A|$ la cantidad de elementos de $A$ y la llamaremos el {\em cardinal de $A$}\
    \pause
    \vskip .6cm
    Por  ejemplo, los conjuntos
    \begin{equation*}
        A = \{ a, b, z, x, 1\}, \qquad B = \{ 1, 2, 3, 4, 5\}
    \end{equation*}
    tienen 5 elementos cada uno. Es decir $|A| =5$ y $|B|= 5$.

    \vskip .6cm\pause

    Conjuntos como $\mathbb Z$, $\mathbb N$ o $\mathbb R$ son infinitos y por lo tanto no tiene sentido hablar de la cantidad de elementos de estos conjuntos.
\end{frame}

\begin{frame}\frametitle{El principio de adición}        
    Dadas dos actividades $X$ e $Y$, si se puede realizar $X$ de $n$ formas distintas o, alternativamente, se puede realizar  $Y$ de $m$ formas distintas. Entonces el número de formas de realizar ``$X$ o $Y$'' es $n + m$.

    \pause
    \vskip .4cm
    {\color{blue} Ejemplo}
    \vskip .2cm 
    Supongamos que una persona va a salir a pasear  y puede ir al cine donde hay $3$ películas en cartel o al teatro donde hay $4$ obras posibles. Entonces, tendrá un total de $3+4=7$ formas distintas de elegir el paseo. 
    \vskip .3cm


    
    

\end{frame}


\begin{frame}
    Este principio, el {\it principio de adición}, es el más básico del conteo y más formalmente dice que si $A$ y $B$ son conjuntos finitos disjuntos, entonces 
\begin{equation*}\label{padd}
|A \cup B| =|A|+|B|.
\end{equation*}
\vskip .4cm  \pause
Se generaliza fácilmente:  
Sean $A_1,\ldots,A_n$ conjuntos finitos tal que $A_i \cap A_j = \emptyset$ cuando $i\not=j$, entonces 
\begin{equation*}
|A_1 \cup \cdots \cup A_n| =|A_1|+\cdots+|A_n|.
\end{equation*}
\pause

\vskip .2cm

Remarcamos que para aplicar el principio de adición es necesario que los eventos se { \bf excluyan mutuamente}. El caso general es
\begin{equation*}
    |A \cup B| =|A|+|B| - |A \cap B|.
\end{equation*}

\end{frame}

\begin{frame}\frametitle{El principio de multiplicación} 
        Suponga que una actividad consiste de $2$ etapas y la primera etapa puede ser realizada de $n$ maneras y la etapa $2$  puede realizarse de $m$  maneras, independientemente de como se ha hecho la etapa $1$. 
        \vskip .2cm
        {\it Principio de multiplicación:} la actividad puede ser realizada de $n\cdot m$  formas distintas.
        
    \pause     
    \vskip .6cm
    {\color{blue} Ejemplo}
    \vskip .2cm
            Supongamos que la persona del ejemplo anterior tiene suficiente tiempo y dinero para ir primero al cine (3 posibilidades) y luego al teatro (4 posibilidades). 
            \vskip .2cm
            Entonces tendrá  $3 \cdot 4=12$ formas distintas de hacer el paseo.

\end{frame}

\begin{frame}
        
    Formalmente, si $A,B$ conjuntos y definimos el {\em producto cartesiano}\index{producto cartesiano} entre $A$ y $B$ por
    $$
    A \times B = \{(a,b): a \in A, b \in B\}.
    $$
        \vskip .5cm
        
        
    Entonces si $A$ y $B$ son conjuntos finitos se cumple que
    $$
    |A \times B| = |A|\cdot|B|.  
    $$
    
            \vskip .3cm         
    {\bf Caso  especial:} $A = B$.
    \pause 
    \vskip .3cm
    {\color{blue} Ejemplo}
    \vskip .2cm
    
    ¿Cuántas palabras de dos letras hay? (26 letras,  no importa si las palabras tienen significado)
    \pause
    
    \vskip .2cm
    
    
    {\color{blue} Respuesta:} $26 \cdot 26$. \qed
    \vskip .2cm
    
\end{frame}

\begin{frame}\frametitle{Selecciones ordenadas con repetición}


{\color{blue} Ejemplo}
\vskip .2cm

Sea  $X = \{ 1, 2, 3 \}$.
\vskip .2cm
¿De cuántas formas se pueden elegir dos de estos números en forma ordenada?
    \vskip .2cm\pause
{\bf Notación:} si elegimos $a$ y $b$ en forma ordenada, denotamos $ab$. 
\vskip .2cm
Entonces, las posibilidades son 
    \begin{align*}
    &11&\quad &12&\quad &13 \\
    &21&\quad &22&\quad &23\\
    &31&\quad &32&\quad &33
    \end{align*}
    
\vskip .1cm \pause
    Es decir, hay $9 = 3^2$ formas posibles. \pause
    


\end{frame}

\begin{frame}    

    
Avancemos un poco más y ahora elijamos en forma ordenada $3$ elementos de  ${1,2,3}$, es claro que estas elecciones son
\begin{align*}
&1 1 1&\quad &211&\quad &311 \\
&1 1 2&\quad & 212&\quad & 312\\
&1 1 3&\quad & 213&\quad & 313\\
&1 2 1&\quad & 221&\quad & 321\\
&1 2 2&\quad & 222&\quad & 322\\
&1 2 3&\quad & 223&\quad & 323\\
&1 3 1&\quad & 231&\quad & 331\\
&1 3 2&\quad & 232&\quad & 332\\
&1 3 3&\quad & 233&\quad & 333.
\end{align*}
El total de elecciones posibles $27 = 3^3$. 




\end{frame}

\begin{frame}    
    

Un diagrama arbolado ayuda a pensar. \pause
\begin{center}
    \includegraphics[scale=0.50]{images/arbol_pos.jpg}
\end{center}

%\vskip 8cm
\vskip .2cm ¿Cómo justificamos esto? \pause Por el principio de multiplicación
\end{frame}

\begin{frame}    
    
    De los dos ejemplos anteriores se podría inferir que hay $3^4$ formas posibles de elegir en forma ordenada $4$ elementos del conjunto $1$, $2$, $3$. También, que cuando elijamos $5$ habrá $3^5$ posibilidades y así sucesivamente para enteros más grandes. 

    \pause 

    \vskip .2cm ¿Cómo justificamos esto? \pause     Por el principio de multiplicación

    \pause 

    \vskip .2cm Hemos visto en las página anteriores que hay $3^3$ formas posibles de elegir $3$ números entre $1$, $2$, $3$.

    \vskip .2cm Ahora,  para elegir el cuarto número hay $3$ posibilidades, por lo tanto, por el principio de multiplicación, el número de posibilidades  con cuatro elecciones es
    \begin{equation*}
        3^3 \times 3 = 3^4.
    \end{equation*}
    \end{frame}

\begin{frame}    

    
El razonamiento anterior  se puede extender:

%\begin{proposicion}
\vskip .3cm
{\color{blue} Proposición}
\vskip .2cm
    {\it  Sean  $m,n \in \mathbb N$. Hay   $n^m$ formas posibles de elegir ordenadamente $m$ elementos de un conjunto de $n$ elementos.}
%\end{proposicion}
\vskip .5cm \pause
{\color{blue} Idea de la prueba.} \pause
\vskip .2cm
%\begin{proof}[Idea de la prueba]
    La prueba de esta proposición se basa en aplicar el principio de multiplicación $m-1$ veces, 
    \vskip .2cm
    A nivel formal,  debemos hacer inducción sobre $m$ y usar el principio de multiplicación en el paso inductivo. 

    \qed
%\end{proof}

\end{frame}



\begin{frame}    


%\begin{ejemplo}
    \vskip .3cm
    {\color{blue} Ejemplo}
    \vskip .2cm
    ¿Cuántos números de cuatro dígitos pueden formarse con
    los dígitos $1, 2, 3, 4, 5, 6$?
    \vskip .2cm \pause
    Por la proposición anterior es claro que hay $6^4$ números posibles.
%\end{ejemplo}

%\begin{ejemplo}
\pause
\vskip .8cm
{\color{blue} Ejemplo}
\vskip .2cm
    ¿Cuántos números de $5$ dígitos y capicúas pueden formarse
    con los dígitos $1, 2, 3, 4, 5, 6, 7, 8$? 
    \vskip .2cm \pause
    Un número
    capicúa de cinco dígitos es de la forma
    $$xyzyx$$
    Se reduce a ver cuántos números de tres dígitos pueden
    formarse con aquéllos dígitos. \pause
    Exactamente $8^3$.
%\end{ejemplo}

\end{frame}

\begin{frame}
    
    
        {\color{blue} Ejemplo}
        \vskip .2cm
Sea $X$ un conjunto de $n$ elementos. ¿Cuántos subconjuntos tiene este conjunto?
\vskip .2cm
Por ejemplo, si $X = \{ a, b, c \}$ los subconjuntos de $X$ son:\pause
$$
\emptyset, \{ a \} , \{ b \}, \{ c \}, \{ a, b \}, \{ a, c \}, \{ b, c \}, \{ a, b, c\}.
$$ 
Es decir, si $X$ es un conjunto  de 3 elementos,  entonces tiene $8$ subconjuntos. 
\pause


\vskip .2cm
\begin{tabular}{lll}
    Sea $A \subseteq X$ &$\to$ & $ a \in A$ o  $ a \not\in A$ (2 posibilidades) \\
    &$\to$ & $ b \in A$ o  $ b \not\in A$ (2 posibilidades) \\
    &$\to$ & $ c \in A$ o  $ c \not\in A$ (2 posibilidades) 
\end{tabular}


\vskip .2cm
Luego hay
$$
2 \cdot 2 \cdot 2 = 2^3 = 8
$$
posibles subconjuntos de $X$.  

\end{frame}

\begin{frame}
Razonando de manera análoga obtenemos nuestro primer resultado ``no sencillo'' de conteo. 
\vskip .4cm
%\begin{proposicion} 
{\color{blue}Proposición}
\vskip .2cm
{\it La cantidad de subconjuntos de  un conjunto de $n$ elementos es $2^n$.}
%\end{proposicion}
\vskip .4cm
\pause
Dado  $X$ un conjunto, denotamos $\mathcal P(X)$ el  conjunto  formado por todos los subconjuntos de $X$, por ejemplo
$$
\mathcal P(\{1,2\}) = \{\emptyset,\{1\},\{2\},\{1,2\}\}.
$$  
Si $X$ es un conjunto finito la proposición anterior nos dice que
$$
\mathcal |P(X)| = 2^{|X|}
$$
    
\end{frame}




\end{document}

