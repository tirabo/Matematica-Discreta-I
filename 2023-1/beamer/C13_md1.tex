%\documentclass{beamer} 
\documentclass[handout]{beamer} % sin pausas
\usetheme{CambridgeUS}
%\setbeamertemplate{background}[grid][step=8 ] % cuadriculado

\usepackage{etex}
\usepackage{t1enc}
\usepackage[spanish,es-nodecimaldot]{babel}
\usepackage{latexsym}
\usepackage[utf8]{inputenc}
\usepackage{verbatim}
\usepackage{multicol}
\usepackage{amsgen,amsmath,amstext,amsbsy,amsopn,amsfonts,amssymb}
\usepackage{amsthm}
\usepackage{calc}         % From LaTeX distribution
\usepackage{graphicx}     % From LaTeX distribution
\usepackage{ifthen}
%\usepackage{makeidx}
\input{random.tex}        % From CTAN/macros/generic
\usepackage{subfigure} 
\usepackage{tikz}
\usepackage[customcolors]{hf-tikz}
\usetikzlibrary{arrows}
\usetikzlibrary{matrix}
\tikzset{
    every picture/.append style={
        execute at begin picture={\deactivatequoting},
        execute at end picture={\activatequoting}
    }
}
\usetikzlibrary{decorations.pathreplacing,angles,quotes}
\usetikzlibrary{shapes.geometric}
\usepackage{mathtools}
\usepackage{stackrel}
%\usepackage{enumerate}
\usepackage{enumitem}
\usepackage{tkz-graph}
\usepackage{polynom}
\polyset{%
    style=B,
    delims={(}{)},
    div=:
}
\renewcommand\labelitemi{$\circ$}
\setlist[enumerate]{label={(\arabic*)}}
\setbeamertemplate{itemize item}{$\circ$}
\setbeamertemplate{enumerate items}[default]
\definecolor{links}{HTML}{2A1B81}
\hypersetup{colorlinks,linkcolor=,urlcolor=links}


\newcommand{\Id}{\operatorname{Id}}
\newcommand{\img}{\operatorname{Im}}
\newcommand{\nuc}{\operatorname{Nu}}
\newcommand{\im}{\operatorname{Im}}
\renewcommand\nu{\operatorname{Nu}}
\newcommand{\la}{\langle}
\newcommand{\ra}{\rangle}
\renewcommand{\t}{{\operatorname{t}}}
\renewcommand{\sin}{{\,\operatorname{sen}}}
\newcommand{\Q}{\mathbb Q}
\newcommand{\R}{\mathbb R}
\newcommand{\C}{\mathbb C}
\newcommand{\K}{\mathbb K}
\newcommand{\F}{\mathbb F}
\newcommand{\Z}{\mathbb Z}
\newcommand{\N}{\mathbb N}
\newcommand\sgn{\operatorname{sgn}}
\renewcommand{\t}{{\operatorname{t}}}
\renewcommand{\figurename }{Figura}

\include{definiciones}

\newcommand{\nc}{\newcommand}

%%%%%%%%%%%%%%%%%%%%%%%%%LETRAS

\nc{\FF}{{\mathbb F}} \nc{\NN}{{\mathbb N}} \nc{\QQ}{{\mathbb Q}}
\nc{\PP}{{\mathbb P}} \nc{\DD}{{\mathbb D}} \nc{\Sn}{{\mathbb S}}
\nc{\uno}{\mathbb{1}} \nc{\BB}{{\mathbb B}} \nc{\An}{{\mathbb A}}

\nc{\ba}{\mathbf{a}} \nc{\bb}{\mathbf{b}} \nc{\bt}{\mathbf{t}}
\nc{\bB}{\mathbf{B}}

\nc{\cP}{\mathcal{P}} \nc{\cU}{\mathcal{U}} \nc{\cX}{\mathcal{X}}
\nc{\cE}{\mathcal{E}} \nc{\cS}{\mathcal{S}} \nc{\cA}{\mathcal{A}}
\nc{\cC}{\mathcal{C}} \nc{\cO}{\mathcal{O}} \nc{\cQ}{\mathcal{Q}}
\nc{\cB}{\mathcal{B}} \nc{\cJ}{\mathcal{J}} \nc{\cI}{\mathcal{I}}
\nc{\cM}{\mathcal{M}} \nc{\cK}{\mathcal{K}}

\nc{\fD}{\mathfrak{D}} \nc{\fI}{\mathfrak{I}} \nc{\fJ}{\mathfrak{J}}
\nc{\fS}{\mathfrak{S}} \nc{\gA}{\mathfrak{A}}
%%%%%%%%%%%%%%%%%%%%%%%%%LETRAS




\title[Clase 12 - MCD (2)]{Matemática Discreta I \\ Clase 13 - Máximo común divisor (2)}
%\author[C. Olmos / A. Tiraboschi]{Carlos Olmos / Alejandro Tiraboschi}
\institute[]{\normalsize FAMAF / UNC
    \\[\baselineskip] ${}^{}$
    \\[\baselineskip]
}
\date[02/05/2023]{2 de mayo  de 2023}

\newcommand{\mcd}{\operatorname{mcd}}
\newcommand{\mcm}{\operatorname{mcm}}


\begin{document}
    
    \frame{\titlepage} 
    
    
    
    \begin{frame}\frametitle{Algoritmo de Euclides}\label{film-alg-eucl}
        Para calcular el mcd de enteros $a$ y $b$, con $b >0$, 
        definimos $q_i$ y $r_i$ recursivamente  de la siguiente manera: $r_0 = a$, $r_1 = b$,  y \pause
        \begin{align*}
            &\text{($e_{1}$)}\qquad& r_0&=r_1 q_1 + r_2& &(0 < r_2<r_1)\\
            &\text{($e_{2}$)}\qquad& r_1&=r_2q_2 + r_3\quad{}\quad{}\quad{}& &(0 < r_3<r_2)  \\
            &\text{($e_{3}$)}\qquad& r_2&=r_3q_3 + r_4\quad{}\quad{}\quad{}& &(0 < r_4<r_3)  \\
            &\cdots&&\\
            &\text{($e_{i}$)}\qquad& r_{i-1}&=r_{i}q_{i} + r_{i+1}& &(0 < r_{i+1} <r_{i}) \\
            &\cdots&& \\
            &\text{($e_{k-1}$)}\qquad& r_{k-2}&=r_{k-1}q_{k-1} + r_{k}& &(0 < r_{k} <r_{k-1}) \\
            &\text{($e_{k}$)}\qquad& r_{k-1}&=r_{k}q_{k} + 0 ,&&  
        \end{align*}\pause
        
        Entonces $r_k = \mcd(a,b)$ {\color{gray} \qquad (Se usa en filmina \ref{film-alg-eucl-2})}
    \end{frame}

    

    \begin{frame}
        Ejemplifiquemos el algoritmo de Euclides.\pause
        
        \vskip .4cm
        
        \begin{ejemplo} Encuentre el mcd de 2406 y 654.
        \end{ejemplo}\pause
        \begin{solucion}\pause Tenemos
            \begin{alignat*}3
                2406&=654\cdot3+444,&\quad&\text{ entonces  }&\quad (2406,654)&= (654,444) \\
                654&=444\cdot1+210, &\quad&\text{ entonces  }&(654,444)&= (444,210)\\
                444&=210\cdot2+24,&\quad&\text{ entonces  }&(444,210)&= (210,24)\\
                210&=24\cdot8+18,&\quad&\text{ entonces  }&(210,24)&= (24,18)\\
                24 &=18\cdot1+6,&\quad&\text{ entonces  }&(24,18)&= (18,6)\\
                18&=6\cdot3 + 0&\quad&\text{ entonces  }& (18,6) &= (6,0) = 6 
            \end{alignat*}\pause
            Por lo tanto $(2406,654)=6$.
        \end{solucion}
    \end{frame}

    
    \begin{frame}[fragile=singleslide]
        
        \begin{itemize}
            \item[$\bullet$] El algoritmo de Euclides es fácilmente implementable en un lenguaje de programación.
            \vskip .4cm
            \item[$\bullet$]  A continuación una versión del mismo en pseudocódigo (estilo Python).
        \end{itemize}
        
        
        \vskip .6cm
        
        {\quad\color{blue} Algoritmo de Euclides }
        %\vskip .2cm
        \begin{verbatim}
        # pre: a y b son números positivos
        # post: obtenemos d = mcd(a,b)
        i, j = a, b
        while j != 0:
            # invariante: mcd(a, b) = mcd(i, j)
            resto = i % j  # i = q * j + resto
            i, j = j, resto
        d = i
        \end{verbatim}
        
        
    \end{frame}
    
    \begin{frame}\label{film-alg-eucl-2}
        
        \begin{teorema}\label{prop-d-comb-lin}
            Sean $a,b \in \Z$, alguno de ellos no nulo. Entonces,  existen $s, t \in \Z$ tal que 
            \begin{equation*}
                \operatorname{mcd}(a,b) = sa +tb.
            \end{equation*}
        \end{teorema}
        \vskip -.5cm
        {\color{blue}Demostración.}\pause 
        \vskip .2cm
        Supongamos que $b>0$ y sea $d=\mcd(a,b)$. Entonces, $d$ es el último resto no nulo del algoritmo (ver p. \ref{film-alg-eucl}).
        \vskip .2cm \pause
        Cada resto puede ser calculado en base a los dos anteriores:
        \begin{equation}\label{eq-alg-euclides}
            (e_{i-1})\quad r_{i-2}=r_{i-1}q_{i-1} + r_{i}\quad \Rightarrow \quad  r_{i} = r_{i-2} - r_{i-1}q_{i-1}.
        \end{equation}
        \vskip .2cm \pause
        Aplicando repetidamente (\ref{eq-alg-euclides}) a los restos correpondientes, $\Rightarrow$  cada resto puede ser escrito como combinación lineal entera de $r_0=a$ y $r_1=b$.
        \vskip .2cm \pause
        En particular,  $d=r_k$ puede ser escrito como combinación lineal entera de $r_0=a$ y $r_1=b$.
    \end{frame}

    
    \begin{frame}\label{fil-obtencion-de-r-s}
        
    Más precisamente, de la filmina \ref{film-alg-eucl} :
    \begin{equation*}
        (e_{i-1})\quad r_{i-2}=r_{i-1}q_{i-1} + r_{i}.
    \end{equation*}\pause
        Esto implica que 
        $$
        ({i})\quad r_{i} = r_{i-2} - r_{i-1}q_{i-1}.
        $$\pause
        Lo cual nos dice que $r_{i}$  puede ser calculado usando $r_{i-1}$ y $r_{i-2}$.
        \pause
        \begin{align*}
            d = r_{k} &\stackrel{(k)}{=} r_{k-2}-r_{k-1}q_{i} = s_kr_{k-2}+ t_k r_{k-1} \\
            &\stackrel{(k-1)}{=} s_kr_{k-2}+ t_k (r_{k-3}-r_{k-2}q_{k-2}) =s_{k-1}r_{k-3}+ t_{k-1} r_{k-2} \\
            &\;\cdots\;\cdots\\
            &\stackrel{(3)}{=} s_4r_{2}+ t_k (r_{1}-r_{2}q_{2}) =s_{3}r_{1}+ t_{3} r_{2} \\
            &\stackrel{(2)}{=} s_3r_{1}+ t_3 (r_{0}-r_{1}q_{1}) =s_{2}r_{0}+ t_{2} r_{1} =s_{2}a+ t_{2}b.
        \end{align*}
        
        \qed
        
    \end{frame}


    \begin{frame}
        \frametitle{}

        \begin{corolario}
            Sean $a$ y $b$ enteros, $b$ no nulo, entonces
            $$
            (a,b) = 1 \Leftrightarrow \text{existen $s,t \in \mathbb Z$ tales que $1 = sa+tb$.}
            $$
        \end{corolario}
        \pause
        
    \begin{definicion}
        Si  $(a,b)=1$ entonces decimos que $a$ y $ b$ son {\em coprimos}\index{coprimos}.
    \end{definicion}


    \begin{observacion}
        \textit{NO} es cierto  que si existen $s,t \in \mathbb Z$ tales que $d = sa+tb$ $\Rightarrow$ $d = (a,b)$.
\vskip .4cm
        Por ejemplo, $4 = 2 \cdot 6 + (-2) \cdot 4$ y $(6,4) =2$. 
    \end{observacion}
        
        
    
    \end{frame}

        
    \begin{frame}
        \begin{ejemplo} Encuentre $d$, el mcd de 174 y 72 y  escribir $d = s \cdot 174 + t \cdot 72$.
        \end{ejemplo}\pause
        {\color{blue} Solución}
        \pause 
        \begin{alignat*}5
            174&=72\cdot 2+30,& \;\quad&\Rightarrow&  \;\quad 30 &=     174-72\cdot 2& &\; &\quad\quad\quad&(1)\\
            72&=30\cdot 2+12,& \;&\Rightarrow& 12&=72-30\cdot 2& \; & \; &&(2) \\
            30&=12\cdot 2+6,& \;&\Rightarrow& 6&=    30-12\cdot 2& \; & \; &&(3)\\
            12&=6\cdot 2+0.& \;&&  && &&\\
        \end{alignat*} \pause \vskip -1.0cm
        Por lo tanto,     $(174,72) = 6$ y,
    \end{frame}
    
    
    
    \begin{frame}
        \begin{align*}
            6&\stackrel{(3)}{=}    30-12\cdot 2\\
            &\stackrel{(2)}{=}    30-(72-30\cdot 2)\cdot 2 \\  
            &=    5 \cdot 30 + (-2) \cdot 72\\  
            &\stackrel{(1)}{=} 5 \cdot (    174-72\cdot 2) + (-2) \cdot 72\\   
            &=  5\cdot 174+ (-12) \cdot 72  
        \end{align*}
        \pause
        Concluyendo: 
        \begin{itemize}
            \item     $(174,72) = 6$ y,
            \item  $6= 5\cdot 174+ (-12) \cdot 72$.
        \end{itemize}
        \qed
        
        \vskip 2cm
        %    72&=30\cdot 2+12,& \;&\Rightarrow& 12&=72-30\cdot 2& \; &\Rightarrow&  \; 12&= (-2)\cdot 174+ 5 \cdot 72 \\
        %30&=12\cdot 2+6,& \;&\Rightarrow& 6&=    30-12\cdot 2& \; &\Rightarrow&  \; 6&= 5\cdot 174+ (-12) \cdot 72\\
        
    \end{frame}
    
    \begin{frame}
        \begin{ejemplo} Encuentre $d$, el mcd de 470 y 55 y escribir $d = s \cdot 470 + t \cdot 55$.
        \end{ejemplo} \pause
        \begin{solucion}\pause
            Por el algoritmo de Euclides obtenemos
            \begin{alignat*}4
                470&=55 \cdot 8 +30&\quad\Rightarrow\quad &30 &=&470 + (-8)\cdot 55&\qquad (1)&\\
                55&=30 \cdot 1 + 25&\quad\Rightarrow\quad &25 &=&55 +(-1)\cdot 30&\qquad (2)&\\
                30&=25 \cdot 1+5&\quad\Rightarrow\quad &5 &=&30 +(-1) \cdot 25&\qquad (3)& \\
                25&=5\cdot 5+0.&&&&&&
            \end{alignat*}\pause
            Luego
            \begin{align*}
                5 &\stackrel{(3)}{=} 30 +(-1) \cdot 25\\
                &\stackrel{(2)}{=} 30 + (-1) \cdot (55 +(-1)\cdot 30) = 2 \cdot 30 + (-1) \cdot 55 \\   
                &\stackrel{(1)}{=}  2 \cdot (470 + (-8)\cdot 55) + (-1) \cdot 55 = 2 \cdot 470 +(-17)\cdot 55    \qed\\    
            \end{align*}
        
        \end{solucion}    
        
    \end{frame}
    
    
    \begin{frame}\frametitle{Mínimo común múltiplo}
        
        
        \begin{definicion}
            Si $a$ y $b$ son enteros decimos que un entero no negativo $m$ es el {\em
                mínimo común múltiplo}\index{mínimo común múltiplo}, o {\em mcm}, de $a$ y $b$ si
            \begin{enumerate}
                \item[a)] $ a|m$ y $b|m$;
                \item[b)] si $ a|n $ y $b|n$ entonces $ m|n$.
            \end{enumerate}
        \end{definicion}\pause
        \vskip 0.6cm
        \begin{itemize}
            \item La condición (a) nos dice que $m$ es múl\-ti\-plo común de $a$ y $b$. 
            \item La condición (b) nos dice que cualquier otro múltiplo de $a$ y $b$ también debe ser múltiplo de $m$.
        \end{itemize}
        
    \end{frame}
    
    
    \begin{frame}
        
        \begin{ejemplo}
            Hallemos el mínimo común múltiplo entre $8$ y $14$.
        \end{ejemplo}\pause
        
        \begin{solucion}\pause
            Escribamos los múl\-ti\-plos de ambos números y busquemos el menor común a ambos. 
            
            \vskip .4cm \pause
            
            Los primeros múltiplos de $8$ son: $8,16,24,32,40,48,56,\ldots$. 
            
            \vskip .4cm 
            Los primeros múltiplos de $14$ son: $14,28,42,56,72,\ldots$. 
            
            \vskip .4cm \pause
            
            Luego se tiene $\mcm(8,14)=56$. Nos faltaría  comprobar que cualquier múltiplo de $8$ y $14$ es múltiplo de 56, pero eso se deduce fácilmente de los resultados que veremos a continuación.
        \end{solucion}
        
        
        
        
    \end{frame}
    


    
    
    \begin{frame}
        \begin{teorema}\label{t1.7.2} Sean $a$ y $b$ enteros no nulos, entonces
            $$
            \mcm(a,b)=\frac{|a b|}{\mcd(a,b)}.
            $$
        \end{teorema}\pause
        
        En particular este resultado implica que si $a$ y $b$ son naturales coprimos, entonces $\mcm(a,b)=ab$.
        \pause
        \vskip .5cm
        
        \begin{ejemplo} Encontrar el  mcm de $8$ y $14$.\pause
            \begin{solucion}\pause
                Es claro que $2 = \mcd(8,14)$, luego $\mcm(8,14) = 8 \cdot 14 / 2 = 56$. 
            \end{solucion}
        \end{ejemplo}
        
    \end{frame}
    
    
    
    \begin{frame}
        \begin{ejercicio}
            Demostrar que si $a$, $b$ y $n$ son enteros no nulos, entonces
            $\mcd(na,nb)=n\mcd(a,b)$.
        \end{ejercicio}\pause
        \begin{solucion}\pause
            Sea $d = (a,b)$, debemos probar que $ nd = (na,nb)$. Es decir, \pause
            \vskip .2cm
            \begin{enumerate}
                \item[{\color{blue}a)}] $ d|a$  y $d|b$;
                \item[{\color{blue}b)}]  si $ c|a $ y $c|b$ entonces $ c|d$.
            \end{enumerate}\pause
            
            \qquad\qquad $\Downarrow$
            
            \begin{enumerate}\pause
                \item[{\color{blue}a')}] $ nd|na$  y $nd|nb$;
                \item[{\color{blue}b')}]  si $ c|na $ y $c|nb$ entonces $ c|nd$.
            \end{enumerate}
        \end{solucion}
    \end{frame}
    
    
    
    \begin{frame}
        
        {\color{blue}a')}  
        
        Por {\color{blue}a)}, $a = d \cdot q_1$, $b = d \cdot q_2$ , luego
        $$na = d \cdot nq_1, \quad nb = d \cdot nq_2,$$ es decir
        $$ nd|na, \quad nd|nb.$$\pause
        
        \vskip .2cm
        {\color{blue}b')}  
        
        Sea $c$ tal que  $ c|na $ y $c|nb$.  
        \vskip .2cm
        Ahora bien
        $$
        d = ra +sb \Rightarrow nd = s(na) +t(nb),
        $$\pause
        Luego,
        $$
        c|na ,c|nb  \Rightarrow  c|s(na) +t(nb) = nd.
        $$
        Esto prueba     {\color{blue}b')}. \qed
    \end{frame}
    
    
    
    
    
\end{document}

