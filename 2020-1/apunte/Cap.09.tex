
\appendix
\setcounter{chapter}{2}
\renewcommand{\thechapter}{\Alph{chapter}}
 \chapter[La función de Euler]{La función de Euler}


 \begin{section}{La función de Euler} \label{A2.1 }


 En esta sección probaremos un útil e importante teorema, usando sólo
 los conceptos de conteo más básicos.

 El teorema se refiere a las
 propiedades de divisibilidad de los enteros. Recordemos que dos enteros $x$ e
 $y$ son {\it coprimos} si el $\mcd(x,y) = 1$. Por cada $n \ge 1$ sea
 $\phi(n)$ el número de
 enteros $x$ en el rango $1 \le x \le n$ tal que $x$ y $n$ son coprimos.
 Podemos
 calcular los primeros valores de $\phi(n)$ haciendo una tabla (tabla
 \ref{tablaA2.1.1}).

La función es llamada {\em función de Euler}, debido a Leonhard
Euler  \index{Euler, Leonhard} ($1\,707-1\,783$). Cuando $n$ es primo,
digamos $n=p$, cada uno de los enteros $1,2,\ldots,p-1$ es coprimo
con $p$, entonces tenemos
$$
\phi(p)=p-1,\qquad\text{ $p$ primo.}
$$




% \vskip 3.cm
 \begin{table}
 %Tabla A2.1.1

 \begin{alignat*}3
 &n& \text{\quad Coprimos a $n$\quad }& &\phi(n)&\\
&&&&&\\
 &1&1&&1& \\
 &2&1&&1& \\
 &3&1,2&&2& \\
 &4&1,3&&2& \\
 &5&1,2,3,4&&4& \\
 &6&1,5&&2& \\
 &7&1,2,3,4,5,6&&6& \\
 &8&1,3,5,7&&4&
 \end{alignat*}

\caption{} \label{tablaA2.1.1}
\end{table}




Nuestra tarea ahora es probar un resultado respecto a la suma de
los valores $\phi(d)$, donde los $d$ son todos los divisores de un
número positivo $n$ dado. Por ejemplo, cuando $n=12$, lo divisores
$d$ son $1$, $2$, $3$, $4$, $5$, $6$ y $12$, podemos ver que
\begin{align*}
&\quad\phi(1)+\phi(2)+\phi(3)+\phi(4)+\phi(6)+\phi(12)\\
&= 1 +1+2+2+2+4 \\
&=12.
\end{align*}

Debemos demostrar que la suma es siempre igual al entero $n$ dado.


\begin{teorema}\label{tA2.1b} Para cualquier $n$ entero positivo,
$$
\sum_{d|n} \phi(d)=n.
$$
\end{teorema}
\begin{proof} Sea $S$ el conjunto de pares de enteros $(d,f)$ que
satisfacen
$$
d|n, \qquad 1\le f \le d, \qquad \mcd(f,d)=1.
$$


\begin{table}
%Tabla A2.1.2


\begin{align*}
&  &f &&1  &&2 &&3 &&4 &&5  &&6 &&7 &&8 &&9 &&10 &&11 &&12 && &\phi(d)\\
&d &  &&   &&  &&  &&  &&   &&  &&  &&  &&  &&   &&   &&   && &   \\
&1 &  &&12 &&  &&  &&  &&   &&  &&  &&  &&  &&   &&   &&   && &1  \\
&2 &  &&6  &&  &&  &&  &&   &&  &&  &&  &&  &&   &&   &&   && &1  \\
&3 &  &&4  &&8 &&  &&  &&   &&  &&  &&  &&  &&   &&   &&   && &2  \\
&4 &  &&3  &&  &&9 &&  &&   &&  &&  &&  &&  &&   &&   &&   && &2  \\
&6 &  &&2  &&  &&  &&  &&10 &&  &&  &&  &&  &&   &&   &&   && &2  \\
&12&  &&1  &&  &&  &&  &&5  &&  &&7 &&  &&  &&   &&11 &&   && &4  \\
&  &  &&   &&  &&  &&  &&   &&  &&  &&  &&  &&   &&   &&   && &12
\end{align*}

\caption{} \label{tablaA2.1.2}
\end{table}


La tabla \ref{tablaA2.1.2} muestra $S$ cuando $n=12$; la ``marca''
que indica que $(d,f)$ pertenece a $S$ es un número cuya
importancia explicaremos en seguida. Por lo general, el número de
``marcas'' en la fila $d$ es el número de $f$'s en el rango $1\le
f\le d$ que satisfacen que el $\mcd(d,f)=1$; esto es $\phi(d)$.
Por lo tanto, contando $S$ por el método de las filas obtenemos
$$
|S| = \sum_{d|n} \phi(d).
$$
Para demostrar que $|S|=n$ debemos construir una biyección $\beta$
de $S$ en $\mathbb N_n$. Dado un par $(d,f)$ en $S$, definimos
$$
\beta(d,f) = f n/d.
$$
En la tabla, $\beta(d,f)$ es la ``marca'' en la fila $d$ y la
columna $f$. Como $d| n$, el valor de $\beta$, es un entero y como
$1\le f\le d$, entonces $\beta(d,f)$ pertenece a $\mathbb N_n$.

Para probar que $\beta$ es una inyección observemos que
$$
\beta(d,f) = \beta(d',f') \quad \Rightarrow \quad fn/d = f'n/d'
\quad \Rightarrow \quad fd'=f'd.
$$
Pero $f$ y $d$ son coprimos, así como también lo son $f'$ y $d'$,
así que podemos concluir que $d=d'$ y $f=f'$.

Para demostrar que $\beta$ es una suryección, supongamos que nos
dan un $x$ que pertenece a $\mathbb N_n$. Sea $g_x$ el mcd de $x$
y $n$, y sea
$$
d_x = n/g_x, \qquad f_x = x /g_x.
$$
Puesto que $g_x$ es un divisor de $x$ y $n$, entonces $d_x$ y
$f_x$ son enteros, y como $g_x$ es el mcd, $d_x$ y $f_x$ son
coprimos. Ahora
$$
\beta(d_x,f_x) = f_x n/d_x = x,
$$
y por lo tanto $\beta$ es suryectiva.

Luego $\beta$ es biyectiva y $|S|=n$, como queríamos demostrar.
\end{proof}

%\begin{subsection}{Ejercicios}
\subsection*{\Large $\S$ Ejercicios}
%\addcontentsline{toc}{subsection}{Ejercicios}
\begin{enumerate}[1)]
\item Encontrar los valores de $\phi(19),\quad \phi(20),\quad \phi(21)$.

\item Probar que si $x$ y $n$ son coprimos, entonces lo son $n-x$ y
$n$. Deducir que $\phi(n)$ es par para todo $n \ge 3$.

\item Probar que, si $p$ es un primo y $m$ es un entero positivo,
entonces un entero $x$ en el rango $1 \le x \le p^m$ {\it no} es
coprimo a $p^m$ si y solo si es un múltiplo de $p$. Deducir que
$\phi(p^m) = p^m - p^{m-1}$.

\item Encontrar un contraejemplo que confirme que es falsa la conjetura
$\phi(ab)= \phi(a)\phi(b)$, para enteros cualesquiera $a$ y $b$.
Trate de modificar la conjetura de tal forma que no pueda
encontrar un contraejemplo.

\item Probar que para cualesquiera enteros positivos $n$ y $m$ se cumple:
$$
\phi(n^m) =n^{m-1}\phi(n).
$$

\item Calcular $\phi(1\,000)$ y $\phi(1\,001)$.
\end{enumerate}
%\end{subsection}


\end{section}



\begin{section}{Una aplicación aritmética del principio del
tamiz}\label{Ap2.2} Por cientos de años los matemáticos han
estudiado problemas sobre números primos y la fac\-to\-ri\-za\-ción de los
enteros. Nuestra breve discusión sobre estos temas en los primeros
capítulos debería haber convencido al lector de que estos
problemas son difíciles, porque los primos mismos se encuentran
irregularmente distribuidos, y porque no hay una forma directa de
encontrar la factorización en primos de un entero dado. De todos
modos, si se nos da la factorización en primos de un entero, es
relativamente fácil responder ciertas preguntas sobre sus
propiedades aritméticas. Supongamos, por ejemplo que queremos
listar todos los divisores de un entero $n$ y sabemos que la
factorización de $n$ es
$$
n=p_1^{e_1}p_2^{e_2}\cdots p_r^{e_r}.
$$
Entonces un entero $d$ es divisor de $n$ si y solo si no tiene
divisores primos distintos de los de $n$, y ningún primo divide
más veces a $d$ que a $n$. Visto así, los divisores son
precisamente los enteros que pueden escribirse de la forma
$$
d=p_1^{f_1}p_2^{f_2}\cdots p_r^{f_r},
$$
donde cada $f_i$ ($1\le i \le r$) satisface $0\le f_i \le e_i$.
Por ejemplo dado que $60= 2^2 \cdot 3 \cdot 5$ podemos listar
rápidamente todos los divisores de 60.


Un problema similar es encontrar el número de enteros $x$ en el
rango $1 \le x \le n$ que son coprimos con $n$. En la sección
\ref{A2.1 } denotamos este número con $\phi(n)$, el valor de la
función $\phi$ de Euler en $n$. Ahora demostraremos que si la
factorización en primos de $n$ es conocida, entonces $\phi(n)$
puede ser calculado por el principio del tamiz.

\begin{ejemplo}?`Cuál es el valor de $\phi(60)$? En otras
palabras, ?`cuántos enteros $x$ en el rango $1 \le x \le 60$
satisfacen $\operatorname{mcd}(x,60)=1$?
\end{ejemplo}
\begin{proof} Sabemos que $60 =2^2 \cdot 3 \cdot 5$, así que
podemos contar el números de enteros $x$ en el rango $1 \le x \le
60$ que no son divisibles por $2$, $3$ o $5$. Con $A(2)$ denotemos el
subconjunto de $\mathbb N_{60}$ que contiene los enteros que \it
son \rm divisibles por $2$, con $A(2,3)$ aquellos que \it son \rm
divisibles por $2$ y $3$, y así sucesivamente, entonces tenemos
$$\begin{aligned}
\phi(60)&=60-|A(2) \cup A(3) \cup A(5)| \\
&= 60-|A(2) + A(3) + A(5)| \\
&\qquad+(|A(2,3) + |A(2,5)| + |A(3,5)|)-|A(2,3,5)|,
\end{aligned}
$$
por el principio del tamiz. Ahora $|A(2)|$ es el número de
múltiplos de $2$ en $\mathbb N_{60}$ que es $60 /2 = 30$. Del mismo
modo $|A(2,3)|$ es el número de múltiplos de $2 \cdot 3$, que es
$60 /(2\cdot 3) = 10$, y así siguiendo, por lo tanto
$$
\phi(60) = 60 -(30+20+10)+(10+6+4)-2=16.
$$
\end{proof}

El mismo método puede ser usado para dar una fórmula explícita
para $\phi(n)$ en el caso general.


\begin{teorema}\label{tA2.2} Sea $n \ge 2$ un entero cuya factorización
es $n=p_1^{e_1}p_2^{e_2}\ldots p_r^{e_r}$. Entonces
$$
\phi(n)=n\left(1-\frac{1}{p_1}\right)\left(1-\frac{1}{p_2}\right)\cdots\left(1-\frac{1}{p_r}\right).
$$
\end{teorema}
\begin{proof} Denotemos $A_j$ el subconjunto de $\mathbb N_n$
que contiene los múltiplos de $p_j$ ($1\le j \le r$). Entonces
$$
\begin{aligned}
\phi(n) &= n- |A_1 \cup A_2 \cup \cdots \cup A_r| \\
       &= n -\alpha_1+ \alpha_2- \cdots +(-1)^r\alpha_r
\end{aligned}
$$
donde $\alpha_i$ es la suma de los cardinales de las
intersecciones de los conjuntos tomados de a $i$. Una intersección
típica como
$$
A_{j_1}\cup A_{j_2}\cup \cdots \cup A_{j_i}
$$
contiene los múltiplos de $P= p_{j_1}\cdot p_{j_2}\cdots p_{j_i}$ en $\mathbb N_n$, y estos son los números
$$
P,2P,3P,\ldots,\left(\frac{n}{p}\right)P.
$$
Luego la cardinalidad de una intersección típica es $n/P$, y
$\alpha_i$ es la suma de términos como
$$
\frac{n}{P}=
n\left(\frac{1}{p_{j_1}}\right)\left(\frac{1}{p_{j_2}}\right)\cdots
\left(\frac{1}{p_{j_i}}\right).
$$
Se sigue que
$$
\begin{aligned} \phi(n) = n - n\left(\frac{1}{p_1} + \frac{1}{p_2} +
\cdots +\frac{1}{p_r}\right) +n\left(\frac{1}{p_1p_2}
+ \frac{1}{p_1p_3}+\cdots\right) &+ \cdots \\
\cdots &+ (-1)^r \left(\frac{1}{p_1p_2 \cdots p_r}\right) \\
=n\left(1-\frac{1}{p_1}\right)&\left(1-\frac{1}{p_2}\right)\cdots\left(1-\frac{1}{p_r}\right).
\end{aligned}
$$
\end{proof}



\end{section}

