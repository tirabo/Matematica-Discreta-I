%\documentclass{beamer} 
\documentclass[handout]{beamer} % sin pausas
\usetheme{CambridgeUS}

\usepackage{etex}
\usepackage{t1enc}
\usepackage[spanish,es-nodecimaldot]{babel}
\usepackage{latexsym}
\usepackage[utf8]{inputenc}
\usepackage{verbatim}
\usepackage{multicol}
\usepackage{amsgen,amsmath,amstext,amsbsy,amsopn,amsfonts,amssymb}
\usepackage{amsthm}
\usepackage{calc}         % From LaTeX distribution
\usepackage{graphicx}     % From LaTeX distribution
\usepackage{ifthen}
%\usepackage{makeidx}
\input{random.tex}        % From CTAN/macros/generic
\usepackage{subfigure} 
\usepackage{tikz}
\usepackage[customcolors]{hf-tikz}
\usetikzlibrary{arrows}
\usetikzlibrary{matrix}
\tikzset{
	every picture/.append style={
		execute at begin picture={\deactivatequoting},
		execute at end picture={\activatequoting}
	}
}
\usetikzlibrary{decorations.pathreplacing,angles,quotes}
\usetikzlibrary{shapes.geometric}
\usepackage{mathtools}
\usepackage{stackrel}
%\usepackage{enumerate}
\usepackage{enumitem}
\usepackage{tkz-graph}
\usepackage{polynom}
\polyset{%
	style=B,
	delims={(}{)},
	div=:
}
\renewcommand\labelitemi{$\circ$}
\setlist[enumerate]{label={(\arabic*)}}
%\setbeamertemplate{background}[grid][step=8 ] % cuadriculado
\setbeamertemplate{itemize item}{$\circ$}
\setbeamertemplate{enumerate items}[default]
\definecolor{links}{HTML}{2A1B81}
\hypersetup{colorlinks,linkcolor=,urlcolor=links}


\newcommand{\Id}{\operatorname{Id}}
\newcommand{\img}{\operatorname{Im}}
\newcommand{\nuc}{\operatorname{Nu}}
\newcommand{\im}{\operatorname{Im}}
\renewcommand\nu{\operatorname{Nu}}
\newcommand{\la}{\langle}
\newcommand{\ra}{\rangle}
\renewcommand{\t}{{\operatorname{t}}}
\renewcommand{\sin}{{\,\operatorname{sen}}}
\newcommand{\Q}{\mathbb Q}
\newcommand{\R}{\mathbb R}
\newcommand{\C}{\mathbb C}
\newcommand{\K}{\mathbb K}
\newcommand{\F}{\mathbb F}
\newcommand{\Z}{\mathbb Z}
\newcommand{\N}{\mathbb N}
\newcommand\sgn{\operatorname{sgn}}
\renewcommand{\t}{{\operatorname{t}}}
\renewcommand{\figurename }{Figura}

\include{definiciones}

\newcommand{\nc}{\newcommand}

%%%%%%%%%%%%%%%%%%%%%%%%%LETRAS

\nc{\FF}{{\mathbb F}} \nc{\NN}{{\mathbb N}} \nc{\QQ}{{\mathbb Q}}
\nc{\PP}{{\mathbb P}} \nc{\DD}{{\mathbb D}} \nc{\Sn}{{\mathbb S}}
\nc{\uno}{\mathbb{1}} \nc{\BB}{{\mathbb B}} \nc{\An}{{\mathbb A}}

\nc{\ba}{\mathbf{a}} \nc{\bb}{\mathbf{b}} \nc{\bt}{\mathbf{t}}
\nc{\bB}{\mathbf{B}}

\nc{\cP}{\mathcal{P}} \nc{\cU}{\mathcal{U}} \nc{\cX}{\mathcal{X}}
\nc{\cE}{\mathcal{E}} \nc{\cS}{\mathcal{S}} \nc{\cA}{\mathcal{A}}
\nc{\cC}{\mathcal{C}} \nc{\cO}{\mathcal{O}} \nc{\cQ}{\mathcal{Q}}
\nc{\cB}{\mathcal{B}} \nc{\cJ}{\mathcal{J}} \nc{\cI}{\mathcal{I}}
\nc{\cM}{\mathcal{M}} \nc{\cK}{\mathcal{K}}

\nc{\fD}{\mathfrak{D}} \nc{\fI}{\mathfrak{I}} \nc{\fJ}{\mathfrak{J}}
\nc{\fS}{\mathfrak{S}} \nc{\gA}{\mathfrak{A}}
%%%%%%%%%%%%%%%%%%%%%%%%%LETRAS



\title[Clase 16 - Congruencia ]{Matemática Discreta I \\ Clase 16 - Congruencia 1}
%\author[C. Olmos / A. Tiraboschi]{Carlos Olmos / Alejandro Tiraboschi}
\institute[]{\normalsize FAMAF / UNC
	\\[\baselineskip] ${}^{}$
	\\[\baselineskip]
}
\date[12/05/2020]{12 de mayo de 2020}




\begin{document}
	
	\frame{\titlepage} 
	
	\begin{frame}
		
		\begin{definicion} Sean $a$ y $b$ enteros y $m$ un
			entero positivo. Diremos que $a$ es {\em congruente} a $b$
			{\textit{módulo}} $m$, y escribimos 
			$$
			a \equiv b \pmod{m}
			$$
			si $a-b$ es divisible or $m$.
		\end{definicion}
		
		Observar que 
		
		$$a\equiv 0 \pmod{m} \Leftrightarrow m|a\qquad\qquad$$ 
		y que 
		$$\qquad\qquad a\equiv b \pmod{m}\Leftrightarrow a-b\equiv 0 \pmod{m}.$$ 
	\end{frame}
	
	\begin{frame}
		
		
		\begin{proposicion}
			Sean $a$ y $b$ enteros y $m$ un entero positivo. Entonces $a\equiv b \pmod{m}$ si  y sólo si
			$a$ y $b$ tienen el mismo resto en la división por $m$.
		\end{proposicion}
		\begin{proof}
			Si $a=mh+r$ y $b=mk+s$, con $0 \le r,s <m$, podemos suponer,
			sin perdida de generalidad, que $r \le s$, luego
			$$
			b-a= m(k-h) + (s-r) \qquad \text{con $0\le s - r < m$}.
			$$
			Se sigue que $s-r$ es el resto de dividir $b-a$ por $m$.
			
			Luego si $a\equiv b \pmod{m}$, el resto de dividir   $b-a$ por $m$ es 0, y por lo tanto $s-r=0$ y $s=r$.
			
			Si $a$ y $b$ tienen el mismo resto en la división por $m$, entonces  $a=mh+r$ y $b=mk+r$, luego $a-b = m(h-k)$ que es divisible por $m$. \qed
		\end{proof}
	\end{frame}
	
	
	\begin{frame}
		Así como separamos $\mathbb Z$ en los números pares e
		impares, la propiedad anterior nos permite expresar $\mathbb Z$
		como una unión disjunta de $m$ subconjuntos. 
		
		\vskip .4cm
		
		Es decir si 
		
		$$
		\mathbb  Z_{[r]} =\{x \in \mathbb Z: \text{el resto de dividir $x$ por $m$ es $r$}\},
		$$
		\vskip .4cm
		entonces dado $m \in \mathbb N$, 
		
		$$
		\mathbb Z= \mathbb Z_{[0]}\cup \mathbb Z_{[1]}\cup \cdots\cup \mathbb Z_{[m-1]}.
		$$
	\end{frame}
	
	
	\begin{frame}
		Es fácil verificar que la {congruencia módulo} $m$ verifica las
		siguientes propiedades\vskip .2cm
		\begin{enumerate}
			\item[a)]
			Es {\it {reflexiva}} es decir $x\equiv x\pmod{m}$.\vskip .2cm
			\item[b)]
			Es {\it {simétrica}}, es decir si $x \equiv y \pmod{m}$, entonces\vskip .2cm
			$y \equiv x \pmod{m}$.
			\item[c)]
			Es {\it { transitiva}}, es decir si $x\equiv y \pmod{m}$ e
			$y\equiv z \pmod{m}$, entonces $x\equiv z \pmod{m}$.
		\end{enumerate}
		\begin{proof}
			a)  $x-x = 0$  y por lo tanto divisible por $m$. 
			\vskip .2cm
			b)  $x-y=km$, entonces 	$y-x=(-k)m$.
			\vskip .2cm
			c)  $x-y=km$ y $y-z=lm$, $\Rightarrow$
			$x-z=(x-y)+(y-z)=(k+l)m$.
			
			\qed
		\end{proof}
	\end{frame}
	
	
	\begin{frame}
		La utilidad de las congruencias reside principalmente en el hecho
		de que son compatibles con las operaciones aritméticas.
		Específicamente, tenemos el siguiente teorema.
		
		\begin{teorema}\label{t4.1} Sea $m$ un entero positivo y sean $x_1$, $x_2$,
			$y_1$, $y_2$ enteros tales que
			$$
			x_1 \equiv x_2 \pmod{m}, \qquad y_1 \equiv y_2 \pmod{m}.
			$$
			Entonces
			\begin{enumerate}
				\item[a)] $ x_1+ y_1 \equiv x_2+ y_2 \pmod{m}$,
				\item[b)] $x_1 y_1 \equiv x_2 y_2 \pmod{m}$,
				\item[c)] Si $x \equiv y \pmod{m}$  y $j \in  \mathbb N$, entonces $x^j \equiv y^j \pmod{m}$.
			\end{enumerate}
		\end{teorema}
		
	\end{frame}
	
	
	\begin{frame}
		\begin{proof}
			
			\noindent(a) Por hipótesis $\exists x,y$ tq $x_1-x_2=mx$ e $y_1-y_2=my$. Luego, 
			$$
			\begin{aligned}
				(x_1+y_1)-(x_2+y_2) &= (x_1-x_2)+ (y_1 -y_2) \\
				&= mx +my \\
				&= m(x+y),
			\end{aligned}
			$$
			y por consiguiente el lado izquierdo es divisible por $m$.
			\vskip .2cm
			\noindent (b)  Aquí tenemos
			$$
			\begin{aligned}
				x_1y_1-x_2y_2 &=  x_1y_1-x_2y_1+ x_2y_1-x_2y_2 \\
				&= (x_1-x_2)y_1+ x_2(y_1 -y_2) \\
				&= mxy_1 +x_2my \\
				&= m(xy_1+x_2y),
			\end{aligned}
			$$
			y de nuevo el lado izquierdo es divisible por $m$.
			
		\end{proof}
	\end{frame}
	
	
	\begin{frame}
		
		\noindent (c)  Lo haremos por inducción sobre $j$. 
		\vskip .2cm
		Es claro que si $j=1$ el resultado es verdadero. 
		\vskip .2cm
		Supongamos ahora que el resultado vale para $j-1$, es decir
		$$
		x^{j-1} \equiv y^{j-1} \pmod{m}.
		$$
		Como $x \equiv y \pmod{m}$,  por   { (b)} tenemos que 
		$$
		x^{j-1}x \equiv y^{j-1}y  \pmod{m},
		$$
		es decir 
		$$
		x^j \equiv y^j \pmod{m}.
		$$
		\qed
	\end{frame}
	
	
	\begin{frame}\frametitle{Regla del  nueve}
		\begin{proposicion}
			Sea $(x_nx_{n-1}\ldots x_0)_{10}$ la
			representación del entero positivo $x$ en base $10$, entonces
			$$
			x \equiv x_0+x_1+\cdots+x_n \pmod{9}
			$$
		\end{proposicion}
		\begin{proof}
			Observemos:  $10\equiv 1\pmod{9}$ $\Rightarrow$  $10^k\equiv 1^k \equiv 1\pmod{9}$. 
			\vskip .2cm
			Por la definición de representación en base $10$, tenemos que 
			$$x=x_0 + 10x_1+ \cdots+10^nx_n,$$ 
			\vskip .2cm
			Luego,  $x_k10^k \equiv x_k \pmod{9}$ y  entonces $x \equiv
			x_0+x_1+\cdots+x_n \pmod{9}$.
		\end{proof}
		\qed
	\end{frame}
	
	
	\begin{frame}
		\begin{corolario} 	Sea $x = (x_nx_{n-1}\ldots x_0)_{10}$, entonces
			$$
			9|x\quad  \Leftrightarrow\quad 9|  x_0+x_1+\cdots+x_n.
			$$
		\end{corolario}
		\begin{proof}
			Por la proposición anterior 
			\begin{equation}
				x \equiv x_0+x_1+\cdots+x_n \pmod{9} \tag{*}
			\end{equation}
			Entonces,
			\begin{align*}
				9|x \quad& \Leftrightarrow \quad	x \equiv 0 \pmod{9}& &\text{(por hipótesis)} \\
				& \Leftrightarrow \quad	x_0+x_1+\cdots+x_n  \equiv 0 \pmod{9}& &\text{(por (*))} \\	
				& \Leftrightarrow  \quad 9 |	x_0+x_1+\cdots+x_n.  & &\quad\quad\quad\quad\quad\qed
			\end{align*}
		\end{proof}
	\end{frame}
	
	
	\begin{frame}
		
		\begin{ejemplo}Probar que \quad $54\,321 \cdot 98\,765\not= 5\,363\,013\,565$.
		\end{ejemplo}
		\begin{proof}
			Módulo 9:
			\begin{equation*}
				\begin{matrix*}[l]
					54\,321 & \equiv& 5 +4 +3+2+1 & \equiv& 15 & \equiv& 6 \pmod{9} \\
					98\,765 & \equiv& 9+8+ 7+6+5 & \equiv&  35  & \equiv& 8 \pmod{9} 
				\end{matrix*}
			\end{equation*}
			Entonces 
			\begin{equation*}
				54\,321 \cdot 98\,765 \equiv 6\cdot 8  \equiv 48  \equiv 4+8 \equiv 12 \equiv 3 \pmod{9} 
			\end{equation*}
			Mientras que 
			$$
			5\,363\,013\,565  \equiv 5+3+6+3+0+1+3+5+6+5  \equiv37   \equiv 1 \pmod{9} 
			$$
			Luego  $54\,321 \cdot 98\,765\not= 5\,363\,013\,565$. \qed
			
		\end{proof}
		
	\end{frame}
	
	
	\begin{frame}
		\begin{ejemplo}
			Sea $x = (x_nx_{n-1}\ldots x_0)_{10}$, entonces
			$$
			4 |x\quad  \Leftrightarrow\quad 4 |  2x_1 + x_0.
			$$
		\end{ejemplo}\vskip -.5cm
		\begin{proof}
			Aquí tenemos que $10  \equiv 2 \pmod{4}$, luego  $10^2  \equiv 4  \equiv 0 \pmod{4}$. 
			\vskip .2cm
			Por lo tanto $10^k    \equiv 0 \pmod{4}$ para $k > 1$. 
			\vskip .2cm
			Luego 
			\begin{equation*}
				x \equiv x_n10^n + x_{n-1}10^{n-1}+\cdots+ x_{2}10^{2}+ x_{1}10 + x_0 \equiv 2 x_{1} + x_0 \pmod{4}.
			\end{equation*}
			Luego
			\begin{equation*}
				4 | x  \;\Leftrightarrow \; x \equiv 0 \pmod{4}  \; \Leftrightarrow  \; 2 x_{1} + x_0  \equiv 0 \pmod{4}   \; \Leftrightarrow  \;4 | 2 x_{1} + x_0 .\qed
			\end{equation*}
			
		\end{proof}
	\end{frame}
	
	
	
\end{document}

