
\appendix
\setcounter{chapter}{0}
\renewcommand{\thechapter}{\Alph{chapter}}
\chapter[Permutaciones]{Permutaciones}




\begin{section}{Permutaciones}
Recordemos  que una \textit{permutación} de un conjunto finito no vacío $X$ es una biyección de $X$ en $X$. (Frecuentemente tomamos $X$ como $ [[1,n]]=\{1,2,\ldots,n\}$.) Por ejemplo una permutación típica de $[[1,5]]$ es la función $\alpha$ definida por las ecuación
$$
\alpha(1)=2,\quad \alpha(2)=4,\quad \alpha(3)=5,\quad
\alpha(4)=1,\quad \alpha(5)=3.
$$

Denotemos el conjunto de todas las permutaciones de $[[1,n]]$ con $S_n$. Por ejemplo, $S_3$ contiene las $3!=6$ permutaciones siguientes:
$$
\begin{matrix} 1&2&3 \\ \downarrow&\downarrow&\downarrow\\1 &2 &
3\end{matrix}\qquad
\begin{matrix} 1&2&3 \\ \downarrow&\downarrow&\downarrow\\1 &3 &2 \end{matrix}\qquad
\begin{matrix} 1&2&3 \\ \downarrow&\downarrow&\downarrow\\2 & 1&3 \end{matrix}
\qquad
\begin{matrix} 1&2&3 \\ \downarrow&\downarrow&\downarrow\\2 & 3&1
\end{matrix}\qquad
\begin{matrix} 1&2&3 \\
\downarrow&\downarrow&\downarrow\\3 &1 &2 \end{matrix}\qquad
\begin{matrix} 1&2&3 \\ \downarrow&\downarrow&\downarrow\\3 &2 &1
\end{matrix}
$$


En la práctica, usualmente asignamos alguna interpretación concreta a un elemento de $S_n$. Como vimos en la sección \ref{permutaciones}, podemos usar la interpretación ``selecciones ordenadas sin repetición''  \index{selección ordenada sin repetición} donde, en este caso seleccionamos los elementos de $\{1,2,3,\ldots,n\}$ en algún orden hasta que no queda ninguno. Una interpretación relacionada es que una permutación efectúa un \textit{reacomodamiento} de $\{1,2,3,\ldots,n\}$; por ejemplo, la permutación $\alpha$ vista más arriba  efectúa el reacomodamiento de $12\,345$, en $24\,513$, así:
$$
\begin{matrix} 1&2&3&4&5 \\
\downarrow&\downarrow&\downarrow&\downarrow&\downarrow\\2 &4 &5 &1
& 3
\end{matrix}
$$
En algunas circunstancias es conveniente mirar una permutación y el correspondiente reacomodamiento como la misma cosa, pero esto puede traer dificultades si debemos considerar sucesivos reacomodamientos. Es importante tener en cuenta que 
$$
\textit{una permutación es una función con ciertas características.}
$$

Cuando las permutaciones son tratadas como funciones es claro como deben combinarse. Consideremos $\alpha$ la permutación de $[[1,5]]$ antes mencionada, y sea $\beta$ la permutación de $[[1,5]]$ dada por 
$$
\beta(1)=3,\quad \beta(2)=5,\quad \beta(3)=1,\quad
\beta(4)=4,\quad \beta(5)=2.
$$
La función compuesta $\beta\alpha$ es la permutación definida por $(\beta\alpha)(i)= \beta(\alpha(i))$ ($1\le i\le 5$), esto es 
$$
\beta\alpha(1)=5,\quad \beta\alpha(2)=4,\quad
\beta\alpha(3)=2,\quad \beta(4)\alpha=3,\quad \beta\alpha(5)=1.
$$
(Recordemos que, como siempre, $\beta\alpha$ significa ``primero $\alpha$, entonces $\beta$''.) En términos de reacomodamientos tenemos
$$\begin{aligned}
\alpha\quad&\quad\begin{matrix} 1&2&3&4&5 \\
\downarrow&\downarrow&\downarrow&\downarrow&\downarrow\\2 &4 &5 &1
& 3
\end{matrix} \\
\beta \quad&\quad \begin{matrix} 1&2&3&4&5 \\
\downarrow&\downarrow&\downarrow&\downarrow&\downarrow\\5 &4 &2 &3
& 1
\end{matrix}
 \end{aligned}
$$



Existen cuatro características de la composición de permutaciones de gran importancia, y están listadas en el próximo teorema.

\begin{teorema}\label{tA3} Las siguientes propiedades valen en el conjunto $S_n$ de todas las permutaciones de $\{1,2,3,...,n\}$.
\begin{enumerate}[label=(\alph*)]
\item  Si $\pi$ y $\sigma$ pertenecen a $S_n$, entonces $\pi\sigma$ también.
\item  Para cualesquiera permutaciones $\pi$, $\sigma$, $\tau$ en $S_n$,
$$
(\pi\sigma)\tau=\pi(\sigma\tau).$$
\item  La función identidad, denotada por $\operatorname{id}$ y definida por $\operatorname{id}(r) =r$ para todo $r$ en $\mathbb N_n$, es una permutación y para cualquier $\sigma$ en $S_n$,
tenemos
$$
\operatorname{id}\sigma=\sigma\operatorname{id}=\sigma.$$
\item  Para toda permutación $\pi$ en $S_n$ hay una permutación inversa $\pi^{-1}$ en $S_n$ tal que
$$
\pi\pi^{-1} = \pi^{-1}\pi = \operatorname{id}.
$$
\end{enumerate}
\end{teorema}
\begin{proof} Todas las afirmaciones se deducen de propiedades conocidas de funciones en general y funciones biyectivas en particular. Por otro lado, es fácil convencerse de la validez de las mismas mirando las permutaciones como reacomodamiento de elementos. 
\end{proof}


Es conveniente tener una notación más compacta para las permutaciones. Consideremos otra vez la permutación $\alpha$ de $\{1,2,3,4,5\}$, y notemos en particular que
$$
\alpha(1)=2,\qquad \alpha(2)=4,\qquad \alpha(4)=1.
$$
Así $\alpha$ lleva $1$ a $2$, $2$ a $4$ y $4$ a $1$, y por esta razón decimos que los símbolos $1, 2, 4$ forma un \textit{ciclo } (de longitud $3$). Del mismo modo, los símbolos $3$ y $5$ forman un ciclo de longitud $2$, y
escribimos:
$$
\alpha=(1\,2\,4)(3\,5).
$$
Esta es la \textit{notación cíclica} para $\alpha$. Cualquier \index{notación cíclica} permutación $\pi$ puede ser escrita cíclicamente de la siguiente manera:
\begin{itemize}
\item comencemos con algún símbolo (digamos el $1$) y veamos el efecto de $\pi$ sobre él y sus sucesores hasta que alcancemos el $1$
nuevamente;
\item elijamos un símbolo que todavía no haya aparecido y construyamos el ciclo que se deriva de él; 
\item repitamos el procedimiento hasta que se terminen los símbolos.
\end{itemize}
Por ejemplo, la permutación $\beta$ definida antes tiene la notación cíclica
$$
\beta=(1\,3)(2\,5)(4),
$$
donde observamos que el símbolo $4$ forma un ciclo ``degenerado'' por sí solo, puesto que $\beta(4)=4$. En algunas ocasiones podemos omitir estos ciclos de longitud 1 cuando escribimos una permutación en notación cíclica, puesto que corresponden a símbolos que no son afectados por la permutación. Sin embargo, usualmente es útil \textit{no} adoptar esta convención hasta que uno se familiariza con la notación.


Aunque la representación de una permutación en notación cíclica es esencialmente única, hay dos manera obvias en las que podemos cambiar la notación sin alterar la permutación. Primero, cada ciclo puede empezar en cualquiera de sus símbolos; por ejemplo $(7\,8\,2\,1\,3)$ y $(1\,2\,7\,8\,2)$ describen el mismo ciclo. Segundo, el orden de los ciclos no es importante; por ejemplo $(1\,2\,4) (3\,5)$ y $(3\,5) (1\,2\,4)$ denotan la misma permutación. Pero las características importantes son el número de ciclos, la longitud del ciclo, y la disposición de los símbolos dentro de los ciclos, y éstas están determinadas de manera única. Por eso, la rotación cíclica nos dice bastantes cosas útiles sobre una permutación.

\begin{ejemplo}\label{cartas} Cartas numeradas del $1$ al $12$ son distribuidas en una mesa en la manera en que se muestra en la parte izquierda de la tabla que sigue. Luego las cartas son levantadas  por filas (de izquierda a derecha y de arriba hacia abajo) y se redistribuyen con el mismo arreglo, pero por columnas, no por filas (de arriba hacia abajo y de izquierda a derecha), apareciendo como se muestra en la parte derecha de la tabla.
$$
\begin{matrix} 1& 2& 3\\
4 &5 &6 \\
7 &8 & 9\\
10 &11 & 12 \end{matrix}\qquad \qquad\qquad
\begin{matrix}1 &5 &9 \\
2 &6 &10 \\
3& 7& 11\\
4&8 & 12 \end{matrix}
$$
?`Cuántas veces debe repetirse este procedimiento hasta que las cortas aparezcan dispuestas como estaban inicialmente?
\end{ejemplo}
\begin{proof}[Solución] Sea $\pi$ la permutación que efectúa el reordenamiento; esto es $\pi(i) =j$ si la carta $j$ aparece en la posición previamente ocupada por la carta $i$. Trabajando con la notación cíclica para $\pi$ encontramos que
$$
\pi=(1)(2\,\,5\,\,6\,\,10\,\,4)(3\,\,9\,\,11\,\,8\,\,7)(12).
$$
Los ciclos degenerados $(1)$ y $(12)$ indican que las cartas 1 y 12 nunca cambian de posición. Las otros ciclos tienen longitud 5, así que cuando el proceso se haya realizado 5 veces las cartas
reaparecerán en sus posiciones originales. Otra forma de expresar el resultado es decir que $\pi^5= \operatorname{id}$, donde $\pi^5$ representa las cinco repeticiones de la permutación $\pi$.
\end{proof}
\end{section}


\subsection*{$\S$ Ejercicios}
\begin{enumerate}
\item Escribir en notación cíclica la permutación que realiza el siguiente reordenamiento:
$$
\begin{matrix} 1&2&3&4&5&6&7&8&9 \\
\downarrow&\downarrow&\downarrow&\downarrow&
\downarrow&\downarrow&\downarrow&\downarrow&\downarrow
\\ 3&5 &7 &8 &4 &6 &1 &2 &9
\end{matrix}
$$

\item Sean $\sigma$ y $\tau$ las permutaciones de $[[1,8]]$ cuyas representaciones en la notación cíclica son
$$
\sigma= (1\,2\,3)(4\,5\,6)(7\,8),\qquad
\tau=(1\,3\,5\,7)(2\,6)(4)(8).
$$
Escribir en notación cíclica $\sigma\tau$, $\tau\sigma$, $\sigma^2$, $\sigma^{-1}$, $\tau^{-1}$. 

\item Resolver el problema presentado en el ejemplo \ref{cartas} cuando hay $20$ cartas acomodadas en $5$ filas de $4$.

\item Probar que hay exactamente tres elementos de $S_4$ que tienen dos ciclos de longitud $2$, escritos en la notación cíclica. 

\item Sea $K$ el subconjunto de $S_4$ que contiene la identidad y las tres permutaciones descritas en el ejercicio previo. Escribir la ``tabla de multiplicación'' para $K$, interpretando la multiplicación como la composición de permutaciones.

\item Calcular en número total de permutaciones $\sigma$ de $\mathbb [[1,6]]$ que satisfacen $\sigma^2=\text{id}$ y $\sigma\not=\text{id}$.
\item Sean $\alpha$ y $\beta$ permutaciones de $[[1,9]]$ cuyas representaciones en la notación cíclica son:
$$
\alpha= (1237)(49)(58)(6),\qquad \beta=(135)(246)(789).
$$
Escribir en notación cíclica $\alpha\beta$, $\beta\alpha$, $\alpha^2$, $\beta^2$, $\alpha^{-1}$, $\beta^{-1}$.

\item Sea $X_1=\{0,1\}$, y para $i\ge 2$ definamos $X_i$ como el conjunto de subconjuntos de $X_{i-1}$. Encontrar el valor más pequeño para el cual $|X_i|>10^{100}$.

\item Por cada entero $i$ en el rango $1 \le i \le n-1$ definimos $\tau_i$ como la permutación de $[[1,n]]$ que intercambia $i$ e $i+1$ y no afecta los otros elementos. Explícitamente 
$$
\tau_i = (1)(2)\cdots(i-1)(i\ i+1)(i+2)\cdots(n).
$$
Probar que toda permutación de $[[1,n]]$ puede ser expresada en términos de $\tau_1,\tau_2,\ldots,\tau_{n-1}$. 

\item Una permutación de $[[1,n]]$ que tenga solo un ciclo (necesariamente de longitud $n$) es llamada \textit{cíclica}  \index{permutación cíclica}. Probar que hay $(n-1)!$ permutaciones cíclicas de $[[1,n]]$.

\item Un mazo de $52$ cartas es dividido en dos partes iguales y luego se alternan las cartas de una y otra parte. Es decir si la numeración original era $1,2,3,\ldots,54$, el nuevo orden es $1,27,2,28,\ldots$ ¿Cuántas veces se debe repetir este procedimiento para obtener de nuevo el mazo original? 
\end{enumerate}
