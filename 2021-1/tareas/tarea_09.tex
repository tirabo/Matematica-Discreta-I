% PDFLaTeX
\documentclass[a4paper,12pt,twoside,spanish]{amsbook}

\input{tareas_definiciones} 


\begin{document}


\tarea{9}


\begin{ejercicio}[1]
	
		(60 pts)
		\begin{enumerate}
			\item[(i)] Encontrar todas las soluciones de la ecuación en congruencia
			$$30\,x\equiv 2 \quad (67)$$
			usando el método visto en clase.
			\item[(ii)] Dar todas las soluciones $x$ de la ecuación anterior tales que $0 < x < 300$.
		\end{enumerate}
	\end{ejercicio}
    \begin{ejercicio}[2]
	 (40 pts)  Encontrar los últimos dos dígitos de $2^{338}$. (\textit{Ayuda:} Se cumple que $2^{22} \equiv 4 \; (100)$).
	
\end{ejercicio}
	
\begin{comment}^{5}
\begin{solucion}

1(i). (40 pts) 
Primero encontramos el mcd entre $59$ y $10$:
\begin{alignat}3
59 &= 10 \cdot 5 + 9& \qquad&\Rightarrow& \qquad 9 &= 59 -  10 \cdot 5 \label{eq:cle1}\\ 
10 &=  9\cdot 1 +1 & \qquad&\Rightarrow& \qquad 1 &= 10 - 9 \label{eq:cle2}\qquad\\
9 &=  1\cdot 9  +0 & \qquad&& \qquad  &
\end{alignat}
Luego, $1=(59,10)$ y 
\begin{align*}
1 & \overset{(\ref{eq:cle2})}{=} 10 -9  \\
& \overset{(\ref{eq:cle1})}{=}  10 -(59 -  10 \cdot 5) = 10 -59 + 5\cdot  10 \\
&=  6 \cdot  10 + (-1) \cdot 59.
\end{align*}

Por lo tanto
\begin{equation*}
	10 \cdot 6 \equiv 1 (59). 
\end{equation*}
Multiplicando  por $8$,  tenemos
\begin{equation*}
10 \cdot 48 \equiv 8 (59). 
\end{equation*}
Por lo tanto $x_0=48$ es solución y todas las soluciones son de la forma $48+ k\cdot 59$, con $k \in \mathbb Z$.
\vskip .4cm



1(ii). (20 pts) Por (i), las soluciones son de la forma  $48+ k\cdot 59$, con $k \in \mathbb Z$, por lo tanto son soluciones
\begin{equation*}
	48 +(-2)\cdot 59 = -70,\quad 48 +(-1)\cdot 59 = -11,\quad  48 +0 \cdot 59= 48, \quad   48 +1 \cdot 59 =107, \quad   48 +2 \cdot 59 =166.,
\end{equation*}
Luego, las soluciones $x$ tal que $0 < x < 150$ son $x= 48$ y $x=107$. 
\vskip .4cm
\newpage
	2. Por una de la versiones del teorema de Fermat, si  $p$ ves primo  y $p\not= a$, tenemos que
	\begin{equation*}
		a^{p-1} \equiv 1 \,(p).
	\end{equation*}
	Como $13$ es primo y $1=(7,13)$,  tenemos que:
	\begin{equation*}
	7^{12} \equiv 1 \,(13).
	\end{equation*}
	Como $98 = 12 \cdot 8 + 2$, 
	\begin{equation*}
		7^{98} \equiv 7^{12 \cdot 8 + 2}  \equiv 7^{12 \cdot 8}\, 7^{2} \equiv (7^{12})^8\, 7^{2}\equiv 1^8\, 7^{2}\equiv 49 \equiv 3\cdot 13 +10  \equiv 10  \, (13).
	\end{equation*}
	
	
\end{solucion}

\end{comment}
\end{document}


	2. 