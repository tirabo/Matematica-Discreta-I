% PDFLaTeX
\documentclass[a4paper,12pt,twoside,spanish]{amsbook}
%\documentclass[a4paper,11pt,twoside]{book}
%\documentclass[a4paper,11pt,twoside,spanish]{amsbook}

%%%---------------------------------------------------


\usepackage{etex}
\tolerance=10000
\renewcommand{\baselinestretch}{1.3}

\renewcommand{\familydefault}{\sfdefault} % la font por default es sans serif

% Para hacer el  indice en linea de comando hacer
% makeindex main
%% En http://www.tug.org/pracjourn/2006-1/hartke/hartke.pdf hay ejemplos de packages de fonts libres, como los siguientes:
%\usepackage{cmbright}
%\usepackage{pxfonts}
%\usepackage[varg]{txfonts}
%\usepackage{ccfonts}
%\usepackage[math]{iwona}
%\usepackage[math]{kurier}


\usepackage{t1enc}
%\usepackage[spanish]{babel}
\usepackage{latexsym}
\usepackage[utf8]{inputenc}
\usepackage{verbatim}
\usepackage{multicol}
\usepackage{amsgen,amsmath,amstext,amsbsy,amsopn,amsfonts,amssymb}
\usepackage{amsthm}
\usepackage{calc}         % From LaTeX distribution
\usepackage{graphicx}     % From LaTeX distribution
\usepackage{ifthen}
\input{random.tex}        % From CTAN/macros/generic
\usepackage{subfigure}
\usepackage{tikz}
\usetikzlibrary{arrows}
\usetikzlibrary{matrix}
%\usetikzlibrary{graphs}
%\usepackage{tikz-3dplot} %for tikz-3dplot functionality
%\usepackage{pgfplots}
\usepackage{mathtools}
\usepackage{stackrel}
\usepackage{enumerate}
\usepackage{tkz-graph}
%\usepackage{makeidx}
\makeindex

%%%----------------------------------------------------------------------------
\usepackage[a4paper, top=3cm, left=3cm, right=2cm, bottom=2.5cm]{geometry}
%% CONTROLADORES DE.
% Tamaño de la hoja de impresión.
% Tamaños de los laterales del documento.
%%%%%%%%%%%%%%%%%%%%%%%%%%%%%%%%%%%%%%%%%%%%%%%%%%%%%%%%%%%%%%%%%%%%%%%%%%%%%%%%%
%%% \theoremstyle{plain} %% This is the default
%\oddsidemargin 0.0in \evensidemargin -1.0cm \topmargin 0in
%\headheight .3in \headsep .2in \footskip .2in
%\setlength{\textwidth}{16cm} %ancho para apunte
%\setlength{\textheight}{21cm} %largo para apunte
%%%%\leftmargin 2.5cm
%%%%\rightmargin 2.5cm
%\topmargin 0.5 cm
%%%%%%%%%%%%%%%%%%%%%%%%%%%%%%%%%%%%%%%%%%%%%%%%%%%%%%%%%%%%%%%%%%%%%%%%%%%%%%%%%%%

\usepackage{hyperref}
\hypersetup{
	colorlinks=true,
	linkcolor=blue,
	filecolor=magenta,
	urlcolor=cyan,
}
\usepackage{hypcap}


\renewcommand{\thesection}{\thechapter.\arabic{section}}
\renewcommand{\thesubsection}{\thesection.\arabic{subsection}}

\newtheorem{teorema}{Teorema}[section]
\newtheorem{proposicion}[teorema]{Proposici\'on}
\newtheorem{corolario}[teorema]{Corolario}
\newtheorem{lema}[teorema]{Lema}
\newtheorem{propiedad}[teorema]{Propiedad}

\theoremstyle{definition}

\newtheorem{definicion}{Definici\'on}[section]
\newtheorem{ejemplo}{Ejemplo}[section]
\newtheorem{problema}{Problema}[section]
\newtheorem{ejercicio}{Ejercicio}[section]
\newtheorem{ejerciciof}{}[section]

\theoremstyle{remark}
\newtheorem{observacion}{Observaci\'on}[section]
\newtheorem{nota}{Nota}[section]

\renewcommand{\partname }{Parte }
\renewcommand{\indexname}{Indice }
\renewcommand{\figurename }{Figura }
\renewcommand{\tablename }{Tabla }
\renewcommand{\proofname}{Demostraci\'on}
\renewcommand{\appendixname }{}
\renewcommand{\contentsname }{Contenidos }
\renewcommand{\chaptername }{}
\renewcommand{\bibname }{Bibliograf\'\i a }



\newcommand{\tarea}[1]{
	\begin{center}
		{\Large Matemática Discreta I - 2021/1} \vskip.4cm
		{\Large Tarea #1}\vskip .4cm
\end{center}}

\renewenvironment{ejercicio}% environment name
{% begin code
	\par\vskip .5cm%
	{\noindent\color{blue}Ejercicios}%
	\vskip .2cm
}%
{%
	\vskip .2cm}% end code


\newenvironment{solucion}% environment name
{% begin code
	\par\vskip .2cm%
	{\noindent\color{blue}Solución}%
	\vskip .2cm
}%
{%
	\vskip .2cm}% end code


\begin{document}
	
	%\frame{\titlepage}
	
	
	\tarea{4}
    \begin{ejercicio}
        \begin{enumerate}
            \item[1.] (80 pts)  ¿De cuántas formas distintas se pueden escoger $5$  cartas de una baraja de $52$ cartas?
            \begin{enumerate}
                \item (20 pts) Si no hay restricciones.
                \item (30 pts) Si debe haber tres picas y dos corazones.
                \item (30 pts) Si debe haber al menos una carta de cada palo.
                %\item (30 pts) Si de haber al menos $3$ picas.
            \end{enumerate}

           \vskip .2cm


            \item[2.] (20 pts) Sea $n\in\mathbb{N}$.  Probar la siguiente igualdad:
            $$
            n = \frac{n! + (n+2)! - n(n+2)(n-1)!}{n!+(n-1)!+(n+1)!}.
            $$
        \end{enumerate}
    \end{ejercicio}
	
	\begin{comment}
	\begin{solucion}
		\noindent (a) Elegir 5 entre 8 es el número combinatorio
		$$
		\binom{8}{5} = \frac{8!}{(8-5)!5!} =\frac{8!}{3!5!} = \frac{8 \cdot 7\cdot 6}{3 \cdot 2\cdot 1} = 8 \cdot 7 =  \colorbox{lightgray}{56}.
		$$
		(Dejando indicado el número combinatorio alcanzaba).
		
		\vskip .4cm
		
		\noindent (b) Para elegir 3 mujeres entre 5 hay $\binom{5}{3}$ posibilidades. Para elegir 2 hombres entre 3 hay $\binom{3}{2}$ posibilidades. Luego  el  número total de posibilidades es
		$$
		\binom{5}{3} \binom{3}{2} = \frac{5!}{2!3!}3 =  \frac{5\cdot 4}{2} 3 =  \colorbox{lightgray}{30}.
		$$
		
			\vskip .4cm
		
		\noindent (c) Primero  elegimos los cargos a ser ocupados por los 3 hombres: para el primer hombre hay 5 posibilidades, para el segundo 4 y para el tercero 3. Es decir el total de posibles asignaciones de cargos a los 3 hombres es $5\cdot 4\cdot 3= 60$.
		
		
		Quedan dos cargos libres (los que dejaron libres los hombres). Para el  primer cargo hay 5 posibilidades y para el segundo hay 4. Luego el total de posibilidades para los dos cargos restantes es $5 \cdot 4 = 20$.
		
		Concluyendo, por el principio de multiplicación, hay
		$$
		60 \cdot 20 =  \colorbox{lightgray}{1200}
		$$
		posibilidades.
		
			\vskip .4cm
		
		\noindent (d)\textit{ Primera forma.} El  hecho de que \textbf{A} y \textbf{E} no están en el equipo se puede partir en tres casos: 1) \textbf{A} y \textbf{E} no están en el equipo, 2) \textbf{A} está en el equipo y \textbf{E} no está en el equipo, y 3) \textbf{A} no está en el equipo y \textbf{E}  está en el equipo.
		
		1) Debemos considerar que el grupo tiene 6 personas (pues \textbf{A} y \textbf{E} nunca pueden ser seleccionados) y estos 6 deben ocupar 5 cargos distintos. Luego la cantidad de elecciones posibles es $6  \cdot 5 \cdot 4 \cdot 3 \cdot 2 = 720$.
		
		2) Ahora debemos pensar que \textbf{A} ocupa uno de los 5 cargos, por lo tanto hay 5 posibilidades  y los 6 restantes (pues \textbf{E} no está) ocupan los 4 cargos restante: $6  \cdot 5 \cdot 4 \cdot 3 =360$ posibilidades. Por lo tanto el total de posibilidades es $5  \cdot  360 = 1800$.
		
		
		3) Es completamente análogo a 2), por lo tanto hay 	1800 posibilidades.
	
		
		Como los tres casos son mutuamente excluyentes (no hay intersección entre ellos), se aplica el principio de adición y por lo tanto hay $$720 + 1800 + 1800 = \colorbox{lightgray}{4320}$$ posibilidades.
		\vskip .4cm
		
		(d)\textit{ Segunda forma. }En  este ejercicio podemos pensar en  el  "complemento" es decir (1) calculamos todas las posibilidades,  después (2) calculamos cuando \textbf{A} y \textbf{E} están juntos y luego el resultado será (1) - (2).

 (1) Todas las posible elecciones es elegir 5 entre 8, en orden sin repetición (el primero director,  el segundo subdirector, etc.) la solución es entonces
 $$
 \frac{8!}{(8-5)!} = \frac{8!}{3!} = 8 \cdot 7 \cdot 6 \cdot 5 \cdot 4 = 6720.
 $$

 (2) Primero elegimos un cargo para \textbf{A}, son  5 posibilidades, después un cargo para \textbf{B}, son 4 posibilidades. Luego quedan tres cargos para 6 personas (selecciones de 3 entre 6, en orden, sin repetición), son $6 \cdot 5 \cdot 4 =120$ posibilidades. el  total es entonces
 $$
5 \cdot 4 \cdot 120 = 2400.
 $$

 De (1) y (2) se deduce que las posibles formas de elegir el comité (con cargos) donde \textbf{A} y \textbf{E} no estén juntos es
 $$
 6720 - 2400 = \colorbox{lightgray}{4320}.
 $$
		
	\end{solucion}
	
	\end{comment}
\end{document}

