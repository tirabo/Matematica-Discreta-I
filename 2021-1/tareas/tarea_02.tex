% PDFLaTeX
\documentclass[a4paper,12pt,twoside,spanish]{amsbook}

\input{tareas_definiciones} 


\begin{document}

\tarea{2}

\begin{ejercicio}
    Sea $\{a_n\}_{n\in\mathbb N}$ la sucesión definida recursivamente por
    $$\begin{cases}
    a_1=1, \\a_2=2, \\a_{n} = (n-2)a_{n-1}+2(n-1)a_{n-2}, \text{ para $n\geq 3$}.
    \end{cases}$$
	\begin{enumerate}
		\item[1.]  Calcule $a_3$ y $a_4$ usando  recursión. (30 pts)
		\item[2.] Pruebe por inducción que $a_n=n!$ para todo $n\in \mathbb N$. (70 pts)
	\end{enumerate}
	\vskip .2cm
\end{ejercicio}
\begin{comment}
	\begin{solucion}
	{\noindent 1.} Por definición $u_0 = 1, \; u_1 = 0$, luego:
	\begin{equation*}
	u_2 = 5u_{2-1} - 6 u_{2-2} = 5u_{1} - 6 u_{0} = 5 \cdot 0 - 6 \cdot 1 = -6 
	\end{equation*}
	Ahora  $u_1 = 0, \; u_2 = -6$, luego:
	\begin{equation*}
	u_3 = 5u_{3-1} - 6 u_{3-2} = 5u_{2} - 6 u_{1} = 5 \cdot (-6) - 6 \cdot 0 = -30. 
	\end{equation*}
	
	\vskip .2cm
	
	
	{\noindent 2.} \textbf{Caso base.} 
	Por un lado $u_0 =1$, por definición, por otro lado  si calculamos con la fórmula: $u_0 = 3 \cdot 2^0 - 2 \cdot 3^0 = 3- 2 = 1$ y listo.
	\vskip .2cm
	Por un lado $u_1 =0$, por definición, por otro lado  si calculamos con la fórmula: $u_1 = 3 \cdot 2^1 - 2 \cdot 3^1 = 3 \cdot 2- 2 \cdot 3 =  6-6 = 0$ y listo.
	
	\vskip .8cm
	
	\textbf{Paso inductivo.} Debemos probar que si  vale
	\begin{equation*}
	u_h = 3\cdot 2^h - 2 \cdot 3^h \text{ \;para\; }  0 \le h \le k \text{\; con\; }  k \ge 0, \tag{HI}
	\end{equation*}
	eso implica que
	\begin{equation*}
	u_{k+1} = 3\cdot 2^{k+1} - 2 \cdot 3^{k+1}. \tag{*}
	\end{equation*}
	
	Comenzamos por el lado izquierdo de (*):
	
	
	\begin{align*}
	u_{k+1} &=   5u_{k+1-1} - 6 u_{k+1-2}&&\text{ \;(por definición de $u_n$)\; } \\
	&=   5u_{k} - 6 u_{k-1}&&\\
	&=   5( 3\cdot 2^{k} - 2 \cdot 3^{k}) - 6 ( 3\cdot 2^{k-1} - 2 \cdot 3^{k-1})&&\text{ \;(por HI)\; } \\
	&=   5\cdot  3\cdot 2^{k} - 5 \cdot 2 \cdot 3^{k} - 6 \cdot  3\cdot 2^{k-1} + 6 \cdot  2 \cdot 3^{k-1}&& \\
	&=   5\cdot  3\cdot 2\cdot 2^{k-1} - 5 \cdot 2 \cdot 3 \cdot  3^{k-1} - 6 \cdot  3\cdot 2^{k-1} + 6 \cdot  2 \cdot 3^{k-1}&& \\
	&=   30\cdot 2^{k-1} -30 \cdot  3^{k-1} - 18\cdot 2^{k-1} + 12 \cdot 3^{k-1}&&\\
	&=   (30-18)\cdot 2^{k-1} +(-30+12) \cdot  3^{k-1}&& \text{ \;(factor común $ 2^{k-1}$ y $ 3^{k-1}$ )\; } \\
	&=   12\cdot 2^{k-1} -18 \cdot  3^{k-1}&&\\
	&=   3 \cdot 2^2 \cdot 2^{k-1} - 2 \cdot 3^2 \cdot  3^{k-1}&&\\
	&=   3 \cdot 2^{2+k-1} - 2  \cdot  3^{2+k-1}&&\\
	&=   3 \cdot 2^{k+1} - 2  \cdot  3^{k+1}&&.
	\end{align*}
	
	Esto  prueba (*).
	\end{solucion}
\end{comment}



\end{document}

