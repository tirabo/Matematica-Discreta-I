% PDFLaTeX
\documentclass[a4paper,12pt,twoside,spanish]{amsbook}

\input{tareas_definiciones} 


\begin{document}

\tarea{7}

\begin{ejercicio}
	\begin{enumerate}
		%\item[1.] (60 pts) Probar que para $n \ge 0$ se satisface $3 | 2^{3n+2} + 5^{n+3}$.  
		\item[1.] (50 pts) Probar que no existen $m,n \in \mathbb Z$ tal que $9\, m^3=n^3$. 
		\item[2.]  (50 pts) Calcular el máximo común divisor y el mínimo común múltiplo de $2772$ y $33000$ usando la descomposición en números primos.
	\end{enumerate}
\end{ejercicio}
	
\begin{comment}
\begin{solucion}

1. Debemos probar que $\sqrt[3]{9}$ no es cociente de dos números naturales. Lo haremos por el absurdo: supongamos que $\sqrt[3]{9} = \frac{m}{n}$  con $m, n$ números naturales. 
Luego,
\begin{equation*}
	\sqrt[3]{9} = \frac{m}{n} \quad \Rightarrow  \quad  (\sqrt[3]{9})^3 = \left(\frac{m}{n}\right)^3  \quad \Rightarrow  \quad  9 = \frac{m^3}{n^3}  \quad \Rightarrow  \quad 9 {n^3}= {m^3}.
\end{equation*}
Veamos entonces que no es posible que 	$ 9 {n^3}= {m^3}$. Podemos escribir,
\begin{align*}
	 m &= 3^{k}p_1^{e_1}p_2^{e_2}\ldots p_r^{e_r},\\
	 n &= 3^{h}p_1^{f_1}p_2^{f_2}\ldots p_r^{f_r},
\end{align*}
con $p_1,\ldots,p_r$ primos distintos entre sí y distintos de 3 y cada exponente ($k,h,e_i,f_i$) no negativo.

Luego
\begin{align*}
m^3 &= 3^{3k}p_1^{3e_1}p_2^{3e_2}\ldots p_r^{3e_r},\\
n^3 &= 3^{3h}p_1^{3f_1}p_2^{3f_2}\ldots p_r^{3f_r},
\end{align*}
y por lo tanto
\begin{align*}
9 {n^3}&= {m^3}&\quad &\Leftrightarrow \\
	3^2  3^{3h}p_1^{3f_1}p_2^{3f_2}\ldots p_r^{3f_r}&= 3^{3k}p_1^{3e_1}p_2^{3e_2}\ldots p_r^{3e_r}&\quad &\Leftrightarrow  \\
		  3^{3h+2}p_1^{3f_1}p_2^{3f_2}\ldots p_r^{3f_r}&= 3^{3k}p_1^{3e_1}p_2^{3e_2}\ldots p_r^{3e_r}. &&
\end{align*}
Por el teorema fundamental de la aritmética, el exponente de $3$  a la izquierda de la igualdad debe ser igual al exponente de $3$  a la derecha de la igualdad,  es decir $3h+2 = 3k$, lo cual implica que $2= 3(k-h)$ $\Rightarrow$ $3|2$, absurdo.

El absurdo vino de suponer que  $\sqrt[3]{9}$ es racional. Por lo tanto   $\sqrt[3]{9}$ no es racional.

\vskip .4cm

	2. Primero  hagamos la descomposición prima de cada número:
	\begin{align*}
		1176  &= 2 \cdot 588 = 2 \cdot 2 \cdot 294=  2 \cdot  2 \cdot 2 \cdot 147 =  2 \cdot  2 \cdot 2 \cdot 3  \cdot 49 = \; 2^3 \cdot 3  \cdot 7^2.\\
		450 &= 10 \cdot 45 =  2 \cdot 5 \cdot 3 \cdot 15  = 2 \cdot 5 \cdot 3 \cdot 3 \cdot 5  =  2 \cdot 5^2 \cdot 3^2 .
	\end{align*}
	
	Podemos los números anteriores incluyendo los primos que participan en ambos:
	\begin{align*}
	1176  &=\; 2^3 \cdot 3^1 \cdot 5^0  \cdot 7^2.\\
	450 &= \; 2^1 \cdot 3^2 \cdot 5^2  \cdot 7^0.
	\end{align*}
	
	Luego 
	\begin{align*}
	\operatorname{mcd}(1176,450)  &=\; 2^1 \cdot 3^1 \cdot 5^0  \cdot 7^0 = 6 .\\
	\operatorname{mcm}(1176,450)  &= \; 2^3 \cdot 3^2 \cdot 5^2  \cdot 7^2 = 88200.
	\end{align*}
	
\end{solucion}

%\end{comment}
\end{document}

	1. Lo haremos por inducción.
\vskip .3cm 
Caso base $n=0$. 

En  este caso  $2^{3n+2} + 5^{n+3} =  2^{2} + 5^{3} = 4 +125  = 129 = 3 \cdot 43$,  es decir $3| 2^{2} + 5^{3}$.

\vskip .3cm 

Paso  inductivo. Supongamos que $3 | 2^{3k+2} + 5^{k+3}$ para un $k \ge 0$ (HI),  debemos probar que 
\begin{equation}\label{eq-q}
3 | 2^{3(k+1)+2} + 5^{(k+1)+3}. \tag{*}
\end{equation}
Ahora bien, 
\begin{align*}
2^{3(k+1)+2} + 5^{(k+1)+3} &= 2^{3k+3+2} + 5^{k+1+3}\\
&= 2^{3k+2}\cdot 2^{3} + 5^{k+3}\cdot 5^{1} \\
&= 8 \cdot 2^{3k+2} + 5\cdot 5^{k+3}\\
&= 5( 2^{3k+2} + 5^{k+3})  +3 \cdot 2^{3k+2}.
\end{align*}
Es decir
\begin{equation}
2^{3(k+1)+2} + 5^{(k+1)+3} = 5( 2^{3k+2} + 5^{k+3})  +3 \cdot 2^{3k+2}. \tag{**}
\end{equation}
Por hipótesis inductiva (HI), $3 | 2^{3k+2} + 5^{k+3}$ y por lo tanto $ 3|5( 2^{3k+2} + 5^{k+3})$. Por otro lado es claro  que $3 | 3 \cdot 2^{3k+2}$. Entonces,  3 divide a la suma de  $5( 2^{3k+2} + 5^{k+3})$ y $3 \cdot 2^{3k+2}$:
\begin{equation*}
3|5( 2^{3k+2} + 5^{k+3}) + 3 \cdot 2^{3k+2} \overset{(**)}{=}2^{3(k+1)+2} + 5^{(k+1)+3},
\end{equation*}
y esto  prueba (*), y por lo tanto prueba el ejercicio.
\vskip .4cm 


	2. 