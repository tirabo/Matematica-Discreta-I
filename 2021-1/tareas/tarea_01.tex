% PDFLaTeX
\documentclass[a4paper,12pt,twoside,spanish]{amsbook}

\input{tareas_definiciones} 


\begin{document}

\tarea{1}

\begin{ejercicio}
	 Demostrár por inducción la siguiente fórmula
    \begin{equation*}
		\sum_{j=1}^{n}( 4j -1)  = n(2n+1) , \; n \in \mathbb N.
	\end{equation*}
    Debés hacer una demostración por inducción mostrando detalladamente cada paso. 

    
\end{ejercicio}

\begin{comment}
\begin{solucion}
	Se demostrará la fórmula por inducción sobre $n$.
	
	\textbf{Caso base.} $n=0$. En este caso  $\sum_{k=0}^{0} (2k +1)  = 2\cdot 0 +1 = 1$ y por otro lado $(0 +1)^2 = 1^2 =1$. Es decir el resultado vale para $n=0$.
	
	
	\vskip .4cm
	
	\textbf{Paso inductivo.} Debemos probar que si  vale
	\begin{equation*}
	\sum_{k=0}^{h} (2k +1) = (h+1)^2, \;  h \ge 0, \tag{HI}
	\end{equation*}
	entonces  
	\begin{equation*}
	\sum_{k=0}^{h+1} (2k +1) = ((h+1)+1)^2 = (h+2)^2.\tag{*}
	\end{equation*}
	\vskip .4cm
	Comenzamos por el lado izquierdo de (*):
	
	\begin{align*}
	\sum_{k=0}^{h+1} (2k +1)  &=   \sum_{k=0}^{h} (2k +1)  +  (2(h+1) +1)&&\text{ \;(por definición de sumatoria)\; } \\	
	&=   \sum_{k=0}^{h} (2k +1)  +  (2h+3)&&\\	
	&=   (h+1)^2 +  (2h+3)&&\text{ \;(por HI)\; } \\	
	&=   h^2 + 2h +1 + 2h+3&& \\	
	&=   h^2 + 4h +4&& \\
	&=   (h+2)^2&& \\	
	\end{align*}
	Esto  prueba (*).
\end{solucion}
	
\end{comment}


\end{document}

