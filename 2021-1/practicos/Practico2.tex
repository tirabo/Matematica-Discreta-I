% PDFLaTeX
\documentclass[a4paper,12pt,twoside,spanish,reqno]{amsbook}
%%%---------------------------------------------------

%\renewcommand{\familydefault}{\sfdefault} % la font por default es sans serif
%\usepackage[T1]{fontenc}

% Para hacer el  indice en linea de comando hacer 
% makeindex main
%% En http://www.tug.org/pracjourn/2006-1/hartke/hartke.pdf hay ejemplos de packages de fonts libres, como los siguientes:
%\usepackage{cmbright}
%\usepackage{pxfonts}
%\usepackage[varg]{txfonts}
%\usepackage{ccfonts}
%\usepackage[math]{iwona}
\usepackage[math]{kurier}

\usepackage{etex}
\usepackage{t1enc}
\usepackage{latexsym}
\usepackage[utf8]{inputenc}
\usepackage{verbatim}
\usepackage{multicol}
\usepackage{amsgen,amsmath,amstext,amsbsy,amsopn,amsfonts,amssymb}
\usepackage{amsthm}
\usepackage{calc}         % From LaTeX distribution
\usepackage{graphicx}     % From LaTeX distribution
\usepackage{ifthen}
\input{random.tex}        % From CTAN/macros/generic
\usepackage{subfigure} 
\usepackage{tikz}
\usetikzlibrary{arrows}
\usetikzlibrary{matrix}
\usepackage{mathtools}
\usepackage{stackrel}
\usepackage{enumitem}
\usepackage{tkz-graph}
%\usepackage{makeidx}
\usepackage{hyperref}
\hypersetup{
    colorlinks=true,
    linkcolor=blue,
    filecolor=magenta,      
    urlcolor=cyan,
}
\usepackage{hypcap}
\numberwithin{equation}{section}
% http://www.texnia.com/archive/enumitem.pdf (para las labels de los enumerate)
\renewcommand\labelitemi{$\circ$}
\setlist[enumerate, 1]{label={(\arabic*)}}
\setlist[enumerate, 2]{label=\emph{\alph*)}}


%%% FORMATOS %%%%%%%%%%%%%%%%%%%%%%%%%%%%%%%%%%%%%%%%%%%%%%%%%%%%%%%%%%%%%%%%%%%%%
\tolerance=10000
\renewcommand{\baselinestretch}{1.3}
\usepackage[a4paper, top=3cm, left=3cm, right=2cm, bottom=2.5cm]{geometry}
\usepackage{setspace}
%\setlength{\parindent}{0,7cm}% tamaño de sangria.
\setlength{\parskip}{0,4cm} % separación entre parrafos.
\renewcommand{\baselinestretch}{0.90}% separacion del interlineado
\setlist[1]{topsep=8pt,itemsep=.4cm,partopsep=4pt, parsep=4pt} %espacios nivel 1 listas
\setlist[2]{itemsep=.15cm}  %espacios nivel 2 listas
%%%%%%%%%%%%%%%%%%%%%%%%%%%%%%%%%%%%%%%%%%%%%%%%%%%%%%%%%%%%%%%%%%%%%%%%%%%%%%%%%%%
%\end{comment}
%%% FIN FORMATOS  %%%%%%%%%%%%%%%%%%%%%%%%%%%%%%%%%%%%%%%%%%%%%%%%%%%%%%%%%%%%%%%%%

\newcommand{\rta}{\noindent\textit{Rta: }} 

\begin{document}
    \baselineskip=0.55truecm %original
    


{\bf \begin{center} \large  Práctico 2 \\ Matemática Discreta I -- Año 2021/1 \\ FAMAF \end{center}}


\begin {enumerate}
\setlength\itemsep{1.1em}

\item La cantidad de dígitos o cifras de un número se cuenta a partir del primer dígito distinto de cero. Por ejemplo, $0035010$ es un número de $5$ dígitos.
\begin{enumerate}
\item ¿Cuántos números de 5 dígitos hay?
\item ¿Cuántos números pares de 5 dígitos  hay?
\item ¿Cuántos números de 5 dígitos existen con sólo un 3?
\item ¿Cuántos números capicúas de 5 dígitos existen?
\item ¿Cuántos números capicúas de a lo sumo 5 dígitos hay?
\end{enumerate}


\item ¿Cuántos números de 6 cifras pueden formarse con los dígitos de 112200?


\item ¿Cuántos números impares de cuatro cifras hay?


\item ¿Cuántos números múltiplos de  5 y menores que 4999 hay?


\item En los boletos viejos de ómnibus, aparecía un {\em número} de 5 cifras (en este caso podían empezar con 0), y uno tenía un {\it boleto capicúa} si el número lo era.
\begin{enumerate}
\item ¿Cuántos boletos capicúas había?
\item ¿Cuántos boletos había en los cuales no hubiera ningún dígito repetido?
\end{enumerate}


\item Las antiguas patentes de auto tenían una letra indicativa de la provincia y luego 6 dígitos. (En algunas provincias, Bs. As. y Capital, tenían 7 dígitos, pero ignoremos eso por el momento). Luego  vinieron patentes que tienen 3 letras y luego 3 dígitos. Finalmente, ahora las patentes tienen 2 letras, luego 3 dígitos y a continuación dos letras más ¿Cuántas patentes pueden hacerse con cada uno de los sistemas?


\item Si uno tiene 8 CD distintos de Rock, 7 CD distintos de música clásica y 5 CD distintos de cuartetos,
\begin{enumerate}
    \item ¿Cuántas formas distintas hay de seleccionar un CD?

    \item ¿Cuántas formas hay de seleccionar tres CD, uno de cada tipo?

    \item Un sonidista en una fiesta de casamientos planea poner 3 CD, uno a continuación de otro. ¿Cuántas formas distintas tiene de hacerlo si le han dicho que no mezcle más de dos estilos?
\end{enumerate}


\item Mostrar que si uno arroja un dado $n$ veces y suma todos los resultados obtenidos, hay $\dfrac{6^n}{2}$ formas distintas de obtener una suma par.


\item ¿Cuántos enteros entre 1 y 10000 tienen exactamente un 7 y exactamente un 5 entre sus cifras?


\item ¿Cuántos subconjuntos de $\{0,1,2,\dots,8,9\}$ contienen al menos un impar?


\item El truco se juega con un mazo de 40 cartas, y se reparten 3 cartas a cada jugador. Obtener el 1 de espadas (el {\it macho}) es muy bueno. También lo es, por otros motivos, obtener un 7 y un 6 del mismo palo ({\it tener 33}). ¿Qué es más probable: obtener el macho, o tener 33?


\item ¿Cuántos comités pueden formarse de un conjunto de 6 mujeres y 4 hombres, si el comité debe estar compuesto por 3 mujeres y 2 hombres?


\item ¿De cuántas formas puede formarse un comité de 5 personas tomadas de un grupo de 11 personas entre las cuales hay 4 profesores y 7 estudiantes, si:
\begin{enumerate}
    \item No hay restricciones en la selección?

    \item El comité debe tener exactamente 2 profesores?

    \item El comité debe tener al menos 3 profesores?

    \item El profesor $X$ y el estudiante $Y$ no pueden estar juntos en el comité?
\end{enumerate}


\item Si en un torneo de fútbol participan $2n$ equipos, probar que el número total de opciones posibles para la primera fecha es $1\cdot 3\cdot 5 \cdots (2n - 1)$. sugerencia: use un argumento por inducción. 


\item En una clase hay $n$ chicas y $n$ chicos. Dar el número de maneras de ubicarlos en una fila de modo que todas las chicas estén juntas.


\item ¿De cuántas maneras  distintas pueden sentarse 8 personas en una mesa circular?


\item \begin{enumerate}
    \item ¿De cuántas maneras distintas pueden sentarse 6 hombres y 6 mujeres en una mesa circular si nunca deben quedar dos mujeres juntas?
    \item \'Idem, pero con 10 hombres y 7 mujeres.
\end{enumerate}


\item 
\begin{enumerate}
    \item  ¿De cuántas formas distintas pueden ordenarse las letras de la palabra MATEMATICA
    \item \'Idem con las palabras ALGEBRA, GEOMETRIA.
    \item ¿De cuántas formas distintas pueden ordenarse las letras de la palabra MATEMATICA si se pide que las consonantes y las vocales se alternen?
\end{enumerate}


\item ¿Cuántas diagonales tiene un polígono regular de $n$ lados?


\item Dados $m$, $n$ y $k$ naturales tales que $m \le k \le n$, probar que se verifica
\begin{equation*}
    \binom{n}{k}\binom{k}{m} = \binom{n}{m}\binom{n-m}{k-m}.
\end{equation*}



\item Probar que para todo $i$, $j$, $k \in {\mathbb N}_0$ vale
\begin{equation*}
    \binom{i + j + k}{i}\binom{j+k}{j} = \frac{(i+j+k)!}{i!j!k!}
\end{equation*}


\item Demostrar que para todo $n \in \mathbb N$ vale:
\begin{enumerate}
  \item $\displaystyle{\binom{n}{0} + \binom{n}{1} + \cdots + \binom{n}{n} = 2^n}$.
\medskip
  \item $\displaystyle{\binom{n}{0} - \binom{n}{1} + \cdots + (-1)^n\binom{n}{n} = 0}$
  \end{enumerate}


\item Probar que para todo natural $n$ vale que 
\begin{equation*}
    \binom{2n}{2} = 2 \binom{n}{2} + n^2.
\end{equation*}

\end{enumerate}



\subsection*{$\S$  Ejercicios de repaso}

\begin{enumerate}[resume]

\item Con $20$ socios de un club se desea formar $5$ listas electorales (disjuntas).
Cada lista consta de $1$ Presidente, $1$ Tesorero y $2$ vocales.  ¿De cuántas
formas puede hacerse?


\item ¿De cuántas formas se pueden fotografiar $7$ matrimonios en una hilera,
de tal forma que cada hombre aparezca al lado de su esposa?


\item ¿De cuántas formas pueden distribuirse $14$ libros distintos entre dos
personas de manera tal que cada persona reciba al menos $3$ libros?

\item Cecilia ha olvidado la contraseña de su correo electrónico, la cual est\'a formada por $11$ caracteres: $4$ d\'igitos y $7$ letras todos mezclados. ?`Cu\'al es el m\'aximo n\'umero de intentos que deber\'ia probar para entrar a su correo si:
\begin{enumerate}
    \item  tiene en cuenta que hay 26 letras?
    
    \item la letra B y el d\'igito 2 no pueden estar ambos en la contraseña?
    
    \item  recuerda que ha formado la clave de la siguiente manera: los primeros siete caracteres son una mezcla de las letras de su nombre (CECILIA), y los \'ultimos cuatro caracteres son una permutaci\'on de las cuatro cifras de su a\~no de nacimiento (1998)?
\end{enumerate}

%\item Se está arreglando la plaza principal de la ciudad. Leandro, el jardinero, ha calculado que
%necesita $6$ plantines para cada cantero rectangular. Él dispone de $4$ clases de plantines y quiere
%que en cada cantero haya, como mínimo, dos clases de plantines diferentes. ¿De cuántas maneras puede seleccionar %los plantines a colocar en cada cantero? 

\end{enumerate}
\end{document}
