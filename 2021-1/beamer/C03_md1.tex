%\documentclass{beamer} 
\documentclass[handout]{beamer} % sin pausas
\usetheme{CambridgeUS}

%\setbeamertemplate{background}[grid][step=8 ] % cuadriculado

\usepackage{etex}
\usepackage{t1enc}
\usepackage[spanish,es-nodecimaldot]{babel}
\usepackage{latexsym}
\usepackage[utf8]{inputenc}
\usepackage{verbatim}
\usepackage{multicol}
\usepackage{amsgen,amsmath,amstext,amsbsy,amsopn,amsfonts,amssymb}
\usepackage{amsthm}
\usepackage{calc}         % From LaTeX distribution
\usepackage{graphicx}     % From LaTeX distribution
\usepackage{ifthen}
%\usepackage{makeidx}
\input{random.tex}        % From CTAN/macros/generic
\usepackage{subfigure} 
\usepackage{tikz}
\usepackage[customcolors]{hf-tikz}
\usetikzlibrary{arrows}
\usetikzlibrary{matrix}
\tikzset{
	every picture/.append style={
		execute at begin picture={\deactivatequoting},
		execute at end picture={\activatequoting}
	}
}
\usetikzlibrary{decorations.pathreplacing,angles,quotes}
\usetikzlibrary{shapes.geometric}
\usepackage{mathtools}
\usepackage{stackrel}
%\usepackage{enumerate}
\usepackage{enumitem}
\usepackage{tkz-graph}
\usepackage{polynom}
\polyset{%
	style=B,
	delims={(}{)},
	div=:
}
\renewcommand\labelitemi{$\circ$}
\setlist[enumerate]{label={(\arabic*)}}

\setbeamertemplate{itemize item}{$\circ$}
\setbeamertemplate{enumerate items}[default]
\definecolor{links}{HTML}{2A1B81}
\hypersetup{colorlinks,linkcolor=,urlcolor=links}


\newcommand{\Id}{\operatorname{Id}}
\newcommand{\img}{\operatorname{Im}}
\newcommand{\nuc}{\operatorname{Nu}}
\newcommand{\im}{\operatorname{Im}}
\renewcommand\nu{\operatorname{Nu}}
\newcommand{\la}{\langle}
\newcommand{\ra}{\rangle}
\renewcommand{\t}{{\operatorname{t}}}
\renewcommand{\sin}{{\,\operatorname{sen}}}
\newcommand{\Q}{\mathbb Q}
\newcommand{\R}{\mathbb R}
\newcommand{\C}{\mathbb C}
\newcommand{\K}{\mathbb K}
\newcommand{\F}{\mathbb F}
\newcommand{\Z}{\mathbb Z}
\newcommand{\N}{\mathbb N}
\newcommand\sgn{\operatorname{sgn}}
\renewcommand{\t}{{\operatorname{t}}}
\renewcommand{\figurename }{Figura}

\include{definiciones}

\newcommand{\nc}{\newcommand}

%%%%%%%%%%%%%%%%%%%%%%%%%LETRAS

\nc{\FF}{{\mathbb F}} \nc{\NN}{{\mathbb N}} \nc{\QQ}{{\mathbb Q}}
\nc{\PP}{{\mathbb P}} \nc{\DD}{{\mathbb D}} \nc{\Sn}{{\mathbb S}}
\nc{\uno}{\mathbb{1}} \nc{\BB}{{\mathbb B}} \nc{\An}{{\mathbb A}}

\nc{\ba}{\mathbf{a}} \nc{\bb}{\mathbf{b}} \nc{\bt}{\mathbf{t}}
\nc{\bB}{\mathbf{B}}

\nc{\cP}{\mathcal{P}} \nc{\cU}{\mathcal{U}} \nc{\cX}{\mathcal{X}}
\nc{\cE}{\mathcal{E}} \nc{\cS}{\mathcal{S}} \nc{\cA}{\mathcal{A}}
\nc{\cC}{\mathcal{C}} \nc{\cO}{\mathcal{O}} \nc{\cQ}{\mathcal{Q}}
\nc{\cB}{\mathcal{B}} \nc{\cJ}{\mathcal{J}} \nc{\cI}{\mathcal{I}}
\nc{\cM}{\mathcal{M}} \nc{\cK}{\mathcal{K}}

\nc{\fD}{\mathfrak{D}} \nc{\fI}{\mathfrak{I}} \nc{\fJ}{\mathfrak{J}}
\nc{\fS}{\mathfrak{S}} \nc{\gA}{\mathfrak{A}}
%%%%%%%%%%%%%%%%%%%%%%%%%LETRAS


\title[Clase 3 - Recursión]{Matemática Discreta I \\ Clase 3 - Recursión}
%\author[C. Olmos / A. Tiraboschi]{Carlos Olmos / Alejandro Tiraboschi}
\institute[]{\normalsize FAMAF / UNC
	\\[\baselineskip] ${}^{}$
	\\[\baselineskip]
}
\date[23/03/2020]{23 de marzo  de 2020}



\begin{document}




\frame{\titlepage} 

%\frame{\frametitle{Índice}\tableofcontents} 


\begin{frame}\frametitle{Definiciones recursivas} 

    \pause

Sea $\mathbb N$ el conjuntos de enteros positivos, esto es
$$
\mathbb N = \{ n \in \mathbb Z | n\ge 1\},
$$
y denotemos $\mathbb N_0$ el conjunto $\mathbb N \cup \{0\}$, esto
es
$$
\mathbb N_0 = \{ n \in \mathbb Z | n\ge 0\}.
$$\pause
$\mathbb{N}$  es llamado el conjunto de  \textit{números naturales}.     \pause
\vskip .4cm
\begin{itemize}
    \item $X \subset \mathbb N$ (o de $\mathbb N_0$) $\Rightarrow$ $X$ tiene cota inferior ($0$ o un $x <0$).\vskip .4cm    \pause
    \item As{í}, en este caso el axioma del buen orden toma la
    forma\vskip .4cm 

    \setbeamercolor{postit}{fg=black,bg=example text.fg!75!black!10!bg}
    \begin{beamercolorbox}[wd=1.0\textwidth,rounded=true,shadow=true]{postit}%
        \begin{center}\textit{\qquad \quad si $X \subset \mathbb N \vee X \subset\mathbb N_0$, no vacío \; $\Rightarrow$ \; $X$ tiene un mínimo.}\vskip .4cm
\end{center}
\end{beamercolorbox}
    

\end{itemize}


\end{frame}



\begin{frame}\frametitle{ }  
	
Una sucesión $u_n$ puede ser dada en forma explícita, por ejemplo  \pause
\begin{itemize}
	\item $u_n=3n+2$,   \pause o
	\item $w_n =(n+1)(n+2)(n+3)$. 
\end{itemize}
\pause
\medspace

En este caso es fácil calcular algún valor de cada sucesión, por ejemplo \pause
\begin{equation*}
	u_5 = 3\cdot 5 +2= 17 \; \text{ o } \; w_3 = 4\cdot 5 \cdot 6 = 120
\end{equation*}

\medspace

\pause
Cuando una sucesión puede expresar se como combinación de un número determinado de operaciones elementales, diremos que tiene una \textit{fórmula cerrada.}
	
\end{frame}



\begin{frame}\frametitle{ }  

A veces la sucesión que nos interesa no tiene una fórmula \textit{cerrada} como las anteriores y podemos expresarla en forma \textit{recursiva}.  \pause Por ejemplo,

\medspace

\begin{itemize}
	\item $u_1=1$, $u_2=2$, (casos base)   \pause
	\item $u_n =u_{n-1} +u_{n-2}$, para $n\ge 3$ (caso recursivo).   \pause
\end{itemize} 

\medspace

La anterior es la \textit{sucesión de Fibonacci.} Entonces podemos calcular los término $n \ge 3$  de la sucesión usando la recursión:

\medspace  \pause

\begin{itemize}
	\item $u_3 =  u_2 + u_1 =  2 +1=3$,  \pause
	\item $u_4 =  u_3 + u_2 = 3 +2 =5$  \pause
	\item $u_5 =  u_4 + u_3 = 5 +3 =8$
\end{itemize} 
y así sucesivamente.
\end{frame}


\begin{frame}\frametitle{} 
	 \begin{ejemplo} Sea $u_n$ definida
	 	\begin{itemize}
	 		\item $ u_1=3$, $u_2=5$ y
	 		\item $u_n=3 u_{n-1} - 2 u_{n-2}$ para $n\geq 3$.
	 	\end{itemize}
 	Calcular $u_n$ para $n \le 5$.\pause 
	 \end{ejemplo}
	\begin{solucion} \pause Los valores para $n=1$ y $n=2$ ya los conocemos. La fórmula recursiva es la que nos servirá para conocer los siguientes términos
	\begin{itemize}
	\pause \item $u_{3} = 3 u_{2} - 2 u_{1}\pause = 3 \cdot 5 - 2 \cdot 3 = 9$,
	\pause \item $u_{4} = 3 u_{3} - 2 u_{2}\pause = 3 \cdot 9 - 2 \cdot 5 = 17$,
	\pause \item $u_{5} = 3 u_{4} - 2 u_{3}\pause = 3 \cdot 17 - 2 \cdot 9  = 33$,

	\end{itemize}
	\end{solucion}
	
	 
	
\end{frame}

\begin{frame}\frametitle{} 
 En base al resultado podemos,  en este caso, adivinar una fórmula general para $u_n$. Escribamos de nuevo los valores de la sucesión para $n \le 5$. 
 
 \medspace \pause 
 
 \begin{itemize}
  	\item $ u_1=3$, \pause 
 	\item $u_2=5$,\pause 
 	\item $u_{3} = 9$,\pause 
 	\item $u_{4} = 17$,\pause 
 	\item $u_{5} = 33$,
\end{itemize}

\medspace 
\pause 
\pause Observando cuidadosamente estos valores podemos darnos cuenta que:
	
\end{frame}

\begin{frame}\frametitle{} 

	\begin{itemize}
		\item $ u_1=3 = 2 +1 = 2^1 + 1$, \pause 
		\item $u_2=5 = 4 +1 = 2^2 + 1$,\pause 
		\item $u_{3} = 9 = 8 +1 = 2^3 + 1$,\pause 
		\item $u_{4} = 17 = 16 +1 = 2^4 + 1$,\pause 
		\item $u_{5} = 33 = 32 +1 = 2^5 + 1$,\pause 
	\end{itemize}
\medspace 

Ahora sí,  claramente podemos deducir que (posiblemente) 
$$u_{n} =  2^n + 1.$$ \pause 
La clase que viene aprenderemos un método (principio de inducción) que nos permitirá probar este tipo de afirmaciones. 
	
\end{frame}



\begin{frame}\frametitle{} 
	El método de definición recursiva aparecerá bastante seguido en la materia. Existen otras formas de este procedimiento que se ``esconden'' por su notación. \pause ¿Qué significan las siguientes expresiones?
	$$
	\sum_{r=1}^{n} r,\qquad 1+2+3+\cdots +n.
	$$\pause 
	Ambas significan que sumamos los primeros $n$ números naturales, pero cada uno contiene un misterioso símbolo, $\sum$ y $\cdots$,
	respectivamente. 
	
	\medspace 
	\pause 
	Lo que deberíamos decir es que cada uno de ellos
	es equivalente a la expresión $s_n$, dada por la siguiente
	definición recursiva:\pause 
	$$
	s_1= 1, \qquad s_n = s_{n-1} + n, \qquad n\ge 2.
	$$
	
	
\end{frame}


\begin{frame}\frametitle{} 
\begin{definicion} Sea $n \in \mathbb N$ sean $a_i$,  $1 \le i \le n$ una secuencia de números (enteros, reales, etc.).\pause  Entonces $\sum_{i=1}^{n} a_i$  denota la función recursiva definida  \pause 
	$$
	\sum_{i=1}^{1} a_i= a_1, \qquad \sum_{i=1}^{n} a_i = \sum_{i=1}^{n-1} a_i+ a_{n} \quad (n\ge 2).
	$$\pause 
	En  este caso  decimos que  $\displaystyle\sum_{i=1}^{n} a_i$ es la \textit{sumatoria} de los $a_i$ de $i=1$  a $n$. 
\end{definicion}
	
\end{frame}


\begin{frame}\frametitle{} 
\begin{definicion}
	El símbolo $\displaystyle\prod_{i=1}^{n} a_i$ denota la función recursiva definida  \pause 
	$$
	\prod_{i=1}^{1} a_i= a_1, \qquad \prod_{i=1}^{n} a_i = \prod_{i=1}^{n-1} a_i \cdot  a_{n} \quad (n\ge 2).
	$$\pause 
	En  este caso  decimos que  $\displaystyle\prod_{i=1}^{n} a_i$ es la \textit{productoria} de los $a_i$ de $i=1$  a $n$. 
\end{definicion}
	


\end{frame}



\begin{frame}\frametitle{} 
	Una definición muy utilizada es la de  $n!$ (que se lee $n$ {\it factorial}). Intuitivamente podemos definirlo como \pause 
	$$
	n!=1 \cdot 2 \cdot 3 \cdots n,
	$$\pause 
	y el significado es bastante claro para cualquiera.
	
	\medspace \pause 
	
	Pero para precisar (y hacerlo claro para una computadora) debemos usar una definición recursiva:
	\pause 
	\begin{itemize}
		\item $1! = 1$, \pause 
		\item $n! = (n-1)! \cdot n$.
	\end{itemize}
	
	
	
\end{frame}


\begin{frame}\frametitle{} 	
	Observar que también podemos definir 
	\pause 
	$$n! = \prod_{i=1}^{n} i.$$ 
	\pause 
	Pese a que esta última fórmula parece cerrada, oculta la definición recursiva de $\prod$.
	
	\medspace 
	
	\pause 
	Es un resultado conocido  que la función $n!$ \textit{no} admite  una fórmula cerrada. 
	
\end{frame}

 

\medspace 
 

 \medspace 




 Finalmente definiremos  la ``$n$-ésima potencia'' de un número: sea $x$ un  número, si $n \in \mathbb N $ definimos\pause 
 $$
 x^1=x,\qquad x^n= x \cdot x^{n-1} \quad (n\ge 2).
 $$\pause 
 Por completitud,  definimos $x^0=1$ cuando $x \ne 0$ y  dejamos indefinido $0^0$. 



\end{document}

