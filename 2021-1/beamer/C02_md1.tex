%\documentclass{beamer} 
\documentclass[handout]{beamer} % sin pausas
\usetheme{CambridgeUS}


%\setbeamertemplate{background}[grid][step=8 ] % cuadriculado

\usepackage{etex}
\usepackage{t1enc}
\usepackage[spanish,es-nodecimaldot]{babel}
\usepackage{latexsym}
\usepackage[utf8]{inputenc}
\usepackage{verbatim}
\usepackage{multicol}
\usepackage{amsgen,amsmath,amstext,amsbsy,amsopn,amsfonts,amssymb}
\usepackage{amsthm}
\usepackage{calc}         % From LaTeX distribution
\usepackage{graphicx}     % From LaTeX distribution
\usepackage{ifthen}
%\usepackage{makeidx}
\input{random.tex}        % From CTAN/macros/generic
\usepackage{subfigure} 
\usepackage{tikz}
\usepackage[customcolors]{hf-tikz}
\usetikzlibrary{arrows}
\usetikzlibrary{matrix}
\tikzset{
	every picture/.append style={
		execute at begin picture={\deactivatequoting},
		execute at end picture={\activatequoting}
	}
}
\usetikzlibrary{decorations.pathreplacing,angles,quotes}
\usetikzlibrary{shapes.geometric}
\usepackage{mathtools}
\usepackage{stackrel}
%\usepackage{enumerate}
\usepackage{enumitem}
\usepackage{tkz-graph}
\usepackage{polynom}
\polyset{%
	style=B,
	delims={(}{)},
	div=:
}
\renewcommand\labelitemi{$\circ$}
\setlist[enumerate]{label={(\arabic*)}}
\setbeamertemplate{itemize item}{$\circ$}
\setbeamertemplate{enumerate items}[default]
\definecolor{links}{HTML}{2A1B81}
\hypersetup{colorlinks,linkcolor=,urlcolor=links}


\newcommand{\Id}{\operatorname{Id}}
\newcommand{\img}{\operatorname{Im}}
\newcommand{\nuc}{\operatorname{Nu}}
\newcommand{\im}{\operatorname{Im}}
\renewcommand\nu{\operatorname{Nu}}
\newcommand{\la}{\langle}
\newcommand{\ra}{\rangle}
\renewcommand{\t}{{\operatorname{t}}}
\renewcommand{\sin}{{\,\operatorname{sen}}}
\newcommand{\Q}{\mathbb Q}
\newcommand{\R}{\mathbb R}
\newcommand{\C}{\mathbb C}
\newcommand{\K}{\mathbb K}
\newcommand{\F}{\mathbb F}
\newcommand{\Z}{\mathbb Z}
\newcommand{\N}{\mathbb N}
\newcommand\sgn{\operatorname{sgn}}
\renewcommand{\t}{{\operatorname{t}}}
\renewcommand{\figurename }{Figura}

\include{definiciones}


\newcommand{\nc}{\newcommand}

%%%%%%%%%%%%%%%%%%%%%%%%%LETRAS

\nc{\FF}{{\mathbb F}} \nc{\NN}{{\mathbb N}} \nc{\QQ}{{\mathbb Q}}
\nc{\PP}{{\mathbb P}} \nc{\DD}{{\mathbb D}} \nc{\Sn}{{\mathbb S}}
\nc{\uno}{\mathbb{1}} \nc{\BB}{{\mathbb B}} \nc{\An}{{\mathbb A}}

\nc{\ba}{\mathbf{a}} \nc{\bb}{\mathbf{b}} \nc{\bt}{\mathbf{t}}
\nc{\bB}{\mathbf{B}}

\nc{\cP}{\mathcal{P}} \nc{\cU}{\mathcal{U}} \nc{\cX}{\mathcal{X}}
\nc{\cE}{\mathcal{E}} \nc{\cS}{\mathcal{S}} \nc{\cA}{\mathcal{A}}
\nc{\cC}{\mathcal{C}} \nc{\cO}{\mathcal{O}} \nc{\cQ}{\mathcal{Q}}
\nc{\cB}{\mathcal{B}} \nc{\cJ}{\mathcal{J}} \nc{\cI}{\mathcal{I}}
\nc{\cM}{\mathcal{M}} \nc{\cK}{\mathcal{K}}

\nc{\fD}{\mathfrak{D}} \nc{\fI}{\mathfrak{I}} \nc{\fJ}{\mathfrak{J}}
\nc{\fS}{\mathfrak{S}} \nc{\gA}{\mathfrak{A}}
%%%%%%%%%%%%%%%%%%%%%%%%%LETRAS


\title[Clase 2 - Orden]{Matemática Discreta I \\ Clase 2 - Ordenando los enteros}
%\author[A. Tiraboschi]{Alejandro Tiraboschi}
\institute[]{\normalsize FAMAF / UNC
	\\[\baselineskip] ${}^{}$
	\\[\baselineskip]
}
\date[18/03/2020]{18 de marzo  de 2021}



\begin{document}


\frame{\titlepage} 

%\frame{\frametitle{Índice}\tableofcontents} 


\begin{frame}
    \frametitle{}

    \begin{itemize}
        \item El orden natural de los enteros es tan importante como sus propiedades aritméticas.\pause \vskip .4cm
        \item Expresamos la idea de orden formalmente diciendo que existe una relación que indicamos ``$<$''.\vskip .4cm \pause
        \item Solo cuatro axiomas se necesitan para especificar las propiedades básicas del símbolo $<$. 
    \end{itemize}\pause

    \vskip .4cm\vskip .4cm

    \begin{observacion}
    Si $a, b \in \Z$, $a < b$ se lee: 
    \begin{center}
        \textit{$a$ es menor que $b$ } \;o también \; \textit{$b$ es mayor que $a$}.
    \end{center}
    \end{observacion}


    

    

\end{frame}


\begin{frame}
    \frametitle{Axiomas de orden}
    \pause 
    \begin{enumerate}[label=\textbf{I\arabic*)}, ref=\textbf{I\arabic*}]
        \item[\textbf{I8)}] \label{axioma-i8} {\em Ley de tricotomía.}\, Vale una y sólo una de las relaciones
    siguientes:
    $$
    a<b, \qquad a = b, \qquad b < a.
    $$\pause 
    \item[\textbf{I9)}] \label{axioma-i9} {\em Ley transitiva.}\, Si $a< b$ y $b < c$, entonces $a<c$.\pause 
    \item[\textbf{I10)}] \label{axioma-i10} {\em Compatibilidad de la suma con el orden.}\, Si $a < b$, entonces $a+c < b+c$. \pause 
    \item[\textbf{I11)}] \label{axioma-i11} {\em Compatibilidad del producto con el orden.}\, Si $a< b$ y $0< c$, entonces $ac < bc$. 
    \end{enumerate}

    \vskip 2cm
\end{frame}


\begin{frame}
    \frametitle{}
    Esta claro que podemos definir los otros símbolos de orden $>$, $\le$ y $\ge$, en términos de los símbolos $<$ e $=$. 
    \vskip 1cm\pause 
    \setbeamercolor{postit}{fg=black,bg=example text.fg!75!black!10!bg}
    \begin{beamercolorbox}[wd=\textwidth,rounded=true,shadow=true]{postit}%
        \begin{itemize}
            \item \textbf{($>$)} Diremos que $m>n$ si  $n<m$.\vskip .4cm\pause 
            \item \textbf{($\le$)} Diremos que $m \le n$ si $m<n$ o $m=n$. \vskip .4cm\pause 
            \item \textbf{($\ge$)} Diremos que $m \ge n$ si $m > n$ o $m=n$.\vskip .4cm
        \end{itemize}
    \end{beamercolorbox}
    \vskip 2cm


    
\end{frame}


\begin{frame}
    \frametitle{}
    
    Es importante notar que el  axioma (\textbf{I11}) tiene una versión valedera para estos nuevos símbolos.\vskip 0.5cm\pause 
    \setbeamercolor{postit}{fg=black,bg=example text.fg!75!black!10!bg}
    \begin{beamercolorbox}[wd=\textwidth,rounded=true,shadow=true]{postit}%
    \begin{itemize}
    \item \textbf{($>$)} Si $a > b$ y $c>0$, entonces $ac > bc$.\vskip .4cm\pause 
    \item \textbf{($\le$)} Si $a \le b$ y $0 \le c$, entonces $ac \le bc$.\vskip .4cm\pause 
    \item \textbf{($\ge$)} Si $a\ge b$ y $c\ge 0$, entonces $ac \ge bc$.\vskip .4cm
    \end{itemize}
    \end{beamercolorbox}
    \vskip 0.5cm\pause 
    Usando las definiciones de $\ge$, $<$, $>$ y el axioma (\textbf{I11}) original es muy sencillo demostrar estas variantes.
\end{frame}


\begin{frame}
    \frametitle{}
    
    El axioma (\textbf{I11}) tiene nuevas variantes cuando consideramos la multiplicación de una desigualdád por enteros negativos.
    \pause 
    \vskip .5cm
    
\begin{proposicion}\label{prop-compatibilidad-negativa}
    Sean $a, b, c \in \Z$.  
    \begin{enumerate}[label=\textit{\alph*)}]
        \item\label{it.cmenor0} Si $c < 0$, entonces $0 < -c$.
        \item\label{it.amenorb} Si $a< b$ y $c< 0$, entonces $ac > bc$. 
    \end{enumerate}
\end{proposicion}
\pause 
\vskip .5cm

La demostración pueden verla en el apunte (proposición 1.2.1). 
\pause 
\vskip 0.5cm

Usando  esta proposición y la definición  de $>$, $\le$, $\ge$ podemos hacer más variantes del  axioma (\textbf{I11}) (¡$16$ en total!). Todas ellas bastante obvias.


\end{frame}


\begin{frame}
    \frametitle{}

    \begin{ejemplo}
        Sean $a, b, c \in \Z$. Entonces, 
        $$
        a\ge  b\; \wedge\; c < 0 \quad \Rightarrow \quad ac \le bc. 
        $$
    \end{ejemplo}\pause 

    \begin{proof}\pause 
        Como $a\ge  b$,  tenemos que $ a >b$ o $a=b$. 
        \vskip 0.5cm
        Si $a>b$,  entonces $b < a$. Como  $c < 0$ $\Rightarrow$ $bc > ac$ $\Rightarrow$  $ac < bc$ $\Rightarrow$ $ac \le bc$.
        \vskip 0.5cm
        Si $a=b$,  entonces  $ac = bc$ $\Rightarrow$ $ac \le bc$. \qed
    \end{proof}

\end{frame}


\begin{frame}
    \frametitle{}
    Ya hemos usado (en axioma \textbf{I4}) el símbolo $\not=$ que denota  ``{\em no} es igual a '' o bien ``es distinto a''.   En  general, cuando tachemos un símbolo, estamos indicando la negación de la relación que define. Por ejemplo, $a\not< b$ denota ``$a$ {\em no} es menor que $b$''. 
    \pause 
\begin{observacion*} Demostremos que  $a\not< b$ es equivalente a $a\ge b$: por la ley de tricotomía axioma (\textbf{I8}) tenemos que solo vale una y solo una de las siguientes afirmaciones
$$
a<b, \qquad a = b, \qquad b < a.
$$\pause 
Como  $a\not< b$, entonces vale una de las dos afirmaciones siguientes, $a=b$ o $b<a$, es decir  vale que $a \ge b$. De forma análoga se prueba que $a\not\le b$ si  y sólo si $a>b$, $a\not> b$ si  y sólo si $a \le b$ y $a\not\ge b$ si  y sólo si $a<b$.

\end{observacion*}

\end{frame}


\begin{frame}
    \frametitle{$\le$ es una relación de orden}        
    \pause 
    Sean  $a$, $b$ y $c$  enteros arbitrarios. Es posible demostrar las siguiente propiedades de $\le$:  
    \vskip .5cm
        \begin{enumerate}
        \item[\textbf{O1)}] {\em Reflexividad.}\, $a \le a$.\pause 
        \item[\textbf{O2)}] {\em Antisimetría.}\, Si $a \le b$ y $b \le a$, entonces $a=b$.\pause 
        \item[\textbf{O3)}] {\em Transitividad.}\, Si $a\le b$ y $b\le c$, entonces $a \le c$.\pause 
        \end{enumerate}
        \pause 
        \vskip .5cm

        Las demostraciones no son difíciles y las dejamos como ejercicios (se encuentran en el apunte). 

        \vskip .5cm\pause 

        Una relación que satisfaga las tres propiedades anteriores (reflexividad, antisimetría y transitividad) es llamada {\em una relación de orden}. 
        
        \vskip .5cm\pause 

        Observar que $<$ {\em no} es una relación de orden, en el sentido de la definición anterior. 
\end{frame}


\begin{frame}
    \frametitle{}
    

    A primera vista podría parecer que ya tenemos todas las propiedades que necesitamos de $\mathbb Z$, pero, sorprendentemente, aún falta un axioma de vital importancia. 
    \vskip .5cm
    \pause 
    Observemos, que todos los axiomas que enunciamos también  los cumplen los números racionales $\Q$ y los números  reales $\R$. 
    \vskip .5cm\pause 
    ¿Cuál es la diferencia \textit{fundamental} entre $\Z$ y $\Q$ o  $\R$? 
    \vskip .5cm\pause 
    \begin{figure}[ht]
        \begin{center}
        \begin{tikzpicture}[line width=1pt]    
        \draw (-4.0,0) -- (4.0,0); 
        \foreach \x in {-7,...,7}
        \draw [fill] (\x/2,0) circle [radius=0.05];
        \end{tikzpicture}
        \end{center}
        \caption{El dibujo correcto de $\mathbb Z$.}\label{f1.2}
    \end{figure}
    \pause 
    \begin{figure}[ht]    
        \begin{center}
        \begin{tikzpicture}[line width=1pt]    
        \draw (-4.0,0) -- (4.0,0); 
        \foreach \x in {-7,...,0}
        \draw [fill] (\x/2,0) circle [radius=0.05];
        \foreach \x in {0.05,0.1,0.2,0.4,0.8,1.6,3.2}
        \draw [fill] (\x,0) circle [radius=0.05];
        \end{tikzpicture}
        \end{center}
        \caption{El dibujo incorrecto de $\mathbb Z$.}\label{f1.3}
    \end{figure}
    \label{dibujo-de-z}

\end{frame}


\begin{frame}
    \frametitle{Axioma de buena ordenación}
    \pause 
    Supongamos que $X$ es un subconjunto de $\mathbb Z$; entonces diremos que el entero $b$ es una {\em cota inferior}\index{cota inferior} de $X$ si
$$
b\le x \qquad \text{ para todo } \ x \in X.
$$
\pause 
\vskip .5cm
Algunos subconjuntos no tienen cotas inferiores: por ejemplo, el conjunto de los enteros negativos $-1, -2, -3, \ldots$, claramente no tiene cota inferior. 
\pause 
\vskip .5cm
\begin{definicion}
    Una cota inferior de un conjunto $X$ que es a su vez es un elemento de $X$, es conocido como el {\em mínimo}\index{mínimo} de X.
\end{definicion}

\vskip 1cm
\end{frame}


\begin{frame}
    \frametitle{Axioma de buena ordenación}
    Nuestro último axioma para $\mathbb Z$ afirma algo que es (aparentemente) una propiedad obvia.
    \vskip .5cm\pause 
\begin{enumerate}
\item[\textbf{I12)}] Si $X$ es un subconjunto de $\mathbb Z$ que no es vacío y tiene una cota inferior, entonces $X$ tiene un mínimo.
\end{enumerate}
\vskip .8cm
El axioma (\textbf{I12}) es conocido como el     \textit{axioma de buena ordenación}  o  \textit{axioma del buen orden} o  \textit{principio de buena ordenación}.\pause 
\vskip .8cm\pause 
Observar que $\Q$ o $\R$ con $<$ \textit{no} satisfacen el axioma de buena ordenación. 
\vskip .8cm
\end{frame}


\begin{frame}
    \frametitle{}
    El hecho de que haya espacios vacíos entre los enteros nos lleva a decir que el conjunto $\mathbb Z$ es \textit{discreto} y es esta propiedad la que da origen al nombre ``matemática discreta''. 
    \vskip .5cm
    \pause 
    
    En cálculo y análisis, los procesos de límite son de fundamental importancia, y es preciso usar aquellos sistemas numéricos que son \textit{continuos}, en vez de los discretos.
    \pause 
    \vskip .5cm

    Repitamos los gráficos de la p. \ref{dibujo-de-z}

    \vskip .5cm
    \begin{figure}[ht]
        \begin{center}
        \begin{tikzpicture}[line width=1pt]    
        \draw (-4.0,0) -- (4.0,0); 
        \foreach \x in {-7,...,7}
        \draw [fill] (\x/2,0) circle [radius=0.05];
        \end{tikzpicture}
        \end{center}
        \caption{El dibujo correcto de $\mathbb Z$.}
    \end{figure}
    \pause 
    \begin{figure}[ht]    
        \begin{center}
        \begin{tikzpicture}[line width=1pt]    
        \draw (-4.0,0) -- (4.0,0); 
        \foreach \x in {-7,...,0}
        \draw [fill] (\x/2,0) circle [radius=0.05];
        \foreach \x in {0.05,0.1,0.2,0.4,0.8,1.6,3.2}
        \draw [fill] (\x,0) circle [radius=0.05];
        \end{tikzpicture}
        \end{center}
        \caption{El dibujo incorrecto de $\mathbb Z$.}
    \end{figure}


\end{frame}


\begin{frame}
    \frametitle{}
    El siguiente resultado es obvio, pero  debe ser demostrado. 
    \vskip .5cm

    \pause 


\begin{proposicion}\label{prop-0-menor-que-1}
$1$ es el menor entero mayor que $0$.
\end{proposicion}

\vskip .5cm\pause 

Es posible hacer la demostración con las herramientas que ya poseemos (si no ¡faltaría algún axioma!). 

\vskip .5cm
\pause 
Sin embargo, la demostración es relativamente compleja y el estudiante interesado la puede ver en el apunte. 


\vskip 1cm
\end{frame}



\end{document}

