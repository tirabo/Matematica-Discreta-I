\documentclass[11pt,spanish,makeidx]{amsbook}
\usepackage{etex}
\tolerance=10000
\renewcommand{\baselinestretch}{1.3}

\usepackage{t1enc}
\usepackage[spanish]{babel}
\usepackage{latexsym}
\usepackage[utf8]{inputenc}
\usepackage{verbatim}
\usepackage{multicol}
\usepackage{amsgen,amsmath,amstext,amsbsy,amsopn,amsfonts,amssymb}
\usepackage{amsthm}
\usepackage{calc}         % From LaTeX distribution
\usepackage{graphicx}     % From LaTeX distribution
\usepackage{ifthen}
\input{random.tex}        % From CTAN/macros/generic
\usepackage{subfigure} 
\usepackage{tikz}
\usetikzlibrary{arrows}
\usetikzlibrary{matrix}
%\usetikzlibrary{graphs}
%\usepackage{tikz-3dplot} %for tikz-3dplot functionality
%\usepackage{pgfplots}
\usepackage{mathtools}
\usepackage{stackrel}
\usepackage{enumerate}
\usepackage{tkz-graph}

%% \theoremstyle{plain} %% This is the default
\oddsidemargin 0.0in \evensidemargin -1.0cm \topmargin 0in
\headheight .3in \headsep .2in \footskip .2in
\setlength{\textwidth}{16cm} %ancho para apunte
\setlength{\textheight}{21cm} %largo para apunte
%\leftmargin 2.5cm
%\rightmargin 2.5cm
\topmargin 0.5 cm

\usepackage{hyperref}
\hypersetup{
	colorlinks=true,
	linkcolor=blue,
	filecolor=magenta,      
	urlcolor=cyan,
}
\usepackage{hypcap}

\renewcommand{\thesection}{\thechapter.\arabic{section}}
\renewcommand{\thesubsection}{\thesection.\arabic{subsection}}

\newtheorem{teorema}{Teorema}[section]
\newtheorem{proposicion}[teorema]{Proposici\'on}
\newtheorem{corolario}[teorema]{Corolario}
\newtheorem{lema}[teorema]{Lema}
\newtheorem{propiedad}[teorema]{Propiedad}

\theoremstyle{definition}

\newtheorem{definicion}{Definici\'on}[section]
\newtheorem{ejemplo}{Ejemplo}[section]
\newtheorem{problema}{Problema}[section]
\newtheorem{ejercicio}{Ejercicio}[section]
\newtheorem{ejerciciof}{}[section]

\theoremstyle{remark}
\newtheorem{observacion}{Observaci\'on}[section]
\newtheorem{nota}{Nota}[section]

\renewcommand{\abstractname}{Resumen}
\renewcommand{\partname }{Parte }
\renewcommand{\indexname}{Indice }
\renewcommand{\figurename }{Figura }
\renewcommand{\tablename }{Tabla }
\renewcommand{\proofname}{Demostraci\'on}
\renewcommand{\refname }{Referencias }
\renewcommand{\appendixname }{Ap\'endice }
\renewcommand{\contentsname }{Contenidos }
\renewcommand{\chaptername }{Cap\'\i tulo }
\renewcommand{\bibname }{Bibliograf\'\i a }

\newcommand\Punto[2]{\rput(#1,#2){$\scriptstyle\bullet$}}
\newcommand\Puntotz[2]{\node at (#1/10,#2/10+3) {$\scriptstyle\bullet$};}  %tikz
% coloca un punto grueso en (#1,#2), coordenadas en mm

\newcommand\BPunto[2]{\rput(#1,#2){$\bullet$}}
\newcommand\BPuntotz[2]{\node at (#1/10,#2/10+3) {$\bullet$};}  %tikz
% coloca un punto mas grande que el anterior

% coloca frase #3 en 8pt en (#1,#2). Coordenadas en mm.
\newcommand\poner[3]{\rput(#1,#2){#3}}
\newcommand\ponertz[3]{\node at (#1/10,#2/10+3) {#3};} %tikz

% hace una linea de x,y a x',y'
\newcommand\linea[4]{\psline(#1,#2)(#3,#4)}
\newcommand\lineatz[4]{\draw (#1/10,#2/10+3)-- (#3/10,#4/10+3);} %tikz

% hace una linea mas gruesa de x,y a x',y'
\newcommand\Blinea[4]{\psline[linewidth=2pt](#1,#2)(#3,#4)}
\newcommand\Blineatz[4]{\draw[line width=2pt](#1/10,#2/10+3) -- (#3/10,#4/10+3);}

\def\conc{+\hspace{-1.5ex}+\hspace{0.5ex}}
\def\con{{\rm con}}
\def\hn{\hspace{-0.2cm}}
\def\h4n{\hspace{-0.4cm}}
\def\impli{\Rightarrow}
\def\ssi{\equiv}
\def\disc{\not\ssi}
\def\cons{\Leftarrow}
\def\la{\leftarrow}
\def\lt{\triangleleft}
\def\lv{[\;\,]}
\def\Max{ {\rm Max} }
\def\Min{ {\rm Min} }
\def\N{I\hspace{-0.8ex} N}
\def\noi{\noindent}
\def\R{I\hspace{-0.8ex} R}
\def\ra{\rightarrow}
\def\Rn{\R^{n}}
\def\rt{\triangleright}
%\def\tomar{\uparrow}
\def\tomar{\hspace{-0.6ex}\uparrow\hspace{-0.6ex}}
%\def\tirar{\downarrow}
\def\tirar{\hspace{-0.6ex}\downarrow\hspace{-0.6ex}}
\def\udo{ {\rm\bf\underline{do}} }
\def\ufi{ {\rm\bf\underline{fi}} }
\def\uif{ {\rm\bf\underline{if}} }
\def\uod{ {\rm\bf\underline{od}} }
\def\v3{\vspace{0.3cm}}
\def\var{{\rm var}}
\def\[{|\hspace{-0.2ex} [}
\def\]{]\hspace{-0.2ex} |}
\def\true{\mbox{\it true\ }}
\def\false{\mbox{\it false\ }}

\newcommand{\bag}[1]{ [\hspace{-0.4ex}[ #1 ]\hspace{-0.4ex}] }
\newcommand{\abs}[1]{ [\hspace{-0.4ex}[ #1 ]\hspace{-0.4ex}] }
%\newcommand{\binom}[2]{ \left(\hspace{-1.2ex}\begin{array}{c} #1 \\ #2 \end{array} \hspace{-1.2ex}\right) }
\newcommand{\cau}[2]{\noi #1 \hspace{0.5cm}\{{\sl #2}\} }
\newcommand{\causa}[2]{\vspace{0.15cm} \noi #1 \hspace{0.5cm} \{{\sl #2}\} \vspace{0.15cm}}
\newcommand \RR{{\mathbb R}}
\newcommand \ZZ{{\mathbb Z}}
\newcommand \NN{{\mathbb N}}
%\newcommand \supr{\displaystyle{\ \,{\mbox{\footnotesize S}}\hspace{-1.7ex}\bigcirc\, }}
\newcommand \supr{\displaystyle{\ \,\vee \hspace{-1.9ex}\bigcirc\,}}
\newcommand \infi{\displaystyle{\ \,\wedge\hspace{-1.9ex}\bigcirc\,}}
%\newcommand \infi{\displaystyle{\ \,{\mbox{\footnotesize I}}\hspace{-1.5ex}\bigcirc\, }}
\newcommand \mcd{\operatorname{mcd}}
\newcommand \mcm{\operatorname{mcm}}
\newcommand \sisolosi{\Leftrightarrow}

\newcommand{\varx}{0} % variable para cambiar coordenada x
\newcommand{\vary}{0} % variable para cambiar coordenada y
\newcommand{\varc}{1} % variable para hacer homotecias

\newcommand\flecha{\Rightarrow}

\newcommand{\programa}[1]{  \vskip .5cm \noindent Programa: {\it #1} }
\newcommand{\principio}[1]{\vskip .3 cm \centerline{\sc #1} \vskip .3cm}
\makeindex

\begin{document}
	\baselineskip=0.55truecm %original
	%\baselineskip=0.43truecm

	\pagenumbering{roman}

	\title{Matemática Discreta I \\ FAMAF  - UNC \\ Versión: \today
	 }

	\maketitle

	\newpage

	\author{
		${}^{}$
		\\${}^{}$
		\\${}^{}$
		\center{\textbf{AUTORES/COLABORADORES}} \\${}^{}$\\ 
		\flushleft 
		\begin{itemize}
			\item \textbf{Obra original:} coordinada y escrita por Alejandro Tiraboschi. 
			\item \textbf{Colaboración especial: } Daniel Penazzi, colaborador  de la sección ``El criptosistema RSA'' y autor principal del apéndice de grafos planares.  
			\item \textbf{Correcciones y sugerencias:} Pedro Pury, Romina Arroyo, Leandro Cagliero. 
		\end{itemize}
	}

	%\author{A. L. Tiraboschi}

	%\address{FaMAF}
	%\subjclass{47A56, 47A10; Secondary 47A55, 47A70, 47A68, 47B15}
	%\date{July 19, 1992}
	%\translator{H. H. McFaden}

	\vskip 2cm 
	\thanks{
		\center{\textbf{LEER}} \\
		${}^{}$\\
		{\flushleft 
			Este material es distribuido bajo la licencia Creative Commons} \\
		{\center \textbf{Atribución--CompartirIgual 4.0 Internacional}}
		\\ 
		\center  lo cual significa 
		\\
		\flushleft
		- En cualquier explotación de la obra autorizada por la licencia será necesario reconocer los autores, colaboradores, etc.\\
		- La distribución de la obra u obras derivadas se debe hacer con una licencia igual a la que regula la obra original.\\
		${}^{}$
		\\
		Los detalles de la licencia pueden encontrarse en \href{https://creativecommons.org/licenses/by/4.0/deed.es}{Creative Commons}
	}

	\tableofcontents 

	\pagenumbering{arabic}

	\chapter*{Prefacio} 
	Las siguientes notas se han utilizado para el dictado del curso  “Matemática Discreta I” del primer año de la carrera de ciencias de la computación de FAMAF-UNC. Han sido las notas principales en el dictado del año 2019 y en algunos años anteriores (desde 1995), y se limitan casi exclusivamente al contenido dictado en el curso. Las partes señaladas con (*) y los apéndices son optativos. Debido a problemas de tiempo, aunque no estén marcadas con (*), a veces tampoco es posible dictar las últimas secciones del capítulo de grafos. 

	Las notas están basadas en diversas fuentes, principalmente en los libros “Discrete Mathematics” de N. Biggs y “Notas de Álgebra I” de E. Gentile, pero a lo largo de los años ha habido numerosas modificaciones y  agregados por parte de los diferentes docentes de la cátedra. 

\chapter[Números Enteros]{Números Enteros}

\begin{section}{Aritmética}\label{1.1}

Todo lector de este apunte conoce los {\it enteros}. En una etapa muy temprana de nuestras vidas conocemos los números enteros positivos o ``números naturales'' 
$$1,2,3,4,5,\ldots$$
Más adelante introducimos el 0 (cero), y los enteros negativos 
$$
-1,-2,-3,-4,-5,\ldots 
$$ 

En este curso no nos preocupamos demasiado por el significado lógico y filosófico de estos objetos, pero necesitamos saber las propiedades que se supone que tienen. Si todos parten de las mismas suposiciones entonces todos llegarán a los mismos resultados. Estos supuestos son los llamados axiomas.

El punto de vista adoptado en este apunte es el señalado antes. Aceptamos sin reparo que existe un conjunto de objetos llamados { \it enteros} conteniendo los enteros positivos y los negativos, y el cero, familiares en nuestra temprana educación y experiencia. El conjunto de enteros se denotará por el símbolo especial ${\mathbb Z}$. Las propiedades de ${\mathbb Z}$ serán dadas por una lista de axiomas, a partir de las cuales seremos capaces de deducir todos los resultados sobre números enteros que necesitaremos en las cuestiones subsiguientes. Empezaremos listando aquellos axiomas que tratan la suma y la multiplicación.

Adoptaremos las notaciones usuales $a+b$ para la suma de dos enteros $a$ y $b$, y $a \times b$ (o frecuentemente solo $ab$) para su producto. Pensamos en $+$ y $\times$ como {\it operaciones} que a un par de enteros $a$ y $b$ les hacen corresponder un entero $a+b$ y otro $a\times b$. El hecho de que $a \times b$ y $a+b$ son enteros, y no algún objeto extra\~no como elefantes, es nuestra primera suposición (axioma {\bf I1}). 

Una propiedad que debemos mencionar es la siguiente: si $a,b, c \in \mathbb Z$  y $a=b$, entonces $a+c = b+c$ y $ac = bc$. Esto se debe a que la suma y el producto son operaciones que, como acabamos de decir, toman un par de enteros y  devuelven otro entero. Si $a=b$, entonces el  par $a,c$ es igual al par $b,c$ y por lo tanto devuelven la misma suma y el mismo producto.

En la siguiente lista de axiomas $a$, $b$, $c$ denotan enteros arbitrarios, y 0 y 1 denotan enteros especiales que cumplen las propiedades especificadas más abajo.

\begin{enumerate}
\item[{\bf I1.}] $a+b$ y $ab$ pertenecen a ${\mathbb Z}$.
\item[{\bf I2.}] {\em Conmutatividad.}\, $a+b = b+a$; $ab=ba$. 
\item[{\bf I3.}] {\em Asociatividad.}\, $(a+b)+c = a+(b+c)$; $(ab)c = a(bc)$. 
\item[{\bf I4.}] {\em Existencia de elemento neutro.}\, Existen números $0$, $1 \in \mathbb Z$ con $0\not=1$ tal que $a+0=a$; $a1=a$. 
\item[{\bf I5.}] {\em Distributividad.}\, $a(b+c)=ab+ac$. 
\item[{\bf I6.}] {\em Existencia del inverso aditivo.}\, Por cada $a$ en ${\mathbb Z}$ existe un único entero $-a$ en ${\mathbb Z}$ tal que $a+(-a)=0$. 
\item[{\bf I7.}] {\em Cancelación.}\, Si $a$ es distinto de 0 y $ab=ac$, entonces $b=c$. 
\end{enumerate}

Debido a la ley de asociatividad para la suma (axioma {\bf I3}) $(a+b)+c$ es igual a $a+(b+c)$ y por lo tanto podemos eliminar los paréntesis sin ambigüedad. Es decir, denotamos
$$
a+b+c := (a+b)+c = a+(b+c).
$$
De forma análoga, usaremos la notación
$$
abc = (ab)c = a(bc).
$$
Debido a la ley de conmutatividad (axioma {\bf I2}), es claro que  del axioma {\bf I4} se deduce que  $0+a=a+0=a$ y $1a = a1=a$. Análogamente,  por  {\bf I2} e  {\bf I6} obtenemos que  $-a+a=a+(-a)=0$.

Todos los axiomas corresponden a propiedades familiares de los  enteros que aprendemos en distintos niveles de nuestra educación  matemática. De ellas pueden deducirse la mayoría de las reglas  aritméticas comunes de los enteros como en el siguiente ejemplo.

\begin{ejemplo}\label{Ej.opuesto_opuesto} Demostrar que, para todo $n$ entero, el opuesto de $-n$ es $n$, es decir que 
$$-(-n) = n.$$ 
\end{ejemplo}
\begin{proof} El axioma {\bf I6} nos dice que $-(-n)$ es el único número que sumado a $-n$, da cero.  Por lo tanto, para demostrar que $-(-n) = n$ basta ver que $(-n)+n=0$. Esto se cumple puesto que 
\begin{alignat*}2
(-n)+n&=n+(-n)& &\text{(axioma {\bf I2})} \\
&=0&\quad &\text{(axioma {\bf I6})}
\end{alignat*}
Por lo tanto  $(-n)+n=0$.
\end{proof}

Como ya dijimos, los números enteros vienen provistos con dos operaciones fundamentales, la suma y la multiplicación. A continuación definimos la resta o sustracción. 

\begin{definicion} Si $a,b\in\mathbb{Z}$ definimos $a-b$ como la suma de $a$ más el opuesto de $b$, es decir que  $a-b=a+(-b)$ por definición. 
\end{definicion}

Ahora demostremos una propiedad básica de la resta.

\begin{ejemplo} Demuestre que para dos enteros $m$ y $n$ cualesquiera
$$m-(-n) = m+n.$$ 
\end{ejemplo}
\begin{proof} Por la definición de sustracción, $m-(-n)$ es la suma $m+(-(-n))$, es decir  $m-(-n)=m+(-(-n))$.  Por el Ejemplo \ref{Ej.opuesto_opuesto} sabemos que $-(-n)=n$ y por lo tanto $m-(-n)=m+(-(-n))=m+n$.
\end{proof}

Tanto formalismo, como el usado en las demostraciones realizadas en el ejemplo anterior, puede ser tedioso, pero nos permiten comenzar a comprender la estructura de una demostración formal. 

\begin{ejemplo} Suponga que existen dos enteros $0$ y $0'$ ambos cumpliendo el
 axioma {\bf I4}, esto es
$$
a+0= a, \qquad a+0'=a
$$
para todo $a$ de $\mathbb Z$. Demostrar que esto implica $0=0'$, por lo tanto 0 está en realidad caracterizado de manera única por el axioma {\bf I4} y es llamado el {\em el} elemento neutro de la suma.
\end{ejemplo}
\begin{proof}
\begin{alignat*}2
0 &= 0 + 0'& &\text{(axioma {\bf I4} aplicado a $0$ y con $0'$ como neutro)} \\
&=0'+0&\quad &\text{(axioma {\bf I2})}\\
&= 0' & &\text{(axioma {\bf I4} aplicado a $0'$ y con $0$ como neutro)}.
\end{alignat*}
\end{proof}

\begin{ejercicio} Probar que hay un único elemento neutro del producto.
\end{ejercicio}

\begin{ejercicio} \label{ej0a}
La siguiente es una demostración de la fórmula $0x=0$ usando solo los axiomas planteados antes. Escriba la demostración completa, explicando que axioma es usado en cada paso.
$$\begin{aligned}
0x &= (0+0)x \\  &=0x+0x.
\end{aligned}$$
Luego $0x =0x+0x$. Sumando $-0x$ a ambos miembros de la igualdad, obtenemos 
$$\begin{aligned}
0x -0x &= 0x+0x -0x\\  0 &=0x.
\end{aligned}$$

\end{ejercicio}

\begin{ejemplo} (Regla de los signos) Veamos que  si $a,b \in \mathbb Z$ entonces
$$
(-a)(-b) = ab ,\quad a(-b) = (-a)b = -(ab).
$$
\end{ejemplo}
\begin{proof}
Veremos que  $a(-b) = -(ab)$. Los otros casos se dejan como ejercicio para el lector.

Una forma de demostrar este caso es  observando que $-(ab)$ es el inverso aditivo de $ab$ y comprobando que $a(-b)$ es también inverso aditivo de $ab$. Luego, por unicidad del inverso aditivo, de deduce que $a(-b) = -(ab)$. 
\begin{alignat*}2
ab + a(-b) &=a(b-b)& &\text{(axioma {\bf I5})} \\
&=a0&\quad &\text{(axioma {\bf I4})}\\
&= 0 & &\text{(ejemplo \ref{ej0a})}.
\end{alignat*}
Es decir $a(-b)$ es el inverso aditivo de $ab$, luego por la unicidad del inverso aditivo (axioma {\bf I6}), $a(-b)=-(ab)$.
\end{proof}

Algunos resultados similares pueden encontrarse en los siguientes ejercicios. Como aún no tenemos todos los axiomas correspondientes a los enteros, los resultados no son particularmente interesantes, pero lo que importa es recordar que pueden ser probados sobre la base única de los axiomas.

\begin{subsection}{Ejercicios}
\begin{enumerate} 
\item
Demuestre la regla $(a+b)c=ac+bc$, explicando cada paso.
\item Como siempre $x^2$ denota $xx$. Demuestre que dados dos enteros $a$ y $b$ tal que $a+b \not=0$, entonces existe un único $c$ tal que $(a+b)c=
a^2 - b^2$.
\end{enumerate}
\end{subsection}

\end{section}

\begin{section}{Ordenando los enteros}\label{1.2}

El orden natural de los enteros es tan importante como sus propiedades aritméticas. Desde el comienzo aprendemos los números en el orden 1,2,3,4,5, y el hecho de que 4 es ``mayor'' que 3 se convierte en algo de importancia práctica para nosotros. Expresamos esta idea formalmente diciendo que existe una relación de orden que indicamos ``$<$'' ($a < b$ se lee: $a$ es menor que $b$ o también $b$ es mayor que $a$). 

Solo cuatro axiomas se necesitan para especificar las propiedades básicas del símbolo $<$ , y ellos son listados en lo que sigue. La numeración de los axiomas se continúa de la sección \ref{1.1}. Como antes, $a$, $b$ y $c$ denotan enteros arbitrarios. 
\begin{enumerate}
\item[{\bf I8.}] {\em Ley de tricotomía.}\, Vale una y sólo una de las relaciones
siguientes:
$$
a<b, \qquad a = b, \qquad b < a.
$$
\item[{\bf I9.}] {\em Ley transitiva.}\, Si $a< b$ y $b < c$, entonces $a<c$.
\item[{\bf I10.}] {\em Compatibilidad de la suma con el orden.}\, Si $a < b$, entonces $a+c < b+c$. 
\item[{\bf I11.}] {\em Compatibilidad del producto con el orden.}\, Si $a< b$ y $0< c$, entonces $ac < bc$. 
\end{enumerate}


Observar que el axioma {\bf I10} nos permite hacer \textit{pasaje de término}, por ejemplo si $a < b$, entonces $0 < b -a$, pues $a < b$ implica $a - a < b - a$, es decir $0 < b -a$. 


Esta claro que podemos definir los otros símbolos de orden $>$, $\le$ y $\ge$, en términos de los símbolos $<$ e $=$. Diremos que $m>n$ si  $n<m$, diremos que $m \le n$ si $m<n$ o $m=n$. Finalmente, diremos que $m \ge n$ si $m > n$ o $m=n$.  Es importante notar que los  axiomas {\bf I9}, {\bf I10} e {\bf I11} tienen una versión valedera para estos nuevos símbolos. Explicitemos el caso  del axioma {\bf I11}:
\begin{enumerate}
\item[{\bf I11.}] {\bf ($>$)} Si $a > b$ y $c>0$, entonces $ac > bc$.
\item[{\bf I11.}] {\bf ($\le$)} Si $a \le b$ y $0 \le c$, entonces $ac \le bc$.
\item[{\bf I11.}] {\bf ($\ge$)} Si $a\ge b$ y $c\ge 0$, entonces $ac \ge bc$.
\end{enumerate}
Usando las definiciones de $\ge$, $<$, $>$ y el axioma {\bf I11} original es muy sencillo demostrar estas variantes. También valen consecuencias obvias de estas propiedades como, por ejemplo, si $a > b$ y $c \ge 0$, entonces $ac \ge bc$. 


Cuando tachemos un símbolo, estamos indicando la negación de la relación que define. Por ejemplo, $a\not< b$ denota ``$a$ {\em no} es menor que $b$''. Otro ejemplo, $a\not= b$ denota  ``$a$ {\em no} es igual a $b$'' o bien ``$a$ es distinto a $b$''. 

\begin{observacion} Demostremos que  $a\not< b$ es equivalente a $a\ge b$: por la ley de tricotomía (axioma {\bf I8}) tenemos que solo vale una y solo una de las siguientes afirmaciones
$$
a<b, \qquad a = b, \qquad b < a.
$$
Como  $a\not< b$, entonces vale una de las dos afirmaciones siguientes, $a=b$ o $b<a$, es decir  vale que $a \ge b$. De forma análoga se prueba que $a\not\le b$ si  y sólo si $a>b$, $a\not> b$ si  y sólo si $a \le b$ y $a\not\ge b$ si  y sólo si $a<b$.

\end{observacion}

\begin{ejemplo}\label{relaciondeorden}
Demostrar las siguiente propiedades de $\le$. Sean  $a$, $b$ y $c$  enteros arbitrarios,  entonces
\begin{enumerate}
\item[{\bf O1.}] {\em Reflexividad.}\, $a \le a$.
\item[{\bf O2.}] {\em Antisimetría.}\, Si $a \le b$ y $b \le a$, entonces $a=b$.
\item[{\bf O3.}] {\em Transitividad.}\, Si $a\le b$ y $b\le c$, entonces $a \le c$.
\end{enumerate}
\begin{proof}
{\bf O1.} Como $a=a$, tenemos entonces que $a \le a$ (por definición de $\le$).
\vskip .2cm
{\bf O2.} Como $a \le b$, tenemos que $a<b$ o bien $a=b$ (por tricotomía no pueden valer ambas). Si ocurriera que $a<b$, por la observación anterior, tendríamos que $a\not\ge b$, es decir $b\not\le a$, lo cual es absurdo pues una de nuestras hipótesis es,  justamente, lo contrario:  $b \le a$.  Es decir, la única posibilidad que queda es que $a=b$.     
\vskip .2cm
{\bf O3.} Como $a\le b$, entonces $a <b$ o bien $a=b$. Como $b\le c$, entonces $b<c$ o bien $b=c$. Para hacer la demostración, debemos pensar en todas las posibles combinaciones de estas afirmaciones:

$a<b$ y $b<c$. Es este caso, por  {\bf I9}, $a<c$. Luego $a\le c$.

$a<b$ y $b=c$. Luego $a<c$ y eso implica que $a\le c$.

$a=b$ y $b<c$. Luego $a<c$ y eso implica que $a\le c$.

$a=b$ y $b=c$. Es claro entonces que $a=c$, lo cual implica que $a\le c$.
\end{proof}
\end{ejemplo}

Una relación que satisfaga las tres propiedades anteriores (reflexividad, antisimetría y transitividad) es llamada {\em una relación de orden}. Observar que $<$ {\em no} es una relación de orden, en el sentido de la definición anterior. 

\begin{proposicion}\label{prop-0-menor-1} Sean $a,b,c$ números enteros. 
	\begin{enumerate}
		\item[{\it (a)}] Si  $a < b$ y $c < 0$, entonces $ac > bc$.
		\item[{\it (b)}] Si $a \ne 0$, entonces  $0 < a^2$. 
		\item[{\it (c)}] \label{item-0-menor-1} $0 < 1$.
		\item[{\it (d)}] $n < n+1$ para todo $n$ en $\mathbb Z$.
	\end{enumerate}
\end{proposicion}
\begin{proof}
	{\it (a)} Como $c < 0$, sumando $-c$ a la desigualdad, se deduce que $0 < -c$, luego por {\bf I11}, obtenemos que $a(-c) < b(-c)$, es decir $-ac < -bc$ y haciendo pasajes de término llegamos a $bc < ac$. 
	
	{\it (b)} Como $a \ne 0$, por tricotomía tenemos dos casos: $ 0<a$  y $a <0$. Si $0 < a$, por {\bf I11}, $0 \cdot a < a \cdot a$,  es decir $0 < a^2$. En  el caso que $a <0$, por {\it (a)}, aplicado a $a<0$ y $a<0$, obtenemos $a^2 >0$.
	
	{\it (c)} Como $1 \ne 0$, por {\it (b)}, tenemos que $0 < 1^2 = 1$.
	
	{\it (d)} Por {\it (c)}, $0 < 1$. Por  {\bf I10}, $n + 0 < n + 1$, es decir  $n < n+1$.
\end{proof}

\begin{subsection}{Ejercicios}\label{ejerciciosorden}
\begin{enumerate}
\item Demostrar que $\ge$ es una relación de orden.
\item Demostrar que dados cualesquiera $a,b,c \in \mathbb Z$ vale que si $a< b$ y $0\le c$, entonces $ac \le bc$. 
\end{enumerate}
\end{subsection}

A primera vista podría parecer que ya tenemos todas las propiedades que necesitamos de $\mathbb Z$, pero, sorprendentemente, aún falta un axioma de vital importancia. Supongamos que $X$ es un subconjunto de $\mathbb Z$; entonces diremos que el entero $b$ es una {\em cota inferior}\index{cota inferior} de $X$ si
$$
b\le x \qquad \text{ para todo } \ x \in X.
$$
Algunos subconjuntos no tienen cotas inferiores: por ejemplo, el conjunto de los enteros negativos $-1,-2,-3,\ldots$, claramente no tiene cota inferior. Por otro lado, el conjunto $S$ denotado por los números resaltados en la Fig.\ref{f1.1} tiene muchas cotas inferiores. Una mirada rápida nos dice que $-9$ por ejemplo es una cota inferior, mientras que una inspección más minuciosa revela que $-7$ es la ``mejor'' cota inferior, pues en realidad pertenece a $S$. En general, una cota inferior de un conjunto $X$ que es a su vez es un elemento de $X$, es conocido como el {\em mínimo}\index{mínimo} de X .

%\vskip .3cm

\begin{figure}[ht]
$$
-10,-9,-,8,\mathbf{-7},\mathbf{-6},-5,-4,\mathbf{-3},-2,-1,0,\mathbf{1},\mathbf{2},\mathbf{3}, 4,5,6,7,8,9,10
$$
\caption{El mínimo de $S$ es $-7$.}\label{f1.1}
\end{figure}
%\vskip .3cm

Nuestro último axioma para $\mathbb Z$ afirma algo que es (aparentemente) una propiedad obvia.

\begin{enumerate}
\item[{\bf I12.}] Si $X$ es un subconjunto de $\mathbb Z$ que no es vacío y tiene una cota inferior, entonces $X$ tiene un mínimo.
\end{enumerate}

El axioma {\bf I12} es conocido como el {\em axioma del buen orden}\index{axioma del buen orden} o {\em principio de buena ordenación}\index{principio de buena ordenación}.  Una buena forma de entender su significado es considerar un juego en el cual dos personas eligen alternativamente un elemento de $X$, sujetos a la regla de que cada número debe ser estrictamente menor que el anterior. El axioma nos dice que cuando los números son enteros, el juego terminará; además el final se producirá cuando uno de los jugadores tenga el buen sentido de elegir el mínimo. Esta propiedad aparentemente obvia {\it no} se mantiene necesariamente cuando tratamos con números que no son enteros, porque $X$ puede no tener un mínimo aunque tenga una cota inferior. Por ejemplo supongamos que $X$ es el conjunto de fracciones $3/2,4/3,5/4,\ldots$ teniendo por forma general $(n+1)/n$, $n\ge 2$. Este conjunto tiene una cota inferior (1, por ejemplo) pero no tiene mínimo y por lo tanto los jugadores podrían seguir jugando para siempre, eligiendo fracciones más y más cercanas a 1.

El axioma del buen orden nos da una justificación firme para nuestro intuitivo dibujo de los enteros: un conjunto de puntos regularmente espaciados sobre una linea recta, que se extiende indefinidamente en ambas direcciones (Fig. \ref{f1.2}). En particular dice que no podemos acercarnos más y más a un entero sin alcanzarlo, de forma que el dibujo de la Fig. \ref{f1.3} no es correcto.

\vskip .3cm

\begin{figure}[ht]
	\begin{tikzpicture}[line width=1pt]
	\draw (-4.0,0) -- (4.0,0); 
	\foreach \x in {-7,...,7}
	\draw [fill] (\x/2,0) circle [radius=0.05];
	\end{tikzpicture}
	\caption{El dibujo correcto de $\mathbb Z$.}\label{f1.2}
\end{figure}

\vskip .3cm

\begin{figure}[ht]
	\begin{tikzpicture}[line width=1pt]
	\draw (-4.0,0) -- (4.0,0); 
	\foreach \x in {-7,...,0}
	\draw [fill] (\x/2,0) circle [radius=0.05];
	\foreach \x in {0.05,0.1,0.2,0.4,0.8,1.6,3.2}
	\draw [fill] (\x,0) circle [radius=0.05];
	\end{tikzpicture}
	\caption{El dibujo incorrecto de $\mathbb Z$.}\label{f1.3}
\end{figure}

\vskip .3cm

El hecho de que haya espacios vacíos entre los enteros nos lleva a decir que el conjunto $\mathbb Z$ es {\it discreto} y es esta propiedad la que da origen al nombre ``matemática discreta''. En cálculo y análisis, los procesos de límite son de fundamental importancia, y es preciso usar aquellos sistemas numéricos que son {\it continuos}, en vez de los discretos.

El siguiente resultado es obvio, pero  debe ser demostrado. Sin embargo,  la demostración es bastante compleja y sólo se da por completitud. 
\begin{proposicion}
$1$ es el menor entero mayor que 0.
\end{proposicion}
\begin{proof}
Por proposición \ref{prop-0-menor-1}{\it (c)} sabemos que $0 < 1$. Supongamos que existe $a \in \mathbb Z$ tal que $0<a<1$ y sea 
$$
X=\{a\in\mathbb Z: 0<a<1\}.
$$
La  suposición que hicimos implica que $X$ es no vacío.  Dado que todos los elementos de $X$ son positivos, $X$ es un subconjunto de $\mathbb Z$ acotado inferiormente (0 es cota inferior). Por el axioma del buen orden (\textbf{I12}) resulta que $X$ tiene un elemento mínimo, que llamaremos $a_0$, y cumple
$$
0<a_0<1. 
$$
Usamos ahora la compatibilidad del  producto con  la relación de orden (\textbf{I11}):  por un lado multiplicamos por $a_0$ la desigualdad $0<a_0$ y obtenemos $0<a_0^2$,  y por otro lado multiplicamos por $a_0$ la desigualdad $a_0<1$ y obtenemos $a_0^2<a_0$. Es decir
$$
 0<a_0^2<a_0<1.
$$
La desigualdad $0<a_0^2<1$ dice que $a_0^2\in X$ pero la desigualdad $a_0^2<a_0$ dice que  $a_0$ no es el mínimo elemento de $X$, lo cual es una contradicción pues dijimos que $a_0$ es el mínimo elemento de $X$.  El  absurdo  vino de suponer que existe $a \in \mathbb Z$ tal que   $0<a<1$.
\end{proof}

\vskip 0.1 cm

\begin{subsection}{Ejercicios} {\rm (continuación)}
\begin{enumerate}
\item Demostrar que si un conjunto $X$ tiene mínimo, este es único. Dicho más formalmente: demostrar que si existen $c,c' \in X$ tal que  $c\le x$ y $c'\le x$ para todo $x \in X$, entonces $c=c'$. 
\item En cada uno de los siguientes casos diga si el conjunto $X$ tiene o no una cota inferior, y si la tiene, encuentre el mínimo.
$$
\begin{aligned}
\text{(i)}\quad X &=\{x \in \mathbb Z | x^2\le 16\}. \\
\text{(ii)}\quad X &=\{x \in \mathbb Z | x=2y \text{\ para algún }
y \in
\mathbb Z\}. \\
\text{(iii)}\quad X &=\{x \in \mathbb Z | x\le 100x\} .
\end{aligned}
$$
\item Un subconjunto $Y$ de $Z$ se dice que tiene una {\em cota superior}\index{cota superior} $c$ si $c\ge y$ para todo $y \in Y$. Una cota superior que además es un elemento de $Y$ es llamada el {\em máximo} de $Y$. Use el axioma {\bf I12} para demostrar que si $Y$ es no vacío y tiene una cota superior, entonces tiene máximo. [Ayuda: aplique el axioma al conjunto cuyos elementos son $-y$ ($y \in Y$).] 
\item Los enteros $n$ que satisfacen $1 \le n \le 25$ están acomodados en forma arbitraria en un arreglo cuadrado de cinco filas y cinco columnas. Se selecciona el máximo  de cada fila, y se denota $s$ al mínimo entre ellos. De manera similar, el mínimo de cada columna es seleccionado y $t$ denota al máximo entre ellos. Demuestre que $s\ge t$ y de un ejemplo en el cual $s\not=t$.
\end{enumerate}
\end{subsection}

\end{section}

\begin{section}{Definiciones recursivas}\label{1.3}

Sea $\mathbb N$ el conjuntos de enteros positivos, esto es
$$
\mathbb N = \{ n \in \mathbb Z | n\ge 1\},
$$
y denotemos $\mathbb N_0$ el conjunto $\mathbb N \cup \{0\}$, esto
es
$$
\mathbb N_0 = \{ n \in \mathbb Z | n\ge 0\}.
$$
Si $X$ es un subconjunto de $\mathbb N$ (o de $\mathbb N_0$)  entonces automáticamente tiene una cota inferior, pues cada elemento $x$ de $X$ satisface $x\ge 1$ (o $x\ge 0$). As{í}, en este caso el axioma del buen orden toma la forma
$$
\begin{aligned}
&\text{\it si $X$ es un subconjunto no vacío de $\mathbb N$ o $\mathbb N_0$ entonces $X$  tiene un mínimo.}
\end{aligned}
$$
Esta la forma más usada en la práctica.

Nuestro primer uso del axioma del buen orden será para justificar un procedimiento muy usual. Frecuentemente encontramos una expresión de la forma $u_n$, donde $n$ indica cualquier entero positivo: por ejemplo, podríamos tener $u_n=3n+2$, o $u_n = (n+1)(n+2)(n+3)$. En estos ejemplos $u_n$ es dado por una fórmula explícita y no existe dificultad en explicar como calcular $u_n$ cuando se nos da un valor específico para $n$. Sin embargo en muchos casos no conocemos una fórmula para $u_n$; es más, nuestro problema puede ser encontrarla. En estos casos pueden darnos ciertos valores de $u_n$ para enteros positivos $n$ peque\~nos, y una relación entre el $u_n$ general y algunos de los $u_r$ con $r<n$. Por ejemplo, supongamos nos es dado 
$$
u_1=1, \qquad u_2=2, \qquad u_n =u_{n-1} +u_{n-2} \qquad (n\ge 3).
$$
Para calcular los valores de $u_n$ para todo $n$ de $\mathbb N$ podemos proceder como sigue:
$$
\begin{matrix}
u_3 & = & u_2 + u_1 & = & 2+1 &=& 3, \\
u_4 & = & u_3 + u_2 & = & 3+2 &=& 5, \\
u_5 & = & u_4 + u_3 & = & 5+3 &=& 8,
\end{matrix}
$$
y así siguiendo.  Éste es un ejemplo de una {\it definición recursiva}. Es ``obvio'' que el método dará un valor único de $u_n$ para todo entero positivo $n$. Pero hablando estrictamente necesitamos el axioma del buen orden para justificar la conclusión a través de las siguientes líneas.

Supongamos que existe un entero positivo $n$ para el cual $u_n$ no está definido de manera única. Entonces por el axioma del buen orden existe un entero positivo mínimo $m$ con esta propiedad. Como $u_1$ y $u_2$ están explícitamente definidos, $m$ no es 1 o 2 y la ecuación $u_m =u_{m-1} +u_{m-2}$ es aplicable. Por la definición de $m$, $u_{m-1}$ y $u_{m-2}$ están definidos de manera única, y la ecuación nos da un valor único para $u_m$ , contrariamente a la hipótesis. La contradicción surge de suponer que no está bien definido para algún $n$, y por lo tanto esta suposición debe ser falsa.

El lector no debe desanimarse por el uso de argumentos tan retorcidos para establecer algo que es ``obviamente'' verdadero. En primer lugar, no debemos usar resultados ilegítimamente (sin demostrarlos), y en segundo lugar, el hecho de que el resultado sea ``obvio'' simplemente indica que estamos trabajando con la correcta representación mental de $\mathbb N$ y $\mathbb Z$. Una vez que hemos establecido esa representación sobre bases firmes podemos empezar a extendernos y obtener resultados que no sean tan ``obvios''.

El método de definición recursiva aparecerá bastante seguido en el resto del apunte. Existen otras formas de este procedimiento que se ``esconden'' por su notación. ?`Qué significan las siguientes expresiones?
$$
\sum_{r=1}^{n} 2r-1,\qquad 1+3+5+\cdots +(2n-1).
$$
Claramente no basta decir que uno significa lo mismo que el otro, porque cada uno contiene un misterioso símbolo, $\sum$ y $\cdots$, respectivamente. Lo que deberíamos decir es que cada uno de ellos es equivalente a la expresión $s_n$, dada por la siguiente definición recursiva:
$$
s_1= 1, \qquad s_n = s_{n-1} +(2n-1) \quad (n\ge 2).
$$

Esto hace ver claro que ambos símbolos misteriosos son, en realidad, una forma de acortar una definición recursiva, y que por lo tanto son expresiones definidas para todo $n$ en $\mathbb N$.

Ideas similares pueden aplicarse a la definición de productos tal como $n!$ (que se lee $n$ {\it factorial}). Si decimos que
$$
n! = \Pi_{i=1}^n i , \quad \text{ o } \quad n!=1 \times 2 \times 3
\times \cdots \times n,
$$
el significado es bastante claro para cualquiera. Pero para precisar (y hacerlo claro para una computadora) debemos usar la definición recursiva
$$
1!=1,\qquad n!=n \times (n-1)! \quad (n\ge 2).
$$

\begin{subsection}{Ejercicios}
\begin{enumerate}
\item En el caso siguiente calcule (donde sea posible) los valores de $u_1$, $u_2$, $u_3$ y $u_4$ dados por las ecuaciones. Si no puede calcular los valores explique porque la definición no esta bien.
\begin{enumerate}
\item[(\em i)] $u_1 = 1$, $u_2=1$, $u_n = u_{n-1} +2 u_{n-2}$, ($n \ge 3$). 
\item[(\em ii)] $u_1 = 1$, $u_n = u_{n-1} +2u_{n-2}$ ($n \ge 2$). 
\item[(\em iii)] $u_1 = 0$, $u_n = nu_{n-1}$ ($n \ge 2$).
\end{enumerate}
\item De una definición recursiva de la ``$n$-ésima potencia'' para todo $n\ge 1$.
\item Sea $u_n$ definido por las ecuaciones
$$
u_1=2,\qquad u_n= 2^{u_{n-1}} \quad (n\ge 2).
$$
?`Cuál es el primer valor de $n$ para el cual no se puede calcular $u_n$ usando una calculadora de bolsillo o de su celular?
\item Escriba fórmulas explícitas para las expresiones $u_n$ definidas por las siguientes ecuaciones.
\begin{enumerate}
\item[(\em i)] $u_1 = 1$, $u_n = u_{n-1} +3$ ($n \ge 2$). 
\item[(\em ii)] $u_1 = 1$, $u_n = n^2u_{n-1}$ ($n \ge 2$).
\end{enumerate}
\end{enumerate}
\end{subsection}
\end{section}

\begin{section}{El principio de inducción}\label{1.4}

Supongamos que nos piden que demostremos el resultado
$$
1+3+5+\cdots+(2n-1) = n^2.
$$
En otras palabras, debemos demostrar que la expresión de la izquierda definida recursivamente es igual a la expresión definida explícitamente por la fórmula de la derecha, para todos los enteros positivos $n$. Podemos proceder como sigue.

La fórmula es ciertamente correcta cuando $n=1$ puesto que $1=1^2$.  Supongamos que es correcta para un valor específico de $n$, digamos para $n=k$, de modo que
$$
1+3+5+\cdots+(2k-1) = k^2.
$$
Podemos usar esto para simplificar la expresión definida recursivamente a la izquierda cuando $n$ es igual a $k+1$,
$$
\begin{aligned}
1+3+5+\cdots+(2k+1) &= 1+3+5+\cdots+(2k-1) +(2k+1) \\
&=k^2 +(2k+1) \\
&=(k+1)^2.
\end{aligned}
$$
Por lo tanto si el resultado es correcto cuando $n=k$, entonces lo es cuando $n=k+1$. Se comienza observando que si es correcto cuando $n=1$, debe ser por lo tanto correcto cuando $n=2$. Con el mismo argumento como es correcto cuando $n=2$ debe serlo cuando $n=3$. Continuando de esta forma veremos que es correcto para todos los enteros positivos $n$.

La esencia de este argumento es comúnmente llamada {\it principio de inducción}. Es una técnica poderosa, fácil de aplicar y la aplicaremos frecuentemente. Pero primero debemos examinar sus bases lógicas y para hacerlo necesitamos una formulación más general. Con $S$ denotemos al subconjunto de $\mathbb N$ para el cual el resultado es correcto: por supuesto, nuestra intención es probar que $S$ es todo $\mathbb N$. El primer paso es probar que $1$ pertenece a $S$, y luego demostraremos que si $k$ pertenece a $S$, también $k+1$. Entonces lo pensamos paso a paso y concluimos que $S=\mathbb N$. Afortunadamente el pensarlo paso a paso no es esencial, debido a que el principio de inducción es consecuencia de los axiomas que elegimos tan cuidadosamente para $\mathbb Z$ y $\mathbb N$. Más específicamente es consecuencia del axioma del buen orden.

\begin{teorema}\label{t1.4} Supongamos que $S$ es un subconjunto de $\mathbb N$ que satisface las condiciones \index{principio de inducción}
\begin{enumerate}
\item[(\em i)] $1 \in S$,
\item[(\em ii)] para cada $k \in \mathbb N$, si $ k \in S$ entonces $k+1\in S$.
\end{enumerate}
Entonces se sigue que $S=\mathbb N$.
\end{teorema}
\begin{proof}Si la conclusión es falsa, $S \not= \mathbb N$ y el conjunto complementario $S^{\text{c}}$ definido por
$$
S^{\text{c}}= \{ r \in \mathbb N | r\not\in S\}
$$
es no vacío. Por el axioma del buen orden, $S^{\text{c}}$ tiene un menor elemento $m$. Como 1 pertenece a $S$, $m\not=1$. Se sigue que $m-1$ pertenece a $\mathbb N$ y como $m$ es el mínimo de $S^{\text{c}}$, $m-1$ debe pertenecer a $S$. Poniendo $k=m-1$ en la condición (ii), concluimos que $m$ esta en $S$, lo cual contradice el hecho de que $m$ pertenece a $S^{\text{c}}$. De este modo, la suposición $S \not= \mathbb N$ nos lleva a un absurdo, y por lo tanto tenemos $S= \mathbb N$.
\end{proof}

En la práctica, generalmente presentamos una ``demostración por inducción'' en términos más descriptivos. El hecho de que el resultado es verdadero cuando $n=1$ se llama {\it base de la inducción} o {\it caso base}, ({\em ii}) es llamado  el {\em paso inductivo} y la suposición de que es verdadero cuando $n=k$ es llamada {\it hipótesis inductiva}\index{hipótesis inductiva}. Cuando se utilizan estos términos, no es necesario introducir explícitamente el conjunto $S$.

El principio de inducción es útil para probar la veracidad de propiedades relativas a los números naturales. Por ejemplo, consideremos las siguientes propiedades $P(n)$, $Q(n)$ y $R(n)$:
\begin{enumerate}
\item $P(n)$ es la propiedad: $2n -1 < n^2 + 1$,
\item $Q(n)$ es la afirmación: si $n$ es par entonces $n$ es divisible por 4,
\item $R(n)$ es la afirmación: $2n < n- 1$.
\end{enumerate}
Intuitivamente notamos que $P(n)$ es verdadera para cualquier $n$ natural, $Q(n)$ lo es para algunos valores de $n$ y es falsa para otros y $R(n)$ es falsa para todo valor de $n$. Sin embargo, para verificar realmente que la propiedad $P(n)$ es verdadera para todo $n$ natural no podemos hacerlo probando para cada $n$ en particular. Resulta entonces muy útil la siguiente versión equivalente del principio de inducción.
\begin{teorema}\label{induccion2} Sea $P(n)$ una propiedad para $n \in \mathbb N$ tal que:
\begin{enumerate}
\item[(1)] $P(1)$ es verdadera.
\item[(2)] Para todo $k \in \mathbb N$, $P(k)$ verdadera implica $P(k + 1)$ verdadera.
\end{enumerate}
Entonces $P(n)$ es verdadera para todo $n \in \mathbb N$.
\end{teorema}
\begin{proof} Basta tomar
$$S = \{n \in \mathbb N| P(n) \text{ es verdadera} \}.$$
Entonces $S$ es un subconjunto de $\mathbb N$ y las condiciones (1) y (2) nos dicen que $1 \in S$ y  si $ k \in S$ entonces $k+1\in S$. Por el teorema \ref{t1.4} se sigue que $S= \mathbb N$, es decir que $P(n)$ es verdadera para todo $n$. natural.
\end{proof}

En la notación del teorema anterior, (1) es llamado  el {\em caso base}, (2) es llamado el  {\em paso inductivo} y $P(k)$ es llamada la {\em hipótesis inductiva}. El paso inductivo  consiste en probar que $P(k) \Rightarrow P(k + 1)$ o, equivalentemente, podemos suponer $P(k)$ verdadera y a partir de ella probar $P(k + 1)$. 

\begin{ejemplo}\label{ejemplo141} Sea $a\in \mathbb Z$ tal que $0<a$. Probar que $0<a^n$ para todo $n \in \mathbb N$.
\end{ejemplo}
\begin{proof}

\noindent(\it Caso  base\rm) El resultado es verdadero cuando $n=1$ pues $ 0 < a=a^1$.

\noindent ({\it Paso  inductivo})  Supongamos que el resultado verdadero cuando $n=k$, o sea, que la hipótesis inductiva es $0 < a^k$. Entonces, como $0<a$, multiplicando por $a$ ambos lados de la desigualdad obtenemos, por la compatibilidad de $<$ con el producto, que $a{.}0 < a^k{.}a$, es decir $0<a^{k+1}$. Luego el resultado es verdadero cuando $n=k+1$ y por el principio de inducción, es verdadero para todos los enteros positivos $n$.
\end{proof}

\begin{ejemplo} El entero $x_n$ esta definido recursivamente por
$$
x_1=2, \quad \text{ y } \quad x_n=x_{n-1} +2n \quad (n\ge 2).
$$
Demuestre que
$$
x_n = n(n+1) \qquad \text{ para todo } n\in \mathbb N.
$$
\end{ejemplo}
\begin{proof}

\noindent(\it Caso  base\rm) El resultado es verdadero cuando $n=1$ pues $ 2=1\times2$.

\noindent ({\it Paso  inductivo})  Supongamos que el resultado verdadero cuando $n=k$, o sea, que la hipótesis inductiva es $x_k = k(k+1)$. Entonces
$$
\begin{matrix} 
x_{k+1} &=& x_k + 2(k+1) \hfill &\quad \text{(por la definición recursiva)} \hfill \\
&=& k(k+1)+2(k+1) \hfill &\quad \text{(por hipótesis inductiva)}\hfill \\
&=& (k+1)(k+2). \hfill &
\end{matrix}
$$
Luego el resultado es verdadero cuando $n=k+1$ y por el principio de inducción, es verdadero para todos los enteros positivos $n$.
\end{proof}

Existen varias formas modificadas del principio de inducción. A veces es conveniente tomar como base inductiva el valor $n=0$, por otro lado puede ser apropiado tomar un valor como $2$ o $3$ porque los primeros casos pueden ser excepcionales. Cada problema debe ser tratado según sus características. Otra modificación útil es tomar como hipótesis inductiva la suposición de que el resultado es verdadero para todos los valores $n\le k$, más que para $n=k$ solamente. (Esta formulación es llamada a veces el principio de inducción {\it completa}.) Todas esas modificaciones pueden justificarse con cambios triviales en la demostración del teorema \ref{t1.4}.

El siguiente teorema incorpora todas las modificaciones del principio de inducción mencionadas más arriba.

\begin{teorema} Supongamos que $n_0$ es cualquier entero (no necesariamente positivo), y sea $X$ el conjunto de enteros $n$ tal que $ n_0 \le n$. Sea $S$ un subconjunto de $X$ que satisface las condiciones: \index{principio de inducción completa}
\begin{enumerate}
\item[(i)] $n_0 \in S$,
\item[(ii)] si $h\in S$ para todo $h$ en el rango $n_0 \le h \le k$ entonces $k+1 \in S$.
\end{enumerate}
Entonces se sigue que $S=X$.
\end{teorema}
\begin{proof}
Si la conclusión es falsa, $S \not= X$ y el conjunto complementario (en $X$)  $S^{\text{c}}$ definido por
$$
S^{\text{c}}= \{ r \in X | r\not\in S\}
$$
es no vacío. Como $X$ es un conjunto acotado inferiormente por $n_0$, por el axioma del buen orden, $S^{\text{c}}$ tiene un menor elemento $m$. Como $n_0$ pertenece a $S$, $m\not=n_0$. Se sigue que $m-1$ pertenece a $X$ y como $m$ es el mínimo de $S^{\text{c}}$, $m-1$ debe pertenecer a $S$. Poniendo $k=m-1$ en la condición (ii), concluimos que $m$ esta en $S$, lo cual contradice el hecho de que $m$ pertenece a $S^{\text{c}}$. De este modo, la suposición $S \not= X$ nos lleva a un absurdo, y por lo tanto tenemos $S= X$.
\end{proof}

Como se podrá observar, la demostración es muy similar a  la del teorema \ref{t1.4}. El teorema anterior lo podemos utilizar para la demostración de propiedades dependientes de números enteros.

\begin{teorema}[Inducción completa]\label{ind-completa} Sea $n_0$ número entero y sea $P(n)$ una propiedad para $n \ge n_0$ tal que:
\begin{enumerate}
\item[(1)] $P(n_0)$ es verdadera.
\item[(2)] Si $P(h)$ verdadera para toda $h$ tal que $n_0 \le h \le k$ implica $P(k + 1)$ verdadera.
\end{enumerate}
Entonces $P(n)$ es verdadera para todo $n \ge n_0$.
\end{teorema}
\begin{proof} Ejercicio.
\end{proof}

\begin{ejemplo}
Sea $u_1 = 3$, $u_2 = 5$ y $u_n = 3u_{n-1}  - 2u_{n-2}$ con $n \in  \mathbb N$, $n \ge 3$. Probar que $u_n = 2^n + 1$.
\begin{proof} Aquí usaremos una extensión natural del principio de inducción: en este caso, el caso base es $n=1$ y $n=2$.

\noindent(\it Caso  base\rm) El resultado es verdadero cuando $n= 1$ pues $3 = 2^1+1$ y para $n=2$ pues $ 5 =2^2+1$.

\noindent ({\it Paso  inductivo}) Supongamos que $k \ge 2$ y el resultado  es cierto para los $h$ tales que  $1 \le h \le k$. Es decir que $u_h = 2^h+1$ para $1 \le h \le k$ y $k \ge 2$ (hipótesis inductiva), entonces
$$
\begin{matrix} u_{k+1} &=& 3u_k -2u_{k-1} \hfill &\quad \text{(por definición recursiva)} \hfill \\
&=& 3(2^k+1)-2(2^{k-1}+1) \hfill &\quad \text{(por hipótesis inductiva})\hfill \\
&=& 3\cdot 2^k+3-2\cdot 2^{k-1}-2 \hfill & \\
&=& 3\cdot 2^k+1- 2^{k} \hfill & \\
&=& 2\cdot 2^k+1 \hfill & \\
&=& 2^{k+1}+1. \hfill & 
\end{matrix}
$$
\end{proof}
\end{ejemplo}

\begin{ejemplo}
Sea $a \in \mathbb Z$ y $n \in \mathbb N$. Definimos $a^n$ de la siguiente manera:
\begin{equation}\label{potencia}
a^1 = a, \quad a^{n+1} = a^{n}a \,\,\text{ para $n >1$.}
\end{equation}
Si $n,m \in \mathbb N$ verificar las siguientes propiedades
\begin{enumerate}
\item \label{pot+pot}$a^{n}a^m = a^{n+m}$.
\item \label{potpot}$(a^n)^m = a^{nm}$
\end{enumerate}
\begin{proof}
Veamos la primera afirmación. Se fijará $n$ y se hará inducción sobre $m$. 

\noindent(\it Caso  base\rm) Debemos ver que $a^{n}a^1 = a^{n+1}$, lo cual es verdadero por la definición recursiva (\ref{potencia}). 

\noindent ({\it Paso  inductivo}) Supongamos que el resultado es verdadero para $m=k$, es decir que $a^{n}a^k = a^{n+k}$ (hipótesis inductiva). Veamos que  $a^{n}a^{k+1} = a^{n+k+1}$. Ahora bien, 
\begin{alignat*}2
a^{n}a^{k+1} &= a^{n}a^{k}a&  & \text{(definición (\ref{potencia}))} \\
&= a^{n+k}a& & \text{(hipótesis inductiva)} \\
&= a^{n+k+1}&  & \text{(definición (\ref{potencia}))}. 
\end{alignat*}

Probemos ahora \ref{potpot}. Al igual que antes, Se fijará $n$ y se hará inducción sobre $m$.

\noindent(\it Caso  base\rm) Debemos ver que $(a^n)^1 = a^n$, lo cual es verdadero por la definición recursiva (\ref{potencia}). 

\noindent ({\it Paso  inductivo}) Supongamos que el resultado es verdadero para $m=k$, es decir que  $(a^n)^k = a^{nk}$ (hipótesis inductiva). Veamos que  $(a^n)^{k+1} = a^{n(k+1)}$. 
\begin{alignat*}2
(a^n)^{k+1} &= (a^n)^{k}a^n&  & \text{(definición (\ref{potencia}))} \\
&= a^{nk}a^n& & \text{(hipótesis inductiva)} \\
&= a^{nk+n}&  & \text{(por \ref{pot+pot})}\\
&= a^{n(k+1)}&  &.  
\end{alignat*}

\end{proof}
\end{ejemplo}

\begin{subsection}{Ejercicios}
\begin{enumerate}
\item Use el principio de inducción para demostrar que
$$
1^2+2^2+\cdots +n^2 = \frac16 n(n+1)(2n +1)
$$
para todos los enteros positivos $n$.
\item Haga una tabla de valores de
$$
S_n = 1^3+2^3+\cdots +n^3
$$
para $1 \le n\le 6$. Basándose en su tabla sugiera una fórmula para $S_n$. [Ayuda: los valores de $S_n$ son cuadrados perfectos.] Use el principio de inducción para establecer que la fórmula es correcta para todo $n\ge 1$. (Si el método falla !`su fórmula es equivocada!)
\item Probar que
$$
1^4+2^4+\cdots+n^4= \frac{1}{30}n(n+1)(2n+1)(3n^2+3n+1).
$$
\item Usar el principio de inducción para probar que $2^n>n+1$ para todo entero $n\ge2$.
\item Encuentre el menor entero positivo $n_0$ para el cual sea verdadero que $n! \ge 2^n$. Tomando el caso $n=n_0$ como la base inductiva, demuestre que el resultado vale para $n\ge n_0$.
\item En los siguientes casos encuentre los valores apropiados de $n_0$ para la base inductiva y demuestre que la afirmación es verdadera para todos los $n\ge n_0$.
$$
\text{(i)} \ n^2 +6n + 9 \ge 0, \qquad \text{(ii)} \ n^3 \ge 6n^2.
$$
\end{enumerate}
\end{subsection}
\end{section}

\chapter[Conteo]{Conteo}

Intutivamente, diremos que un conjunto $A$ es finito si podemos contar la cantidad de elementos que tiene. En ese caso denotaremos $|A|$ la cantidad de elementos de $A$ y la llamaremos el {\em cardinal de $A$}\index{cardinal de un conjunto}.  

\begin{section}{Principios básicos}

\noindent\textbf{El principio de adición}

Se puede realizar una acción $X$ de $n$ formas distintas o, alternativamente, se puede realizar otra acción $Y$ de $m$ formas distintas. Entonces el número de formas de realizar la acción ``$X$ o $Y$'' es $n + m$.

\begin{ejemplo}\label{cine} Supongamos que una persona va a salir a pasear  y puede ir al cine donde hay 3 películas en cartel o al teatro donde hay 4 obras posibles. Entonces, tendrá un total de $3+4=7$ formas distintas de elegir el paseo. 
\end{ejemplo}

Este principio es el más básico del conteo y más formalmente dice que si $A$ y $B$ son conjuntos finitos disjuntos, entonces 
\begin{equation}\label{padd}
|A \cup B| =|A|+|B|.
\end{equation}
El principio es fácilmente generalizable a varios conjuntos.

\begin{proposicion}\label{principiodeadicion}
Sean $A_1,\ldots,A_n$ conjuntos finitos tal que $A_i \cap A_j = \emptyset$ cuando $i\not=j$, entonces 
\begin{equation*}
|A_1 \cup \cdots \cup A_n| =|A_1|+\cdots+|A_n|.
\end{equation*}
\end{proposicion}
\begin{proof} 
La  prueba se hace por inducción en $n$. Debemos probar 
\begin{align*}
P(n) =\; &\text{Si $A_1,\ldots,A_n$ conjuntos finitos disjuntos dos a dos, entonces }\\  &|A_1 \cup \cdots \cup A_n| =|A_1|+\cdots+|A_n|.
\end{align*}

({\em Caso base $n=1$}) En este caso no hay nada que probar pues  $|A_1|=|A_1|$.

({\em Paso inductivo}) La hipótesis inductiva es $P(k)$ y debemos probar que $P(k) \Rightarrow P(k+1)$. Si denotamos $B = A_1 \cup \cdots \cup A_k$, entonces 
$$
A_1 \cup \cdots \cup A_{k+1} = B \cup A_{k+1}
$$
Ahora bien, si $x \in B \cap A_{k+1}$, entonces $x \in A_i$ para algún $i < k+1$ y $x \in A_{k+1}$. Como $A_{i} \cap A_{k+1} = \emptyset$, se produce un absurdo  que viene de suponer que existía un elemento en $B \cap A_{k+1}$. Luego   $B \cap A_{k+1}= \emptyset$ y por el principio de adición  $|B \cup A_{k+1}| = |B|+|A_{k+1}|$. 

Por la hipótesis inductiva tenemos que 
$$
|B| = |A_1 \cup \cdots \cup A_k| =|A_1|+\cdots+|A_k|,
$$
Luego
$$
|A_1 \cup \cdots \cup A_k \cup  A_{k+1}| = |B|+|A_{k+1}| = |A_1|+\cdots+|A_k|+|A_{k+1}|.
$$
\end{proof}

Si $A$ y $B$ no son disjuntos, cuando sumamos $|A|$ y $|B|$ estamos contando $A \cap B$ dos veces. Entonces, para obtener la respuesta correcta debemos restar $|A \cap B|$ y obtenemos
$$
|A \cup B| = |A|+|B| - |A \cap B|.
$$
Generalizar la fórmula de arriba a más conjuntos no es del todo sencillo y es el  llamado principio del tamiz o principio de inclusión-exclusión (ver apéndice \ref{principiodeltamiz}). 

\vskip .3cm

\noindent\textbf{El principio de multiplicación}

Suponga que una actividad consiste de 2 etapas y la primera etapa puede ser realizada de $n_1$ maneras y la etapa 2  puede realizarse de $n_2$  maneras, independientemente de como se ha hecho la etapa 1. Entonces toda la actividad puede ser realizada de $n_1\times n_2$  formas distintas.

\begin{ejemplo}
Supongamos que la persona del ejemplo \ref{cine} tiene suficiente tiempo y dinero para ir primero al cine y luego al teatro. Entonces tendrá  $3 \times 4=12$ formas distintas de hacer el paseo.
\end{ejemplo}

Formalmente, si $A,B$ conjuntos y definimos el {\em producto cartesiano}\index{producto cartesiano} entre $A$ y $B$ por
$$
A \times B = \{(a,b): a \in A, b \in B\}.
$$
Entonces si $A$ y $B$ son conjuntos finitos se cumple que
$$
|A \times B| = |A||B|.  
$$

\end{section}

\begin{section}{Selecciones ordenadas con repetición}

Un aplicación inmediata del principio de multiplicación  es que nos permite calcular la cantidad de selecciones ordenadas con repetición. 

\begin{ejemplo} Sea  $X = [[ 1 , 3]] = \{ 1, 2, 3 \}$ ¿de cuántas formas se pueden elegir dos de estos números en forma ordenada? Es decir, debemos elegir dos números $a$ y $b$ teniendo en cuenta que si $a\not=b$ no es lo mismo elegir $a$ y luego $b$ que $b$ y $a$.  

Para no escribir demasiado vamos a adoptar una notación muy conveniente: si elegimos $a$ y $b$ en forma ordenada, denotamos $ab$. Entonces, en muy breve espacio seremos capaces de escribir todas las selecciones ordenadas de 2 elementos del  conjunto  $[[ 1 , 3]]$:
\begin{align*}
&11&\quad &12&\quad &13 \\
&21&\quad &22&\quad &23\\
&31&\quad &32&\quad &33
\end{align*}
Son $9 = 3^2$ formas. ¿Cómo justificamos esto? Es claro que para la primera elección tenemos 3 valores posibles y para la segunda elección tenemos también 3 valores posibles, entonces, por el principio de multiplicación, tenemos en total $3\cdot 3$ elecciones posibles.  

Avancemos un poco más y ahora elijamos en forma ordenada 3 elementos de  $[[ 1 , 3]]$, es claro que estas elecciones son
\begin{align*}
&1 1 1&\quad &211&\quad &311 \\
&1 1 2&\quad & 212&\quad & 312\\
&1 1 3&\quad & 213&\quad & 313\\
&1 2 1&\quad & 221&\quad & 321\\
&1 2 2&\quad & 222&\quad & 322\\
&1 2 3&\quad & 223&\quad & 323\\
&1 3 1&\quad & 231&\quad & 331\\
&1 3 2&\quad & 232&\quad & 332\\
&1 3 3&\quad & 233&\quad & 333.
\end{align*}
El total de elecciones posibles $27 = 3^3$. Esto se justifica usando dos veces el principio de multiplicación: para la primera elección tenemos 3 valores posibles. Para la segunda elección tenemos también 3 valores posibles, entonces, por el principio de multiplicación, tenemos en total $3\cdot 3$ valores posibles para la elección de los dos primeros números. Como para la tercera elección tenemos 3 valores posibles, por el principio de multiplicación nuevamente, tenemos   $3\cdot 3 \cdot 3$ elecciones posibles.

Un diagrama arbolado ayuda a pensar.

\begin{tikzpicture}[line width=1pt]
\lineatz{5}{-95}{28}{-35}
\lineatz{5}{-95}{28}{-95}
\lineatz{5}{-95}{28}{-155}

\ponertz{30}{-35}{1}
\lineatz{32}{-35}{57}{-15}
\lineatz{32}{-35}{57}{-35}
\lineatz{32}{-35}{57}{-50}

\ponertz{60}{-15}{1}
\lineatz{62}{-15}{87}{-10}
\lineatz{62}{-15}{87}{-15}
\lineatz{62}{-15}{87}{-20}
\ponertz{90}{-10}{1}
\ponertz{90}{-15}{2}
\ponertz{90}{-20}{3}
\ponertz{100}{-10}{111}
\ponertz{100}{-15}{112}
\ponertz{100}{-20}{113}

\ponertz{60}{-35}{2}
\lineatz{62}{-35}{87}{-30}
\lineatz{62}{-35}{87}{-35}
\lineatz{62}{-35}{87}{-40}
\ponertz{90}{-30}{1}
\ponertz{90}{-35}{2}
\ponertz{90}{-40}{3}
\ponertz{100}{-30}{121}
\ponertz{100}{-35}{122}
\ponertz{100}{-40}{123}

\ponertz{60}{-55}{3}
\lineatz{62}{-55}{87}{-50}
\lineatz{62}{-55}{87}{-55}
\lineatz{62}{-55}{87}{-60}
\ponertz{90}{-50}{1}
\ponertz{90}{-55}{2}
\ponertz{90}{-60}{3}
\ponertz{100}{-50}{131}
\ponertz{100}{-55}{132}
\ponertz{100}{-60}{133}

\ponertz{30}{-95}{2}
\lineatz{32}{-95}{57}{-75}
\lineatz{32}{-95}{57}{-95}
\lineatz{32}{-95}{57}{-115}

\ponertz{60}{-75}{1}
\lineatz{62}{-75}{87}{-70}
\lineatz{62}{-75}{87}{-75}
\lineatz{62}{-75}{87}{-80}
\ponertz{90}{-70}{1}
\ponertz{90}{-75}{2}
\ponertz{90}{-80}{3}
\ponertz{100}{-70}{211}
\ponertz{100}{-75}{212}
\ponertz{100}{-80}{213}

\ponertz{60}{-95}{2}
\lineatz{62}{-95}{87}{-90}
\lineatz{62}{-95}{87}{-95}
\lineatz{62}{-95}{87}{-100}
\ponertz{90}{-90}{1}
\ponertz{90}{-95}{2}
\ponertz{90}{-100}{3}
\ponertz{100}{-90}{221}
\ponertz{100}{-95}{222}
\ponertz{100}{-100}{223}

\ponertz{60}{-115}{3}
\lineatz{62}{-115}{87}{-110}
\lineatz{62}{-115}{87}{-115}
\lineatz{62}{-115}{87}{-120}
\ponertz{90}{-110}{1}
\ponertz{90}{-115}{2}
\ponertz{90}{-120}{3}
\ponertz{100}{-110}{231}
\ponertz{100}{-115}{232}
\ponertz{100}{-120}{233}

\ponertz{30}{-155}{3}
\lineatz{32}{-155}{57}{-135}
\lineatz{32}{-155}{57}{-155}
\lineatz{32}{-155}{57}{-175}

\ponertz{60}{-135}{1}
\lineatz{62}{-135}{87}{-130}
\lineatz{62}{-135}{87}{-135}
\lineatz{62}{-135}{87}{-140}
\ponertz{90}{-130}{1}
\ponertz{90}{-135}{2}
\ponertz{90}{-140}{3}
\ponertz{100}{-130}{311}
\ponertz{100}{-135}{312}
\ponertz{100}{-140}{313}

\ponertz{60}{-155}{2}
\lineatz{62}{-155}{87}{-150}
\lineatz{62}{-155}{87}{-155}
\lineatz{62}{-155}{87}{-160}
\ponertz{90}{-150}{1}
\ponertz{90}{-155}{2}
\ponertz{90}{-160}{3}
\ponertz{100}{-150}{321}
\ponertz{100}{-155}{322}
\ponertz{100}{-160}{323}

\ponertz{60}{-175}{3}
\lineatz{62}{-175}{87}{-170}
\lineatz{62}{-175}{87}{-175}
\lineatz{62}{-175}{87}{-180}
\ponertz{90}{-170}{1}
\ponertz{90}{-175}{2}
\ponertz{90}{-180}{3}
\ponertz{100}{-170}{331}
\ponertz{100}{-175}{332}
\ponertz{100}{-180}{333}
\end{tikzpicture}

\vskip .3cm

Cada rama del árbol representa una selección ordenada de elementos de $[[1, 3]]$.

\end{ejemplo}

El razonamiento anterior  se puede extender:

\begin{proposicion}
Sean  $m,n \in \mathbb N$. Hay   $n^m$ formas posibles de elegir ordenadamente $m$ elementos de un conjunto de $n$ elementos.
\end{proposicion}
\begin{proof}[Idea de la prueba]
La prueba de esta proposición se basa en aplicar el principio de multiplicación $m-1$ veces, es decir debemos hacer inducción sobre $m$ y usar el principio de multiplicación en el paso inductivo. 
\end{proof}

\begin{observacion} En el ejemplo denotamos $ [[ 1 , 3]] = \{ 1, 2, 3 \}$. En general, si  $n \in \mathbb N$ denotaremos  $[[ 1 , n]]$ al conjunto de los primeros $n$ números naturales. Es decir:
$$
 [[ 1 , n]] = \{ 1, 2, \ldots,n\}.
$$  
\end{observacion}

\begin{ejemplo}
¿Cuántos números de cuatro dígitos pueden formarse con los dígitos 1, 2, 3, 4, 5, 6?

Por la proposición anterior es claro que hay $6^4$ números posibles.
\end{ejemplo}

\begin{ejemplo}
¿Cuántos números de 5 dígitos y capicúas pueden formarse con los números dígitos 1, 2, 3, 4, 5, 6, 7, 8? Un número capicúa de cinco dígitos es de la forma 
$$
xyzyx
$$
Se reduce a ver cuántos números de tres dígitos pueden formarse con aquéllos dígitos. Exactamente $8^3$.
\end{ejemplo}

 

\begin{ejemplo} Sea $X$ un conjunto de $m$ elementos. Queremos contar cuántos subconjuntos tiene este conjunto.  Por ejemplo, si $X = \{ a, b, c \}$ los subconjuntos de $X$ son exactamente
$$
\emptyset, \{ a \} , \{ b \}, \{ c \}, \{ a, b \}, \{ a, c \}, \{ b, c \}, \{ a, b, c\}.
$$ 
Es decir que existen 8 subconjuntos de $X$, un conjunto de 3 elementos. ¿Cómo podemos encontrar razonando este número? Un forma sería la siguiente:  cuando elijo un subconjunto el $a$ puede estar o no estar en el subconjunto, es decir hay dos posibilidades. Con el $b$ pasa lo mismo, puede estar o no estar y por lo tanto hay 2 posibilidades. Con el $c$ se hace un razonamiento análogo y por lo tanto tenemos que hay en total 
$$
2 \cdot 2 \cdot 2 = 2^3 = 8
$$
posibles subconjuntos de $X$.  

Otra forma de verlo: podemos identificar  cada subconjunto de  $X$ con una terna ordenada de $0$'s y $1$'s de la siguiente manera: si $a$ está en el subconjunto la primera coordenada de la terna es 1, si no es 0;  si $b$ está en el subconjunto la segunda coordenada de la terna es 1, si no es 0;  si $c$ está en el subconjunto la tercera coordenada de la terna es 1, si no es 0. Es decir tenemos la identificación
\begin{align*}
&\emptyset& &\leftrightarrow& &000 \\ 
&\{ a \} & &\leftrightarrow& &100 \\ 
&\{ b \}& &\leftrightarrow& &010 \\ 
&\{ c \}& &\leftrightarrow& &001 \\ 
&\{ a, b \}& &\leftrightarrow& &110 \\ 
&\{ a, c \}& &\leftrightarrow& &101 \\ 
&\{ b, c \}& &\leftrightarrow& &011 \\ 
&\{ a, b, c\}& &\leftrightarrow& &111 .
\end{align*}
 
Observar entonces que seleccionar un subconjunto de $X$ es equivalente a elegir en forma ordenada 3 elementos del conjunto $\{ 0, 1 \}$; y sabemos entonces que en ese caso tenemos $2^3$ posibilidades. 

En general, cuando $X$ tiene $n$ elementos,  elegir un subconjunto de $X$ es  equivalente a elegir en forma  ordenada $n$ elementos del conjunto $\{ 0, 1 \}$ y por lo tanto

\begin{proposicion}\label{cardp} La cantidad de subconjuntos de  
un conjunto de $n$ elementos es $2^n$.
\end{proposicion}

Dado  $X$ un conjunto, denotamos $\mathcal P(X)$ el  conjunto  formado por todos los subconjuntos de $X$, por ejemplo
$$
\mathcal P(\{1,2\}) = \{\emptyset,\{1\},\{2\},\{1,2\}\}.
$$  
Si $X$ es un conjunto finito la proposición \ref{cardp} nos dice que
$$
\mathcal |P(X)| = 2^{|X|}
$$
\end{ejemplo}

\end{section}

\begin{section}{Selecciones ordenadas sin repetición}\label{permutaciones}

Sea $n \in \mathbb{N}$. Definimos recursivamente \emph{factorial de $n$} al número denotado 
$$n!,$$
tal que
\begin{align*}
1! &= 1\\
(n + 1)! &= n!  (n + 1)
\end{align*}
Definimos también
$$0! = 1$$
Por ejemplo
\begin{align*}
2! &= 2 \cdot 1 = 2 \\
3! &= 3 \cdot 2 \cdot 1 = 6 \\
4! &= 3!\cdot 4 = 6 \cdot 4 = 24. 
\end{align*}

Ahora estudiaremos las selecciones ordenadas de $m$ elementos entre $n$ donde {\em no} se permite la repetición. Es decir si  el conjunto es $A= \{a_1,a_2,\ldots,a_n\}$, las selecciones deben ser del tipo 
$$
a_{i_1} a_{i_2} \cdots a_{i_m}
$$
donde  $a_{i_j} \not= a_{i_k}$ si $i\not=k$. 

Por ejemplo, las selecciones de 3 elementos en forma ordenada y sin repetición de $[[1, 3]]$  son exactamente
$$
1 2 3,\; 1 3 2,\; 2 1 3,\; 2 3 1,\; 3 1 2,\; 3 2 1
$$
(son las ternas donde los tres números son distintos). O sea hay 6 selecciones ordenadas y sin repetición de  elementos de $[[1, 3]]$.

Notemos que
$$
3 \cdot 2 \cdot 1 = 6 = 3!
$$
Esta forma de escribir nos da la razón de que haya 6 selecciones ordenadas y sin repetición de  elementos de $[[1, 3]]$: para la elección del primer elemento tenemos 3 posibilidades (el 1, 2 o 3). Cuando elegimos el segundo elemento, si queremos que no haya repetición, debemos excluir el valor elegido en primer lugar, o sea que tenemos solo 2 elecciones. Análogamente para la tercera elección solo hay solo una posibilidad, pues debemos descartar los valores elegidos en el primer y segundo lugar. Tenemos entonces  $3 \cdot 2 \cdot 1$ selecciones posibles.

En un diagrama arbolado la selección se puede representar de la siguiente forma:

\vskip .3cm

\begin{tikzpicture}[line width=1pt]
\lineatz{5}{-35}{27}{-15}
\lineatz{5}{-35}{27}{-35}
\lineatz{5}{-35}{27}{-55}

\ponertz{30}{-15}{1}
\lineatz{32}{-15}{57}{-10}
\lineatz{32}{-15}{57}{-20}
\ponertz{60}{-10}{2}
\ponertz{60}{-20}{3}
\lineatz{62}{-10}{87}{-10}
\lineatz{62}{-20}{87}{-20}
\ponertz{90}{-10}{3}
\ponertz{90}{-20}{2}
\ponertz{100}{-10}{123}
\ponertz{100}{-20}{132}

\ponertz{30}{-35}{2}
\lineatz{32}{-35}{57}{-30}
\lineatz{32}{-35}{57}{-40}
\ponertz{60}{-30}{1}
\ponertz{60}{-40}{3}
\lineatz{62}{-30}{87}{-30}
\lineatz{62}{-40}{87}{-40}
\ponertz{90}{-30}{3}
\ponertz{90}{-40}{1}
\ponertz{100}{-30}{213}
\ponertz{100}{-40}{231}

\ponertz{30}{-55}{3}
\lineatz{32}{-55}{57}{-50}
\lineatz{32}{-55}{57}{-60}
\ponertz{60}{-50}{1}
\ponertz{60}{-60}{2}
\lineatz{62}{-50}{87}{-50}
\lineatz{62}{-60}{87}{-60}
\ponertz{90}{-50}{2}
\ponertz{90}{-60}{1}
\ponertz{100}{-50}{312}
\ponertz{100}{-60}{321}
\end{tikzpicture}

El número total es entonces $3 \times 2 \times 1 = 6$.

Pensemos ahora que queremos elegir en forma ordenada y sin repetición 3 elementos entre 5. Entonces para la primera elección tenemos 5 posibilidades, para la segunda 4 posibilidades y para la tercera 3 posibilidades haciendo un total de 
$$
5 \times 4 \times 3
$$
selecciones posibles. 

Se puede demostrar que si $n < m$, no hay ninguna selección ordenada y  sin repetición de $m$ elementos  de un conjunto de $n$ elementos (lo cual se ve muy bien intuitivamente: si hay más personas que asientos, ¡alguien se quedará parado!). Este hecho es llamado el {\em principio de las casillas}\index{principio de las casillas} en la literatura.

\begin{proposicion}\label{prop1}
Si $n \ge m$ entonces existen
\begin{equation}\label{ordsinrep}
 n \cdot (n - 1) \cdots (n - (m - 1)) \quad \text{($m$ factores)}
\end{equation}
selecciones ordenadas y sin repetición de $m$ elementos de un conjunto de $n$ elementos.
\end{proposicion}
\begin{proof} La prueba es una generalización del razonamiento aplicado más arriba a los ejemplos: debemos seleccionar $m$-veces elementos de un conjunto que tiene $n$ elementos. La primera selección puede ser de cualquiera de los $n$ objetos; la segunda selección debe recaer en uno de los $n-1$ elementos restantes. De manera similar, hay $n-2$ posibilidades para la tercera selección, y así sucesivamente. Cuando hacemos la $m$-ésima selección, $m-1$ elementos ya han sido seleccionados, y entonces el elemento seleccionado debe ser uno de los $n-(m-1)$ elementos restantes. Por consiguiente el número total de selecciones es el propuesto. 
\end{proof}

\begin{observacion}
El resultado anterior en particular nos dice que existen
$$
n \cdot (n - 1) \cdots (n - (n - 1)) = n \cdot (n - 1) \cdots 1 = n!
$$
selecciones ordenadas y sin repetición de  $n$ elementos en un conjunto con $n$ elementos y esta podría ser una motivación natural del factorial.

Las selecciones ordenadas y sin repetición de  $n$ elementos en un conjunto con $n$ elementos se denominan {\em permutaciones}\index{permutación} de grado $n$.

Hay, pues, $n!$ permutaciones de grado $n$.
\end{observacion}

Volviendo al resultado de la proposición \ref{prop1}, por ejemplo hay 
\begin{itemize}
\item $7 \cdot 6 \cdot 5$ selecciones ordenadas y sin repetición de 3 elementos de  $[[1,7]]$,
\item $7 \cdot 6 \cdot 5 \cdot 4 \cdot 3$  selecciones ordenadas y sin repetición de 5 elementos de $[[1,7]]$ y
\item $7!$  selecciones ordenadas y sin repetición de todos los elementos de $[[1,7]]$.
\end{itemize}
\vskip .3cm

Notemos que si $n \ge m$ entonces
$$
n \cdot (n - 1) \cdots (n - (m - 1)) = \frac{n!}{(n - m)!}
$$
pues
$$
n! = n \cdot (n - 1 ) \cdots (n -(m - 1 ) ) \cdot (n -m)!
$$

Por lo tanto  la proposición \ref{prop1} se puede reescribir de 
la siguiente manera:

 \vskip .5cm

\begin{proposicion}Si $n \ge m$ entonces existen
\begin{equation}\label{ordsinrep2}
\frac{n!}{(n - m)!}
\end{equation}
selecciones ordenadas y sin repetición de $m$ elementos de un conjunto de $n$ elementos.
\end{proposicion}

\begin{ejercicio} Simplificar las expresiones siguientes ($n \in \mathbb N$)
\begin{align*}
&\text{a) } \frac{n!}{( n - 2 ) !} \quad\text{ si } 2 \le n&  & \text{b) } \frac{(n + 2)!}{n!}  \\
&\text{c) } \frac{(n + 2)!}{( n - 2 ) !} \quad\text{ si } 2 \le n&  & \text{d) } \frac{n!}{(n-2)! 2!}  \quad \text{ si } 2 \le n\\
&\text{e) } \frac{(n-1)!}{(n + 2)!}& \ &
\end{align*}
\end{ejercicio}

\begin{ejemplo}
Si en un colectivo hay 10 asientos vacíos. ¿En cuántas formas pueden sentarse 7 personas? Se trata de ver cuantas selecciones ordenadas y sin repetición de 7 asientos entre 10. 

Este número es
$$
10 \cdot 9 \cdot 8 \cdot 7 \cdot 6 \cdot 5 \cdot 4 \quad \text{(7 factores).}
$$
\end{ejemplo}

\begin{ejemplo}
¿Cuántas permutaciones pueden formarse con las letras de
{\em silvia}?

Afirmamos que se pueden formar  $\displaystyle{\frac{6!}{2!}}$ palabras usando las letras de {\em silvia}.

Si escribo en lugar de {\em silvia},
$$
\text{\em s i l v i' a}
$$
Es decir si cambio la segunda {\em i } por {\em i'}, todas las letras son distintas, luego hay $6!$ permutaciones, pero cada par de permutaciones del tipo
\begin{align*}
\cdots \text{\em i } \cdots  \text{\em i' }  \cdots \\
\cdots \text{\em i' } \cdots  \text{\em i } \cdots
\end{align*}
coinciden, por lo tanto tengo que dividir por 2 el número total de permutaciones.

Tomemos la palabra
$$
\text{\em ramanathan}
$$
el número total de permutaciones es $\displaystyle{\frac{10!}{ 4!2!}}$.

En efecto, escribiendo el nombre anterior así 
$$
r\;a_1\;m\;a_2\;n_1\;a_3\;t\;h\;a_4\;n_1
$$
el número total de permutaciones es $10!$. Pero permutando las $a_i$ y las $n_i$ sin mover las otras letras obtenemos la misma permutación de {\em ramanathan}.

Como hay $4!$ permutaciones de las letras $a_1$, $a_2$, $a_3$, $a_4$, y $2!$ de $n_1$, $n_2$ el número buscado es 
$$
\frac{10!}{ 4!2!}.
$$

Dejamos a cargo del lector probar que el número total de permutaciones de las letras de {\em arrivederci} es
$$
\frac{11!}{3!  2!  2!}
$$
\end{ejemplo}

\end{section}

\begin{section}{Selecciones sin orden}

Consideremos un conjunto $X$ finito de $n$ elementos. Nos proponemos averiguar cuántos subconjuntos de $m$ elementos hay en $X$.

\begin{ejemplo}
Por ejemplo, sea $X = \{ 1, 2, 3, 4, 5 \}$ y nos interesan los subconjuntos de tres elementos. ¿Cuántos habrá? Una forma de individualizar un subconjunto de tres elementos en $X$, consiste en, primero, seleccionar  ordenadamente 3 elementos de $[[ 1 , 5 ]]$.

Habría, a priori, $5 \cdot 4 \cdot 3$ subconjuntos pues ese es el número de selecciones ordenadas y sin repetición de 3 elementos de $[[ 1 , 5 ]]$.

Pero es claro que algunas de las selecciones ordenadas pueden determinar el mismo subconjunto. En efecto, por ejemplo, cualesquiera de las selecciones
\begin{align*}
1 2 3\\
1 3 2\\
2 1 3\\
2 3 1\\
3 1 2\\
3 2 1
\end{align*}
determina el subconjunto $\{ 1, 2, 3\}$ . Es decir las permutaciones de $\{ 1, 2, 3\}$ determinan el mismo subconjunto.  Y así con cualquier otro subconjunto de tres elementos. Por lo tanto, el número total de subconjuntos de 3 elementos debe ser
$$
\frac{5 \cdot 4 \cdot 3}{3!} =  \frac{5!}{3! (5 - 3)!}
$$
\end{ejemplo}

\vskip .3cm

En el caso general de subconjuntos de $m$ elementos de un conjunto de $n$ elementos ($m \le n$) podemos razonar en forma análoga. Cada subconjunto de $m$ elementos está determinado por una selección ordenada y todas las permutaciones de esta selección.

Por lo tanto el número total de subconjunto de $m$ elementos de $X$ es
$$
\frac{n \cdot (n - 1) . . . (n - (m - 1))}{m!} = \frac{n!}{(n - m)!\; m!}
$$

\begin{definicion}
Sean $n, m \in \mathbb N_0$, $m \le n$. Definimos
$$
\binom{n}{m} = \frac{n!}{(n - m)! \; m!}
$$
y por razones que se verán más adelante se denomina el {\em coeficiente binomial}\index{coeficiente binomial} o {\em número combinatorio}\index{número combinatorio} asociado al par $n$, $m$ con $m \le n$.

Definimos también
$$
\binom{n}{m} = 0 \text{ si } m > n.
$$
\end{definicion}

\begin{observacion} Hay unos pocos números combinatorios que son fácilmente calculables: 
$$
\binom{n}{0} = \binom{0}{0} = 1 \;\text{ y }\;  \binom{n}{1} = \binom{n}{n-1} = n. 
$$
Estos resultados se obtienen por aplicación directa de la definición (recordar que  $0! =1$). 
\end{observacion}

En resumen, tenemos la siguiente proposición.

\begin{proposicion}
Sean $n, m \in \mathbb N_0$, $m \le n$, y supongamos que el conjunto $X$ tiene $n$ elementos. Entonces la cantidad de subconjuntos de $X$ con $m$ elementos es
$
\displaystyle\binom{n}{m}
$.
\end{proposicion}

Como vimos anteriormente el número combinatorio suele resultar de utilidad para resolver problemas de conteo. Veamos un ejemplo.

\begin{ejemplo}
 ¿Cuántos comités pueden formarse de un conjunto de 6 mujeres y 4 hombres, si el comité debe estar compuesto por 4 mujeres y 2 hombres?

{\sc Solución.} Debemos elegir 4 mujeres entre 6, y la cantidad de elecciones posibles es   $\binom{6}{4}$. Por otro lado, hay $ \binom{4}{2}$ formas de elegir 2 hombres entre 4. Luego, por el principio de multiplicación,  el resultado es
$$
\binom{6}{3}\binom{4}{2} = \frac{6!}{2!4!}\frac{4!}{2!2!} = \frac{6\times 5}{2}\times\frac{4\times 3}{2} = 15 \times 6 = 90.
$$
\end{ejemplo}

\begin{proposicion}[Simetría del número combinatorio]\label{simcomb}
Sean $m,n \in \mathbb N_0$, tal que $m \le n$. Entonces
$$
\binom{n}{m} = \binom{n}{n-m}.
$$
\end{proposicion}
\begin{proof}
$$
\binom{n}{n-m} = \frac{n!}{(n-(n-m))!\,(n-m)!} =  \frac{n!}{m!\,(n-m)!} =   \frac{n!}{(n-m)!\,m!} = \binom{n}{m}.
$$
\end{proof}

\begin{nota}
El hecho  de que 
$$
\binom{n}{m} = \binom{n}{n-m}.
$$
se puede interpretar en términos de subconjuntos:  $\displaystyle\binom{n}{m}$ es el número de subconjuntos de $m$ elementos de un conjunto de $n$ elementos. Puesto que con cada subconjunto de $m$ elementos hay unívocamente asociado un subconjunto de $n - m$ elementos, su complemento en $X$, es claro que $\displaystyle\binom{n}{m} = \binom{n}{n-m}$.
\end{nota}

\vskip .3cm

\begin{teorema}[Fórmula del triángulo de Pascal] \label{propcomb}
Sean $m,n \in \mathbb N$, tal que $m \le n$. Entonces
$$
\binom{n+1}{m} = \binom{n}{m-1} + \binom{n}{m}  
$$
\end{teorema}
\begin{proof}
El enunciado nos dice que debemos demostrar que 
\begin{equation*}
\frac{(n+1)!}{(n-m+1)!\,m!} = \frac{n!}{(n-m+1)!\,(m-1)!} +  \frac{n!}{(n-m)!\,m!}
\end{equation*}
Hay varias forma de operar algebraicamente las expresiones y obtener el resultado. Nosotros partiremos de la expresión de la derecha y obtendremos la de la izquierda:
\begin{align*}
\frac{n!}{(n-m+1)!\,(m-1)!} +  \frac{n!}{(n-m)!\,m!} &= \frac{n!}{(n-m)!\,(m-1)!}\left(\frac{1}{(n-m+1)} +  \frac{1}{m}\right)\\
&= \frac{n!}{(n-m)!\,(m-1)!}\left(\frac{m+n-m+1}{(n-m+1)m} \right) \\
& = \frac{n!}{(n-m)!\,(m-1)!}\left(\frac{n+1}{(n-m+1)m} \right) \\
& = \frac{n!(n+1)}{(n-m)!(n-m+1)\,(m-1)!\,m}\\
& = \frac{(n+1)!}{(n-m+1)!\,m!}.
\end{align*}

\end{proof}

Aunque por razones de conteo es obvio que los números combinatorios son números naturales, esto no es claro por la definición formal.  
\begin{corolario}
Si $n \in \mathbb N$ y $ 0\le m \le n$ entonces $\displaystyle\binom{n}{m} \in \mathbb N$.
\end{corolario}
\begin{proof}
Haremos inducción en $n$. Si $n = 1$ los posibles números combinatorios son
$$
\binom{1}{1}  = \binom{1}{0}  = 1 \in \mathbb N.
$$

Ahora supongamos que el resultado sea cierto  para $n \in \mathbb N$. Es decir, $\binom{n}{m} \in \mathbb N$ cualquiera sea $m$ tal que $0 \le m \le n$ (hipótesis inductiva). Probaremos entonces que $\binom{n+1}{m} \in \mathbb N$ cualquiera sea $m$ tal que $0 \le m \le n+1$ 

Si $m=0$, $\binom{n+1}{m} = \binom{n+1}{0} =1 \in \mathbb N$.

Si $1 \le m \le n$, entonces por el teorema anterior (en la segunda parte)
$$
\binom{n+1}{m} = \binom{n}{m-1} + \binom{n}{m}  
$$
Como 
$\binom{n}{m-1}$  y $\binom{n}{m}$ pertenecen  a $\mathbb  N$ por la hipótesis inductiva, su suma es también un número natural, o sea 
$
\binom{n+1}{m}   \in \mathbb N.
$

Si $m = n+1$, entonces $\binom{n+1}{n+1} = 1  \in \mathbb N$.

Por lo anterior, se concluye que
$$
\binom{n+1}{m}   \in \mathbb N
$$
cualquiera sea $m$ , $0 \le m \le n + 1$.

Por lo tanto, es válido el paso inductivo y así nuestra afirmación queda probada.
\end{proof}

El teorema precedente permite calcular los coeficientes binomiales inductivamente. Escribamos en forma de triángulo

\begin{align*}
&& && && && && &\binom{0}{0}& && && && && &&  \\
&& && && && &\binom{1}{0}& && &\binom{1}{1}& && && && &&  \\
&& && && &\binom{2}{0}& && &\binom{2}{1}& && &\binom{2}{2}& && && &&  \\
&& && &\binom{3}{0}& && &\binom{3}{1}& && &\binom{3}{2}& && &\binom{3}{3}& && &&  \\
&& &\binom{4}{0}& && &\binom{4}{1}& && &\binom{4}{2}& && &\binom{4}{3}& && &\binom{4}{4}& &&  \\
&\cdot& && &\cdot& && &\cdot& && &\cdot& && &\cdot& && &\cdot& 
\end{align*}

En virtud del teorema \ref{propcomb} cada término interior es suma de los dos términos inmediatos superiores. Los elementos en los lados valen 1 por lo tanto se puede calcular cualquiera
de ellos.
\begin{align*}
&& && && && && &1& && && && && &&  \\
&& && && && &1& && &1& && && && &&  \\
&& && && &1& && &2& && &1& && && &&  \\
&& && &1& && &3& && &3& && &1& && &&  \\
&& &1& && &4& && &6& && &4& && &1& &&  \\
&\cdot& && &\cdot& && &\cdot& && &\cdot& && &\cdot& && &\cdot& 
\end{align*}
(Lector: calcule el valor de la suma total de cada fila del triángulo.)

El  triángulo es denominado \emph{triángulo de Pascal}. Entre las propiedades que cuenta el triángulo de Pascal está la de ser simétrico respecto de su altura, como consecuencia de la simetría de los números combinatorios (ver proposición \ref{simcomb}). 

\end{section}

\begin{section}{El teorema del binomio}

En álgebra elemental aprendemos las formulas
$$
(a+b)^2 = a^2 +2ab +b^2, \qquad (a+b)^3 = a^3 + 3 a^2b +3ab^2 +
b^3,
$$
y a veces nos piden desarrollar la formula para $(a+b)^4$ y potencias mayores de $a+b$. El resultado general que da una formula para $(a+b)^n$ es conocido como el  {\it {teorema del binomio}}.  \index{Teorema del binomio}

\begin{teorema}\label{t3.6} Sea $n$ un entero positivo. El coeficiente del termino $a^{n-r}b^r$ en el desarrollo de $(a+b)^n$ es el número binomial $\binom{n}{r}$. Explícitamente, tenemos
$$
(a+b)^n= \binom{n}{0} a^n + \binom{n}{1} a^{n-1}b+ \binom{n}{2}
a^{n-2}b^2 + \cdots + \binom{n}{n} b^n.
$$
\end{teorema}
\begin{proof}(Primera) Considerar que ocurre cuando multiplicamos $n$ factores
$$
(a+b)(a+b) \cdots (a+b).
$$
Un término en el producto se obtiene seleccionando o bien $a$ o bien $ b$ de cada factor. El número de términos $a^{n-r}b^r$ es solo el número de formas de seleccionar $r$ $b$'s (y consecuentemente $n-r$ a's), y por definición éste es el número binomial $\binom{n}{r}$.
\end{proof}

\begin{observacion}\label{cvar} Antes de hacer una segunda demostración del teorema del binomio veamos el siguiente resultado que nos resultará útil: sea $a_k,a_{k+1},\ldots,a_{m-1},a_m$ una sucesión de números reales ($k \le m$) y sea $r \in \mathbb N_0$.  Entonces
$$
\sum_{i=k}^m a_i = \sum_{i=r}^{m-k+r} a_{i+k-r}.
$$ 
La sumatoria de la derecha es la de la izquierda con un ``cambio de variable'' en el índice. La demostración de este hecho se puede hacer por inducción sobre $m$ (caso base $m=k$) o simplemente escribiendo ambas sumatorias con la notación de puntos suspensivos y verificando que ambas son iguales a
$$
a_k+a_{k+1}+\cdots+a_{m-1}+a_m.
$$  
\end{observacion}

\begin{proof}(Segunda) Se hace por inducción en $n$. Si $n=1$, el resultado es trivial. Supongamos que el resultado es cierto para $n-1$, es decir
$$
(a+b)^{n-1}=\sum_{i=0}^{n-1} \binom{n-1}{i}a ^{n-1-i}b^{i}.
$$
Luego
\begin{alignat*}2
(a+b)^n&= (a+b)(a+b)^{n-1}&& \\
& = (a+b)\{\sum_{i=0}^{n-1} \binom{n-1}{i}a ^{n-1-i}b^{i}\}&&\qquad \text{por hip. inductiva} \\
&=\sum_{i=0}^{n-1} \binom{n-1}{i}a ^{n-i}b^{i}+\sum_{i=0}^{n-1} \binom{n-1}{i}a ^{n-1-i}b^{i+1}&& \qquad \text{propiedad distributiva}\\
&=\sum_{i=0}^{n-1} \binom{n-1}{i}a ^{n-i}b^{i}+\sum_{i=1}^{n} \binom{n-1}{i-1}a ^{n-i}b^{i}&& \qquad \text{ver observación \ref{cvar}}\\
&= a^n + \sum_{i=1}^{n-1}\{ \binom{n-1}{i}+\binom{n-1}{i-1}\}a ^{n-i}b^{i}+ b^n && \qquad \text{agrupar por potencias iguales}\\
&= a^n + \sum_{i=1}^{n-1} \binom{n}{i}a ^{n-i}b^{i}+ b^n&&\qquad \text{por teorema 3.4.1} \\
&= \sum_{i=0}^{n} \binom{n}{i}a ^{n-i}b^{i}.&&\text{}
\end{alignat*}
\end{proof}

Los coeficientes en el desarrollo pueden por lo tanto ser calculados con el método recursivo usado para los números binomiales (triángulo de Pascal) o usando la formula. Por ejemplo,
$$
\begin{aligned} (a+b)^6 &= \binom{6}{0} a^6 + \binom{6}{1} a^5 b
+\binom{6}{2}a^4b^2 +
\binom{6}{3}a^3b^3 \\
&\quad + \binom{6}{4}a^2b^4+\binom{6}{5}ab^5\binom{6}{6}b^6 \\
&=  a^6 + 6 a^5 b +15a^4b^2 + 20a^3b^3 + 15a^2b^4+6ab^5+ b^6.
\end{aligned}
$$
Por supuesto, podemos obtener otras formulas útiles si reemplazamos $a$ y $b$ por otras expresiones. Algunos ejemplos típicos son:
$$\begin{aligned}
(1+x)^4 &= 1 + 4x + 6x^2+ 4x^3+ x^4;\\
(1-x)^7 &= 1 -7x + 21x^2- 35x^3+ 35x^4- 21x^5+ 7x^6 -x^7;\\
(x+2y)^5 &= x^5 + 10 x^4 y + 40 x^3 y^2+80 x^2 y^3+80 x y^4+32 y^5; \\
(x^2+y)^4 &= x^8 +4 x^3 y +6 x^4 y^2 +4 x^2 y^3 + y^4.
\end{aligned}
$$

La expresión $a+b$ es conocida como un expresión {\it binómica} porque tiene dos términos. Como los números  \index{expresión binómica} $\binom{n}{r}$ aparecen como los coeficientes en el desarrollo de $(a+b)^n$, generalmente se los llama, como ya fue dicho,  coeficientes binomiales.   De todos modos esta claro por la prueba del teorema \ref{t3.6} que estos números aparecen en este contexto porque representan el número de formas de hacer ciertas selecciones. Por esta razón continuaremos usando el nombre de números binomiales,  que se aproxima más al concepto que simbolizan.

Además de ser extremadamente útil en manipulaciones algebraicas, el teorema del binomio puede usarse para deducir identidades en que estén involucrados los números binomiales.

\begin{ejemplo}Probar que
\begin{equation}\label{eqcuaddo}
\binom{n}{0}^2+\binom{n}{1}^2+\binom{n}{3}^2+\cdots+\binom{n}{n}^2=
\binom{2n}{n}.
\end{equation}
\end{ejemplo}
\begin{proof}
Usamos la igualdad
\begin{equation*}\label{eqxn}
(1+x)^n(1+x)^n=(1+x)^{2n}.
\end{equation*}
El resultado se demostrará encontrando el coeficiente del termino $x^n$ de ambos términos de esta igualdad.

De acuerdo con el teorema del binomio el miembro izquierdo es el producto de dos factores, ambos iguales a
$$
\binom{n}{0}1+\binom{n}{1}x+\cdots+\binom{n}{r}x^r+\cdots+\binom{n}{n}x^n.
$$
 Cuando los dos factores se multiplican, un termino en $x^n$ se obtiene tomando un termino del primer factor de tipo $x^i$ y un termino del segundo factor de tipo $x^{n-i}$. Por lo tanto los coeficientes de $x^n$ en el producto son
\begin{equation}\label{igcuad}
\binom{n}{0}\binom{n}{n}+\binom{n}{1}\binom{n}{n-1}+\binom{n}{2}\binom{n}{n-2}+\cdots
+\binom{n}{n}\binom{n}{0}.
\end{equation}
Como $\binom{n}{n-r}=\binom{n}{r}$, vemos que (\ref{igcuad}) es el lado izquierdo de la igualdad (\ref{eqcuaddo}). Pero el lado derecho es $\binom{2n}{n}$ que es también el coeficiente de $x^n$ en el desarrollo de $(1+x)^{2n}$, y entonces obtenemos la igualdad que buscábamos.
\end{proof}

\begin{subsection}{Ejercicios}\label{ej3.6.1}
\begin{enumerate}
\item Desarrollar las fórmulas de $(1+x)^8$ y $(1-x)^8$.
\item Calcular los coeficientes de
%Corregir: poner enumerate
(i) $x^5$ en $(1+x)^{11}$;

(ii) $a^2b^8 $ en $(a+b)^{10}$;

(iii) $a^6 b^6$ en $(a^2 +b^3)^5$;

(iv) $x^3$ en $(3+4x)^6$.

\item Usar la identidad $(1+x)^m(1+x)^n=(1+x)^{m+n}$ para probar que
$$
\binom{m+n}{r} =
\binom{m}{0}\binom{n}{r}+\binom{m}{1}\binom{n}{r-1}+\cdots
\binom{m}{r}\binom{n}{0}
$$
donde $m,n$ y $r$ son enteros positivos y, $m\ge r$, y $n \ge r$.
\end{enumerate}
\end{subsection}

\end{section}

\begin{section}{Ejercicios}

\begin{enumerate}
\item Una ficha del juego de dominó puede ser representada con el símbolo $[x|y]$, donde $x$ e $y$ son miembros del conjunto $\{0,1,2,3,4,5,6\}$. Los números $x$ e $y$ pueden ser iguales. Explique por que el número total de fichas de dominó es 28 y no 49.
\item ?`De cuántas formas se pueden elegir un casillero negro y uno blanco en un tablero de ajedrez de tal forma que los dos casilleros no estén ni en la misma fila ni en la misma columna?
\item Supongamos que en clase hay $m$ mujeres y $n$ varones. ?`De cuántas formas puedo ordenar a todos los alumnos en una hilera de manera que todas las mujeres queden juntas?
\item Supongamos que tenemos un dominó generalizado en que las fichas toman su par de valores entre $0$ y $n$. Sea $k$ un entero en el rango $0\le k \le n$. Probar que el número de fichas $[x|y]$ en que $x+y = n-k$ es igual al número de fichas en que  $x+y = n+k$.
\item Denotemos $u_n$ la cantidad de palabras de longitud $n$ en el alfabeto $\{0,1\}$ que tienen la propiedad de no tener dos ceros consecutivos. Probar que
$$
u_1=2,\qquad u_2=3,\qquad u_n =u_{n-1} +u_{n-2}\quad (n \ge 3).
$$
\item Desarrollar $(x+y)^9$ y $(x-y)^9$.
\item Calcular el coeficiente de
$$
\begin{aligned}
\text{(i)}\quad & x^6 \text{ en } (1+x)^{12},\\
\text{(ii)}\quad & a^3 b^7 \text{ en } (a+b)^{10},\\
\text{(iii)}\quad & a^4 b^6 \text{ en } (a^2+b)^8.
\end{aligned}
$$
\item Probar que
$$
\binom{n}{r}\binom{r}{k}=\binom{n}{k}\binom{n-k}{r-k}.
$$
\item Se dibujan todas las posibles diagonales que conectan a un conjunto de $n$ puntos en el círculo y se observa que no hay tres rectas que se cruzan en un mismo punto ?`Cuántos puntos internos de intersección hay?
\item Probar que el número de formas de distribuir $n$ bolas idénticas en $m$ cajas con etiquetas, algunas de las cuales puede quedar vacía es
$$
\binom{n+m-1}{n}.
$$
\item Probar que si $n \ge m$, entonces
$$
\binom{m}{m}+\binom{m+1}{m}+\cdots+\binom{n}{m}=\binom{n+1}{m+1}.
$$
\item  Sea $X$ un $n$-conjunto. Probar que

(i) existe un conjunto de $\binom{n-1}{k-1}$ $k$-subconjuntos de $X$ tales que cada par de ellos tiene intersección no vacía.

(ii) existe un conjunto de $\binom{n}{n^*}$ subconjuntos de $X$ con la propiedad que ninguno contiene a otro. Aquí $n^*$ es igual a $n/2$ si $n$ es par y a $\frac12(n-1)$ si $n$ es impar.
\end{enumerate}

\end{section}

\chapter[Divisibilidad]{Divisibilidad}

\begin{section}{Cociente y resto}\label{1.5}
Cuando somos chicos aprendemos que 6 ``cabe'' cuatro veces en 27 y el resto es 3, o sea
$$
27=6 \times 4 + 3.
$$
Un punto importante es que el resto debe ser menor que 6. Aunque, también es verdadero que, por ejemplo
$$
27=6 \times 3 + 9,
$$
debemos tomar el menor valor para el resto, de forma que ``lo que queda'' sea un número no negativo lo más chico posible. El hecho de que el conjunto de posibles ``restos'' tenga un mínimo es una consecuencia del axioma del buen orden.

\begin{teorema}\label{t1.5} Sean $a$ y $b$ números enteros cualesquiera con $b \in \mathbb N$, entonces existen enteros únicos $q$ y $r$ tales que
$$
a=b \times q + r\qquad\text{ y }\qquad 0\le r<b.
$$
\end{teorema}
\begin{proof} Debemos aplicar el axioma del buen orden al conjunto de los ``restos''
$$ R=\{x \in \mathbb N_0 | a = by + x \ \text{ para algún }\ y \in \mathbb Z\}.
$$
Primero demostraremos que $R$ no es vacío. Si $a\ge 0$ la igualdad
$$
a= b0 + a
$$
demuestra que $a \in R$, mientras que si $a<0$ la igualdad
$$
a= ba + (1-b)a
$$
demuestra que $(1-b)a\in R$ (en ambos casos es necesario controlar que el  elemento es no negativo.)

Ahora, como $R$ es un subconjunto no vacío de $\mathbb N_0$, tiene un mínimo $r$, y como $r$ esta en $R$ se sigue que $a=bq+r$ para algún $q$ en $\mathbb Z$. Además
$$
a=bq+r \Rightarrow a=b(q+1)+(r-b)
$$
de manera que si $r\ge b$ entonces $r-b$ esta en $R$. Pero $r-b$ es menor que $r$, contradiciendo la definición de $r$ como el menor elemento de $R$. Como la suposición $r \ge b$ nos lleva a una contradicción, solo puede ocurrir que $ r<b$, como queríamos demostrar.

\vskip .3cm

Es fácil ver que el cociente $q$ y el resto $r$ obtenidos en el teorema son únicos. Supongamos que $q'$ y $ r'$, también satisfacen las condiciones, esto es
$$
a=bq'+r' \qquad \text{ y } \qquad 0\le r' < b.
$$
Si $q>q'$, entonces $q-q' \ge 1$ y tenemos que
$$
r'=a-bq' = (a-bq)+b(q-q') \ge r+b.
$$
Como $r+b \ge b$, se sigue que $r'\ge b$ contradiciendo la segunda propiedad de $r'$. Por lo tanto la suposición $q'>q$ es falsa. El mismo argumento con $q$ y $q'$ intercambiados demuestra que $q<q'$ también es falsa. Entonces debemos tener $q=q'$, y en consecuencia $ r=r'$, puesto que
$$
r=a-bq = a-bq'=r'.
$$
\end{proof}

\begin{corolario}\label{c1.5} Sean $a$ y $b$ números enteros cualesquiera con $b< 0$, entonces existen enteros únicos $q$ y $r$ tales que
	$$
	a=b \times q + r\qquad\text{ y }\qquad 0\le r<|b|.
	$$
\end{corolario}
\begin{proof}
	$-b = |b|$ es positivo, luego por el teorema anterior,
	\begin{equation*}
		a=(-b) \times q + r\qquad\text{ y }\qquad 0\le r<|b|,
	\end{equation*}  
	por lo tanto 
	\begin{equation*}
	a=b \times (-q) + r\qquad\text{ y }\qquad 0\le r<|b|.
	\end{equation*}  
\end{proof}

Los enunciados del teorema \ref{t1.5} y del corolario \ref{c1.5} son conocidos como el \textit{algoritmo de división}.

\begin{ejemplo} {} 
\vskip .2cm
${}^{}$
\begin{itemize}
\item Si $a=13$ y $b=3$, entonces $13 = 3 \times 4 +1$. Es decir $q= 4$, $r=1$. 
\item Si $a=2$ y $b=5$, entonces $2 = 5 \times 0 +2$. Es decir $q= 0$, $r=2$. 
\item Si $a=-13$ y $b=3$, entonces $-13 = 3 \times (-5) +2$. Es decir $q= -5$, $r=2$. En algunos viejos compiladores del lenguaje $C$, la división entera estaba mal definida, pues consideraban, por ejemplo, $-13 = 3 \times (-4) -1$. Es decir, si el número a ser dividido era negativo, tomaban el resto también como un número negativo, lo cual no está de acuerdo al teorema \ref{t1.5}.  
\item Si $a=-2$ y $b=3$, entonces $-2 = 3 \times (-1) +1$. Es decir $q= -1$, $r=1$. 
\item Si $a=13$ y $b=-3$, entonces $13 = 3 \times 4 +1 = (-3) \times (-4) +1$. Es decir $q= -4$, $r=1$. 
\end{itemize}
\end{ejemplo}

\begin{subsection}{Desarrollos en base $b$, ($b \ge 2$)} Una consecuencia importante del teorema \ref{t1.5} es que justifica nuestro método usual de representación de enteros. 

\begin{ejemplo} Deseamos escribir el número 407 con una expresión de la forma 
$$
407 = r_n5^n +r_{n-1} 5^{n-1}+\cdots + r_1 5 + r_0,
$$
con $0 \le r_i < 5$. Veamos que esto es posible y se puede hacer de forma algorítmica. La forma de hacerlo  es, primero, dividir el número original y los sucesivos cocientes por 5:  
\begin{alignat}2
407 &=5\cdot 81 &+& 2 \label{b3}\\
81 & = 5\cdot 16 &+& 1  \label{b4}\\
16 & = 5\cdot 3 &+& 1  \label{b5}\\
3 & = 5\cdot 0 &+& 3.
\end{alignat}
Observar entonces que
\begin{alignat*}2
407 &=5\cdot 81 + 2  &\quad& \text{por (\ref{b3})}\\
 &= 5\cdot (5\cdot 16 + 1) + 2  &\quad& \text{por (\ref{b4})}\\
 &= 5^2 \cdot 16+ 5\cdot 1 + 2 &\quad& \text{}\\
 &= 5^2 \cdot (5\cdot 3 + 1)+ 5\cdot 1 + 2   &\quad& \text{por (\ref{b5})}\\
 &= 5^3 \cdot 3+ 5^2 \cdot 1 + 5\cdot 1 + 2.  &\quad& \text{}
\end{alignat*}
En este caso diremos que el desarrollo en base 5 de 407 es 3112 o, resumidamente, $407 = (3112)_5$.  Observar que el desarrollo en base 5 de 407 viene dado por los restos de las divisiones sucesiva, leídos en forma ascendente.
\end{ejemplo}

Sea $b \ge 2$ un número entero, llamado {\em base}\index{base (de un sistema de numeración)} para los cálculos. Para cualquier entero positivo $x$ tenemos, por la aplicación repetida del teorema \ref{t1.5},
\begin{alignat*}2
x&=bq_0 &+& r_0 \\
q_0 & = bq_1 &+&r_1 \\
\cdots & \\
q_{n-2} & = bq_{n-1} &+&r_{n-1} \\
q_{n-1} & = bq_n &+&r_n.
\end{alignat*}
Aquí cada resto es uno de los enteros $0, 1,\ldots,b-1$, y paramos cuando $q_n=0$. Reemplazando sucesivamente los cocientes $q_i$, como lo hicimos en el ejemplo, obtenemos
$$
x=r_nb^n +r_{n-1} b^{n-1}+\cdots + r_1 b + r_0.
$$
Hemos representado $x$ (con respecto a la base $b$) por la secuencia de los restos, y escribimos $x=(r_nr_{n-1}\dots r_1 r_0)_b$. Convencionalmente $b=10$ es la base para los cálculos hechos ``a mano'' y omitimos ponerle el subíndice, entonces tenemos la notación usual
$$
1984=(1\cdot 10^3 ) + (9\cdot 10^2 )+(8\cdot 10 ) + 4.
$$
Esta notación posicional requiere símbolos solo para los enteros $0, 1,\ldots,b-1$ .La base $b=2$ es particularmente adaptable para los cálculos en computadoras porque los símbolos 0 y 1 pueden representarse físicamente por la ausencia o presencia de un pulso de electricidad o luz. 

\begin{ejemplo} ?`Cuál es la representación en base 2 de $(109)_{10}$?
\end{ejemplo}
\begin{proof}[Solución] Dividiendo repetidamente por 2 obtenemos
$$
\begin{aligned}
109&=2\cdot 54+1\\ 54&=2\cdot 27+0\\ 27&=2\cdot 13+1\\ 13&=2\cdot 6+1\\
6&=2\cdot 3+0 \\ 3&=2\cdot 1+1 \\1&=2\cdot 0+1
\end{aligned}
$$
Por lo tanto
$$
(109)_{10} = (1101101)_2.
$$

La base 16 también es usada en computación pues se utiliza el byte como unidad básica de memoria y debido a que un byte puede tomar $2^8$ posibles valores, tenemos que $$2^8 = 2^4 \cdot 2^4 = 16 \cdot 16 = 1 \cdot 16^2 + 0 \cdot 16^1 + 0 \cdot 16^0.$$ Luego   un byte puede representar $(100)_{16}$ valores. Más allá de la justificación, es claro que los dígitos disponibles (del 0  al 9) no nos alcanzan  para representar un  número en base 16, pues se requieren 16 símbolos. La convención usada es ${\tt A}=10, {\tt B}=11, {\tt C} =12, {\tt D} = 13, {\tt E} = 14, {\tt F} = 15$.

\begin{ejemplo} Representar  12488 en  base 16.
\begin{alignat*}2
12488 &= 16 \cdot 780 &+&  8\\
780 & = 16 \cdot 48 &+& 12\\
48 & = 16\cdot 3 &+& 0\\
3 & = 16 \cdot 0  &+& 3.
\end{alignat*}
Luego $12488 = (30{\tt C}8)_{16}$.
\end{ejemplo}

\end{proof}

\end{subsection}
\begin{subsection}{Ejercicios}
\begin{enumerate}
\item Encuentre $q$ y $r$ que satisfagan el teorema \ref{t1.5} cuando
$$
\text{(i)}\quad a=1001,\,\,\, b=11; \qquad \text{(ii)}\quad
a=12345,\,\,\, b=234.
$$
\item Encuentre las representaciones de $(1985)_{10}$ en base 2, en base 5 y en base 11.
\item
 Encuentre las representación usual (base 10) de
$$
\text{(i)} \quad (11011101)_2; \qquad \text{(ii)} (4165)_7.
$$
\end{enumerate}
\end{subsection}
\end{section}

\begin{section}{Divisibilidad}\label{1.6}

\begin{definicion}Dados dos enteros $x$ e $y$ decimos que $y$ es un {\em divisor}\index{divisor} de $x$, y escribimos $y|x$, si
$$
x=yq\quad\text{ para algún }\quad q\in \mathbb Z.
$$
También decimos que $y$ es un {\em factor} de $x$, que $y$ {\em divide}\index{divide} a $x$, que $x$ es {\em divisible} por $y$, y que $x$ es {\em múltiplo}\index{múltiplo} de $y$.
\end{definicion}

Cuando $y|x$ podemos usar el símbolo $\frac{x}{y}$ (o $x/y$) para denotar el entero $q$ tal que $ x=yq$. Cuando $y$ no es un divisor de $x$ tenemos que asignar un nuevo significado a la fracción $x/y$, puesto que este número no es un entero. El lector indudablemente, esta familiarizado con las reglas para manejar fracciones, y usaremos esas reglas de tanto en tanto, pero es importante recordar que las fracciones no han sido aún formalmente definidas en el contexto de este apunte. Y es aún más importante recordar que $x/y$ no es un elemento de $\mathbb Z$ a menos que $y$ divida a $x$\footnote{En algunos lenguajes de programación si $x,y$ son enteros, entonces la operación $x/y$ devuelve el cociente entero $q$. No usaremos esa convención en este apunte.}.

Veamos ahora alguna propiedades básicas de la relación ``divide a''. 

\begin{proposicion}Sean $a,b,c$ enteros, entonces
\begin{enumerate}
\item $1|a$, $a|0$, $a|\pm a$;
\item si $a|b$, entonces $a|bc$ para cualquier $c$;
\item si $a|b$ y $a|c$, entonces $a|b+c$;
\item si $a|b$ y $a|c$, entonces $a|rb+sc$ para cualesquiera $r,s \in \mathbb Z$;
\item si $a',b'$ enteros y $a|a'$, $b|b'$, entonces $ab|a'b'$.
\end{enumerate}
\end{proposicion}
\begin{proof}
	La demostración de estos hechos es sencilla, por ejemplo, demostremos (3): como $a|b$, existe $q$ tal que $b = aq$. Análogamente, como $a|c$, existe $q'$ tal que $c = aq'$. Entonces $b+c = aq+aq' = a(q+q')$, luego $a|b+c$.  

	Demostración de (5): como $a|a'$, $b|b'$, existen $r,s$ enteros tal que $a'=ra$ y $b' =sb$. Por lo tanto $a'b' = (ra)(sb) = (rs)ab$. Esto implica que  $ab|a'b'$.

	Las demás demostraciones se dejan como ejercicio para el lector. 

\end{proof}

\begin{ejemplo} Demuestre que si $c$, $d$ y $n$ son enteros tales
que
$$
d|n \quad\text{ y }\quad c|\frac{n}{d}
$$
entonces
$$
c|n \quad\text{ y }\quad d|\frac{n}{c}.
$$
\end{ejemplo}
\begin{proof} Como $d|n$ existe un entero $s$ tal que $n=ds$, y $n/d$ denota al entero $s$. Puesto que $c|n/d$ existe un entero $t$ tal que
$$
s=\frac{n}{d} =ct.
$$
Se sigue que
$$
n=ds=d(ct)=c(dt)
$$
entonces $c|n$ y $n/c$ denota al entero $dt$. Finalmente, como $n/c=dt$ tenemos $d|n/c$, como queríamos demostrar.
\end{proof}

\begin{proposicion}\label{pm} Sean $a$ y $b$ enteros.
\begin{enumerate}
\item Si  $ab=1$ entonces $a=b=1$ o $a=b=-1$. 
\item Si $x$ e $y$ son enteros tales que $x|y$ e $y|x$, entonces $x=y$ o $x=-y$.
\end{enumerate}
\end{proposicion}
\begin{proof} 1. Si $a$ o $b$ valen 0, entonces $ab=0 \not=1$. Luego $a$ y $b$ son distintos de 0. Si $a>0$ y $b<0$ por los axiomas de compatibilidad del orden con el producto $ab<0$. Lo mismo ocurre si $a<0$ y $b>0$.

Es decir podemos suponer que o bien $a>0$ y $b>0$, o bien $a<0$ y $b<0$. 

Si  $a>0$ y $b>0$, entonces  $a\ge 1$ y $b\ge 1$. Si $a=1$, como $ab =1$, tenemos que $b = 1$. Si $a>1$, como  $b>0$ por compatibilidad de $<$ con el producto tenemos que $ab>1$, lo cual no es cierto. Es decir, hemos probado que si  $a>0$ y $b>0$, entonces $a=1$ y $b=1$.

Si  $a<0$ y $b<0$, entonces   $-a>0$ y $-b>0$ y $(-a)(-b) = ab =1$. Luego, por el párrafo de arriba, $-a=-b=1$ y en consecuencia $a=b=-1$.

\vskip 0.1cm

2. Sean $x,y$ tales que  $x|y$ e $y|x$. Como $x|y$, existe $q \in \mathbb Z$ tal que $y = qx$. Análogamente, como $y|x$ existe $q'$ tal que $x = q'y$. Luego
$$
y = qx = q(q'y) = (qq')y.
$$
Por el axioma de cancelación (cancelando $y$) obtenemos que $1 = qq'$. Por lo demostrado más arriba tenemos que, o bien $q=q'=1$ y en consecuencia $x=y$, o bien $q=q'=-1$ y en consecuencia $x=-y$. 
\end{proof}

\begin{subsection}{Ejercicios}
\begin{enumerate}
\item Use el principio de inducción para demostrar que, para todo $n\ge0$,
$$
\text{ (i) } \quad n^2+3n \,\,\text{ es divisible por 2 } \qquad
\text{ (ii) } \quad n^3+3n^2+2n \,\,\text{ es divisible por 6. }
$$
\end{enumerate}
\end{subsection}
\end{section}

\begin{section}{El máximo común divisor y el mínimo común múltiplo}\label{1.7}

\begin{definicion}\label{mcd} Si $a$ y $b$ son enteros algunos de ellos no nulo, decimos que un entero no negativo $d$ es un {\em máximo común divisor}\index{máximo común divisor}, o {\em mcd}, de $a$ y $b$ si
\begin{enumerate}
\item[({\em i})] $ d|a$ y $d|b$;
\item[({\em ii})] si $ c|a $ y $c|b$ entonces $ c|d$.
\end{enumerate}
\end{definicion}
La condición ({\em i}) nos dice que $d$ es un común divisor de $a$ y $b$ y la condición ({\em ii}) nos dice que cualquier divisor común de $a$ y $b$ es también divisor de $d$. Por ejemplo, 6 es un divisor común de 60 y 84, pero no es el mayor divisor común, porque $12|60$ y $12|84$ pero $12{\not|}6$ (el símbolo significa ``no divide''.)

\begin{ejemplo} La definición no afirma que el máximo común divisor existe, pero podemos encontrarlo fácilmente para un par de números $a$, $b$ simplemente por inspección. Por  ejemplo, si $a = 174$ y $b =72$;

Divisores de 174: 1, 2, 3, 6, 29, 58, 87, 174

Divisores de 72: 1, 2, 3, 4, 6, 8, 9, 12, 18, 24, 36, 72 

Luego, $6$ es divisor común de 174 y 72, y todos los demás divisores comunes ($1$, $2$ y $3$) dividen a $6$.
\end{ejemplo}

\begin{teorema}\label{tmcd} 
Sean $a,b$ enteros con alguno de ellos no nulo. Si el máximo común divisor entre  $a$ y $b$ existe, entonces es único.  
\end{teorema}
\begin{proof}
Sean $d$ y $d'$ dos enteros no negativos que satisfacen las propiedades de la definición del máximo común divisor. Es decir
\begin{align}
&\text{({\em i}) $ d|a$ y $d|b$;\quad ({\em ii}) si $ c|a $ y $c|b$ entonces $ c|d$} \label{mcd1}\\
&\text{({\em i}) $ d'|a$ y $d'|b$;\quad ({\em ii}) si $ c|a $ y $c|b$ entonces $ c|d'$} \label{mcd2}
\end{align}
Ahora bien, $d$ satisface la propiedad (\ref{mcd1})({\em i}), aplicando la propiedad (\ref{mcd2})({\em ii}) obtenemos que $d|d'$. Análogamente,  $d'$ satisface la propiedad (\ref{mcd2})({\em i}),  luego aplicando la propiedad (\ref{mcd1})({\em ii}) obtenemos que $d'|d$. Por lo tanto $d, d' \ge 0$ y  se dividen mutuamente y en consecuencia son iguales (ver ejemplo \ref{pm}).
\end{proof}

Cuando  el máximo común divisor entre $a$ y $b$ existe lo denotaremos $\mcd(a,b)$.

Podemos enunciar las propiedades más sencillas del mcd en la siguiente proposición

\begin{proposicion} Sean $a,b$ enteros con $a \not = 0$, entonces
\begin{enumerate}
\item Si $\mcd(a,b)$ existe, entonces $\mcd(b,a) = \mcd(a,b) = mcd(\pm a, \pm b)$,
\item si $a>0$,  $\mcd(a,0) = a$ y $\mcd(a,a) = a$,
\item $\mcd(1,b) = 1$.
\end{enumerate}
\end{proposicion}
\begin{proof}
Estas propiedades son de demostración casi trivial, por ejemplo para demostrar que  $\mcd(1,b) = 1$ comprobamos que 1 cumple con la definición de $\mcd(1,b)$:
\begin{enumerate}
\item[({\em i})] $ 1|1$ y $1|b$;
\item[({\em ii})] si $ c|1 $ y $c|b$ entonces $ c|1$,
\end{enumerate}
propiedades que son obviamente verdaderas.

1. y 2.  se dejan a cargo del lector. 
\end{proof}

La siguiente propiedad no es tan obvia y resulta muy importante. 
 
\begin{propiedad}\label{propiedad1}
Probar que si $a \not=0, b \in \mathbb Z$ y existe $\mcd(a,b-a)$, entonces existe $\mcd(a,b)$ y $\mcd(a,b) = \mcd(a,b-a)$. 
\end{propiedad}
\begin{proof}
Sea $d =  \mcd(a,b-a)$, luego 
\begin{enumerate}
\item[({\em i})] $ d|a$ y $d|b -a$;
\item[({\em ii})] si $ c|a $ y $c|b -a$ entonces $ c|d$.
\end{enumerate}
Ahora bien, como  $ d|a$ y $d|b -a$, entonces $  d|a +(b -a) = b$. Es decir, para recalcar,
 
\noindent {({\em *}) $ d|a$ y $d|b$.} 

\noindent Por otro lado, si  $ c|a $ y $c|b$, entonces  $c|b -a$, luego por ({\em ii}) tenemos que $c|d$. Es decir, 

\noindent{({\em **}) si  $ c|a $ y $c|b$, entonces  $c|d$.}

\noindent Luego por ({\em *}), ({\em **}) y la definición de mcd obtenemos que $d = \mcd(a,b)$.
\end{proof}

La propiedad anterior nos provee un método práctico para encontrar el máximo común divisor entre dos números, como vemos en el siguiente ejemplo.

\begin{ejemplo} Encontrar el mcd entre 72 y 174.
\begin{proof}[Solución] Observar que en caso de existir el mcd para todo número, tenemos
\begin{align*}
\mcd(72, 174) &= \mcd(72,174-72) = \mcd(72,102) = \mcd(72,30) =  \mcd(42, 30) \\&= \mcd(12,30) = \mcd(12,18)= \mcd(12,6)= \mcd(6,6) = 6.  
\end{align*}
Ahora bien, como sabemos que el $\mcd(6,6)=6$ existe, por la propiedad (\ref{propiedad1}), obtenemos que $\mcd(12,6) $ existe y es $6$, lo que implica que 
$\mcd(12,18)$ existe y es $6$, y así sucesivamente hasta probar que $\mcd(72, 174)$ existe y es $6$.
\end{proof}
\end{ejemplo} 

Existe un famoso método para calcular el mcd de dos enteros no negativos $a,b$ con $b \not=0$, basado en la técnica del cociente y el resto. Depende del siguiente hecho.

\begin{proposicion}\label{prop-alg-eucl} Sean  $a,b$ enteros no negativos con $b \not=0$, entonces 
\begin{equation}\label{bec}
a=bq+r\quad \Rightarrow \quad\mcd(a,b)=\mcd(b,r).
\end{equation}
Más precisamente, si $a = bq + r$, entonces el $\mcd(a,b)$ existe si y sólo si  $\mcd(b,r)$ existe y ambos números son iguales.
\end{proposicion}
\begin{proof}
Para demostrar esto debemos observar que si $c$ divide $a$ y $b$, entonces también divide a $a-bq$; y como $a-bq=r$, tenemos que $c|r$. De este modo cualquier divisor común de $a$ y $b$ es también divisor común de $b$ y $r$.  Por otro lado si $c$ divide $b$ y $r$ también divide a $a=bq+r$. Es decir, 
\begin{equation}\label{eqmcd1}
\text{$c$ es divisor común de $a$ y $b$ si y sólo si $c$ es divisor común de $b$ y $r$.} \tag{*}
\end{equation}
Luego, supongamos que  exista $d = mcd(a,b)$, probaremos que $d$ es el mcd entre $b$ y $r$:
\begin{enumerate}
\item[({\em i})] como $d|a$ y $d|b$, entonces por (*) tenemos que $ d|b$ y $d|r$;
\item[({\em ii})] si $c|a $ y $c|r$ entonces por (*)  $c|a$ y $c|b$ y debido a que $d = \mcd(a,b)$, se deduce que $c|d$.
\end{enumerate}
Al satisfacer ({\em i}) y ({\em ii}), obtenemos que $d = mcd(b,r)$.

Análogamente, se demuestra que si existe $d = mcd(b,r)$, entonces  $d = mcd(a,b)$.
\end{proof}

La aplicación repetida de este simple hecho, en combinación con el algoritmo de división, nos da un método para calcular el mcd.

\begin{ejemplo} Encuentre el mcd de 2406 y 654.
\end{ejemplo}
\begin{proof}[Solución] Tenemos
\begin{alignat*}3
\mcd(2406,654)&=\mcd(654,444)& &\text{ porque }& 2406&=654\times3+444,\\
               &=\mcd(444,210)& &\text{ porque }& 654&=444\times1+210,\\
               &=\mcd(210,24)&& \text{ porque } &444&=210\times2+24,\\
               &=\mcd(24,18) && \text{ porque } &210&=24\times8+18,\\
               &=\mcd(18,6)  && \text{ porque } &24 &=18\times1+6,\\
               & =\mcd(6,0) = 6           &&\text{ porque }&18&=6\times3 + 0
\end{alignat*}

\end{proof}

Este ejemplo es un caso particular o una aplicación del algoritmo que nos permite calcular el máximo común divisor.

\vskip .3cm

\begin{center}

\fbox{\begin{minipage}{35em} 
\vskip .3cm

{\flushleft \textbf{Algoritmo de Euclides}}

\vskip .2cm

Por lo general, para calcular el mcd de enteros $a$ y $b$, con $b >0$, 
 definimos $q_i$ y $r_i$ recursivamente  de la siguiente manera: $r_0 = a$, $r_1 = b$,  y 
\begin{align*}
&\text{($e_{1}$)}\qquad& r_0&=r_1 q_1 + r_2& &(0 < r_2<r_1)\\
&\text{($e_{2}$)}\qquad& r_1&=r_2q_2 + r_3\quad{}\quad{}\quad{}& &(0 < r_3<r_2)  \\
&\text{($e_{3}$)}\qquad& r_2&=r_3q_3 + r_4\quad{}\quad{}\quad{}& &(0 < r_4<r_3)  \\
&\cdots&&\\
&\cdots&& \\
&\text{($e_{k-2}$)}\qquad& r_{k-3}&=r_{k-2}q_{k-2} + r_{k-1}& &(0 < r_{k-1} <r_{k-2}) \\
&\text{($e_{k-1}$)}\qquad& r_{k-2}&=r_{k-1}q_{k-1} + r_{k}& &(0 < r_{k} <r_{k-1}) \\
&\text{($e_{k}$)}\qquad& r_{k-1}&=r_{k}q_{k} + 0 ,&&  
\end{align*}

\vskip .3cm

\end{minipage}}

\end{center}

\vskip .2cm 

El proceso se detiene cuando uno de los restos $r_i$  es igual a $0$ y queda claro que el proceso debe detenerse, porque cada resto no nulo es positivo y estrictamente menor que el anterior.

Este procedimiento es conocido como el {\em algoritmo de Euclides}\index{algoritmo de Euclides}, debido al matemático griego Euclides (300 a.~c.). Es extremadamente útil en la práctica, y tiene importantes consecuencias.

\begin{teorema} Sean  $a$ y $b$ enteros con $b >0$, entonces el máximo común divisor existe y es el $r_{k}$ obtenido en el algoritmo de Euclides. 
\end{teorema}
\begin{proof}
Observar que aplicando repetidas veces (\ref{bec}) de proposición  \ref{prop-alg-eucl} obte\-nemos:
\begin{multline*}
r_k = \mcd(r_{k},0) = \mcd(r_{k-1},r_k) =\mcd(r_{k-2},r_{k-1}) = \cdots\\\cdots 
=  \mcd(r_2,r_3) =  \mcd(r_1,r_2)  =  \mcd(r_0,r_1) = \mcd(a,b)  
\end{multline*}
\end{proof}

\begin{observacion}[*] El algoritmo de Euclides es fácilmente implementable en un lenguaje de programación. A continuación una versión del mismo en pseudocódigo. 

\vskip .5cm

\begin{minipage}{200pt}
\noindent {\sc Algoritmo de Euclides }
\vskip .2cm
\begin{small}
\begin{verbatim}
# pre: a y b son números positivos
# post: Obtenemos d = mcd(a,b)
r(0), r(1) = a, b
i = 1 
while r(i) != 0:
    r(i+1) = r(i-1) % r(i)   # a % b = resto de a / b
    i = i + 1
d = r(i-1)
\end{verbatim}
\end{small}
\end{minipage}

\end{observacion}

\vskip .5cm

\begin{teorema}\label{t1.7.1} Sean $a$ y $b$ enteros, $b$ no nulo y sea $d=\mcd(a,b)$. Entonces existen enteros $s$ y $t$ tales que
$$
d=sa+tb.
$$
\end{teorema}
\begin{proof} En el caso que $b >0$, de acuerdo con el cálculo hecho antes $d=r_{k}$ y usando la  ecuación ($e_{k-1}$) tenemos
$$
r_{k}=r_{k-2} -r_{k-1}q_{k-1}.
$$
As{í}, $d$ puede escribirse en la forma $ d = s_{k}r_{k-2} +t_{k}r_{k-1}$,donde $s_{k}=1$ y $t_{k}=-q_{k-1}$ . Usando la ecuación  ($e_{k-2}$), sustituyendo $r_{k-1}$ en términos de $r_{k-3}$ y $r_{k-2}$ obtenemos
$$
d= s_{k}(r_{k-3}-r_{k-2}q_{k-2}) + t_{k}r_{k-3} =  s_{k-1}r_{k-3} +t_{k-1}r_{k-2}
$$
donde $s_{k-1} = s_{k} + t_{k} $ y $t_{k-1}= -s_{k}q_{k-2}$.  Aplicando  repetidas veces las ecuciones del algoritmo de Euclides obtenemos, en general que 
$$
d =  s_{i}r_{i-2} +t_{i}r_{i-1}
$$
con  $s_{i}, t_{i} \in \mathbb Z$, para $2 \le i \le k$. En particular 
$$
d =  s_{2}r_{0} +t_{2}r_{1} = s_{2}a +t_{2}b.
$$
\end{proof}

Por el ejemplo, de los cálculos usados para encontrar el mcd de 2406 y 654 obtenemos
\begin{alignat*}9 6&=&&&&&&&\mathbf{24}-\mathbf{18}&\times1&&=
&1&\times&\mathbf{24}&+&(-1) &\times \mathbf{18 }\\
&=&&&\mathbf{24}&+&(-1)&\times&(\mathbf{210}-\mathbf{24}&\times8)&&=
&(-1)&\times&\mathbf{210}&+&9&\times \mathbf{24}\\
&=&&&\mathbf{-210}&+&9&\times&(\mathbf{444}-\mathbf{210}&\times2)&&=
&9&\times&\mathbf{444}&+&(-19) &\times \mathbf{210}\\
&=&9&\times&\mathbf{444}&+&(-19)&\times&(\mathbf{654}-\mathbf{444}&\times1)&&=
&(-19)&\times&\mathbf{654}&+&28 &\times \mathbf{444}\\
&=&(-19)&\times&\mathbf{654}&+&28&\times&(\mathbf{2406}-\mathbf{654}&\times3)&&=
&28&\times&\mathbf{2406}&+&(-103) &\times \mathbf{654}.
\end{alignat*}
De este modo, la expresión requerida $d=sa+tb$ es
$$
6=28\times2406+(-103)\times 654.
$$

\begin{corolario} Sean $a$ y $b$ enteros, $b$ no nulo, entonces
$$
\mcd(a,b) = 1 \Leftrightarrow \text{existen $s,t \in \mathbb Z$ tales que 1 = sa+tb.}
$$
\end{corolario}
\begin{proof}
($\Rightarrow$) Es consecuencia trivial de la proposición anterior.

($\Leftarrow$) Sea $d = \mcd(a,b)$, entonces $d|a$ y $d|b$ y por lo tanto $d|sa+tb$ para cualesquiera   $s,t \in \mathbb Z$.  En particular, la hipótesis implica que $d|1$ y, en consecuencia $d =1$. 
\end{proof}

\begin{definicion}
Si el $\mcd(a,b)=1$ entonces decimos que $a$ y $ b$ son {\em coprimos}\index{coprimos}.
\end{definicion}

El resultado  anterior  es una herramienta de suma utilidad para trabajar en problemas relacionados con el mcd. Por ejemplo, todos estamos familiarizados con la idea de que una fracción puede reducirse al ``mínimo término'', o sea a la forma $a/b$ con $a$ y $b$ coprimos. El siguiente ejemplo establece que esta forma es única, y como veremos, el hecho clave de la demostración es que podemos expresar a 1 como $sa+tb$.

\begin{ejemplo} Supongamos que $a$, $a'$, $b$, $b'$ son enteros positivos que satisfacen
$$
\text{({\em i}) }\quad ab'=a'b; \qquad\quad\text{(ii) }
\quad\mcd(a,b)=\mcd(a',b')=1.
$$
Entonces $a=a'$ y $b=b' $.

(La condición ({\em i}) podría escribirse como $a/b=a'/b'$, pero preferimos usar esta forma que no asume ningún conocimiento sobre fracciones.)
\end{ejemplo}
\begin{proof} Como el $\mcd (a,b) =1$ existen enteros $s$ y $t$ tales que $sa+tb=1$. En consecuencia
$$
b'=(sa+tb)b' =sab'+tbb' = sa'b + tbb'=(sa'+tb')b,
$$
y por lo tanto $b|b'$. Por un argumento similar y usando el hecho de que el $\mcd(a',b')=1 $ deducimos que $b|b'$, por lo tanto $b=b'$ o $b=-b'$ y como $b $ y $b'$ son ambos positivos debemos tener $b=b'$ . Ahora de (i) deducimos que $a=a'$ y el resultado esta demostrado.
\end{proof}

\begin{observacion}[*]
El  algoritmo explicado anteriormente para obtener el $\mcd(a,b)$ como combinación lineal entera $sa+tb$ no es muy sencillo de programar. Más aún, requiere terminar el cálculo del $\mcd$, usando el algoritmo de Euclides, para comenzar a calcular los coeficientes enteros $s,t$. 

Ahora bien, estos coeficientes pueden ser calculados, mientras se aplica el algoritmo de Euclides, de la siguiente manera:  si $a,b$ enteros, $b>0$, llamemos $r_0=a$, $r_1=b$. Los primeros pasos del algoritmo de Euclides son
\begin{align}
r_{0} &= r_{1} q_{1} + r_{2}  \tag{\emph{e1}} \\ 
r_{1} &= r_{2} q_{2} + r_{3}  \tag{\emph{e2}}
%r_{i-1} = r_{i} q_{i} + r_{i+1}  \tag{\emph{i}}
\end{align}
Observemos que de la primera ecuación se deduce  $r_{2} = r_{0} - r_{1} q_{1}  = a + b(-q_1)$. De la segunda ecuación obtenemos $r_{3} = r_{1} - r_{2} q_{2}  = b -  r_{2} q_{2}$ y reemplazando por el valor de $r_2$ antes obtenido concluimos que $r_3 = a(-q_2) +b(1+q_1q_2)$, por lo tanto hemos obtenido que
\begin{align}
r_{2} &= as_2 + bt_2  \tag{\emph{r2}} \\ 
r_{3} &= as_3 + bt_3   \tag{\emph{r3}}
%r_{i-1} = r_{i} q_{i} + r_{i+1}  \tag{\emph{i}}
\end{align}	   
con ciertos valores de $s_i,t_i$ bien determinados. Por consiguiente, logramos escribir los primeros restos que se obtienen con  el algoritmo de Euclides como combinación lineal entera de  $a$ y $b$. En el siguiente paso tenemos
\begin{align}
r_{2} &= r_{3} q + r_{4},  \tag{\emph{e3}}
%r_{i-1} = r_{i} q_{i} + r_{i+1}  \tag{\emph{i}}
\end{align}
por lo tanto  
\begin{align*}
r_{4} &= r_2 + r_{3}(- q)&\text{(por (\emph{e3}))}& \\
      &= as_2 + bt_2 + (as_3 + bt_3)(-q) &\text{(por (\emph{r3}) y (\emph{r4}))}&  \\
			&= a(s_2-qs_3) + b(t_2-qt_3).  
\end{align*}
Es  decir hemos obtenido
\begin{align}
r_{4} &= as_4 + bt_4  \tag{\emph{r4}} 
\end{align}
con 
\begin{equation*}
s_4 = s_2-qs_3, \quad t_4 = t_2-qt_3. 
\end{equation*}

\vskip .2cm

Procedemos en forma análoga para los siguiente restos, es decir a partir de que conocemos la escritura de  $r_{i-1}$ y $r_i$ como combinación lineal entera de $a$ y $b$:
\begin{align*}
r_{i-1} &= as_{i-1} + bt_{i-1} \\ 
r_{i}   &= as_i + bt_i  
%r_{i-1} = r_{i} q_{i} + r_{i+1}  \tag{\emph{i}}
\end{align*}
calculamos el cociente y el resto de dividir $r_{i-1}$ por $r_{i}$: 
\begin{align*}
r_{i-1} &= r_{i} q + r_{i+1},  
%r_{i-1} = r_{i} q_{i} + r_{i+1}  \tag{\emph{i}}
\end{align*}	  
y obtenemos
\begin{align*}
r_{i+1} &= as_{i+1} + bt_{i+1}
\end{align*}
donde 
\begin{equation*}
s_{i+1} = s_{i-1}-qs_i, \quad t_{i+1} = t_{i-1}-qt_i. 
\end{equation*}

Como  $ r_i  >r_{i+1}\ge  0$, existe un $k$ tal  $r_k \not = 0$ y  $r_{k+1} = 0$, entonces $r_k = \mcd(a,b)$ y  si $s = s_k$ y $t = t_k$, obtenemos que $\mcd(a,b) = sa+tb$.  

Simplificando un poco el algoritmo en el inicio, podemos escribirlo con el siguiente pseudocódigo. 

\vskip .5cm

\begin{minipage}{200pt}
\noindent {\sc Algoritmo de Euclides 2}
\vskip .3cm
\begin{small}
\begin{verbatim}
# pre: a y b son números positivos
# post: Obtenemos s y t tal que mcd(a,b) = a*s +b*t
r(0), r(1) = a, b
s(0), t(0) = 1, 0  # r(0) = a*s(0) + b*t(0)
s(1), t(1) = 0, 1  # r(1) = a*s(1) + b*t(1)
i = 1 
while r(i) != 0:
    q, r(i+1) = r(i-1) / r(i), r(i-1) % r(i)
    # r(i-1) = r(i)*q + r(i+1),  0 <= r(i+1) < r(i)
    s(i+1) = s(i-1) + s(i)*q
    t(i+1) = t(i-1) + t(i)*q
    # r(i+1) = a*s(i+1) + b*t(i+1)
    i = i+1
s, t  = s(i), t(i)
\end{verbatim}
\end{small}
\end{minipage}

\end{observacion}

\vskip .5cm

\begin{ejemplo}[*]
Calcular $\mcd(87,72)$ y escribirlo como combinación lineal entera de $87$ y $72$.
\begin{proof}[Solución]
Los dos primeros pasos del algoritmo de Euclides son
\begin{align*}
87 &= 72 \times 1 + 15   \\ 
72 &= 15 \times 4 + 12.  
\end{align*}
Luego $15 = 87 \times 1 + 72 \times (-1)$ y $12 = 72 \times 1 + 15 \times (-4) = 87 \times(-4) + 72 \times 5$. Es decir
\begin{align}
15 &= 87 \times 1 + 72 \times (-1) \label{alg1}   \\ 
12 &= 87 \times(-4) + 72 \times 5. \label{alg2}
\end{align}
Ahora bien
$$
15 = 12 \times 1 + 3.
$$ 
Luego
$$
3 = 15 \times 1 + 12 \times (-1)
$$
y, usando las ecuaciones (\ref{alg1}) y (\ref{alg2}), obtenemos:
\begin{align}
3 &= 87 \times 5 + 72 \times (-6). \label{alg3}
\end{align}
Como $3|12$, resulta que $3 = \mcd(87,72)$ y la ecuación (\ref{alg3}) nos da la escritura del mcd  como combinación lineal entera de $87$ y $72$.
\end{proof}
\end{ejemplo}

\vskip .3cm

{\flushleft \textbf{Mínimo común múltiplo}}

\begin{definicion}
Si $a$ y $b$ son enteros decimos que un entero no negativo $m$ es el {\em mínimo común múltiplo}\index{mínimo común múltiplo}, o {\em mcm}, de $a$ y $b$ si
\begin{enumerate}
\item[({\em i})] $ a|m$ y $b|m$;
\item[({\em ii})]  si $ a|n $ y $b|n$ entonces $ m|n$.
\end{enumerate}
\end{definicion}
La condición ({\em i}) nos dice que $m$ es múltiplo común de $a$ y $b$, la condición ({\em ii}) nos dice que es mínimo. Por ejemplo hallemos el mínimo común múltiplo entre 8 y 14. Escribamos los múltiplos de ambos números y busquemos el menor común a ambos. Los primeros múltiplos de 8 son: $8,16,24,32,40,48,56,\ldots$. Los primeros múltiplos de 14 son: $14,28,42,56,72,\ldots$. Luego se tiene $\mcm(8,14)=56$.

El siguiente teorema garantiza la existencia del mcm.

\begin{teorema}\label{t1.7.2} Sean $a$ y $b$ enteros no nulos, entonces
$$
\mcm(a,b)=\frac{a b}{\mcd(a,b)}.
$$
\end{teorema}
\begin{proof} Demostraremos que
$$
m=\frac{a b}{\mcd(a,b)}
$$
es el mínimo común múltiplo de $a$, $b$.

Como
$$
m=\frac{a b}{\mcd(a,b)}=\frac{a}{\mcd(a,b)} b
=a\frac{b}{\mcd(a,b)}
$$
resulta que $m$ es múltiplo de $a$ y $b$.

Sea ahora $n$ un múltiplo de $a$ y $b$. Por teorema \ref{t1.7.1} existen enteros $r,s$ tales que 
\begin{equation}\label{clmcd}
\mcd(a,b)=ra+sb
\end{equation} 
y por lo tanto,
dividiendo la ecuación (\ref{clmcd}) por $\mcd(a,b)$ y multiplicando por $n$, obtenemos la siguiente ecuación:
\begin{equation}\label{clmcd2}
n= r\frac{a}{\mcd(a,b)}n + s\frac{b}{\mcd(a,b)}n.
\end{equation} 
Escribiendo $n=b'b=a'a$ ($a',b'$ en $\mathbb Z$) y haciendo los reemplazos en (\ref{clmcd2}), resulta finalmente
\begin{equation}\label{clmcd3}
n= rb'\frac{a b}{\mcd(a,b)}+sa'\frac{a b}{\mcd(a,b)}= \frac{a b}{\mcd(a,b)}(rb'+sa')
\end{equation}
lo cual demuestra que $m$ divide a $n$.
\end{proof}

En particular este resultado implica que si $a$ y $b$ son enteros coprimos, entonces $\mcm(a,b)=ab$.

\begin{subsection}{Ejercicios}
\begin{enumerate}
\item Encontrar el $\mcd$ de 721 y 448 y expresarlo en la forma
$721m+448n$ con $m,n \in \mathbb Z$.
\item Demostar que si $a$, $b$ y $n$ son enteros no nulos, entonces
$\mcd(na,nb)=n\mcd(a,b)$.
\item\label{imp} Demostrar que si existen enteros $m$ y $n$ tales que $mu+nv=1$, entonces el
$\mcd(u,v)=1$.
\item Usar el teorema \ref{t1.7.1} y el Ej. \ref{imp} para demostrar que si el
$\mcd(a,b)=d$ , entonces
$$
\mcd\left(\frac{a}{d},\frac{b}{d}\right) =1.
$$
\item  Sean $a$ y $b$ enteros positivos y sea $d=\mcd(a,b)$. Probar que existen enteros $ x$ e $y$ que satisfacen la ecuación $ax+by=c$ si y solo si $d|c$.
\item  Encontrar enteros $x$ e $y$ que satisfagan
$$
966x+685y=70.
$$
\end{enumerate}
\end{subsection}

\end{section}

\begin{section}{Factorización en primos}

\begin{definicion} Se dice que un entero positivo $p$ es {\em primo}\index{número primo} si $p\ge 2$ y los únicos enteros positivos que dividen $p$ son 1 y $p$ mismo.
\end{definicion}

Luego un entero $m\ge 2$ no es un primo si  y sólo si puede escribirse como $m=m_1m_2$ donde $m_1$ y $m_2$ son enteros estrictamente entre 1 y $m$.

Enfaticemos que de acuerdo a la definición, 1 {\it no} es primo. Los primeros primos (los  menores que 100) son
$$
2,\ 3,\ 5,\ 7,\ 11,\ 13,\ 17,\ 19,\ 23,\ 29,\ 31,\ 37,\ 41,\ 43,\ 47,\ 53,\ 59,\ 61,\ 67,\ 71,\ 73,\ 79,\ 83,\ 89,\ 97.
$$

\begin{observacion}[Criba de Eratóstenes *] Un forma de encontrar números primos es con la \emph{criba de Eratóstenes.} Es un algoritmo que permite hallar todos los números primos menores que un número natural dado $n$. Se forma una lista con todos los números naturales comprendidos entre $2$ y $n$, y se van tachando los números que no son primos de la siguiente manera: comenzando por el 2, se tachan todos sus múltiplos; comenzando de nuevo, cuando se encuentra un número entero que no ha sido tachado, ese número es declarado primo, y se procede a tachar todos sus múltiplos y así sucesivamente. El proceso termina cuando alcanzamos $n$. 

Podemos expresar el algoritmo en pseudocódigo:

\vskip .5cm

\begin{minipage}{200pt}
\noindent {\sc Criba de Eratóstenes}
\vskip .2cm 
\begin{small}
\begin{verbatim}
# pre: n número natural
# post: se obtiene ``primos'' la lista de números primos hasta n
primos = lista vacía
inter = lista de 2 a n  # en inter se tacharán los números compuestos 
for i = 2 to n:
    if i no está tachado en inter:
        agregar i a primos
        k = 2
        while k * i <= n:
            tachar k * i en la lista inter
            k = k +1
\end{verbatim}
\end{small}
\end{minipage}

\end{observacion} 

\vskip .5 cm

El lector debe estar casi totalmente familiarizado con la idea de que cualquier entero positivo puede expresarse como producto de primos: por ejemplo
$$
825=3\times5\times5\times11.
$$
La existencia de esta factorización en primos para cualquier entero positivo es una consecuencia del axioma del buen orden.

\begin{teorema}
Todo  entero  mayor que 1 es producto de números primos. 
\end{teorema}                                                
\begin{proof} Sea $B$ el conjunto de enteros positivos que no tienen una factorización en primos.

Si $B$ no es vacío entonces, por el axioma del buen orden, tiene un mínimo $m$. Si $m$ fuera un primo $p$ entonces tendríamos la factorización trivial $m=p$; por lo tanto $m$ no es primo y existen $m_1,m_2$ enteros positivos con  $1<m_1<m$ y $1<m_2< m$ tal que $m=m_1m_2$.

Como estamos suponiendo que $m$ es el menor entero ($\ge 2$) que no tiene factorización en primos, entonces $m_1$ y $m_2$ tienen factorización en primos. Pero entonces la ecuación $m=m_1m_2$ produce una factorización en primos de $m$, contradiciendo la suposición de que $m$ era un elemento de $B$. Por lo tanto $B$ debe ser vacío, y la afirmación esta probada.
\end{proof}

\begin{ejemplo} 
Encontremos la factorización en números primos de $201\text{.}000$. Esto se hace dividiendo  sucesivamente los números hasta llegar a factores primos:
\begin{align*}
201\text{.}000 &= 201\times 1000 = 3\times 67\times 10\times 10\times 10 =  3\times 67\times 2\times 5 \times 2\times 5 \times 2\times 5 \\&= 2^3\times 3\times 5^3\times 67.
\end{align*}
Como vimos más arriba 2, 3, 5 y 67 son  números primos y por lo tanto hemos obtenido la descomposición prima de $201\text{.}000$.
\end{ejemplo}

Veamos ahora alguna propiedades básicas de los números primos.
\begin{observacion} \label{pdivpp}
Sea $a \in \mathbb Z$ y $p$ primo. Entonces 
\begin{enumerate}
\item  Si $p{\not|}a$, $\mcd(a,p) = 1$.
\item  Si $p$ y $p'$ son primos y $p|p'$ entonces $p=p'$.
\end{enumerate}
\end{observacion}
\begin{proof}1. Como los únicos divisores de $p$ son $p$ y 1, y $p{\not|}a$, el único  divisor común de $p$ y $a$ es 1.

2.  $p'$ es primo, por lo tanto tiene sólo dos divisores positivos $1$ y $p'$. Como $p$ no es 1, tenemos que  $p=p'$.
\end{proof}

Para encontrar la descomposición prima de un número, digamos $n$, debemos ir tomando todos los números menores a $n$ y comprobando si estos lo dividen o no. En lo que sigue veremos el criterio de la raíz, que se utiliza para comprobar si un número es primo en menos pasos que la comprobación directa.  

\begin{lema}\label{lema-criterio-raiz} Si $n>0$ no es primo, entonces existe $m>0$ tal que $m|n$ y $m \le \sqrt{n}$.  
\end{lema}
\begin{proof}
Si $n$ no es primo, entonces $n = m_1m_2$ con $1\le m_1,m_2 < n$. Supongamos que $m_1,m_2 > \sqrt n$, entonces $n = m_1m_2 >  \sqrt n\sqrt n = n$, lo cual es una contradicción. Por lo tanto, $m_1$ o $m_2$ debe ser menor o igual que $\sqrt n$ y por consiguiente encontramos un divisor de $n$ menor o igual a  $\sqrt n$. 
\end{proof}

\begin{proposicion}[Criterio de la raíz]\label{craiz}Sea $n\ge 2$. Si para todo $m$ tal que $1<m \le \sqrt{n}$ se cumple que $m{\not|}n$, entonces $n$ es primo.
\end{proposicion}
 \begin{proof}  Supongamos que $n$ no es primo, luego, por el lema anterior, existe $m$ tal que  $m|n$ y $1 < m \le \sqrt n$ y esto contradice nuestras hipótesis. La contradicción se produce al suponer que $n$ no es primo, por lo tanto $n$ es primo. 
\end{proof}

\begin{corolario}\label{coro-criterio-raiz}
	Sea $n\ge 2$. Si para todo primo  $p$ tal que $1<p \le \sqrt{n}$ se cumple que $p{\not|}n$, entonces $n$ es primo.
\end{corolario}
\begin{proof}
	Si $n$ no fuera primo existiría $1 <m \le \sqrt{n}$ tal que $m|n$ (lema \ref{lema-criterio-raiz}). Sea $p$ primo  tal que $p|m$, luego $p \le \sqrt{n}$ y $p|n$, absurdo que vino de suponer que $n$ no es primo.
\end{proof}
Este criterio reduce enormemente la cantidad de pruebas que debemos hacer para verificar si un número es primo.

\begin{ejemplo} Verificar si 467 es primo o no.
\begin{proof}[Solución]
	 Observar primero que si no utilizamos el criterio de la raíz deberíamos hacer 465 divisiones: deberíamos comprobar si $m|467$ con  $1<m <467$. 
	
	Como $\sqrt{467} < 22$, por el criterio de la raíz, sólo debemos comprobar si $m|467$ para $2\le m \le 21$. Un sencilla comprobación (dividiendo) muestra que los números $2,3,\cdots,20,21$ no  dividen a 467 y por  lo tanto 467 es primo.
	
	Usando el corolario \ref{coro-criterio-raiz} y nuestro conocimientos de los primos pequeños,  la economía de cuentas es aún mayor: como $\sqrt{467} < 22$ y 467 no es divisible por 2, 3, 5, 7, 11, 13, 17 y 19, se deduce que 467 es primo. 
\end{proof}
\end{ejemplo}

La facilidad con la que establecemos la existencia de la factorización de primos conlleva dos dificultades importantes. Primero el problema de encontrar los factores primos no es de ningún modo directo; y segundo no es obvio que exista una {\it única} factorización en primos para todo entero dado $n\ge 2$ . El siguiente resultado es un paso clave en la demostración de la unicidad.

\begin{teorema}\label{t1.8} Sea $p$  un número  primo.

\begin{enumerate} 
\item Si $p|xy$ entonces $p|x$ o $p|y$.
\item $x_1,x_2,\ldots,x_n$ son enteros tales que
$$
p|x_1x_2\ldots x_n
$$
entonces $p|x_i$ para algún $x_i$ ($1\le i \le n$).
\end{enumerate}
\end{teorema}
\begin{proof} 1)  Si $p|x$ ya está probado el resultado. Si $p{{\not|}}x$ entonces tenemos $\mcd(x,p)=1$. Por el teorema \ref{t1.7.1} existen enteros $r$ y $s$ tales que $rp+sx=1$. Por lo tanto tenemos
$$
y =(rp+sx)y =(ry)p+s(xy).
$$
Como $p|p$ y $p|xy$, entonces divide a ambos términos y se sigue que $p|y$. 

2) Usemos el principio de inducción. El resultado es obviamente verdadero cuando $n=1$ (base inductiva). 

Ahora, supongamos que el resultado es verdadero cuando $n=k$, es decir si $p|x_1x_2\ldots x_k$, entonces  $p|x_i$ para algún $i$ con $1\le i \le k$ (hipótesis inductiva).

Debemos probar que si $p|x_1x_2\ldots x_{k}x_{k+1}$, entonces  $p|x_i$ para algún $x_i$ ($1\le i \le k+1$).

Supongamos $p|x_1x_2\ldots x_{k}x_{k+1}$ y sea $x=x_1x_2\ldots x_k$. Si $p|x$ entonces, por la hipótesis inductiva, $p|x_i$ para algún $x_i$ en el rango $1\le i \le k$. Si $p{{\not|}}x$ entonces, por 1), se sigue que $p|x_{k+1}$. De este modo, en ambos casos $p$ divide uno de los $x_i$ ($1\le i\le k+1$). 
\end{proof}

Un error común es asumir que el teorema \ref{t1.8} se mantiene verdadero cuando reemplazamos el primo $p$ por un entero arbitrario . Pero esto claramente falso: por ejemplo
$$
6| 3\times 8 \quad \text{ pero } \quad 6{\not|} 3 \quad \text{ y }
\quad 6{\not|}8.
$$
Ejemplos como éste nos ayudan a entender que el teorema \ref{t1.8} expresa una propiedad muy significativa de los números primos. Además veremos que esta propiedad juega un papel crucial en el siguiente resultado, que a veces es llamado el {\it Teorema Fundamental de la Aritmética}.

\begin{teorema}\label{t1.8.2} La factorización en primos de un entero positivo $n\ge 2$ es única, salvo el orden de los factores primos.
\end{teorema}
\begin{proof} Por el axioma del buen orden, si existe un entero para el cual el teorema es falso, entonces hay un entero mínimo $n_0\ge 0$ con esta propiedad. Supongamos entonces que
$$
n_0= p_1p_2\ldots p_k\quad\text{ y }\quad n_0= p'_1p'_2\ldots p'_l,
$$
donde los $p_i$ ($1\le i \le k$) son primos, no necesariamente distintos, y los $p'_i$ ($1\le i \le l$) son primos, no necesariamente distintos. La primera ecuación implica que $p_1|n_0$, y la segunda ecuación implica que $p_1 | p'_1p'_2\ldots p'_l$. Por consiguiente por teorema \ref{t1.8} tenemos que $p_1|p'_j$ para algún $j$ ($1\le j \le l$). Reordenando la segunda factorización podemos asumir que $p_1 | p'_1$, y puesto que $p_1$ y $p'_1$ son primos, se sigue que $p_1=p'_1$ (observación \ref{pdivpp} 3)). Luego por el axioma {\bf I7}, podemos cancelar los factores $p_1$ y $p'_1$, y obtener
$$
p_2p_3 \ldots p_k = p'_2p'_3 \ldots p'_l,
$$
y llamemos a esto $n_1$. Pero supusimos que $n_0$ tenía dos factorizaciones diferentes, y hemos cancelado el mismo número ($p_1=p'_1$) en ambas factorizaciones, luego $n_1$ tiene también dos factorizaciones primas diferentes. Esto contradice la definición de $n_0$ como el mínimo entero sin factorización única. Por lo tanto el teorema es verdadero para $n\ge 2$.
\end{proof}

En la práctica a menudo reunimos los primos iguales en la factorización de $n$ y escribimos
\begin{equation}\label{factpri}
n=p_1^{e_1}p_2^{e_2}\ldots p_r^{e_r},
\end{equation}
donde $p_1,p_2,\ldots ,p_r$ son primos distintos y $e_1,e_2,\ldots,e_r$ son enteros positivos. Por ejemplo $ 7000 = 2^3 \times 5^3 \times 7$.

La factorización prima nos dice que los números primos son los  ``ladrillos'' esenciales para ``construir'' los números enteros usando multiplicaciones. Ahora bien, podría ocurrir que haya un número finitos de ellos y que podamos escribir cada número como producto de primos en forma muy sintética. Pero este no es el caso.

\begin{proposicion} Existen infinitos números primos. 
\end{proposicion} 
\begin{proof} Haremos la demostración por el absurdo: supongamos que existen en total $r$ números primos $p_1,p_2,\ldots, p_r$. Sea $n =  p_1p_2\ldots p_r+1$. Sea $p$ primo tal que $p|n$. Como la lista de primos es exhaustiva, existe $i$ con $1 \le i \le r$ tal que $p=p_i$. Ahora bien $p_i| n$ y $p_i|p_1p_2\ldots p_r$, luego $p_i|n-p_1p_2\ldots p_r =1$, lo cual es un absurdo que vino de suponer que el número de primos es finito.  
\end{proof}

\begin{ejemplo} Probar que si $m$ y $n$ son enteros tales que $m\ge 2$ y $n\ge 2$, entonces $m^2 \not=2n^2$.
\end{ejemplo}
\begin{proof} Supongamos que la factorización prima de $n$ contiene al 2 elevado a la $x$ (donde $x$ es cero si 2 no es factor primo de $n$). Entonces $n=2^xh$, donde $h$ es producto de primos más grandes que 2, luego
$$
2n^2=2(2^xh)^2= 2^{2x+1}h^2.
$$
Por lo tanto 2 está elevado a una potencia {\it impar} en la factorización prima de $2n^2$.

Por otro lado, si $m=2^yg$, donde $g$ es producto de primos mayores que 2, entonces
$$
m^2= (2^yg)^2 = 2^{2y}g^2,
$$
luego 2 está elevado a una potencia {\it par} (posiblemente cero) en la factorización prima de $m^2$. se sigue entonces que de ser $m^2 = 2n^2$ deberíamos tener dos factorizaciones primas diferentes del mismo número entero, contradiciendo al teorema \ref{t1.8.2}. Entonces $m^2 \not= 2n^2$.
\end{proof}

Es claro que la conclusión del {ejemplo}  vale también si nosotros permitimos que alguno de los enteros $m$ o $n$ valga 1. Luego podemos expresar el resultado diciendo que no hay enteros positivos $m$ y $n$ que cumplan
$$
\left(\frac{m}{n}\right)^2 =2
$$
o equivalentemente, diciendo que la raíz cuadrada de $2$ no puede ser expresada como una fracción $m/n$, es decir, hemos probado: 
\begin{center}
	\textit{$\sqrt{2}$ no es un  número racional.}
\end{center}


Una notación conveniente para nuestros propósitos será la siguiente: sean $m$ y $n$ dos enteros positivos, a veces es conveniente escribir la factorización prima de ambos números usando los mismos primos, y los primos que usamos son los que se encuentran en la factorización prima de ambos. Es decir  escribimos
$$
m=p_1^{e_1}p_2^{e_2}\ldots p_r^{e_r},\qquad
n=p_1^{f_1}p_2^{f_2}\ldots p_r^{f_r}.
$$
con $e_i,f_i \ge 0$ para $i=1,\ldots,r$ y $e_i$ o $f_i$ distinto de cero. 
 
\begin{ejercicio}
	Demostrar por inducción que si $a_1,\ldots,a_r$, $b_1,\ldots,b_r$ son números enteros tales que $a_i| b_i$ para $1 \le i \le r$,  entonces   $a_1\ldots a_r|b_1\ldots b_r$. 
\end{ejercicio} 
 
 
Veremos ahora un resultado que se puede deducir fácilmente del Teorema Fundamental de la Aritmética (TFA).

\begin{proposicion} Sean $m,n \ge2$ con
$$
m=p_1^{e_1}p_2^{e_2}\ldots p_r^{e_r},\qquad
n=p_1^{f_1}p_2^{f_2}\ldots p_r^{f_r}.
$$
donde $p_i$ primo y $e_i,f_i \ge 0$ para $i=1,\ldots,r$. 

Entonces $m|n$ si y sólo si $e_i \le f_i$ para todo $i$.
\end{proposicion}
\begin{proof}
($\Rightarrow$) Por la descomposición de $m$ es claro que $p^{e_i}|m$. Como $m|n$ entonces   $p^{e_i}|n$. Es decir $n =  p^{e_i}u$. Es claro por TFA entonces que $e_i \le f_i$.
\vskip .2cm
($\Leftarrow$) Como $e_i \le f_i$, tenemos que $p^{e_i}|p^{f_i}$, para $1 \le i \le r$.  Luego  $$p_1^{e_1}p_2^{e_2}\ldots p_r^{e_r}| p_1^{f_1}p_2^{f_2}\ldots p_r^{f_r}.$$ Es decir $m|n$.
\end{proof}

\begin{ejercicio} Sean $m,n$ enteros con $m,n\ge 2$. Entonces  $m$ y  $n$ son coprimos si y sólo si no comparten ningún primo en la factorización. 

En otras palabras, sean  
$$
m=p_1^{e_1}p_2^{e_2}\ldots p_r^{e_r},\qquad
n=q_1^{f_1}q_2^{f_2}\ldots q_s^{f_s},
$$ 
las descomposiciones primas de $m$ y $n$. Entonces  $\operatorname{mcd}(m,n) =1$ si y sólo si con $p_i \not= q_j$ para todos los $i,j$.  
\end{ejercicio}

Ahora veremos que es es posible calcular el $\mcd$ y el $\mcm$ de un par de números sabiendo sus descomposiciones primas.

\begin{proposicion}\label{prop-mcd}
Sean $m$ y $n$ enteros positivos cuyas factorizaciones primas son
$$
m=p_1^{e_1}p_2^{e_2}\ldots p_r^{e_r},\qquad
n=p_1^{f_1}p_2^{f_2}\ldots p_r^{f_r}.
$$
\begin{enumerate}
\item El mcd de $m$ y $n$ es $d=p_1^{k_1}p_2^{k_2}\ldots p_r^{k_r}$ donde, para cada $i$ en el rango $1\le i \le r$, $k_i$ es el mínimo entre $e_i$ y $f_i$.
\item El mcm de $m$ y $n$ es $u=p_1^{h_1}p_2^{h_2}\ldots p_r^{h_r}$ donde, para cada $i$ en el rango $1\le i \le r$, $h_i$ es el máximo entre $e_i$ y $f_i$.
\end{enumerate}
\end{proposicion}
\begin{proof} 1. Sea $c$ tal que $c|n$ y $c|m$, entonces los primos que intervienen en la factorización de $c$ son $p_1,\ldots,p_r$ y por lo tanto $c =  p_1^{t_1}p_2^{t_2}\ldots p_r^{t_r}$. Además, como $c|n$ y $c|m$ tenemos que $t_i \le e_i,f_i$ y por lo tanto $t_i \le k_i = \min(e_i,f_i)$. De esto se deduce que $c|p_1^{k_1}p_2^{k_2}\ldots p_r^{k_r}=d$. Por otro lado, es claro  que  $p_1^{k_1}p_2^{k_2}\ldots p_r^{k_r}$ divide a $m$ y $n$ y se deduce el resultado.

2. Ejercicio. 
\end{proof}

\begin{observacion} La proposición anterior nos puede llevar a pensar que una forma sencilla de encontrar el mcd y mcm  es usando la descomposición en factores primos de los números involucrados. Esto, en general, no es así para números grandes: no hay un método eficiente para encontrar la descomposición prima de un número grande. Esencialmente, el mejor método para encontrar un divisor de un número grande es el criterio de la raíz, es decir probando si algún número menor que la raíz del número original lo divide. El criterio de la raíz baja el número de comprobaciones de $n$ a $\sqrt{n}$ y eso no ayuda mucho cuando $n$ es grande.

 Ahora bien, ¿qué es un ``número grande''? En la actualidad, por ejemplo, con todos los recursos computacionales de que se disponen no es posible factorizar números de 200 dígitos o más.     
\end{observacion}

\begin{ejemplo}
Encontraremos el mcd y el mcm  de 825 y 385.

Como $825 =  3\times5^2\times11$ y $385 = 5\times 7\times11$, tenemos que
$$
\text{mcd}(825,385) = 5\times11 = 55, \qquad \text{mcm}(825,385) = 3\times5^2\times 7\times11 = 5775.
$$
\end{ejemplo}

\begin{observacion}
	Una propiedad importante, que se deduce fácilmente de la proposición \ref{prop-mcd}, es la siguiente: si  existe un primo $p$ tal que $p|a$ y $p|b$,  entonces $p|\mcd(a,b)$,  en particular $\mcd(a,b) >1$. La contra-recíproca de este resultado es utilizada frecuentemente para probar que dos números son coprimos:  si $a,b$ números enteros tales que no existe primo $p$ tal que $p|a$ y $p|b$,  entonces $\mcd(a,b) =1$.
\end{observacion}

\begin{subsection}{Ejercicios} {\rm (continuación)}
\begin{enumerate}
\item Probar que si $m$ y $n$ son enteros positivos, tales que $m\ge 2$ y $n \ge 2$, y $m^2 = kn^2$, entonces $k$ es el cuadrado de un entero.
\item Use la identidad
$$
2^{rs} -1 = (2^r-1) (2^{(s-1)r}+2^{(s-2)r}+\cdots +2^r+1)
$$
para probar que si $2^n-1$ es primo, entonces $n$ es primo.
\item Encontrar el mínimo $n$ para el cual la recíproca del ejercicio anterior es falsa: esto es, $n$ es primo pero $2^n-1$ no lo es.
\end{enumerate}
\end{subsection}
\end{section}

\begin{section}{Ejercicios}
\begin{enumerate}
\item Probar que $4^{2n}-1$ es divisible por $15$ para todo entero $n\ge 1$.
\item Encontrar el mcd entre 1320 y 714, y expresar el resultado en la forma $1320x+714y$ ($x,y \in \mathbb Z$).
\item Probar que 725 y 441 son coprimos y encontrar enteros $x$ e $y$ tales que $725x+441y=1$.
\item Encontrar una solución con números enteros de la ecuación
$$ 325x+26y=91.$$
\item El entero $f_n$ es definido recursivamente por la ecuación
$$
f_1=1,\qquad f_2=1, \qquad f_{n+1} = f_n + f_{n-1} \quad
\text{($n\ge 2$)}.
$$
Probar que $\mcd(f_{n+1},f_n)=1$ para todo $n\ge2$.
\item Probar que si $\mcd(a,x)=d$ y $\mcd(b,x)=1$, entonces $\mcd(ab,x)=d$.
\item Usted tiene a disposición una cantidad ilimitada de agua, un gran contenedor y dos jarras de 7 y 9 litros respectivamente. ?`Cómo se las arreglaría usted para poner un litro de agua en el contenedor? Explique la relación entre su método y el teorema \ref{t1.7.1}.
\item Siguiendo la definición de mcd de dos enteros, defina el mcd de $n$ enteros $a_1,a_2,\ldots,a_n$. Probar que si $d=\mcd(a_1,a_2,\ldots,a_n)$, entonces existen enteros $x_1,x_2,\ldots,x_n$ tales que
$$
d=x_1a_1+x_2a_2+\cdots+x_na_n.
$$
\item Sea $n$ un entero con las siguientes propiedades: (1) la descomposición prima de $n$ no tiene factores repetidos (es decir $n$ es de cuadrado libre) y (2) si $p$ cualquier primo, entonces $p|n$ si  y sólo si $p-1|n$.

Encuentre el valor de $n$.
\item El entero $u_n$ es definido por las ecuaciones
$$
u_1=2,\qquad u_{n+1}=u_n^2-u_n+1 \,\,\, \text{($n\ge1$)}.
$$
Encontrar el menor valor de $n$ para el cual $u_n$ no es primo y encontrar los factores de este $u_n$. ?`Es $u_6$ primo?.
\item Probar que los enteros definidos en el ejercicio anterior satisfacen
$$
u_{n+1}= 1 + u_1u_2 \ldots u_n.
$$
Deducir que $u_{n+1}$ tiene un factor primo que es diferente de todo factor primo que aparece en la descomposición de los $u_1,u_2,\ldots,u_n$. Con esto probar que el conjunto de primos no tiene máximo.
\item ?`Es 65537 primo?.
\item Probar que no existen enteros $x$, $y$, $z$, $t$ para los cuales valga la relación
$$
x^2+y^2-3z^2-3t^2=0.
$$
\item Probar que si y $xy=z^2$ para algún entero $z$, entonces $x=m^2$ y $y=n^2$ para ciertos enteros $m,n$.
\item Probar que si $\mcd(a,b)=1$, entonces $\mcd(a+b,a-b)$ es 1 o 2.

\item Probar que   $\mcd(mx,my)= m \mcd(x,y)$, para $m$ entero positivo  y $x,y$ números enteros arbitrarios.
\end{enumerate}
\end{section}

\chapter[Aritmética Modular]{Aritmética Modular}

\begin{section}{Congruencias}
Una de las más familiares particiones de un conjunto es la partición de $\mathbb Z$ en enteros pares y enteros impares. Es decir $\mathbb Z$ es la unión disjunta del conjunto de números pares y el de los números impares. Es claro que dos números $a,b$ tienen la misma paridad si $a-b$ es divisible por 2. Para expresar este hecho es usual la notación
$$
a\equiv b \pmod2
$$
y se dice que $a$ es {\it congruente } a $b$ {\it módulo} 2. Es decir $a$ y $b$ son ambos pares o ambos impares si y solo si $a$ es congruente a $b$ módulo 2.

Claramente esta definición se puede extender a cualquier entero positivo $m$.

\begin{definicion} Sean $a$ y $b$ enteros y $m$ un entero positivo. Diremos que $a$ es {\em congruente}\index{congruencia} a $b$ \index{congruencia} {\em módulo} $m$, y escribimos  \index{módulo $m$}
$$
a \equiv b \pmod{m}
$$
si $a-b$ es divisible or $m$ o, equivalentemente, el resto de dividir $a-b$ por $m$ es 0.
\end{definicion}

Observar que $a\equiv 0 \pmod{m}$ si  y sólo si $m|a$ y que $a\equiv b \pmod{m}$ si y sólo si $a-b\equiv 0 \pmod{m}$. 

\begin{proposicion}
Sean $a$ y $b$ enteros y $m$ un entero positivo. Entonces $a\equiv b \pmod{m}$ si  y sólo si $a$ y $b$ tienen el mismo resto en la división por $m$.
\end{proposicion}
\begin{proof}
Si $a=mh+r$ y $b=mk+s$, con $0 \le r,s <m$, podemos suponer, sin perdida de generalidad, que $r \le s$, luego
$$
b-a= m(k-h) + (s-r) \quad \text{con $0\le s - r < m$}.
$$
Se sigue que $s-r$ es el resto de dividir $b-a$ por $m$.

Luego si $a\equiv b \pmod{m}$, el resto de dividir   $b-a$ por $m$ es 0, y por lo tanto $s-r=0$ y $s=r$.

Si $a$ y $b$ tienen el mismo resto en la división por $m$, entonces  $a=mh+r$ y $b=mk+r$, luego $a-b = m(h-k)$ que es divisible por $m$.
\end{proof}

Así como antes podíamos separar $\mathbb Z$ en los números pares e impares, la propiedad anterior nos permite expresar $\mathbb Z$ como una unión disjunta de $m$ subconjuntos. Es decir si 
\begin{equation*}
	\mathbb  Z_{[i]} =\{x \in \mathbb Z:\text{ el resto de dividir $x$ por $m$ es $i$}\},
\end{equation*}
entonces dado $m \in \mathbb N$, 
\begin{equation*}
	\mathbb Z= \mathbb Z_{[0]}\cup \mathbb Z_{[1]}\cup \cdots\cup \mathbb Z_{[m-1]}.
\end{equation*}

Es fácil verificar que la {congruencia módulo} $m$ verifica las  propiedades de una relación de equivalencia:
\begin{enumerate}
\item[(1)] Es {\it {reflexiva}} es decir $x\equiv x\pmod{m}$.
\item[(2)] Es {\it {simétrica}}, es decir si $x \equiv y \pmod{m}$, entonces $y \equiv x \pmod{m}$.
\item[(3)] Es {\it { transitiva}}, es decir si $x\equiv y \pmod{m}$ e $y\equiv z \pmod{m}$, entonces $x\equiv z \pmod{m}$.
\end{enumerate}
La primera propiedad es verdadera debido a que $x-x$ es cero y por lo tanto divisible por $m$. La segunda se debe a que si $x-y=km$, entonces $y-x=(-k)m$. Finalmente, podemos demostrar la tercera de la siguiente forma, puesto que $x-y=km$ y $y-z=lm$, tenemos que $x-z=(x-y)+(y-z)=(k+l)m$.

\vskip .3cm 

La utilidad de las congruencias reside principalmente en el hecho de que son compatibles con las operaciones aritméticas. Específicamente, tenemos el siguiente teorema.

\begin{teorema}\label{t4.1} Sea $m$ un entero positivo y sean $x_1$, $x_2$, $y_1$, $y_2$ enteros tales que
$$
x_1 \equiv x_2 \pmod{m}, \qquad y_1 \equiv y_2 \pmod{m}.
$$
Entonces
\begin{enumerate}
\item[{\em (i)}] $ x_1+ y_1 \equiv x_2+ y_2 \pmod{m}$,
\item[{\em (ii)}] $x_1 y_1 \equiv x_2 y_2 \pmod{m}$,
\item[{\em (iii)}] Si $x \equiv y \pmod{m}$  y $j \in  \mathbb N$, entonces $x^j \equiv y^j \pmod{m}$.
\end{enumerate}
\end{teorema}
\begin{proof} ({\em i}) Por hipótesis tenemos que existen enteros $x,y$ tales que $x_1-x_2=mx$ e $y_1-y_2=my$. Se sigue que
$$
\begin{aligned}
(x_1+y_1)-(x_2+y_2) &= (x_1-x_2)+ (y_1 -y_2) \\
&= mx +my \\
&= m(x+y),
\end{aligned}
$$
y por consiguiente el lado izquierdo es divisible por $m$, como queríamos demostrar.

({\em ii})  Aquí tenemos
$$
\begin{aligned}
x_1y_1-x_2y_2 &=  x_1y_1-x_2y_1+ x_2y_1-x_2y_2 \\
&= (x_1-x_2)y_1+ x_2(y_1 -y_2) \\
&= mxy_1 +x_2my \\
&= m(xy_1+x_2y),
\end{aligned}
$$
y de nuevo el lado izquierdo es divisible por $m$.

({\em iii}) Lo haremos por inducción sobre $j$. 

Es claro que si $j=1$ el resultado es verdadero. Supongamos ahora que el resultado vale para $j-1$, es decir que si  $x \equiv y \pmod{m}$, entonces 
$$
x^{j-1} \equiv y^{j-1} \pmod{m}.
$$
Como $x \equiv y \pmod{m}$,  por   {\em (ii)} tenemos que 
$$
x^{j-1}x \equiv y^{j-1}y  \pmod{m},
$$
es decir 
$$
x^j \equiv y^j \pmod{m}.
$$
\end{proof}

\begin{proposicion}\label{prop412} Sea $(x_nx_{n-1}\ldots x_0)_{10}$ la representación del entero positivo $x$ en base 10, entonces
$$
x \equiv x_0+x_1+\cdots+x_n \pmod{9}
$$
\end{proposicion}
\begin{proof}
 Observemos primero que como $10\equiv 1\pmod{9}$, entonces  $10^k\equiv 1^k \equiv 1\pmod{9}$. Esto es debido  al teorema \ref{t4.1}{\em (iii)} 

Por la definición de representación en base 10, tenemos que 
$$
x=x_0 + 10x_1+ \cdots+10^nx_n,
$$ 
por el párrafo anterior y teorema \ref{t4.1}{\em (ii)}  obtenemos que $x_k10^k \equiv x_k \pmod{9}$ y por teorema \ref{t4.1}{\em (i)} se deduce que $x \equiv x_0+x_1+\cdots+x_n \pmod{9}$.
\end{proof}

El procedimiento anterior a veces es llamado ``regla del nueve''  \index{regla del nueve} y no es más que el tradicional método para chequear si un número es divisible por 9: $n$ es divisible por 9 si y sólo si la suma de sus dígitos es divisible por 9.  

\begin{ejercicio}
	Usando congruencias encontrar criterios similares  a la ``regla del nueve'' para verificar si un número es divisible por 3. Hacer lo mismo con 5. 
\end{ejercicio}

\begin{ejemplo} Verificar  el siguiente cálculo
$$
54\,321 \times 98\,765= 5\,363\,013\,565.
$$
\end{ejemplo}
\begin{proof}
En la notación de la proposición \ref{prop412} escribamos $\Sigma x$ en vez de $ \sum_{i=0}^n x_i = x_0+x_1+\cdots+x_n$. Hemos visto que $\Sigma x \equiv x \pmod{9}$. Por la parte (ii) del teorema \ref{t4.1} tenemos
$$
\Sigma x\;\Sigma y \equiv xy \pmod{9},
$$
y por consiguiente si $xy=z$ debemos tener $\Sigma x\Sigma y \equiv\Sigma z \pmod{9}$. En el cálculo que se tiene en el ejemplo
$$
\Sigma 54\,321=15,\quad \Sigma 98\,765=35,\quad
\Sigma 5\,363\,013\,565=37,
$$
y
$$
\Sigma 15=6, \quad \Sigma 35=8,\quad \Sigma 37=10.
$$
Puesto que $6 \times 8$ no es congruente a $10 \pmod{9}$ se sigue que $15 \times 35$ no es congruente a $37 \pmod{9}$ y que $54\,321 \times 98\,765$ no es congruente a $5\,363\,013\,565\pmod{9}$. En consecuencia el cálculo está errado.
\end{proof}

\vskip .3cm

\begin{subsection}{Ejercicios}
\begin{enumerate}
\item Sin hacer ninguna ``multiplicación larga'' probar que
$$
\begin{aligned}
\text{(i)} \quad&1\,234\,567 \times 90\,123 \equiv 1 \pmod{10} \\
\text{(ii)} \quad &2468 \times 13\,579 \equiv -3 \pmod{25}
\end{aligned}
$$
\item Usar la regla del nueve para verificar que dos de las siguientes ecuaciones son falsas. ?`Qué se puede decir de la otra ecuación?
$$\begin{aligned}
\text{(i)}  \quad &5783 \times 40\,162  = 233\,256\,846 , \\
\text{(ii)} \quad &9787 \times 1258  = 12\,342\,046 , \\
\text{(iii)} \quad & 8901 \times 5743  = 52\,018\,443 .
\end{aligned}
$$
\item Encontrar el resto de dividir $3^{15}$ por 17 y el de dividir $15^{81}$ por 13.
\item  Sea $(x_nx_{n-1}\ldots x_0)_{10}$ la representación en base 10 de un entero positivo $x$. Probar que
$$
x\equiv x_0-x_1+x_2+\cdots +(-1)^nx_n \pmod{11},
$$
y use este resultado para verificar si 1\,213\,141\,516\,171\,819 es divisible por 11.
\end{enumerate}
\end{subsection}

\end{section}

\begin{section}{Ecuación lineal de congruencia}
 \index{Ecuación lineal de congruencia}
Se trata primero de estudiar en general el problema de resolución de la ecuación en $x$
\begin{equation}\label{ecuacionlineal}
 ax \equiv b \pmod{m}.
\end{equation}
Es fácil ver que el problema no admite siempre solución, por ejemplo $2x\equiv 3 \pmod{2}$ no posee ninguna solución en $\mathbb Z$, pues cualquiera se $k \in \mathbb Z$, $2k-3$ es impar, luego no es divisible por $2$.

Notemos además que si $x_0$ es solución de la ecuación (\ref{ecuacionlineal}), también lo es $x_0+km$ de manera que si la ecuación posee una solución, posee infinitas soluciones. Para evitar la ambigüedad de infinitas soluciones, nos limitaremos a considerar las soluciones tales que $0\le x < m$.

\begin{ejemplo} La solución general de la ecuación $3x\equiv 7
\pmod{11}$ es $6+k7$ con $k \in \mathbb Z$.
\end{ejemplo}
\begin{proof} Si probamos con los enteros $x$ tal que  $0\le x < 11$, veremos que la ecuación admite una única solución, a saber $x=6$. Otras soluciones se obtienen tomando $6+11k$. Por otra parte si $u$ es también solución de la ecuación, entonces $3u\equiv 7 \pmod{11}$ y  como   $3 \times 6 \equiv 7 \pmod{11}$, se tiene que $3u \equiv 3\times 6 \pmod{11}$ y por lo tanto $3(u-6)$ es múltiplo de 11. Como 11 no divide a 3 se tiene que $11|(u-6)$, o sea $u=6+11k$.
\end{proof}

Analicemos ahora la situación general de la ecuación $ ax\equiv b \pmod{m}$. Si $\mcd(a,m)=1$, entonces sabemos que existen enteros $r$ y $s$ tales que $1=ra+sm$ y por lo tanto $b=(rb)a +(sb)m$, o sea que
$$
a(rb) \equiv b \pmod{m},
$$
es decir $rb$ es solución de la ecuación. Veremos que el caso general se hace en forma análoga a lo anterior.

\begin{teorema}\label{th-elc} Sean $a,b$ números enteros y $m$ un entero positivo y denotemos $d = \mcd(a,m)$ . La  ecuación 
\begin{equation}\label{eq-elc}
ax \equiv b \pmod{m}
\end{equation}
admite solución si y sólo si $d|b$, y en este caso dada $x_0$ una solución, todas las soluciones son de la forma $x = x_0 + k n$ con $k \in \mathbb Z$ y $n = \displaystyle{\frac{m}{d}}$ 
\end{teorema}
\begin{proof} Como $d =\mcd(a,m)$, existen $r,s \in \mathbb Z$ tales que
$$
d = ra+sm.
$$
Si $d|b$, entonces existe $h\in \mathbb Z$ tal que $b = dh$. Si multiplicamos por $h$ la ecuación de arriba obtenemos
$$
dh = (rh)a+(sh)m.
$$
Luego $a(rh) \equiv a(rh)+(sh)m \equiv dh \equiv b \pmod{m}$, y por lo tanto $rh$ es solución de la ecuación lineal de congruencia.    

Por otro lado si $ax\equiv b\pmod{m}$, entonces $ax-b=km$ para algún $k$, o sea
$$
b=ax+(-k)m
$$
de la cual se sigue que si $d|a$ y $d|m$, entonces $d|b$ y por lo tanto $\mcd(a,m)|b$.

Por lo tanto hemos demostrado que la condición necesaria y suficiente para que la ecuación $ax\equiv b \pmod{m}$ admita una solución es que $\mcd(a,m)|b$.

En el caso que $d|b$ veamos ahora cuales son todas las soluciones posibles de la ecuación (\ref{eq-elc}). Sean $x_1,x_2$ soluciones, es decir
\begin{align*}
ax_1 &\equiv b \pmod{m} \\
ax_2 &\equiv b \pmod{m},
\end{align*}
entonces, restando miembro a miembro, obtenemos
$$
ax_1 -ax_2 \equiv b - b \equiv 0 \pmod{m}.
$$
Es decir, $x_1,x_2$ son soluciones de la ecuación (\ref{eq-elc}) si y sólo si  $y = x_1 -x_2$ es solución de la ecuación lineal de congruencia \begin{equation}\label{elc0}
ay \equiv 0 \pmod{m}.
\end{equation}

Si $\mcd(a,m) =1$ es claro que la ecuación $ay \equiv 0 \pmod{m}$ tiene como solución todos los $y$ tales que $m|ay$. Como $m$ y $a$ son  coprimos, las soluciones son todos los $y$ tal $m|y$, es decir todos los múltiplos de $m$.

Si $\mcd(a,m) =d > 1$,  la ecuación $ay \equiv 0 \pmod{m}$ tiene como solución todos los $y$ tales que $ay=mk$ para algún $k$. Si dividimos por $d$, podemos decir que las soluciones son todos los $y$ tales que $(a/d)y = (m/d)k$, es decir todos los $y$ tal que $(m/d)|(a/d)y$. Como $m/d$ y $a/d$ son coprimos, las soluciones son todos los múltiplos de $m/d$.

Sean $x_0$ y $x$ tal que $ax_0 \equiv b \pmod{m}$ y $ax \equiv b \pmod{m}$, entonces $a(x_0-x) \equiv 0 \pmod{m}$ y por lo tanto $x_0-x = kn$ para algún $k$. Es decir, cualquier $x$ que es solución lineal de congruencia es de la forma $x_0 = x+kn$ para algún $k$.
\end{proof}

De las demostraciones podemos obtener un método general para encontrar soluciones de la ecuación lineal de congruencia
$$
ax \equiv b \pmod{m}.
$$
con $\mcd(a,m)|b$
\begin{enumerate}
\item Encontrar, usando el algoritmo de Euclides, $r,s$ tales que 
\begin{equation}\label{elc2}
d =\mcd(a,m) = ra+sm.
\end{equation}
\item Como $d|b$, tenemos que $b = td$ y multiplicamos la ecuación (\ref{elc2}) por $t$: $$dt =  (rt)a+(st)m.$$
\item $b = dt = (rt)a+(st)m \equiv (rt)a \pmod{m}$. 

Luego $x_0 = rt$ es solución de la ecuación lineal de congruencia. 
\item Toda solución de la ecuación lineal de congruencia es $x= x_0+ k(m/d)$ con $k \in \mathbb Z$.
\end{enumerate}

\vskip .3cm

Observar que, en las hipótesis del teorema, si $\mcd(a,m) =1$, entonces siempre existen soluciones a la ecuación $ax\equiv b\pmod m$ y todas las soluciones son de la forma $x_0+km$, donde $x_0$ es una solución particular. Más aún, debido a esto, hay una única solución $x$, con $ 0\le x < m$. 

\vskip .3cm

\begin{ejemplo} Hallar las soluciones de la ecuación $13x\equiv 7 \pmod{15}$ con $0\le x< 15$.
\end{ejemplo}
\begin{proof}[Solución]
Hagamos, paso a paso, el procedimiento explicado anteriormente.
\begin{enumerate}
\item 
Usando  el algoritmo de Euclides obtenemos el  $\mcd(13,15)$.
\begin{align*}
15 &=  13 \times 1 + 2 \\
13 &= 2 \times 6 + 1 \\
2 &= 1 \times 2 + 0
\end{align*}
Luego $1 = \mcd(13,15)$. Como 1 divide a cualquier número, en este caso  la ecuación tiene solución. Del algoritmo de Euclides deducimos 
\begin{align*}
1 &= 13 - 2 \times 6\\
&= 13 - (15-13) \times 6  \\
&= 13\times 7 - 15\times 6. 
\end{align*}
Es decir
\begin{equation}\label{eq1315}
1 = 13\times 7 - 15\times 6. 
\end{equation}

\item Multiplicando la ecuación (\ref{eq1315}) por 7 obtenemos
$$
7 = 13\times 49 - 15\times 42. 
$$
\item Luego $13 \times 49 \equiv 7 \pmod{15}$, es decir 49 es solución de la ecuación y todas las soluciones son de la forma 
 $x=49 + 15k$.
\end{enumerate}
Debemos ver ahora cuales soluciones $x$ cumplen $0\le x< 15$. La forma más sencilla de hacerlo es buscando por tanteo: $49+15(-1)=34$, $49+15(-2)=19$, , $49+15(-3)=4$, $49+15(-4)=-11$. Es decir la solución que buscamos es $x=4$.  
\end{proof}

\begin{ejemplo} Hallar las soluciones de la ecuación $42x\equiv
50 \pmod{76}$ con $0\le x< 76$.
\end{ejemplo}
\begin{proof}[Solución] Como antes, hagamos paso a paso el procedimiento explicado anteriormente.
\begin{enumerate}
\item 
Usando  el algoritmo de Euclides obtenemos el  $\mcd(42,76)$.
\begin{align*}
76 &=  42 \times 1 + 34 \\
42 &= 34 \times 1 + 8 \\
34 &= 8 \times 4 + 2 \\
8 &= 2 \times 4 
\end{align*}
Luego $2 = \mcd(42,76)$. Como $2|50$ la ecuación tiene solución. Del algoritmo de Euclides deducimos 
\begin{align*}
2 &= 34 - 8 \times 4\\
&= 34 - (42-34) \times 4 = 34\times 5 - 42\times 4\\
&= (76-42)\times 5 - 42\times 4 \\
&= 76\times 5 - 42\times 9. 
\end{align*}
Es decir
$$
2 = (-9) \times 42 + 5 \times 76.
$$

\item $50 = 2\times 25$ y tenemos  que 
\begin{align*}
50 &= (-9\times 25) \times 42 + (5 \times 25) \times 76 \\
50 &= (-225) \times 42 + 125 \times 76
\end{align*}
\item Luego $x_0 = -225$ es una solución de la ecuación lineal de congruencia y todas las soluciones son de la forma $-225 + (76/2)k$, es decir $x = -225 + 38k$.
\end{enumerate}
Debemos ver ahora cuales soluciones $x$ cumplen $0\le x< 76$. Como en el caso anterior podemos hacer esto por tanteo, pero en aquí la forma más sencilla de hacerlo es escribir las inecuaciones
\begin{align*}
0&\le -225 + 38k < 76 \\
225&\le 38k < 76+225= 301 \\
225/38 &\le k \le 301/38 \\
5.9 &\le k \le 7.9
\end{align*}
Luego $k = 6$ o $k =7$ y entonces $x_1 = -225 + 38\times 6 = 3$ y $x_2 = -225 + 38\times 7 = 41$ son las soluciones que buscamos. 
\end{proof}

\begin{ejemplo} Hallar las soluciones del sistema de ecuaciones
	\begin{align*}
		&4x\equiv 11 \pmod{15} \\
		&7x\equiv 8 \pmod{12}.	
	\end{align*}
\end{ejemplo}
\begin{proof}[Solución] 
	Es fácil verificar que  $-1$ es solución de la primera ecuación y como $\mcd(4,15)=1$, las soluciones  de la primera ecuación son de la forma $x= -1 + 15k$ con $k \in \ZZ$. 
	
	Ahora, debemos encontrar los $k \in \ZZ$ soluciones de la ecuación
	\begin{equation}\label{eq-sistema}
		7(-1 + 15k)\equiv 8 \pmod{12}.
	\end{equation}
	Expandiendo el lado izquierdo de la ecuación obtenemos
	\begin{equation*}
		7\times (-1) + 7\times 15 k\equiv -7 + 105 k\equiv 5 + 9k \pmod{12}.
	\end{equation*}
	Por lo tanto la ecuación  (\ref{eq-sistema}) es equivalente a $5 + 9k \equiv 8 \pmod{12}$. Pasando el $5$ a la derecha obtenemos:
	\begin{equation}
		9k \equiv 3 \pmod{12}.
	\end{equation}
	Usando el método del teorema \ref{th-elc} obtenemos que las soluciones son  $k = 3 + 6t$, con $t \in \ZZ$. 
	
	Por lo tanto, las soluciones del sistema son $x =  -1 + 15k =  -1 + 15(3 + 6t)= 44 + 90t$, con $t \in \ZZ$. 
\end{proof}


\begin{subsection}{Ejercicios}
\begin{enumerate}
\item  Resolver las siguientes ecuaciones lineales de congruencia
$$ \text{(i)}\quad 2x \equiv 1 \pmod{7}\qquad
 \text{(ii)}\quad 3970x \equiv 560 \pmod{2755}.
$$
\end{enumerate}
\end{subsection}

\end{section}

\begin{section}{Teorema de Fermat} \index{Teorema de Fermat} 
El siguiente lema nos sirve de preparación para la demostración del Teorema (o fórmula) de Fermat.

\begin{lema} \label{l4.3} Sea $p$ un número primo, entonces
\begin{enumerate}
\item[({\em i})] $p|\binom{p}{r}$, con $0< r <p$,
\item[({\em ii})] $(a+b)^p \equiv a^p+b^p \pmod{p}$.
\end{enumerate}
\end{lema}
\begin{proof} ({\em i})  Escribamos el número binomial de otra forma: 
$$
\binom{p}{r}=\frac{p!}{r!(p-r)!}=p\frac{(p-1)!}{r!(p-r)!}
$$ 
es un número entero, digamos $k$, luego 
\begin{equation}\label{binp}
p r!(p-r)! = k(p-1)!.
\end{equation}
Como $p-1$, $r$ y $p-r$ son menores que $p$, entonces  $(p-1)!$, $r!$ y $(p-r)!$ son producto de números menores que $p$ y por lo tanto son producto de primos menores que $p$.  Por lo tanto, el primo $p$ no aparece en la descomposición prima de $(p-1)!$, $r!$ y $(p-r)!$. Por la igualdad de la ecuación (\ref{binp}), $p$ debe ser factor de $k = \binom{p}{r}$, luego $p|\binom{p}{r}$.  ({\em ii}) Por el teorema del binomio (teorema \ref{t3.6}) sabemos que
$$
(a+b)^p =\sum_{i=0}^{p} \binom{p}{i} a^ib^{p-i}.
$$
Por ({\em i}) es claro que $ \binom{p}{i} a^ib^{p-i}\equiv 0 \pmod{p}$, si $0<i<p$. Luego se deduce el resultado.
\end{proof}

El siguiente es el llamado teorema de Fermat.

\begin{teorema}\label{t4.3} Sea $p$ un número primo y $a$ número entero. Entonces
$$
a^p\equiv a\pmod{p}.
$$
\end{teorema}
\begin{proof} Supongamos que $a\ge 0$, entonces hagamos inducción en $a$. Si $a=0$, el resultado es trivial. Supongamos el resultado probado para $k$, es decir $k^p \equiv k \pmod{p}$. Entonces $(k+1)^p \equiv k^p +1^p \equiv k+1 \pmod{p}$. La primera congruencia es debido al lema \ref{l4.3} ({\em ii}) y la segunda es válida por hipótesis inductiva. Luego $a^p\equiv a\pmod{p}$ cuando $a >0$.

Si $a<0$, entonces $-a>0$ y ya vimos que $(-a)^p \equiv -a \pmod{p}$, es decir que $(-1)^pa^p \equiv (-1)a \pmod{p}$. Si $p\not=2$, entonces $(-1)^p=-1$ y se deduce el resultado. Si $p=2$, entonces $(-1)^p=1$, pero como $1\equiv -1 \pmod{2}$, obtenemos también $a^p\equiv a\pmod{p}$.
\end{proof}


\begin{corolario}  Sea $p$ un número primo y $a$ número entero tal que $p$ y $a$ son coprimos. Entonces
	$$
	a^{p-1}\equiv 1\pmod{p}.
	$$
	content
\end{corolario}
\begin{proof}
	Como que $a$ y $p$ son coprimos, por Fermat $p|(a^p -a)=a(a^{(p-1)} -1)$. Como $p$ no divide a $a$, tenemos que $p|(a^{(p-1)} -1)$ y el resultado  está probado.
\end{proof}

Este último enunciado es también conocido como teorema de Fermat.

La función de Euler  \index{función de Euler} $\phi(n)$, para $n\ge 1$, está definida como el cardinal del conjunto de los $x$ entre 1 y $n$ que son coprimos con $n$. El (segundo) teorema de Fermat admite la siguiente generalización.

\begin{teorema}[Teorema de Euler]\index{Teorema de Euler} 
	Sea $n$ un entero positivo y $a$ un número entero coprimo con $n$, entonces
	\begin{equation*}
		a^{\phi(n)} \equiv 1\pmod{n}
	\end{equation*}
\end{teorema}
\begin{proof}[Demostración (*)] 

Sea $k=\phi(n)$ y sean $x_1,\ldots,x_k$ los $k$ números coprimos con $n$ comprendidos entre $1$ y $n$. 

Como $a$ coprimo con $n$, entonces existe $u$  tal que   $au \equiv 1 \pmod{n}$.   Sea $x'_i$  es el resto de dividir $x_ia$ por $n$, para $1 \le i \le k$, es decir  $x_ia \equiv x'_i \pmod{n}$ y $0 \le x'_i < n$. Veamos que la lista $x'_1,\ldots,x'_k$ es un a lista de números coprimos con $n$ y en la lista no hay elementos repetidos.  
Como $x_i$ y $a$ son coprimos con $n$, también lo es $x_ia$ y por lo tanto $x'_i$  es coprimo con $n$. Por  otro lado, si $x'_i = x'_j$,  entonces $x_ia = x_ja $, luego $x_iau \equiv x_jau \pmod{n}$, por lo tanto  $x_i \equiv x_j \pmod{n}$ y en consecuencia $x_i = x_j$ e $i=j$ (pues $1 \le x_i,x_j < n$).

Como $x'_1,\ldots,x'_k$ es una lista de $k$ números coprimos con $n$ y en la lista no hay elementos repetidos tenemos que $x'_1,\ldots,x'_k$ es una permutación de   $x_1,\ldots,x_k$ y, por lo tanto $\{x'_1\ldots x'_k\} = \{x_1 \ldots x_k\}$.

Ahora bien,  
\begin{equation*}
	a^{\phi(n)}x_1\ldots x_k =ax_1\ldots ax_k \equiv x'_1\ldots x'_k \equiv x_1\ldots x_k \pmod{n}.
\end{equation*}
 Como $x_1\ldots x_k$ coprimo con $n$, existe $v$ tal que $x_1\ldots x_kv\equiv 1\pmod{n}$, por lo tanto
 \begin{equation*}
 	1 \equiv  x_1\ldots x_kv  \equiv a^{\phi(n)}x_1\ldots x_k v \equiv a^{\phi(n)} \pmod{n}.
 \end{equation*}
\end{proof}

\begin{subsection}{Ejercicios}
\begin{enumerate}
\item Usar el teorema de Fermat para calcular el resto de dividir $3^{47}$ por 23.
\item Si $m$ coprimo con $n$, entonces $ma\equiv mb \pmod{n}$ si y solo si $a\equiv b\pmod{m}$.
\end{enumerate}
\end{subsection}

\end{section}

\begin{section}{El criptosistema RSA (*)}

Una de las aplicaciones más elementales y difundidas de la aritmética es en el diseño de sistemas criptográficos. El RSA es el más conocido de ellos y será presentado en esta sección. \index{RSA}

Por criptosistema nos referimos a sistemas de encriptamiento o codificación esencialmente pensados para proteger la \index{encriptar}  \index{codificar} confidencialidad de datos que se desean transmitir. Entre los criptosistemas encontramos los simétricos \index{criptografía simétrica} y los de clave pública o asimétricos. \index{criptografía asimétrica} \index{criptografía de clave pública} 

Los sistemas criptográficos simétricos son aquellos en que tanto el emisor como el receptor conocen una función, digamos $f$ y una palabra, digamos $x$ (la clave), tanto la función como la clave  deben ser confidenciales o más comúnmente solo la clave debe ser confidencial. Cuando el emisor desea enviar un mensaje $M$, entonces aplica la función a $M$ y $x$, es decir $M'=f(M,x)$, envía $M'$ y el receptor aplica la función inversa y recupera $M$, es decir $M=f^{-1}(M',x)$. Es llamada \emph{simétrica} porque tanto el emisor como el receptor manejan las mismas claves y el emisor puede pasar a receptor y viceversa usando la misma encriptación.   

En los sistemas de clave pública el receptor conoce una clave privada $d$ (no compartida por nadie) y publicita una clave pública $x$, de la misma manera que antes,  si alguien desea enviar un mensaje $M$ al receptor debe hacer $M'=f(M,x)$, pero el receptor para decodificar debe hacer $M=g(M',y)$, donde $g$ es una función adecuada. Una ventaja evidente de los sistemas de clave pública es que no es necesario poner en conocimiento del emisor ninguna clave confidencial, más aún cualquier persona puede enviar en forma confidencial datos a otra persona que ha publicitado su clave.

Rivest, Shamir y Adleman descubrieron el primer criptosistema práctico de clave pública, que es llamado RSA. La seguridad del RSA se basa en la dificultad de factorizar números enteros grandes. Este sistema es el más comúnmente recomendado para uso en sistemas de clave pública. La mayor ventaja del RSA es que puede ser usado para proveer privacidad y autenticación (firma digital) en las comunicaciones. Su principal desventaja es que su implementación se basa en exponenciación de números enteros grandes, una operación que consume recursos de la computadora, aunque esto es cada vez menos significativo.

Antes de describir el RSA digamos que se basa fuertemente en el teorema de Fermat visto en la sección anterior. 

En el sistema RSA deben realizarse algunos paso previos para fijar ciertos parámetros que luego nos permitirán encriptar y desencriptar los mensajes.

\vskip .4cm 

\noindent\textbf{Idea del  algoritmo}

Supongamos que la persona $B$  quiere enviar a la persona $A$ un mensaje $m$ pero encriptado de tal forma que sólo $A$ pueda leer su contenido. Por su parte $A$ hace públicos dos números $e$ y $n$ que son los que se utilizarán para encriptar los mensajes que le envíen. 

Entonces a partir de $m$ la persona $B$ genera un mensaje cifrado $c$ mediante la siguiente operación:
$$
    c\equiv m^e\ \pmod{n}\ ,
$$
donde $e$ y $n$ es la clave pública de $A$.

Ahora $A$ recupera le mensaje $m$ a partir del mensaje en clave $c$ mediante la operación inversa dada por
$$
    m\equiv c^d\ \pmod{n}\ ,
$$
donde $d$ es la clave privada que solo $A$ conoce.

\vskip .3cm 

\noindent \textbf{Elección de claves}

\vskip .3cm 

Dados primos distintos $p$ y $q$ suficientemente grandes tomamos $n = pq$. 

- La \emph{clave pública} es $n$ y $e$ con $1 < e < (p-1)(q-1)$ tal que $\operatorname{mcd}(e, (p-1)(q-1)) = 1$. 

- La \emph{clave privada} es un $d$ tal que $ed \equiv 1 \pmod{(p-1)(q-1)}$ y $0 \le d <(p-1)(q-1)$.

\begin{observacion} Algunos comentarios sobre la elección de $p,q,e,d$.
\begin{itemize}
\item
Los dos primos $p$ y $q$ deberían tener alrededor de 100 dígitos cada uno (longitud considerada segura en este momento).
\item
El número $e$ puede elegirse pequeño y se selecciona haciendo prueba y error con el algoritmo de Euclides, es decir probando hasta encontrar un $e$ tal que $\operatorname{mcd}(e, (p-1)(q-1)) = 1$.
\item
La existencia de $d$ está garantizada por el teorema \ref{th-elc} (ecuación lineal de congruencia), pues $e$ y $(p-1)(q-1)$ son coprimos.
\end{itemize}

\end{observacion}

\vskip .4cm 

\noindent \textbf{Encriptar y desencriptar mensajes}

\vskip .3cm 
El receptor de mensajes publicita la clave pública $(e,n)$. Obviamente no da a conocer ni $p$, ni $q$ y mantiene segura la clave privada $d$. Como mencionamos anteriormente, el envío del mensaje y su decodificación requiere dos pasos
\begin{enumerate}
\item El  emisor desea \emph{encriptar }un número $m \in \{0,\ldots,n-1\}$ y para ello calcula $c \equiv m^e \pmod{n}$ y  envía $c$ al receptor.
\item El receptor desea \emph{desencriptar} el mensaje, es decir usando la clave pública $(e,n)$ y $c$ desea recuperar $m$: calcula $c^d \pmod{n}$ y veremos a continuación que este número es $m$. 
\end{enumerate}

En la siguiente proposición se encuentra toda la información necesaria para implementar el algoritmo RSA. 

\begin{proposicion} \label{rsa}
	\begin{enumerate}
		\item Sea $n \in \mathbb N$,  tal que  $n = pq$ producto de dos números primos ($n$ es clave pública).
		\item Sea $e \in \mathbb N$ tal que $e$ y  $u := (p-1)(q-1)$  son coprimos ($e$ es clave pública). 
		\item Como $e$ y $u$ son coprimos, existe $d \in \mathbb N$ tal que $ed \equiv 1 \pmod{u}$ ($d$ es clave privada).
	\end{enumerate}
	 

	Sean $m \in \{0,\ldots,n-1\}$ ($m$  es el mensaje), entonces si 
	\begin{equation*}
		c \equiv m^e \pmod{n} 
	\end{equation*}
	obtenemos
	\begin{equation*}
		m \equiv c^d \pmod{n}.
	\end{equation*}

\end{proposicion} 
\begin{proof}
Como $ed \equiv 1 \pmod{(p - 1)(q - 1)}$, entonces existe $k$ tal que  
\begin{equation}\label{rsa1}
ed = 1 + k(p - 1)(q - 1).
\end{equation}
Consideremos el mensaje $m$ y si es o no coprimo con $p$.

Si $\operatorname{mcd}(m, p) = 1$, el Teorema de Fermat dice que $m^{p - 1}  \equiv 1\pmod{p}$.
Entonces $(m^{p - 1})^x \equiv 1\pmod{p}$ para cualquier $x$. En particular, para 
$x = k(q-1)$. Asi que tenemos:
\begin{equation*}
m^{k(p-1)(q-1)} = (m^{p-1})^{k(q-1)} \equiv 1\pmod{p}.
\end{equation*}
Multiplicando esta ecuación por $m$ obtenemos
\begin{equation}\label{rsa2}
m^{1+k(p-1)(q-1)} \equiv m\pmod{p}. 
\end{equation}
Usando las ecuaciones (\ref{rsa1}) y (\ref{rsa2}) obtenemos:
\begin{equation}\label{rsa3}
m^{ed} \equiv m\pmod{p} 
\end{equation}
Esto, si  $\operatorname{mcd}(m, p) = 1$. Pero si esto último no es cierto, entonces al ser $p$ primo debemos tener $m \equiv 0\pmod{p}$ y en ese caso (\ref{rsa3}) es trivial (dice que $0 \equiv 0$) Con lo cual (\ref{rsa3}) es verdadero para todo $m$.

Obviamente la elección u orden de $p$ o $q$ es arbitraria, así que (\ref{rsa3}) también es verdadera si reemplazamos $p$ por $q$. Así que tenemos: $$p|(m^{ed}-m) \qquad \text{ y } \qquad q|(m^{ed}-m).$$ Como $p$ y $q$ son primos distintos, entonces concluimos que $pq|(m^{ed}-m)$, es decir, $$m^{ed} \equiv m \pmod{pq}.$$
\end{proof}

\begin{ejemplo} Probemos el sistema en forma práctica usando primos pequeños, por ejemplo $p=31$, $q=73$. En este caso $n = pq = 2263$. 

Busquemos ahora un $e$:  tenemos que $ (p-1)(q-1)= 30\times 72 = 2160$. Vemos que $2$, $3$, $5$ dividen a $2160$, pero $7$ es coprimo con $2160$. Tomemos entonces $e = 7$.

Usando el algoritmo de Euclides obtenemos $1 = 2\times 2160 + (-617)\times 7$. Luego, podemos tomar $d = 2160 -617 =1543$. 

Por lo tanto el receptor tiene clave pública $(7,2263)$ y conserva en secreto  su clave privada $1543$

- Supongamos que el emisor ha nacido en el año 1993 y quiere enviarle en secreto al receptor su año de nacimiento. Entonces encripta el año haciendo 
$$
1993^7 \equiv 1417 \pmod{2160},
$$   
y envía, por una vía insegura, por ejemplo un email, el número 1417 al receptor.

- El receptor calcula
$$
1417^{1543} \pmod{2160}
$$ 
y obtiene nuevamente 1993 (si quiere convencerse de esto ingrese {\tt (1417 \^{} 1543) mod 2160} a la ventana de búsqueda de \href{https://www.wolframalpha.com}{Wolfram Alpha}).

Pese a que el cálculo de desencriptado (en este caso $1417^{1543} \pmod{2160}$) puede parecer costoso computacionalmente, hay métodos eficientes para hacerlo. 

\end{ejemplo}

Una propiedad importante del RSA es que puede ser usado para firma digital o autenticación. En las hipótesis de la proposición \ref{rsa}, es claro que lo que probamos es que 
$$
(m^e)^d \equiv m \pmod{n},
$$
para $m \in \{0,\ldots,n-1\}$. Ahora bien 
$$
(m^e)^d  = m^{ed} = (m^d)^e,  
$$
es decir
$$
(m^d)^e \equiv m \pmod{n}, 
$$
para $m \in \{0,\ldots,n-1\}$. Por lo tanto, el receptor puede codificar un número o mensaje $m$ calculando  $b \equiv m^d  \pmod{n}$ y cualquiera que conozca la clave pública puede obtener el original calculando $b^e \pmod{n}$. 

Lo interesante de esto es que si el receptor envía $m$ (el mensaje) y $b$ (la codificación de $m$), cualquiera puede comprobar que el mensaje ha sido codificado por el receptor (y no por otra persona) verificando  que   $m \equiv b^e \pmod{n}$.     

\begin{ejemplo}
Como ya hemos mencionado, podemos ver que el RSA también puede ser usado para un sistema de \emph{autenticación}, es decir es posible comprobar quien es la persona que envía el mensaje. Veamos una forma de hacerlo: la persona $A$ tiene clave pública $(e,n)$ y clave privada $d$ y   la persona $B$  tiene clave pública $(e',n')$ y clave privada $d'$. 

\begin{enumerate}
\item La persona $B$ desea enviar un mensaje $m$ (en forma segura) a la persona $A$ y quiere certificar que el mensaje fue enviado por él.
\item $B$ calcula $x \equiv m^{d'} \pmod{n'}$. Es decir encripta su mensaje usando su  clave privada.
\item Ahora $B$ codifica $m$ y $x$ con la clave pública de $A$, es decir calcula $c \equiv m^{e} \pmod{n}$ e  $y \equiv x^e \pmod{n}$. 
\item $B$ envía $c$ e $y$ al receptor $A$.
\item La persona $A$ recupera $m$ y $x$ calculando  $m \equiv c^d \pmod{n}$ y  $x \equiv y^d \pmod{n}$. 
\item $A$ comprueba que el mensaje  proviene de $B$ o, mejor dicho, proviene de la persona con clave pública $(e',n')$, verificando que $m \equiv x^{e'} \pmod{n'}$. 
\end{enumerate}

\end{ejemplo}
\end{section}

\begin{section}{Ejercicios}
\begin{enumerate}
\item Determine todas las posibles soluciones de las congruencias
$$
\text{(i)}\quad 5x\equiv1 \pmod{11},\qquad\, \text{(ii)}\quad 5x\equiv 7 \pmod{15}.
$$
\item Sin hacer demasiadas cuentas verifique que $192\,837\,465\,564\,738\,291$ es divisible por 11.
\item Resolver la ecuaciones
$$
\text{(i)}\quad 5x\equiv12 \pmod{13},\qquad\, \text{(ii)}\quad x^2-x \equiv 1 \pmod{11}.
$$
\item ?`Cuál es el último dígito de la representación en base 10 de $7^{93}$.
\item Usar que $1001=7\times 11 \times 13$ para construir una prueba para la división para los número 7, 11 y 13 similar a la prueba del 9.
\end{enumerate}
\end{section}

\chapter[Grafos]{Grafos}

\begin{section}{Grafos y sus representaciones}\label{5.1}

Los objetos a los cuales llamaremos {\it {grafos} } son muy útiles en matemática discreta. Su nombre se deriva del hecho de que pueden ser entendidos con una notación gráfica (o pictórica), y en este aspecto solamente se parecen a los familiares gráficos de funciones que son estudiados en matemática elemental. Pero nuestros grafos son bastante diferentes de los gráficos de funciones y están más relacionados con objetos que en el lenguaje diario llamamos ``redes'' (networks).  \index{redes}

Usaremos la siguiente definición en lo que sigue: dado un conjunto $X$ un {\em $2$-subconjunto} es un subconjunto de $X$ de dos elementos.  

\begin{definicion} Un {\em grafo} $G$ consiste de un  \index{grafo} conjunto finito $V$, cuyos miembros son llamados  {\em vértices},  \index{vértices de un grafo} y un conjunto de $2$-subconjuntos de $V$, cuyos miembros son llamados {\em aristas}.  \index{aristas de un grafo} Nosotros usualmente escribiremos $G=(V,E)$ y diremos que $V$ es el {\em conjunto de vértices} y $E$ es el {\em conjunto de aristas}.
\end{definicion}

La restricción a un conjunto finito no es esencial, pero es conveniente para nosotros debido a que no consideraremos ``grafos'' infinitos en este apunte.

Un ejemplo típico de un grafo $G=(V,E)$ es dado por los conjuntos
\begin{equation}\label{grafosimple}
V=\{a,b,c,d,z\}, \qquad\quad
E=\{\{a,b\},\{a,d\},\{b,z\},\{c,d\},\{d,z\}\}.
\end{equation}
Este ejemplo y la definición misma no son demasiado esclarecedores, y solamente cuando consideramos la {\it representación pictórica} de un grafo es cuando se hace la luz.  \index{representación pictórica (de un grafo)}

\begin{figure}[ht]
		\begin{tikzpicture}[scale=1]
		%\SetVertexSimple[Shape=circle,FillColor=white]
		\Vertex[x=0.00, y=2.00]{$a$}
		\Vertex[x=1.90, y=0.62]{$b$}
		\Vertex[x=1.18, y=-1.62]{$c$}
		\Vertex[x=-1.18, y=-1.62]{$d$}
		\Vertex[x=-1.90, y=0.62]{$z$}
		\Edges($c$, $d$,$a$,$b$,$z$,$d$)
		\end{tikzpicture}
	\caption{Una representación pictórica del grafo definido en (\ref{grafosimple}).}\label{f5.1}
\end{figure}

Nosotros representamos los vértices como puntos, y unimos dos puntos con una linea siempre y cuando el correspondiente par de vértices está en una arista. Luego la Fig. \ref{f5.1} es una representación pictórica del grafo dado en el ejemplo arriba. Esta clase de representación es extremadamente conveniente para trabajar ``a mano'' con grafos relativamente pequeños. Más aún, esta representación es de gran ayuda para formular y comprender argumentos abstractos. Nosotros damos a continuación un ejemplo frívolo.

\begin{ejemplo} Mario y su mujer Abril dan una fiesta en la cual hay otras cuatro parejas de casados. Las parejas, cuando arriban, estrechan la mano a algunas personas, pero, naturalmente, no se estrechan la mano entre marido y mujer. Cuando la fiesta finaliza el profesor pregunta a los otros a cuantas personas han estrechado la mano, recibiendo 9 respuestas diferentes. ?`Cuántas personas estrecharon la mano de Abril?
\end{ejemplo}
\begin{proof}[Solución] Construyamos un grafo cuyos vértices son las personas que asisten a la fiesta. Las aristas del grafo son las  $\{x,y\}$ siempre y cuando $x$ e $y$ se hayan estrechado las manos. Puesto que hay nueve personas aparte de Mario, y que una persona puede estrechar a lo sumo a otras 8 personas, se sigue que las 9 respuestas diferentes que ha recibido el profesor deben ser 0, 1, 2, 3, 4, 5, 6, 7, 8. Denotemos los vértices con estos números y usemos $M$ para Mario. Así obtenemos la representación pictórica de la Fig. \ref{f5.2}

\begin{figure}[h]
	\begin{tikzpicture}[scale=2.5]
	\draw[-,line width=1pt] (0.81,0.59) -- (0.7*0.81,0.7*0.59);
	\draw[-,line width=1pt] (0.31,0.95) -- (0.04, 0.81) -- (0.31,0.95) -- (0.45, 0.68) -- (0.31,0.95);
	\draw[-,line width=1pt] (-0.31,0.95) -- (-0.45, 0.68) -- (-0.31,0.95) -- (-0.22, 0.66) -- (-0.31,0.95) -- (-0.04, 0.81) -- (-0.31,0.95);
	\draw[-,line width=1pt] (-0.81,0.59) -- (-0.76, 0.29) -- (-0.81,0.59) -- (-0.62, 0.36) -- (-0.81,0.59) -- (-0.53, 0.48) -- (-0.81,0.59) -- (-0.51, 0.64) -- (-0.81,0.59);
	\draw[-,line width=1pt] (-1.0,-0.0) -- (-0.79, -0.21) -- (-1.0,-0.0) -- (-0.72, -0.12) -- (-1.0,-0.0) -- (-0.7, -0.0) -- (-1.0,-0.0) -- (-0.72, 0.11) -- (-1.0,-0.0) -- (-0.79, 0.21) -- (-1.0,-0.0);
	\draw[-,line width=1pt] (-0.81,-0.59) -- (-0.51, -0.64) -- (-0.81,-0.59) -- (-0.51, -0.54) -- (-0.81,-0.59) -- (-0.54, -0.45) -- (-0.81,-0.59) -- (-0.6, -0.38) -- (-0.81,-0.59) -- (-0.67, -0.32) -- (-0.81,-0.59) -- (-0.76, -0.29) -- (-0.81,-0.59);
	\draw[-,line width=1pt] (-0.31,-0.95) -- (-0.04, -0.81) -- (-0.31,-0.95) -- (-0.09, -0.75) -- (-0.31,-0.95) -- (-0.15, -0.7) -- (-0.31,-0.95) -- (-0.22, -0.66) -- (-0.31,-0.95) -- (-0.29, -0.65) -- (-0.31,-0.95) -- (-0.37, -0.66) -- (-0.31,-0.95) -- (-0.45, -0.68) -- (-0.31,-0.95);
	\draw[-,line width=1pt] (0.31,-0.95) -- (0.45, -0.68) -- (0.31,-0.95) -- (0.38, -0.66) -- (0.31,-0.95) -- (0.32, -0.65) -- (0.31,-0.95) -- (0.25, -0.66) -- (0.31,-0.95) -- (0.18, -0.68) -- (0.31,-0.95) -- (0.13, -0.71) -- (0.31,-0.95) -- (0.08, -0.76) -- (0.31,-0.95) -- (0.04, -0.81) -- (0.31,-0.95);

	\GraphInit[vstyle=Welsh]
	\Vertices[]{circle}{0,1,2,3,4,5,6,7,8,$M$}
	\draw (0.6,-0.45) node {?};
	\end{tikzpicture} 
	\caption{La fiesta de Abril}\label{f5.2}
\end{figure}

Ahora, el vértice 8 alcanza a todos los otros vértices excepto uno, el cual debe por lo tanto representar a la esposa de 8. Este vértice debe ser el 0 el cual por cierto que no está unido al 8 (ni obviamente a ningún otro). Luego 8 y 0 son una pareja de casados y 8 está unido a 1, 2, 3, 4, 5, 6, 7 y $M$. En particular el 1 está unido al 8 y ésta es la única arista que parte del 1. Por consiguiente 7 no esta unido al 0 y al 1 (únicamente), y la esposa de 7 debe ser 1, puesto que 0 esta casado con 8. Continuando con este razonamiento vemos que 6 y 2, y 5 y 3 son parejas de casados. Se sigue entonces que $M$ y 4 están casados, luego el vértice 4 representa a Abril, quien estrechó la mano de cuatro personas.
\end{proof}

Aunque la representación pictórica es intuitivamente atractiva para los seres humanos, es claramente inútil cuando deseamos comunicarnos con una computadora. Para lograr esto debemos representar el grafo mediante cierta clase de lista o tabla. Diremos que dos vértices $x$ e $y$ de un grafo son {\em adyacentes} cuando $\{x,y\}$ es una arista.  \index{vértices adyacentes} Entonces  podemos representar un grafo $G=(V,E)$ por su {\em lista de adyacencia},  \index{lista de adyacencia} donde cada vértice $v$ encabeza una lista de aquellos vértices que son adyacentes a $v$. El grafo de Fig. \ref{f5.1} tiene la siguiente lista de adyacencia:

\vskip.4cm

\begin{center}
\begin{tabular}{ccccc}
$a$&$b$&$c$&$d$&$z$ \\ \hline
$b$&$a$&$d$&$a$&$b$ \\
$d$&$z$&&$c$&$d$\\
&&&$z$&
\end{tabular}
\end{center}

Las listas de adyacencia son redundantes (cada arista está representada dos veces) pero como todo lenguaje de programación de alto nivel maneja la estructura tipo lista,  preferimos esta representación pues  un grafo  resulta ser como una lista de listas  o un  arreglo de listas.  

\begin{ejemplo}
Por cada entero positivo $n$ definimos el {\em grafo completo  \index{grafo completo} $K_n$} como el grafo con $n$ vértices y en el cual cada par de vértices es adyacente. 

¿Cuántas aristas tiene $K_n$? De cada vértice ``salen'' $n-1$ aristas, las que van a otros vértices. Si  sumamos $n$-veces las $n-1$ aristas es claro que estamos contando cada arista dos veces, luego el número total de aristas es $n(n-1)/2$ (observar que esta es una demostración, usando  grafos, de que $\displaystyle\sum_{i=1}^n i = n(n-1)/2$).

\end{ejemplo}
\begin{subsection}{Ejercicios}
\begin{enumerate}
\item A tres casas $A,B,C$ se les debe conectar el gas, el agua y la electricidad: $G,W,E$. Escribir la lista de adyacencia para el grafo que representa este problema y construir una representación pictórica del mismo. ¿Puede usted encontrar un dibujo en el cual las líneas que representan las aristas no se crucen?
\item Los senderos de un jardín han sido diseñados dándoles forma de {\em grafo rueda} $W_n$,  cuyos vértices son $V=\{0,1,2,\ldots,n\}$ y sus aristas son
$$
\begin{aligned}
\{0,1\},\qquad &\{0,2\},\ldots,\{0,n\}, \\
\{1,2\}.\qquad &\{2,3\},\ldots,\{n-1,n\},\qquad \{n,1\}.
\end{aligned}
$$
\begin{figure}[ht]
	\begin{tabular}{llll}
		&
		\begin{tikzpicture}[scale=0.97]
		\SetVertexSimple[Shape=circle,FillColor=white,MinSize=8 pt]
		\Vertex[x=0.00, y=2.00]{1}
		\Vertex[x=1.90, y=0.62]{2}
		\Vertex[x=1.18, y=-1.62]{3}
		\Vertex[x=-1.18, y=-1.62]{4}
		\Vertex[x=-1.90, y=0.62]{5}
		\Vertex[x=0.00, y=0.0]{6}
		\Edges(1,2,3,4,5,1)
		\Edges(6,1,2,6,3,4,6,5)
		\draw (0,-2.45) node {$W_6$};
		\end{tikzpicture}
		&
		\qquad\qquad
		& 
		\begin{tikzpicture}[scale=0.65]
		\SetVertexSimple[Shape=circle,FillColor=white,MinSize=8 pt]
		\Vertex[x=3.00, y=0.00]{1}
		\Vertex[x=1.50, y=2.60]{2}
		\Vertex[x=-1.50, y=2.60]{3}
		\Vertex[x=-3.00, y=0.00]{4}
		\Vertex[x=-1.50, y=-2.60]{5}
		\Vertex[x=1.50, y=-2.60]{6}
		\Edges(1,2,3,4,5,6,1)
		\Edges(1,4) \Edges(3,6) \Edges(2,5)
		\Vertex[x=0, y=0]{7}
		\draw (0,-3.8) node {$W_7$};
		\end{tikzpicture}
	\end{tabular}
	\caption{Grafos rueda $W_6$ y $W_7$}\label{figure-grafos-rueda}
\end{figure}
(ver figura \ref{figure-grafos-rueda}). Describa una ruta por los senderos de tal forma que empiece y termine en le vértice 0 y que pase por cada vértice una sola vez. 
\item  ?`Para cuales valores de $n$ podemos hacer una representación pictórica de $K_n$ con la propiedad que las líneas que representan las aristas no se corten?
\item Un {\it {3-ciclo}} en un grafo es un conjunto de tres vértices mutuamente adyacentes. Construir un grafo con cinco vértices y seis aristas que no
contenga 3-ciclos.
\end{enumerate}
\end{subsection}

\end{section}

\begin{section}{Isomorfismo de grafos} \label{5.2}
En este punto nosotros debemos enfatizar que un grafo esta definido como una entidad matemática abstracta. Es en este contexto que nosotros discutiremos el importante problema de que queremos decir cuando decimos que dos grafos son ``el mismo''.

Claramente lo importante de un grafo no son los nombres con que designamos a los vértices, ni su representación pictórica o cualquier otra representación. La propiedad característica de un grafo es la manera en que los vértices están conectados por aristas. 

Antes de definir isomorfismo de grafos repasaremos el  concepto de función o aplicación biyectiva. Dado  dos conjuntos $X,Y$ diremos que una aplicación $f: X \to Y$ es {\em biyectiva} si para cada $y \in Y$ existe un  único $x \in X$ tal que $f(x) =y$. Un propiedad importante, de las funciones biyectivas es que $f$ es biyectiva si y sólo sí  $f$ tiene {\em inversa}, es decir existe $f^{-1}: Y \to X$, tal que $f(f^{-1}(y)) = y$, $\forall \,y \in Y$ y $f^{-1}(f(x)) = x$, $\forall \,x \in X$.

\begin{ejemplo}
La función  
\begin{align*}
f&: \{1,2,3\}\to\{a,b,c\} \quad \text{definida } f(1) = c, f(2) = b, f(3) = a
\end{align*}
es biyectiva y su  inversa es 
$$
f^{-1}(a) = 3,\;f^{-1}(b) = 2,\;f^{-1}(c) =1.
$$
También es biyectiva la aplicación
\begin{align*}
g&: \{x,y\}\times \{u,w,z\} \to \{1,2,3,4,5,6\} \quad \text{definida } 
\end{align*}
$$
g(x,u)= 1,\, g(x,w) =2,\, g(x,z) =3,\, g(y,u) =4,\, g(y,w) =5,\, g(y,z) =6. 
$$
\end{ejemplo}

\begin{definicion} Dos grafos $G_1$ y $G_2$ se dicen que son {\em isomorfos} cuando  existe una biyección $\alpha$ entre el  \index{grafos isomorfos}  \index{isomorfismo de grafos} conjunto de vértices de $G_1$ y el conjunto de vértices de $G_2$ tal que  si $\{x,y\}$ es una arista de $G_1$ entonces $\{\alpha(x),\alpha(y)\}$ es una arista de $G_2$ y recíprocamente si  $\{z,w\}$ es una arista de $G_2$ entonces $\{\alpha^{-1}(z),\alpha^{-1}(w)\}$ es una arista de $G_1$. La biyección $\alpha$ es llamada un {\em isomorfismo}.
\end{definicion}

\begin{observacion}
	Si $G_1$, $G_2$ grafos y $\alpha$ es un isomorfismo de grafos entre ellos, entonces dados $x,y  \in V_1$,  tenemos que $\{\alpha(x),\alpha(y)\}$ es un 2-subconjunto de  $V_2$. Es decir, podemos definir
	\begin{equation*}
	\begin{matrix}
		\overline{\alpha}: &\text{2-subconjuntos de $V_1$}& \to &\text{2-subconjuntos de $V_2$} \\
		&\{x,y \}& \to &\{\alpha(x),\alpha(y)\}
	\end{matrix}, 
	\end{equation*}
	y el hecho de ser $\alpha$ un isomorfismo de grafos nos dice que $\overline{\alpha}(E_1) \subseteq E_2$ y que
	\begin{equation*}
		\overline{\alpha}: E_1 \to E_2
	\end{equation*}
	es una biyección.
\end{observacion}

Por ejemplo, considere los dos grafos de la Fig. \ref{f5.3}. En este caso hay una biyección entre el conjunto de vértices de $G_1$ y el conjunto de vértices de $G_2$ la cual tiene la propiedad requerida; esta biyección es dada por
$$
\alpha(a)=t,\quad \alpha(b)=v,\quad \alpha(c)=w,\quad \alpha(d)=u.
$$
Podemos comprobar que a cada arista de $G_1$ le corresponde una arista de $G_2$ y vi\-ce\-ver\-sa. Por ejemplo, a la arista $bc$ de $G_1$ le corresponde la arista $vw$ de $G_2$, y así siguiendo. (Usaremos la abreviación $xy$ para la arista $\{x,y\}$, recordando que una arista es un par desordenado, es decir $xy$ es lo mismo que $yx$.)

\begin{figure}[ht]
	\begin{tabular}{llll}
		&
		\begin{tikzpicture}[scale=1]
		%\SetVertexSimple[Shape=circle,FillColor=white]
		\Vertex[x=0,y=0]{$a$}
		\Vertex[x=2,y=0]{$b$}
		\Vertex[x=2,y=-2]{$c$}
		\Vertex[x=0,y=-2]{$d$}
		\Edges($a$, $b$,$c$,$d$,$a$,$b$,$d$)
		\draw (1,-3) node {$G_1$};
		\end{tikzpicture}
		&
		\qquad
		& 
		\begin{tikzpicture}[scale=1]
		%\SetVertexSimple[Shape=circle,FillColor=white]
		\Vertex[x=1,y=0]{$t$}
		\Vertex[x=1,y=-1.3]{$w$}
		\Vertex[x=2,y=-2]{$v$}
		\Vertex[x=0,y=-2]{$u$}
		\Edges($v$, $t$,$u$,$v$,$w$,$u$)
		\draw (1,-3) node {$G_2$};
		\end{tikzpicture}
	\end{tabular}
	\caption{$G_1$ y $G_2$ son isomorfos} \label{f5.3}
\end{figure}

Cuando, como en la Fig. \ref{f5.3}, dos grafos $G_1$ y $G_2$ son isomorfos usualmente nos referiremos a ellos como que son ``el mismo'' grafo.

Para mostrar que dos grafos no son isomorfos, nosotros debemos demostrar que no hay una biyección entre el conjunto de vértices de uno con el conjunto de vértices de otro, que lleve las aristas de uno en las aristas del otro.
Si dos grafos tienen diferente número de vértices, entonces no es posible ninguna biyección, y los grafos no pueden ser isomorfos. Si los grafos tienen el mismo número de vértices, pero diferente número de aristas, entonces hay biyecciones de vértices  pero ninguna de ellas puede ser un isomorfismo. 

\begin{definicion} 
Sea $G=(V,E)$ un grafo. Se dice que $G^{\prime}=(V^{\prime},E^{\prime})$ es {\em subgrafo} de $G=(V,E)$, si  $V^{\prime} \subset V$, $E^{\prime} \subset E$ y $G'$ es grafo.
\end{definicion}

Es claro, pero  no lo demostraremos aquí, que un isomorfismo lleva un subgrafo a un subgrafo isomorfo. Este resultado es una herramienta que puede ser útil para ver si dos grafos no son isomorfos. 
\begin{figure}[ht]
	\begin{tabular}{llll}
		&
		\begin{tikzpicture}[scale=1]
		%\SetVertexSimple[Shape=circle,FillColor=white]
		\Vertex[x=0.00, y=2.00]{$a$}
		\Vertex[x=1.90, y=0.62]{$b$}
		\Vertex[x=1.18, y=-1.62]{$c$}
		\Vertex[x=-1.18, y=-1.62]{$d$}
		\Vertex[x=-1.90, y=0.62]{$e$}
		\Edges($c$, $b$,$a$,$e$,$d$,$b$,$a$,$d$)
		\Edges($e$,$b$)
		\draw (0,-2.2) node {$G_1$};
		\end{tikzpicture}
		&
		\qquad
		& 
		\begin{tikzpicture}[scale=1]
		%\SetVertexSimple[Shape=circle,FillColor=white]
		\Vertex[x=0.00, y=2.00]{1}
		\Vertex[x=1.90, y=0.62]{2}
		\Vertex[x=1.18, y=-1.62]{3}
		\Vertex[x=-1.18, y=-1.62]{4}
		\Vertex[x=-1.90, y=0.62]{5}
		\Edges(1,2,3,4,5,1)
		\Edges(4,2,5)
		\draw (0,-2.2) node {$G_2$};
		\end{tikzpicture}
	\end{tabular}
	\caption{$G_1$ y $G_2$ no son isomorfos} \label{f5.4}
\end{figure}


Por ejemplo, los dos grafos de la Fig. \ref{f5.4} tienen cada uno cinco vértices y siete aristas pero no son isomorfos. Una manera de ver esto es observar que los vértices $a$, $b$, $d$, $e$ forman un subgrafo completo de $G_1$ (cada par de ellos está conectado por una arista). Cualquier isomorfismo debe llevar estos vértices en cuatro vértices de $G_2$ con la misma propiedad, y puesto que no hay tal conjunto de vértices en $G_2$ no puede haber ningún isomorfismo.

\begin{figure}[h]
	\begin{tabular}{llll}
		&
		\begin{tikzpicture}[scale=1]
		\SetVertexSimple[Shape=circle,FillColor=white,MinSize=8 pt]
		\Vertex[x=0.00, y=2.00]{a}
		\Vertex[x=2., y=-1.50]{b}
		\Vertex[x=-2., y=-1.50]{c}
		\Edges(a,b,c,a)
		\Vertex[x=0.00, y=0.85]{1}
		\Vertex[x=1., y=-0.9]{2}
		\Vertex[x=-1., y=-0.9]{3}
		\Edges(1,2,3,1)
		\Edges(a,1,3,c,b,2)
		\draw (0,-2.2) node {$G_1$};
		\end{tikzpicture}
		&
		\qquad
		& 
		\begin{tikzpicture}[scale=0.65]
		\SetVertexSimple[Shape=circle,FillColor=white,MinSize=8 pt]
		%
		\Vertex[x=3.00, y=0.00]{1}
		\Vertex[x=1.50, y=2.60]{2}
		\Vertex[x=-1.50, y=2.60]{3}
		\Vertex[x=-3.00, y=0.00]{4}
		\Vertex[x=-1.50, y=-2.60]{5}
		\Vertex[x=1.50, y=-2.60]{6}
		\Edges(1,2,3,4,5,6,1)
		\Edges(1,4) \Edges(3,6) \Edges(2,5)
		\draw (0,-3.8) node {$G_2$};
		\end{tikzpicture}
	\end{tabular}
	\caption{$G_1$ y $G_2$ no son isomorfos}\label{f5.5}
\end{figure}

Si $G=(V,E)$ grafo,  el \textit{grafo complemento}  es $G' = (V,E')$, donde $E'$ son todos los 2-subconjuntos de $V$ que no están en $E$. Es decir, el grafo complemento tiene los mismos vértices que el grafo original y todas las aristas que le faltan a $G$ para ser grafo completo. Es muy sencillo probar entonces que dos grafos son isomorfos si y sólo sí sus complementos son isomorfos. Esto puede resultar útil a la hora de decidir si dos grafos son isomorfos. Por ejemplo, consideremos los grafos $G_1$ y $G_2$  de la Fig. \ref{f5.5}, entonces,  es fácil ver que el complemento de $G_1$ es un ciclo y  el complemento de $G_2$ no  lo es y por lo tanto  no son isomorfos.   


\begin{subsection}{Ejercicios} \label{ejercicios5.2}
\begin{enumerate}
\item Probar usando subgrafos que los grafos mostrados en la Fig. \ref{f5.5} no son isomorfos.


\item \label{ejercicio5.2.2} Encontrar un isomorfismo entre los grafos definidos por las siguientes listas de adyacencias. (Ambas listas especifican versiones de un grafo famoso conocido como {\em grafo de Petersen}). \index{grafo de Petersen}

$$
\begin{matrix}
a&b&c&d&e&f&g&h&i&j\\ \hline
b&a&b&c&d&a&b&c&d&e\\
e&c&d&e&a&h&i&j&f&g\\
f&g&h&i&j&i&j&f&g&h
\end{matrix}
\qquad \begin{matrix}
0&1&2&3&4&5&6&7&8&9\\ \hline
1&2&3&4&5&0&1&0&2&6\\
5&0&1&2&3&4&4&3&5&7\\
7&6&8&7&6&8&9&9&9&8
\end{matrix}
$$

\item Sea $G=(V,E)$ el grafo definido como sigue. El conjunto de vértices $V$ es el conjunto de todas las palabras de longitud tres en el alfabeto $\{0,1\}$, y el conjunto de aristas $E$ contiene aquellos pares de palabras que difieren exactamente en una posición. Probar que $G$ es isomorfo al grafo formado por las esquinas y aristas de un cubo.
\end{enumerate}
\end{subsection}

\end{section}

\begin{section}{Valencias}\label{5.3}
La {\em {valencia}} de un vértice $v$ en un grafo $G=(V,E)$ es el \index{valencia de un vértice} número de aristas de $G$ que contienen a $v$. Usaremos la notación $\delta(v)$ para la valencia de $v$, formalmente
$$
\delta(v)=|D_v|, \quad \text{ donde } \quad D_v=\{e \in E| v\in
e\}.
$$
El grafo descrito en Fig. \ref{f5.1} tiene $\delta(a)=2$, $\delta(b)=2$, $\delta(c)=1$, $\delta(d)=3$, $\delta(z)=2$. El primer teorema de la teoría de grafos nos dice que la suma de estos números es dos veces el número de aristas.

\begin{teorema}\label{t5.3} La suma de los valores de las valencias $\delta(v)$, tomados sobre todos los vértices $v$ del grafo $G=(V,E)$, es igual a dos veces el número de aristas:
$$
\sum_{v \in V} \delta(v) = 2|E|.
$$
\end{teorema}
\begin{proof} La valencia de un vértice $v$ indica la cantidad de ``extremos'' de aristas que ``tocan'' a $v$. Es claro que hay $2|E|$ extremos de aristas, luego la suma total de las valencias de los vértices es $2|E|$.
\end{proof}

Hay un útil corolario de este resultado. Diremos que un vértice de $G$ es {\em impar} si su  \index{vértice impar}  \index{vértice par} valencia es impar, y {\em par} si su valencia es par. Denotemos $V_i$ y $V_p$ los conjuntos de vértices impares y pares respectivamente, luego $V=V_i \cup V_p$ es una partición de $V$. Por teorema \ref{t5.3}, tenemos que
$$
\sum_{v \in V_i} \delta(v) + \sum_{v \in V_p} \delta(v)= 2|E|.
$$
Ahora cada término en la segunda suma es par, luego esta suma es un número par. Puesto que el lado derecho también es un número par, la primera suma debe ser también par. Pero la suma de números impares solo puede ser par si el número de términos es par. En otra palabras:

\begin{teorema} El número de vértices impares es par.
\end{teorema}

Este resultado es a veces llamado el ``handshaking lemma'' (handshake=estrechar la mano, darse la  \index{handshaking lemma} mano), debido a que se puede interpretar en términos de gente y darse la mano: dado un conjunto de personas, el número de personas que le ha dado la mano a un número impar de miembros del conjunto es par. 

Un grafo en el cual todos los vértices tienen la misma valencia $r$ se llama {\em regular}  \index{grafo regular} (con valencia $r$), o {\em $r$-valente.} En este caso, el resultado del teorema \ref{t5.3} se traduce
$$
r|V|=2|E|.
$$

Muchos de los grafos que aparecen en las aplicaciones son regulares. Ya conocemos los  grafos completos $K_n$; ellos son regulares, con valencia $n-1$. De geometría elemental conocemos los polígonos de $n$ lados, los cuales en teoría de grafos son llamados {\em {grafos cíclicos}}  \index{grafo cíclico} $C_n$. Formalmente, podemos decir que el conjunto de vértices de $C_n$ es $\{1,2,\ldots,n\}$, y las aristas son $\{1,2\},\{2,3\},\ldots,\{n-1,n\},\{n,1\} $. Claramente, $C_n$ es un grafo regular con valencia 2, si $n\ge 3$.

Una aplicación importante de la noción de valencia es en el problema de determinar si dos grafos son o no isomorfos. Si $\alpha:V_1 \to  V_2$ es un isomorfismo entre $G_1$ y $G_2$, y $\alpha(v)=w$, entonces cada arista que contiene a $v$ se transforma en una arista que contiene a $w$. En consecuencia $\delta(v)=\delta(w)$. Por otro lado, si $G_1$ tiene un vértice $x$, con valencia $\delta(x)=\delta_0$, y $G_2$ no tiene vértices con valencia $\delta_0$, entonces $G_1$ y $G_2$ no pueden ser isomorfos. Esto nos da otra manera para distinguir los grafos de la Fig \ref{f5.4}, puesto que el primer grafo tiene un vértice de valencia 1 y el segundo no.

Una extensión de esta idea se da en la siguiente proposición.

\begin{proposicion}\label{criterioiso}Sean  $G_1$ y $G_2$ grafos isomorfos. Para cada $k\ge 0$ sea $n_i(k)$ el número de vértices de $G_i$ que tienen valencia $k$ ($i=1,2$). Entonces $n_1(k)=n_2(k)$.
\end{proposicion}
\begin{proof} Hemos visto más arriba que si $\alpha:V_1 \to  V_2$ es un isomorfismo entre $G_1$ y $G_2$ y $v\in V_1$, entonces $\delta(v)=\delta(\alpha(v))$. Luego la cantidad de vértices con valencia $k$ en $G_1$ es igual  a la cantidad de vértices con valencia $k$ en $G_2$.     
\end{proof}

\begin{ejemplo} Revisemos los grafos de la Fig. \ref{f5.4} y la Fig. \ref{f5.5} de la sección anterior. 

Los dos grafos de la Fig. \ref{f5.4}  no son isomorfos debido a que en el primer grafo existen tres vértices con valencia 3 mientras que en el segundo existen sólo dos.

Observar que los criterios vistos hasta ahora relativos a cantidad de vértices,  cantidad de aristas y valencias, incluyendo el de la proposición \ref{criterioiso}, no son útiles para determinar si los grafos de  la Fig. \ref{f5.5} son isomorfos o no: ambos tienen 6 vértices, 9 aristas y todos los vértices son de valencia 3. Sin embargo, en el caso de la Fig. \ref{f5.5} podemos determinar que los grafos no son isomorfos observando los subgrafos de cada uno. Ahora bien, no  hay ningún criterio general eficiente para determinar si dos grafos son isomorfos o no: en los casos difíciles esencialmente debemos probar con todas las biyecciones posibles de los vértices de un grafo a los vértices del otro y eso es no computable para casos no demasiado  grandes.   
  
 
\end{ejemplo}

\begin{subsection}{Ejercicios}\label{ejercicios5.3}
\begin{enumerate}
\item ?` Es posible que las siguientes listas sean las valencias de todos los vértices de un
grafo? Si así lo fuera, dar una representación pictórica de tal
grafo. (Recuerde que hay a lo más una arista que una un par de
vértices dados.)
$$
\begin{aligned}
\text{(i)} \,\,\, 2,2,2,3. \qquad &\text{(ii)} \,\,\, 1,2,2,3,4. \\
\text{(iii)} \,\,\,2,2,4,4,4. \qquad &\text{(iv)} \,\,\, 1,2,3,4.
\end{aligned}
$$
\item  Si $G$ es un grafo que tiene $n$ vértices y sus valencias son $d_1,d_2,\ldots,d_n$, ?`cuáles son las valencias de $G^c$, el grafo complemento de $G$?
\item Encuentre todos los grafos posibles (no isomorfos) que pueda, que sean regulares, 4-valentes y con 7 vértices. [Ayuda: considere el complemento de esos grafos.]
\item Probar que si $G$ es un grafo con al menos dos vértices, entonces $G$ tiene dos vértices con la misma valencia.
\end{enumerate}

\end{subsection}

\end{section}

\begin{section}{Caminos y ciclos}\label{5.4}
Comenzaremos esta sección analizando el \textit{problema de los puentes de Königsberg}, resuelto por Leonhard Euler en 1736 y cuya resolución dio origen a la teoría de grafos. Simplificaremos el enunciado del problema diciendo que tenemos 4 islas unidas por 7 puentes como se muestra en la figura \ref{figure-7-puentes}.
\begin{figure}[ht]
\begin{tikzpicture}
[lineDecorate/.style={-,thick},%                                                
nodeDecorate/.style={shape=circle,inner sep=2pt,draw,thick}]
%% nodes or vertices                                                            
\foreach \nodename/\x/\y/\direction/\navigate in {
	a/0/0/left/west, b/0/2/left/west, c/0/4/left/west, d/4/2/right/east}
{
	\node (\nodename) at (\x,\y) [nodeDecorate] {};
	\node [\direction] at (\nodename.\navigate) {\footnotesize$\nodename$};
}
%% edges or lines                                                               
\path
\foreach \startnode/\endnode in {a/d, b/d, c/d}
{
	(\startnode) edge[lineDecorate] node {} (\endnode)
}
\foreach \startnode/\endnode in {a/b, b/c, c/b, b/a}
{
	(\startnode) edge[lineDecorate,bend left] node {} (\endnode)
};
\end{tikzpicture}
\caption{Los 7 puentes de Königsberg} \label{figure-7-puentes}
\end{figure}

Euler se preguntó entonces, si era posible dar un paseo comenzando desde alguna de estas islas, pasando por todos los puentes, recorriendo sólo una vez cada uno, y regresando al mismo punto de partida. Podríamos resolver el problema por \textit{fuerza bruta}, probando todos los recorridos posibles, pero este método es demasiado costoso y, hecho manualmente,  propenso al error.

Euler determinó que tal recorrido no era posible usando el siguiente razonamiento: las islas intermedias de un recorrido posible necesariamente han de estar conectadas a un número par de puentes. En efecto, si llegamos a una isla desde algún puente, entonces el único modo de salir de esa isla es por un puente diferente. Esto significa que tanto la isla inicial como la isla final serían las únicas que podrían estar conectadas con un número impar de puentes. Sin embargo, el requisito adicional del problema dice que partimos de una isla y regresamos a ella al final del recorrido, por lo que no podría existir ninguna isla conectada con un número impar de puentes. Como todas las islas están conectadas con un número impar de puentes, no hay solución posible.

Euler generalizó el razonamiento anterior, ya usando el lenguaje de vértices y aristas, y también enunció, y dio por cierta,  la recíproca: si a todo vértice concurren un número par de aristas,  entonces existe una caminata que pasa por todas las aristas una sola vez.  Estos resultados fueron publicados en la revista  \textit{Commentarii academiae scientiarum Petropolitanae} (v.8, 1741) en el trabajo \textit{``Solutio problematis ad geometriam situs pertinentis''}, el cuál dio origen a la teoría de grafos. En ese trabajo también se prueba el ``handshaking lemma'' mencionado en la sección anterior.  La  recíproca fue probada por  Hierholzer en el año 1871.

Aunque el grafo correspondiente al problema de los puentes de Königsberg es un grafo multigrafo, un grafo donde puede haber varias aristas entre dos vértices, el razonamiento de Euler puede aplicarse a los grafos por nosotros definidos (teorema \ref{teo-caminata-euleriana}).

\vskip .3cm

Cómo en el caso problema anterior, frecuentemente usamos grafos como modelos de situaciones prácticas que involucran rutas: los vértices representan ciudades, islas o cruces, y cada arista representa una ruta o cualquier otro forma de comunicación. Las definiciones de esta sección se comprenderán mejor con esta clase de ejemplo en mente.

\begin{definicion} \rm Una {\em caminata} en un grafo $G$ es  \index{caminata} una secuencia de vértices
$$
v_1,v_2,\ldots,v_k,
$$
tal que $v_i$ y $v_{i+1}$ son adyacentes ($1 \le i \le k-1$). Si todos los vértices son distintos, una caminata es llamada un {\em camino}.  \index{camino}

Llamaremos {\em ciclo} a una caminata  \index{ciclo} $v_1,v_2,\ldots,v_{r+1}$ cuyos vértices son distintos exceptuando los extremos, es decir que $v_1=v_{r+1}$ y  a menudo diremos que es un {\em $r$-ciclo}, o un ciclo de {\em longitud} $r$ en $G$.  \index{longitud de un ciclo}
\end{definicion}

Es decir, una caminata especifica una ruta en $G$: del primer vértice vamos a uno adyacente, de éste a otro adyacente y así siguiendo. En una caminata podemos visitar cualquier vértice varias veces, y en particular, podemos ir de un vértice $x$ a otro $y$ y luego tomar la dirección contraria y regresar a $x$. En un camino, cada vértice es visitado solo una vez.

Escribamos $x \sim y$ siempre y cuando los vértices $x$ e $y$ de $G$ puedan ser unidos por un camino en $G$: hablando en forma rigurosa, esto significa que hay un camino $v_1,v_2,\ldots,v_k$ en $G$ con $x=v_1$ e $y=v_k$. 

\begin{lema} Sea $G$ un grafo. $x \sim y$ si  y sólo si $x$ e $y$ pueden ser unidos por una caminata 
\end{lema}
\begin{proof}Es claro que si $x$ e $y$ están unidos por un camino, como  un camino es un caso especial de caminata, $x$ e $y$ están unidos por una caminata.

Veamos que si  $x$ e $y$ están unidos por una caminata, entonces están unidos por un camino. Sea 
$$
x=x_1,x_2,\ldots,x_k=y,
$$
una caminata entre  $x$ e $y$. Si ninguno de los $x_i$ se repite, entonces tenemos un camino y terminado el problema. Si hay repetición, entonces existe $j$ tal que $x_j = x_{j+r}$ con $r >0$, es decir tenemos una caminata
$$
x=x_1,x_2,\ldots,x_j,\ldots,x_{j+r},\ldots, x_k=y,
$$
Como $x_j = x_{j+r}$ podemos eliminar la sub-caminata $x_{j+1},\ldots,x_{j+r}$ (un ``bucle'' dentro de la caminata) y nos queda 
$$
x=x_1,x_2,\ldots,x_j,x_{j+r+1},\ldots, x_k=y,
$$
una caminata, más corta,  entre $x$ e $y$. Podemos repetir este procedimiento hasta eliminar todos los ``bucles'' y obtener un camino.
\end{proof}

Debido a este lema es sencillo verificar la siguientes propiedades: sea $G$ y $x,y,z$ vértices de $G$, entonces
\begin{enumerate}
\item[({\em i})] $x \sim x$ (reflexividad de $\sim$).
\item[({\em ii})] $x \sim y$, entonces $y \sim x$ (simetría de $\sim$).
\item[({\em iii})]  $x \sim y$,  $y \sim z$, entonces  $x \sim z$ (transitividad  de $\sim$).
\end{enumerate}

En un lenguaje formal, una relación que  cumple las tres propiedades anteriores es llamada una  {\em relación de equivalencia} del conjunto, en este caso tenemos una relación de equivalencia del conjunto de vértices $V$ de $G$. 

Estas tres propiedades nos permiten partir a $V$ en conjuntos disjuntos: dos vértices están en el mismo conjunto si ellos pueden ser unidos por un camino, y están en conjuntos diferentes si no podemos encontrar tal camino. llamaremos a estos conjuntos disjuntos las {\em clases de equivalencia de $\sim$}.

\begin{definicion}Supongamos que $G=(V,E)$ es un grafo y que la partición de $V$ en las clases de equivalencia de $\sim$ es
$$
V= V_1 \cup V_2 \cup \cdots \cup V_r.
$$
Denotemos con $E_i$ ($1\le i \le r$) al subconjunto de $E$ que contiene todas las aristas cuyos finales están en $V_i$. Entonces los grafos $G_i=(V_i,E_i)$ son llamados las {\em componentes}  \index{componente de un grafo}  \index{grafo conexo} de $G$. Si $G$ tiene solo una componente diremos entonces que el grafo es {\em conexo}.
\end{definicion}

\begin{figure}[t]
\begin{tikzpicture}[scale=1.1]
\SetVertexSimple[Shape=circle,FillColor=white,MinSize=8 pt]
\Vertex[x=6, y=-0.5]{1}
\Vertex[x=5.5, y=-2]{2}
\Vertex[x=7, y=-3.5]{3}
\Vertex[x=7, y=-2]{4}
\Vertex[x=11, y=-2]{5}
\Vertex[x=6.50, y=-1.2]{6}
\Vertex[x=6.50, y=-2.5]{7}
\Vertex[x=9.5, y=-0.8]{8}
\Vertex[x=8.5, y=-2.5]{9}
\Vertex[x=10, y=-3.3]{a}
\Edges(5,1,2,4,5,3,2,4,3,5,4)
\Edges(a,8,9,7,6,8,9)
\end{tikzpicture}
\caption{Un grafo con dos componentes} \label{f5.6}
\end{figure}

La terminología casi explica por si misma el significado de estas definiciones. El grafo mostrado en la Fig. \ref{f5.6} tiene dos componentes, y por consiguiente no es conexo. La descomposición de un grafo en componentes es muy útil, puesto que muchas propiedades de los grafos pueden ser establecidas considerando las componentes separadamente. Por esta razón, teoremas acerca de grafos a menudo son probados solo para la clase de grafos conexos.

Cuando tenemos una representación pictórica de un grafo de moderado tamaño, es bastante fácil determinar si es o no conexo. Sin embargo, cuando un grafo viene definido por una lista de adyacencia y hay una gran número de vértices y aristas, determinar si el grafo es conexo no es un problema sencillo, aunque existen  algoritmos eficientes para resolverlo.

\begin{ejemplo}\label{chunner} Leandro y Juan, dos amigos, planean tomar sus vacaciones en determinada isla. La Fig. \ref{f5.7} representa los lugares de interés turístico de la isla y las carreteras que los unen. Leandro es un turista por naturaleza, y desea visitar cada lugar una vez y volver al punto de partida. Juan es un explorador, y desea atravesar todos los caminos solo una vez, a él lo tiene sin cuidado si regresa o no al lugar del cual partió. ?`Podrán encontrar las rutas que desean Leandro y Juan?
\end{ejemplo}

\begin{figure}[ht]
	\begin{tikzpicture}[scale=1]
	%\SetVertexSimple[Shape=circle,FillColor=white]
	\def\rvar{1.2}
	\Vertex[x=0.00, y=-2.00]{$u$}
	\Vertex[x=\rvar*1.90, y=-0.62]{$t$}
	\Vertex[x=\rvar*1.18, y=1.62]{$q$}
	\Vertex[x=-1.18*\rvar, y=1.62]{$p$}
	\Vertex[x=-1.90*\rvar, y=-0.62]{$r$}
	\Vertex[x=0, y=0]{$s$}
	\Edges($u$,$t$,$q$,$p$,$r$,$u$,$s$,$t$,$r$,$s$,$q$,$r$,$p$,$t$,$s$,$u$,$s$,$p$)
\end{tikzpicture}
	\caption{El gran tour} \label{f5.7}
\end{figure}

\begin{proof}[Solución] Leandro puede usar diferentes rutas para alcanzar su objetivo: una posibilidad es el ciclo $p,q,t,s,u,r,p$.

Sin embargo, Juan está en un apuro. Llamemos $x$ al punto de partida y llamemos $y$ al punto de llegada, y supongamos por el momento que $x \not= y$. Entonces él usa una arista con extremo en $x$ para partir y cada vez que vuelve a $x$ debe arribar y partir por nuevas aristas. Luego, usa un número impar de aristas con extremo en $x$, y por consiguiente $x$ debe ser un vértice impar. De manera análoga, $y$ debe ser también un vértice impar, puesto que Juan usa dos aristas cada vez que pasa por $y$, y una más al finalizar en $y$. Los restantes vértices deben ser pares, puesto que cada vez que Juan llega a un vértice intermedio parte de nuevo, y por consiguiente usa dos aristas.

Resumiendo, una ruta para Juan que empiece y finalice en vértices distintos $x$ e $y$, es solo posible si hay dos vértices impares (que son $x$ e $y$) y el resto de los vértices es par.  Pero en el grafo de la Fig. \ref{f5.7} el valor de las valencias es: $\delta(p)=4$, $\delta(q)=4$, $\delta(r)=5$, $\delta(s)=5$, $\delta(t)=5$, y $\delta(u)=3$. Luego hay demasiados vértices impares, y por lo tanto no existe la ruta que Juan desea. Si permitimos la posibilidad de que $x=y$ , la situación es aún peor, pues en este caso todos los vértices deberían ser pares.
\end{proof}

En general, la ruta de Leandro es un ciclo que contiene todos los vértices del grafo dado. Tales ciclos fueron estudiados por el matemático irlandés W.R. Hamilton (1805-65),  \index{Hamilton, W. R.} y en consecuencia un ciclo con esta propiedad es llamado un {\em ciclo hamiltoniano}. En nuestro ejemplo, fue muy fácil \index{ciclo hamiltoniano} encontrar un ciclo hamiltoniano, pero este fue un caso muy especial y no representativo. Para ciertos grafos, puede ser un problema difícil decidir si un ciclo hamiltoniano existe o no.

Por otro lado, el problema de Juan puede ser fácilmente resuelto. Una caminata que use cada arista de un grafo solo una vez es llamada una {\em caminata euleriana}, debido a que Euler \index{caminata euleriana} fue el primero en estudiar estas caminatas y encontró que si $x\not= y$, una condición necesaria para que exista una caminata euleriana que comience en $x$ y finalice en $y$ es que $x$ e $y$ deben ser vértices impares y el resto debe ser par, mientras que si $x=y$ la condición es que todos los vértices deben ser pares. Es decir que una condición necesaria para que exista una caminata euleriana en un grafo $G$ es que $G$ debe tener a lo más dos vértices impares. Más aún, puede probarse que esta condición es también suficiente. Puesto que es sencillo calcular las valencias de los vértices de un grafo, es relativamente sencillo decidir si un grafo tendrá o no una caminata euleriana. 

Resumiendo las definiciones de más arriba:

\begin{definicion}
Un {\em ciclo hamiltoniano} en un grafo $G$ es un ciclo que contiene a todos los vértices del grafo.

Una {\em caminata euleriana} en un grafo $G$ es un caminata que usa todas las aristas de $G$ exactamente una vez. Una caminata euleriana que comienza y termina en un mismo vértice se llama también {\em circuito euleriano}.
\end{definicion}

El siguiente teorema resume los resultados sobre caminatas eulerianas. La demostración no es demasiada complicada, pero excede los alcances de este curso.  

\begin{teorema}\label{teo-caminata-euleriana} Un grafo conexo con más de un vértice posee una caminata euleriana de $v$ a $w$, con $v \not= w$ si y sólo si $v$ y $w$ son los únicos vértices de grado impar. Un grafo conexo con más de un vértice tiene un circuito euleriano si y sólo si todos los vértices tienen grado par.
\end{teorema}
\begin{proof}[Demostración (*)]
	La existencia de un circuito euleriano implica que todos los vértices tienen valencia par y esto se demuestra con un razonamiento completamente análogo al que vimos al comienzo de la sección en el problema de los puentes de Königsberg. 
	
	
	Ahora demostraremos la existencia de circuitos eulerianos en grafos donde todas las valencias son pares (un grafo par). 
	
	Observemos lo siguiente: 1) dado un vértice $v$ de un grafo par (conexo o no conexo), si hacemos una caminata que no repite aristas, solo se puede ``trabar'' en $v$, pues la paridad de las valencias nos garantiza que cuando entramos a un vértice $w \not= v$ siempre habrá una arista disponible para salir. 2) Si eliminamos las aristas recorridas en la caminata anterior nos queda un subgrafo par, posiblemente no conexo. Esto se debe a que de cada vértice estamos eliminando un número par de aristas. 
	
	En  base a estas dos observaciones se puede hacer un algoritmo (el \textit{algoritmo de  Hierholzer}) que nos permite encontrar un circuito euleriano.
	\begin{enumerate}[(a)]
		\item Elija un vértice de partida $v$ y vaya recorriendo el grafo por aristas no visitadas hasta que no pueda elegir una arista no visitada. Cuando esto ocurra usted estará, por 1)  de más arriba, de nuevo en $v$, pero puede que no haya visitado todas las aristas. 
		\item Mientras exista un vértice $u$ que pertenezca a la caminata pero del cual salga una arista no visitada, comience otra caminata en $u$ que vaya recorriendo aristas no visitadas hasta quedar trabado en $u$. Las observaciones 1) y 2) nos garantizan que esta caminata terminará en $u$. Inserte esta nueva caminata a la anterior. 
	\end{enumerate} 
	Finalizadas estas iteraciones, se obtiene una caminata euleriana. 
	
	En  el caso que haya dos vértices, $v$ y $w$, de valencia impar y todos los demás de valencia par, podemos reducirlo al caso anterior, eliminando   $\{v,w\}$ si esta arista está en el grafo o  agregando  $\{v,w\}$ si la arista no lo está. Sin embargo, es más fácil pensarlo directamente en forma algorítmica, siendo el algoritmo exactamente el mismo (en este caso en (a) la caminata se traba en $w$ en vez de $v$).   
\end{proof}


Más allá del resultado y la demostración del teorema, repetiremos el algoritmo, eficiente y fácilmente implementable,  para encontrar caminatas o circuitos eulerianos:
\begin{itemize}
	\item elija un vértice de partida $v$ y vaya recorriendo el grafo por aristas no visitadas hasta que no pueda elegir una arista no visitada. 
	\item Mientras exista un vértice $u$ que pertenezca a la caminata pero del cual salga una arista no visitada, comience otra caminata en $u$ que vaya recorriendo aristas no visitadas hasta quedar atascado en $u$. Inserte esta nueva caminata a la anterior. 
\end{itemize}


\begin{ejemplo} Dado  el  grafo de de la  Fig. \ref{f5.7.1}, encontrar una caminata euleriana con origen en $p$ y final en $r$. 

\begin{figure}[ht]
	\begin{tikzpicture}[scale=1]
	%\SetVertexSimple[Shape=circle,FillColor=white]
	\def\rvar{1.2}
	\Vertex[x=0.00, y=-2.00]{$u$}
	\Vertex[x=\rvar*1.90, y=-0.62]{$t$}
	\Vertex[x=\rvar*1.18, y=1.62]{$q$}
	\Vertex[x=-1.18*\rvar, y=1.62]{$p$}
	\Vertex[x=-1.90*\rvar, y=-0.62]{$r$}
	\Vertex[x=0, y=0]{$s$}
	\Edges($u$,$t$,$q$,$p$)
	\Edges($r$,$u$)
	\Edges($s$,$t$)
	\Edges($r$,$s$,$q$,$r$)
	\Edges($p$,$t$,$s$)
	\Edges($s$,$p$)
	\end{tikzpicture}
	\caption{El gran tour, reducido} \label{f5.7.1}
\end{figure}
Debemos primero observar que la caminata euleriana debe existir pues $\delta(p)=3$, $\delta(q)=4$, $\delta(r)=3$, $\delta(s)=4$, $\delta(t)=4$, y $\delta(u)=2$. 

Como fue dicho más arriba,  el algoritmo que podemos utilizar es: partimos de $p$ y recorremos el grafo sin repetir aristas. Cuando  no podamos avanzar más (bajo estas condiciones) habremos obtenido la caminata euleriana o,  en caso contrario, iteramos el procedimiento en el subgrafo que se obtiene eliminando las aristas de la caminata previa. Una posible solución es, por ejemplo,
$$
p,s,q,p,t,u,r,s,t,q,r
$$  

Existen muchas posibles caminatas eulerianas de $p$ a $r$, por ejemplo le proponemos que encuentre una cuya primera arista sea $\{p,t\}$.
\end{ejemplo}

\begin{observacion}(*) Escribiremos en pseudocódigo, basado en Python, el algoritmo para hallar una caminata euleriana. Partimos de un grafo $G$ con $n$ vértices tal que o bien 1) todos los vértices son de valencia par, o bien 2) solo hay dos vértices de valencia impar. Tomamos como comienzo del recorrido a un vértice arbitrario $v$, con la restricción en el caso 2) de que $\delta(v)$ sea  impar.  
\vskip .4cm
\begin{minipage}{400pt}
\noindent {\sc Caminata euleriana }
\vskip .2cm
\begin{small}
\begin{verbatim}
# pre: G grafo, v vértice de G (G y v en las condiciones de arriba)
# post: devuelve 'recorrido' la lista de vertices de la caminata
#       euleriana. La caminata empieza en v.
recorrido = [v]  # caminata
libres = lista de adyacencia de G # libres[u] = lista de adyacentes a u
while libres != emptyset: # libres[u] no vacío para algún u
   w = primer vértice de recorrido tal que libres[w] no vacío
   pos = w  
   rec_aux = [w] 
   while len(libres[pos]) > 0:
      prox = libres[pos][0]  # {pos, prox} es una arista no utilizada
      rec_aux.append(prox)  # agrega prox a rec_aux 
      libres = quitar de libres la arista 'pos, prox'
      pos = prox  # nos ubicamos en el próximo vértice
   recorrido = inserta rec_aux en w en recorrido
\end{verbatim}
\end{small}
\end{minipage}
\vskip .4cm
En todo  el programa {\tt libres} es una lista de adyacencia que nos va dando las aristas no utilizadas. Es decir si $w$ es un vértice {\tt libres[w]} es una lista de los vértices $u$ tal que la arista $wu$ no ha sido utilizada. 



\end{observacion}

\begin{subsection}{Ejercicios}\label{ejercicios5.4}
\begin{enumerate}
\item Encontrar el número de componentes de el grafo cuya lista de adyacencia es
$$
\begin{matrix}
a&b&c&d&e&f&g&h&i&j\\ \hline
f&c&b&h&c&a&b&d&a&a\\
i&g&e&&g&i&c&&f&f\\
j&&g&&&j&e&&&
\end{matrix}
$$

\item ?`Cuántas componentes conexas tiene el grafo de la fiesta de Abril (sección \ref{5.1})?
\item Encontrar un ciclo hamiltoniano en el grafo formado por los vértices y aristas de un
cubo.
\item El año que viene el Leandro y Juan desean visitar otra isla, donde los lugares interesantes y las caminos que los unen están representados por el grafo que tiene la siguiente lista de adyacencia
$$
\begin{matrix}
0&1&2&3&4&5&6&7&8\\ \hline
1&0&1&0&3&0&1&0&1\\
3&2&3&2&5&4&5&2&3\\
5&6&7&4&&6&7&6&5\\
7&8&&8&&8&&8&7.
\end{matrix}
$$
?`Es posible encontrar rutas para Leandro y Juan que satisfagan lo pedido en el  Ejemplo \ref{chunner}?
\item Un ratón intenta comer un $3\times 3\times 3$ cubo de queso. Él comienza en una esquina y come un subcubo de $1\times 1\times 1$, para luego pasar a un subcubo  adyacente. ?`Podrá el ratón terminar de comer el queso en el centro?
\end{enumerate}
\end{subsection}

\end{section}

\begin{section}{Árboles}\label{5.5}
\begin{definicion} Diremos que un grafo $T$ es un {\em árbol} si cumple 
\index{árbol}
\begin{enumerate}
\item[(T1)] \label{T1}$T$ es conexo y no hay ciclos en $T$.
\end{enumerate}
\end{definicion}

Algunos árboles típicos han sido dibujados en la Fig. \ref{f5.8}. A causa de su particular estructura y propiedades, los árboles aparecen en diversas aplicaciones de la matemática, especialmente en investigación operativa y ciencias de la computación. Comenzaremos el estudio de ellos estableciendo algunas propiedades sencillas.

\begin{figure}[ht]
	\begin{tabular}{llllllll}
		&
		\begin{tikzpicture}[scale=1]
		\SetVertexSimple[Shape=circle,FillColor=white,MinSize=8 pt]
		\Vertex[x=0.00, y=0]{a}
		\Vertex[x=0, y=-1]{b}
		\Vertex[x=0., y=-2]{c}
		\Vertex[x=0, y=-3]{d}
		\Vertex[x=0., y=-4]{e}
		\Edges(a,b,c,d,e)
		\end{tikzpicture}
		&
		\qquad
		& 
		\begin{tikzpicture}[scale=1]
		\SetVertexSimple[Shape=circle,FillColor=white,MinSize=8 pt]
		%
		\Vertex[x=0.00, y=0]{a}
		\Vertex[x=-1.5, y=-0.5]{b}
		\Vertex[x=1.5, y=-0.5]{c}
		\Vertex[x=-1.5, y=-1.5]{d}
		\Vertex[x=1.5, y=-1.5]{e}
		\Vertex[x=0, y=-1.5]{f}
		\Vertex[x=-0.7, y=-1]{g}
		\Vertex[x=0.7, y=-1]{h}
		\Vertex[x=0, y=-4]{i}
		\Edges(d,b,a,c,e)
		\Edges(g,f,h)
		\Edges(a,f,i)
		\end{tikzpicture}
		&
		\qquad
		& 
		\begin{tikzpicture}[scale=1]
		\SetVertexSimple[Shape=circle,FillColor=white,MinSize=8 pt]
		%
		\Vertex[x=0.00, y=0]{a}
		\Vertex[x=0, y=-1.0]{b}
		\Vertex[x=0, y=-2.5]{c}
		\Vertex[x=1.2, y=-2]{e}
		\Vertex[x=-1.2, y=-2]{f}
		\Vertex[x=-1.2, y=-3.5]{g}
		\Vertex[x=1.2, y=-3.5]{h}
		\Edges(a,b,c)
		\Edges(f,b,e)
		\Edges(g,c,h)
		\end{tikzpicture}
		&
		\qquad
		& 
		\begin{tikzpicture}[scale=0.65]
		\SetVertexSimple[Shape=circle,FillColor=white,MinSize=8 pt]
		%
		\Vertex[x=0.00, y=0.00]{0}
		\Vertex[x=3.00, y=0.00]{1}
		\Vertex[x=2.12, y=2.12]{2}
		\Vertex[x=0.00, y=3.00]{3}
		\Vertex[x=-2.12, y=2.12]{4}
		\Vertex[x=-3.00, y=0.00]{5}
		\Vertex[x=-2.12, y=-2.12]{6}
		\Vertex[x=0.00, y=-3.00]{7}
		\Vertex[x=2.12, y=-2.12]{8}
		\Edges(1,0,5) \Edges(3,0,7) \Edges(2,0,6)\Edges(4,0,8)
		\end{tikzpicture}
	\end{tabular}
	\caption{Algunos árboles} \label{f5.8}
\end{figure}

\vskip .3cm

El siguiente lema nos resultará útil para probar una parte del teorema fundamental de esta sección.

\begin{lema}\label{conv} Sea $G=(V,E)$ un grafo conexo, entonces $|E| \ge |V| -1$.  
\end{lema}
\begin{proof} Como $G$ es conexo existe una caminata que recorre todos los vértices de $G$:
$$
v_1,v_2,\ldots,v_r.
$$
Renombremos los vértices de $G$ con números naturales de tal forma que el primer vértice de la caminata sea 1, el segundo 2 y cada vez que aparece un vértice que no ha sido renombrado se le asigna el número siguiente. Luego la caminata comienza en 1 y termina en $n$, donde $n = |V|$.  Observar que cada vez que renombramos un vértice (excepto el primero) su antecesor es menor, es decir dado $i$ tal que $1 < i \le n$ tenemos que la caminata tiene la forma
$$
1,\ldots,j_i,i,\ldots,j_n,n
$$ 
donde $j_i < i$, luego es claro  que 
$$
\{j_{2},2\}, \{j_{3},3\}, \ldots, \{j_{n},n\}
$$
forman un conjunto de $n-1$ aristas distintas en $G$. 
\end{proof}


\begin{lema}\label{lema-T1->T3}
	Sea T es un árbol,  entonces  el grafo obtenido a partir de $T$ removiendo una arista tiene dos
	componentes, cada una de las cuales es un árbol.
\end{lema}
\begin{proof}
	Supongamos que $uv$ es una arista en $T$, y sea $S=(V,E')$ el grafo con el mismo conjunto de vértices que $T$ y con el conjunto de aristas $E'=E-uv$. Sea $V_1$ el conjunto de los vértices $x$ de $T$ para los cuales existe un único camino en $T$ de $x$ a $v$ que pasa por $u$. Claramente, este camino debe finalizar con la arista $uv$, pues sino $T$ tendría un ciclo. Sea $V_2$ el complemento de $V_1$ en $V$.
	
	Cada vértice en $V_1$ se une por un camino en $S$ a $u$, y cada vértice en $V_2$ se une por un camino en $S$ a $v$, pero no existe camino de $u$ a $v$ en $S$. Se sigue entonces que $V_1$ y $V_2$ son las dos componentes del conjunto de vértices de $S$. Cada componente es conexa (por definición), y no contiene ciclos, pues sino habría ciclos en $T$. Es decir que las dos componentes son árboles.
	
\end{proof}

\begin{teorema}\label{teo-arboles} Si $T=(V,E)$ es un grafo conexo con al menos dos vértices, entonces son equivalentes las siguientes propiedades
\begin{enumerate}
\item[(T1)] T es un árbol.
\item[(T2)] \label{T2} Para cada par $x$, $y$ de vértices existe un único camino en $T$ de $x$ a
$y$.
\item[(T3)] \label{T4} $|E|=|V|-1$.
\end{enumerate}
\end{teorema}
\begin{proof} (T1) $\Rightarrow$ (T2): Puesto que $T$ es conexo, existe un camino de $x$ a $y$, digamos
$$
x=v_0,v_1,\ldots,v_r=y.
$$
Si existiera otro camino, digamos
$$
x=u_0,u_1,\ldots,u_s=y,
$$
consideremos $i$ el más pequeño subíndice para el cual se cumple que $u_{i+1}\not=v_{i+1}$ (Fig. \ref{f5.9}).

\begin{figure}[ht]
	\begin{tikzpicture}[scale=1]
	\SetVertexSimple[Shape=circle,FillColor=white,MinSize=5 pt]

	%
	%\ponertz{-7}{-30}{$x=$}
	\draw (-0.6,0) node {$x = $};
	\draw (0,0.4) node {$v_0$};
	\draw (0,-0.4) node {$u_0$};
	\Vertex[x=0.00, y=0]{v0}
	\Vertex[x=1, y=0]{v1}
	\draw (1,0.4) node {$v_1$};
	\draw (1,-0.4) node {$u_1$};
	\Vertex[x=3, y=0]{vi}
	\draw (2.9,0.4) node {$v_i$};
	\draw (2.9,-0.4) node {$u_i$};
	\Vertex[x=3.7, y=1]{vi1}
	\draw (3.7,1.4) node {$v_{i+1}$};
	\Vertex[x=3.7, y=-1]{ui1}
	\draw (3.7,-1.4) node {$u_{i+1}$};
	\Edges(v0,v1)
	\Edges(vi,vi1)
	\Edges(vi,ui1)
	\Vertex[x=6, y=1]{vj1}
	\draw (6,1.4) node {$v_{j-1}$};
	\Vertex[x=6, y=-1,style=white]{uk1}
	\draw (6,-1.4) node {$u_{k-1}$};
	\Vertex[x=6.7, y=0]{vj}
	\draw (6.8,0.4) node {$v_j$};
	\draw (6.8,-0.4) node {$u_k$};
	\Vertex[x=8.7, y=0]{vr}
	\draw (8.7,0.4) node {$v_r$};
	\draw (8.7,-0.4) node {$u_s$};
	\Edges(vj1,vj)
	\Edges(uk1,vj)
	\draw (9.3,0) node {$=y$};

	\SetVertexNormal[LineColor=white]
	\Vertex[x=1.8, y=0]{s1}
	\Vertex[x=2.2, y=0]{s2}
	\Edges(v1,s1)
	\Edges(vi,s2)
	\draw (1.7,0) node {$\scriptstyle\bullet$};
	\draw (2,0) node {$\scriptstyle\bullet$};
	\draw (2.3,0) node {$\scriptstyle\bullet$};
	\Vertex[x=4.5, y=1]{s11}
	\Vertex[x=4.5, y=-1]{s12}
	\Edges(vi1,s11)
	\Edges(ui1,s12)
	\Vertex[x=5.2, y=1]{s21}
	\Vertex[x=5.2, y=-1]{s22}
	\Edges(vj1,s21)
	\Edges(uk1,s22)
	\draw (4.45,1) node {$\scriptstyle\bullet$};
	\draw (4.85,1) node {$\scriptstyle\bullet$};
	\draw (5.25,1) node {$\scriptstyle\bullet$};
	\draw (4.45,-1) node {$\scriptstyle\bullet$};
	\draw (4.85,-1) node {$\scriptstyle\bullet$};
	\draw (5.25,-1) node {$\scriptstyle\bullet$};
	\Vertex[x=7.5, y=0]{s31}
	\Vertex[x=7.9, y=0]{s32}
	\Edges(vj,s31)
	\Edges(vr,s32)
	\draw (7.4,0) node {$\scriptstyle\bullet$};
	\draw (7.7,0) node {$\scriptstyle\bullet$};
	\draw (8.0,0) node {$\scriptstyle\bullet$};
	\end{tikzpicture}
	%fig 5.10
	\caption{Dos caminos diferentes determinan un ciclo} \label{f5.9}
\end{figure}

Puesto que ambos caminos finalizan en $y$ ellos se encontrarán de nuevo, y entonces podemos definir $j$ como el más pequeño subíndice tal que
$$
j>i \quad \text{ y } \quad v_j=u_k \quad \text{ para algún } k.
$$
Entonces $v_i,v_{i+1},\ldots,v_j,u_{k-1},u_{k-2},\ldots,u_{i+1},v_i$ es un ciclo en $T$, y esto contradice a las hipótesis. Por consiguiente solo existe un camino en $T$ de $x$ a $y$.

\vskip .2cm 


(T2) $\Rightarrow$ (T1): Debemos probar que si  para cada par $x$, $y$ de vértices existe un único camino en $T$ de $x$ a $y$,  entonces $T$ es conexo  y no tiene ciclos. Es claro que $T$ es conexo, pues para cualesquiera dos vértices existe un camino que los une. Ahora supongamos que hay un ciclo en $T$
\begin{equation*}
	v_0,v_1,\ldots,v_r,v_0, \quad (v_i \not= v_j \text{ si $i\not=j$}).
\end{equation*}
Luego $v_0,v_1,\ldots,v_r$ y  $v_0,v_r$ son dos caminos de $v_0$  a $v_r$ contradiciendo la hipótesis. Por lo tanto, no puede existir un ciclo en $T$, 


\vskip .2cm 

(T1) $\Rightarrow$ (T3): Demostraremos este resultado por inducción completa en el número de vértices de $T$. El resultado es cierto cuando $|V|=1$, puesto que el árbol de un vértice no tiene aristas.

Supongamos que el resultado es  cierto para árboles con $k$ o menos vértices. Sea $T$ un árbol con $|V|=k+1$, y sea $uv$ una arista en $T$. Por el lema \ref{lema-T1->T3} si quitamos $uv$ de $T$ obtenemos $T_1=(V_1,E_1)$ y $T_2=(V_2,E_2)$ dos árboles y tenemos que
$$
|V_1| + |V_2| = |V|, \qquad |E_1| + |E_2| = |E|-1.
$$
Aplicando la hipótesis inductiva a $T_1$ y $T_2$ obtenemos
$$
|E|=|E_1| + |E_2| + 1 = |V_1|-1 +|V_2|-1+1= |V| -1,
$$
como nosotros deseábamos. Por consiguiente el resultado es cierto para todos lo enteros positivos.

\vskip .2cm  

(T3) $\Rightarrow$ (T1): Supongamos que $T$ satisface (T3) pero no es árbol. Por lo tanto $T$ tiene al menos un ciclo. Si eliminamos una arista del ciclo, el grafo sigue siendo conexo, pero con una arista menos (y la misma cantidad de vértices), es decir obtenemos un grafo $G' = (V',E')$  conexo y con $|E'| = |V'|-2$. Pero esto contradice el resultado obtenido en el lema \ref{conv}.    

Hemos  probado (T1) $\Rightarrow$ (T2), (T2) $\Rightarrow$ (T1), (T1) $\Rightarrow$ (T3), (T3) $\Rightarrow$ (T1). Por la propiedad transitiva de $\Rightarrow$  obtenemos que todas las afirmaciones son equivalentes.
\end{proof}

Las propiedades (T2)-(T3) nos dan maneras alternativas de definir árboles. Por ejemplo la propiedad (T2) puede ser considerada como la propiedad que define un árbol, en vez de (T1).  

\begin{observacion}
El teorema anterior nos muestra un recurso muy usado para probar que una cierta cantidad de afirmaciones son equivalentes. En el caso de 3 afirmaciones $P$, $Q$, $R$, uno debería probar
$$
P \Leftrightarrow Q, P \Leftrightarrow R, Q \Leftrightarrow R,
$$
y eso nos garantizaría la equivalencia entre $P$, $Q$ y $R$. Si embargo, podemos ahorrar trabajo demostrando solamente
$$
P \Rightarrow Q, Q \Rightarrow R, R \Rightarrow P,
$$  
pues si queremos probar, por ejemplo $P \Leftrightarrow Q$, esto es equivalente a probar $P \Rightarrow Q$, ya sabido por hipótesis, y $Q \Rightarrow P$, que se deduce de $Q \Rightarrow R$, $R \Rightarrow P$ y la propiedad transitiva de $\Rightarrow$. En el teorema \ref{teo-arboles}, nos vimos obligados a hacer algunas demostraciones más, pero  el principio es el mismo: la propiedad transitiva del $\Rightarrow$ nos permite demostrar las equivalencias. 

Para el caso de cuatro proposiciones $P_1,P_2,P_3,P_4$ la economía de demostraciones es aún más drástica: para probar todas las equivalencias posibles de cuatro afirmaciones hacen falta 12 demostraciones (seis de ida y seis vuelta), pero, en ciertos casos, podemos hacer sólo 4 demostraciones:
$$
P_1 \Rightarrow P_2, P_2 \Rightarrow P_3, P_3 \Rightarrow P_4, P_4 \Rightarrow P_1. 
$$ 
\end{observacion}

\begin{subsection}{Ejercicios}\label{ejercicios5.5}
\begin{enumerate}
\item \label{ejercicio5.5.1} Hay seis diferentes (es decir, no isomorfos entre si) árboles con seis vértices: haga un dibujo de ellos.
\item Sea $T=(V,E)$ un árbol con $|V| \ge 2$. Usando la propiedad (T4) y el teorema \ref{t5.3} probar que $T$ tiene al menos dos vértices con valencia 1.
\item Una {\em foresta} es un grafo que satisface que no contiene ciclos pero no necesariamente es conexo. Probar que si $F=(V,E)$ es una foresta con $c$ componentes entonces
$$
|E|=|V|-c.
$$
\end{enumerate}
\end{subsection}

\end{section}

\begin{section}{Coloreando los vértices de un grafo} \label{5.6}

Un problema que se nos presenta frecuentemente en la vida moderna es aquel de confeccionar un horario para un conjunto de eventos de tal manera de evitar interferencias. Consideremos ahora un caso muy simple, que nos servirá de ejemplo para mostrar como la teoría de grafos puede ayudar al estudio de este problema.

Supongamos que deseamos hacer un horario con seis cursos de una hora, $v_1,v_2,v_3,v_4,v_5,v_6$. Entre la audiencia potencial hay gente que desea asistir a $v_1$ y $v_2$, $v_1$ y $v_4$, $v_3$ y $v_5$, $v_2$ y $v_6$, $v_4$ y $v_5$, $v_5$ y $v_6$ y $v_1$ y $v_6$. ?`Cuántas horas son necesarias para poder confeccionar un horario en el cual no haya interferencias?

Podemos representar la situación por un grafo (Fig. \ref{f5.10}). Los vértices corresponden a las seis clases, y las aristas indican las interferencias potenciales.

\begin{figure}[ht]
	\begin{tikzpicture}[scale=0.55]
		%\SetVertexSimple[Shape=circle,FillColor=white]
		%
		\Vertex[x=3.00, y=0.00]{$v_3$}
		\Vertex[x=1.50, y=2.60]{$v_2$}
		\Vertex[x=-1.50, y=2.60]{$v_1$}
		\Vertex[x=-3.00, y=0.00]{$v_6$}
		\Vertex[x=-1.50, y=-2.60]{$v_5$}
		\Vertex[x=1.50, y=-2.60]{$v_4$}
		\Edges($v_2$,$v_1$,$v_6$, $v_5$,$v_4$,$v_1$)
		\Edges($v_2$,$v_6$)
		\Edges($v_3$,$v_5$)
	\end{tikzpicture}
	%fig 5.10
\caption{El grafo para un problema de horarios} \label{f5.10}
\end{figure}

Un horario el cual cumple con la condición de evitar interferencias es el siguiente:
$$
\begin{matrix}
\text{Hora 1} & \text{Hora 2} &\text{ Hora 3}& \text{Hora 4} \\
v_1 \text{ y } v_3 & v_2 \text{ y } v_4 & v_5 & v_6
\end{matrix}
$$
En términos matemáticos, tenemos una partición del conjuntos de vértices en cuatro partes, con la propiedad que ninguna parte contiene un par de vértices adyacentes del grafo. Un descripción más gráfica utiliza la función
$$
c: \{ v_1,v_2,v_3,v_4,v_5,v_6\} \to  \{1,2,3,4\}
$$
la cual asigna cada vértice (curso) a la hora que le corresponde. Usualmente, nosotros hablamos de colores asignados a los vértices, en vez de horas, pero claramente la naturaleza exacta de los objetos $1,2,3,4$ no es importante. Podemos usar el nombre de colores reales, rojo, verde, azul , amarillo, o podemos hablar del color 1, color 2, etc. Lo importante es que los vértices que son adyacentes en el grafo deben tener diferentes colores.

\begin{definicion} Una {\em coloración de vértices} de un  \index{coloración de vértices} grafo $G=(V,E)$ es una función $c:V \to  \mathbb N$ con la siguiente propiedad:
$$
c(x)\not= c(y) \quad \text{ si } \quad \{x,y\} \in E.
$$
El {\em número cromático} de $G$, denotado $\chi(G)$, se define \index{número cromático} como el mínimo entero $k$ para el cual existe una coloración de vértices de $G$ usando $k$-colores. En otra palabras, $\chi(G)=k$ si  y sólo si existe una coloración de vértices $c$ la cual es una función de $V$ a $\mathbb N_k$, y $k$ es el mínimo entero con esta propiedad.
\end{definicion}

Volviendo al ejemplo de la Fig. \ref{f5.10}, vemos que nuestro primer intento de horario es equivalente a una coloración de vértices con cuatro colores. El mínimo número de horas necesarias será el número cromático del grafo, y la pregunta es ahora si este número es cuatro o menor que cuatro. Un rápido intento con tres colores nos da la solución de este problema:
$$
\begin{matrix}
\text{Color 1}\quad &\text{Color 2}\quad&\text{Color 3} \\
v_1 &v_2 \text{ y } v_5 \quad & v_3,v_4 \text{ y } v_6 .
\end{matrix}
$$
Más aún, hacen falta por lo menos tres colores, puesto que $v_1$, $v_2$, y $v_6$ son mutuamente adyacentes y por lo tanto deben tener diferentes colores. Luego concluimos que el número cromático del grafo es 3.

En general, para probar que el número cromático de un grafo dado es $k$, debemos hacer dos cosas:
\begin{enumerate}
\item[(i)] encontrar una coloración de vértices usando $k$ colores;
\item[(ii)] probar que ninguna coloración de vértices usa menos de $k$ colores.
\end{enumerate}

\begin{subsection}{Ejercicios}\label{ejercicios5.6}
\begin{enumerate}
\item \label{ejercicio5.6.1} Encontrar el número cromático de los siguientes grafos:

(i) un grafo completo $K_n$;

(ii) un grafo cíclico $C_{2r}$ con un número par de vértices;

(iii) un grafo cíclico $C_{2r+1}$ con un número impar de vértices.

\item  Determine los números cromáticos de los grafos descritos en la Fig. \ref{f5.11}.

\begin{figure}[ht]
\begin{tabular}{llllll}
	& 
	\begin{tikzpicture}[scale=0.50]
	\SetVertexSimple[Shape=circle,MinSize=5 pt,FillColor=white]
	%
	\Vertex[x=0.00, y=0.00]{0}
	\Vertex[x=3.00, y=0.00]{1}
	\Vertex[x=2.12, y=2.12]{2}
	\Vertex[x=0.00, y=3.00]{3}
	\Vertex[x=-2.12, y=2.12]{4}
	\Vertex[x=-3.00, y=0.00]{5}
	\Vertex[x=-2.12, y=-2.12]{6}
	\Vertex[x=0.00, y=-3.00]{7}
	\Vertex[x=2.12, y=-2.12]{8}
	\Edges(1,0,5) \Edges(3,0,7) \Edges(2,0,6)\Edges(4,0,8)
	\Edges(1,2,3,4,5,6,7,8,1)
	\end{tikzpicture}
	&
	\qquad\quad
	& 
	\begin{tikzpicture}[scale=0.8]
	\SetVertexSimple[Shape=circle,MinSize=5 pt,FillColor=white]
	\Vertex[x=0.00, y=0]{0}
	\Vertex[x=0.00, y=2.00]{1}
	\Vertex[x=1.90, y=0.62]{2}
	\Vertex[x=1.18, y=-1.62]{3}
	\Vertex[x=-1.18, y=-1.62]{4}
	\Vertex[x=-1.90, y=0.62]{5}
	\Edges(1,2,3,4,5,1)
	\Vertex[x=0, y=0.62]{a}
	\Vertex[x=-0.59, y=0.19]{b}
	\Vertex[x=0.59, y=0.19]{c}
	\Vertex[x=-0.36, y=-0.49]{d}
	\Vertex[x=0.36, y=-0.49]{e}
	\Edges(5,d,3,c,1,b,4,e,2,a,5)
	\Edges(0,a)
	\Edges(0,b)
	\Edges(0,c)
	\Edges(0,d)
	\Edges(0,e)
	\end{tikzpicture}
	&
	\qquad\quad
	& 
	\begin{tikzpicture}[scale=0.55]
	\SetVertexSimple[Shape=circle,MinSize=5 pt,FillColor=white]
	%
	\Vertex[x=3.00, y=0.00]{1}
	\Vertex[x=1.50, y=2.60]{2}
	\Vertex[x=-1.50, y=2.60]{3}
	\Vertex[x=-3.00, y=0.00]{4}
	\Vertex[x=-1.50, y=-2.60]{5}
	\Vertex[x=1.50, y=-2.60]{6}
	\Edges(1,2,3,4,5,6,1)
	\Edges(1,3) \Edges(1,4) \Edges(1,5)
	\Edges(3,5,6,2)
	\end{tikzpicture}
\end{tabular}
	\caption{Encontrar el número cromático}\label{f5.11}
\end{figure}

\item Describir todos los grafos $G$ tales que $\chi(G)=1$.
\end{enumerate}
\end{subsection}

\end{section}

\begin{section}{El algoritmo greedy para coloración de vértices}
\label{5.7}

Es bastante difícil encontrar el número cromático de un grafo dado. En realidad, no se conoce ningún algoritmo para este problema que trabaje en ``tiempo polinomial'', y la mayoría de la gente cree que tal algoritmo no existe. Sin embargo hay un método simple de hacer una coloración cromática usando un ``razonable'' número de colores.

El método consiste en asignar los colores de los vértices en orden, de tal manera que cada vértice recibe el primer color que no haya sido ya asignado a alguno de sus vecinos. En este algoritmo insistimos en hacer la mejor elección que podemos en cada paso, sin mirar más allá para ver si esta elección nos traerá problemas luego. Un algoritmo de esta clase se llama a menudo un {\em algoritmo greedy (goloso)}.  \index{algoritmo greedy (goloso)}

El algoritmo greedy para coloración de vértices es fácil de programar. Supóngase que hemos dado a los vértices algún orden $v_1,v_2,\ldots,v_n$. Asignemos el color 1 a $v_1$; para cada $v_i$ ($2\le i \le n$) formamos el conjunto $S$ de colores asignados a los vértices $v_j$ ($1\le j <i$) que son adyacentes a $v_i$, y le damos a $v_i$ el primer color que no está en $S$. (En la práctica, pueden ser usados métodos más sofisticados de manejar los datos.)

\vskip .5cm

\begin{minipage}{400pt}
\noindent {\sc Algoritmo greedy para coloración de vértices }
\vskip .2cm
\begin{small}
\begin{verbatim}
# pre: 1,...,n los vértices de un grafo G
# post: devuelve v[1],...,v[n] una coloración de G
v[1] = 1 # asignamos el color 1 al vértice 1
for i = 2 to n:
    S = []  # S conjunto de colores asignados a los vértices j
            # (1 <= j <i) que son adyacentes a i (comienza vacío)
    for j = 1 to i-1:
        if j es adyacente a i:
           S.append(v[j])  # agrega el color de j a  S
    k=1
    while k in S:
        k = k+1
    v[i] = k  # Asigna el color k a i, donde k es el primer color que 
              # no esta en S. 
\end{verbatim}
\end{small}
\end{minipage}

 \vskip .5cm

Debido a que la estrategia greedy es corta de vista, el número de colores que usará será normalmente más grande que le mínimo posible. Por ejemplo, el algoritmo greedy aplicado en el grafo de Fig. \ref{f5.10} da precisamente le coloración de vértices con cuatro colores que fue propuesta anteriormente, luego encontramos otra coloración con tres colores. Por supuesto todo depende del orden que se elige inicialmente para los vértices. Es bastante fácil ver que si se elige el orden correcto, entonces el algoritmo greedy nos da la mejor coloración posible (ejercicio \ref{ejercicio5.7}(2)). Pero hay $n!$ órdenes posibles, y si tuviéramos que controlar cada uno de ellos, el algoritmo requeriría ``tiempo $n!$''.

Más allá de esto, el algoritmo greedy es útil tanto en la teoría como en la práctica. Probaremos ahora dos teoremas por medio de la estrategia greedy.

\begin{teorema}\label{t5.7.1} Si $G$ es un grafo con valencia máxima $k$, entonces
\begin{enumerate}
\item[(i)] $\chi(G)\le k+1$,
\item[(ii)] si $G$ es conexo y no regular , $\chi(G) \le k$.
\end{enumerate}
\end{teorema}
\begin{proof} (i) Sea $v_1,v_2,\ldots,v_n$ un ordenamiento de los vértices de $G$. Cada vértice tiene a lo más $k$ vecinos, y por consiguiente el conjunto $S$ de los colores asignados por el algoritmo greedy a los vértices $v_j$ que son adyacentes a $v_i$ ($1\le j <i$) tiene como máximo cardinal $k$. Por consiguiente al menos uno de los colores $1,2,\dots,k+1$ no está en $S$, y el algoritmo greedy asigna entonces el primero de estos a $v_i$.

(ii) Para probar esta parte debemos elegir un orden especial de los vértices, comenzando con $v_n$ y yendo hacia atrás. Puesto que $G$ tiene valencia máxima $k$ y es no regular, existe al menos un vértice $G$ cuya valencia es menor que $k$: llamémoslo $v_n$. Listemos los vecinos de $v_n$ como $v_{n-1},v_{n-2},\ldots,v_{n-r}$; hay a lo más $k-1$ de ellos. A continuación listemos los vecinos de $v_{n-1}$ (excepto $v_n$ y sus vecinos), y observemos que como la valencia es a lo más $k$ hay a lo más $k-1$ de estos vértices. A continuación listemos los vecinos de $v_{n-2}$ que no hayan sido listados antes, y así siguiendo. Puesto que $G$ es conexo, en determinado momento podremos listar todos los vértices de $G$. Más aún, el método de construcción asegura que cada vértice es adyacente a lo más a $k-1$ de sus predecesores en el orden $v_1,v_2,\ldots,v_n$.

Usando el mismo argumento que en la parte (i) (pero para este orden) se sigue que el algoritmo greedy requerirá a lo más $k$ colores. Luego $\chi(G)\le k$.
\end{proof}

La parte (ii) del teorema es falsa si permitimos que $G$ sea regular. El lector que haya respondido correctamente al ejercicio \ref{ejercicios5.6}(\ref{ejercicio5.6.1}) será capaz de dar dos ejemplos de este hecho: los grafos completos, y los grafos cíclicos de longitud impar, ambos requieren $k+1$ colores. Si embargo, puede ser demostrado que estos son los únicos contraejemplos.

Otra consecuencia útil del algoritmo greedy se refiere a grafos $G$ son $\chi(G)=2$. Para tales grafos, los conjuntos $V_1$ y $V_2$ de vértices de colores 1 y 2 respectivamente, forman una partición de $V$, con la propiedad que cada arista tiene un vértice en $V_1$ y el otro en $V_2$. Por esta razón, cuando $\chi(G)=2$, diremos que $G$ es {\em bipartito}.  \index{grafo bipartito} Una coloración de vértices con dos colores de un cubo se ilustra en la Fig. \ref{f5.12}, junto a un dibujo alternativo que enfatiza la naturaleza bipartita del grafo. Usualmente usaremos esta clase de dibujo cuando trabajemos con grafos bipartitos.

\begin{figure}[ht]
	\renewcommand{\varx}{1} % variable para cambiar coordenada x
	\renewcommand{\vary}{1} % variable para cambiar coordenada y
	\renewcommand{\varc}{1}
	\begin{tabular}{llll}
		& 
		\begin{tikzpicture}[scale=1]
		\SetVertexSimple[Shape=circle,MinSize=5 pt,FillColor=white]
		\Vertex[x=0.00, y=0.00]{0}
		\Vertex[x=2.00, y=0.00]{1}
		\Vertex[x=2.00, y=-2.00]{2}
		\Vertex[x=0.00 , y=-2.00]{3}
		\Vertex[x=0.00 + \varx, y=0.00 + \vary]{4}
		\Vertex[x=2.00 + \varx, y=0.00 + \vary]{5}
		\Vertex[x=2.00 + \varx, y=-2.00 + \vary]{6}
		\Vertex[x=0.00 + \varx, y=-2.00 + \vary]{7}
		\Edges(0,1,2,3,0,4,5,6,7,4)
		\Edges(1,5)
		\Edges(2,6)
		\Edges(3,7)
		\draw (-0.4,0) node {1};
		\draw (-0.4,-2) node {2};
		\draw (-0.4 + \varx, 0.00 + \vary) node {2};
		\draw (-0.4 + \varx,-2.00 + \vary) node {1};
		\draw (2.40, 0.00) node {2};
		\draw (2.40, -2.00) node {1};
		\draw (2.30 + \varx, 0.00 + \vary) node {1};
		\draw (2.30 + \varx, -2.00 + \vary) node {2};
		\end{tikzpicture}
		&
		\qquad\quad
		& 
		\begin{tikzpicture}[scale=1]
		\SetVertexSimple[Shape=circle,MinSize=5 pt,FillColor=white]
		\Vertex[x=0.00, y=0.00]{0}
		\Vertex[x=2.00, y=0.00]{1}
		\Vertex[x=0.0, y=-1.00]{2}
		\Vertex[x=2.00 , y=-1.00]{3}
		\Vertex[x=2.00, y=-2.00]{4}
		\Vertex[x=0.00 , y=-2.00]{5}
		\Vertex[x=2.00, y=-3.00]{6}
		\Vertex[x=0.00, y=-3.00]{7}
		\Edges(0,1,2,3,0,4,5,6,7,4)
		\Edges(1,5)
		\Edges(2,6)
		\Edges(3,7)
		\draw (-0.4,0) node {1};
		\draw (-0.4,-1) node {1};
		\draw (-0.4,-2) node {1};
		\draw (-0.4,-3) node {1};
		\draw (2.4,0) node {2};
		\draw (2.4,-1) node {2};
		\draw (2.4,-2) node {2};
		\draw (2.4,-3) node {2};
		\end{tikzpicture}
	\end{tabular}
\caption{El cubo es un grafo bipartito} \label{f5.12}
\end{figure}

\begin{teorema}\label{t5.7.2} Un grafo es bipartito si  y sólo si no contiene ciclos de longitud impar.
\end{teorema}
\begin{proof} Si hay un ciclo de longitud impar, entonces se requieren tres colores, solamente para colorear este ciclo, y el número cromático del grafo es por ende al menos tres. Luego si el grafo es bipartito, no puede tener ciclos de longitud impar.

Recíprocamente, supongamos que $G$ es un grafo sin ciclos de longitud impar. Construiremos un orden de $G$ para el cual el algoritmo greedy producirá una coloración de vértices con dos colores. Elijamos cualquier vértice y llamémoslo $v_1$; diremos que $v_1$ esta en el {\it nivel 0}. A continuación, listemos la lista de vecinos de $v_1$ (excepto $v_1$), llamémoslos $v_2,v_3,\dots,v_r$; diremos que estos vértices están en el {\it nivel 2}. Continuando de esta manera, definimos el {\it nivel $l$} como todos aquellos vértices adyacentes a los del {\it nivel $l-1$}, exceptuando aquellos previamente listados en el {\it nivel $l-2$}. Cuando ningún nuevo vértice puede ser agregado de esta forma, obtenemos la componente $G_0$ de $G$ (si $G$ es conexo $G_0=G$).

El hecho crucial producido por este orden es que un vértice del nivel $l$ solo puede ser adyacente a vértices de los niveles $l-1$ y $l+1$, y no a vértices del mismo nivel. Supongamos que $x$ e $y$ son vértices en el mismo nivel; entonces ellos son unidos por caminos de igual longitud $m$ a algún vértice $z$ de un nivel anterior, y los caminos pueden ser elegidos de tal manera que $z$ sea el único vértice común (Fig. \ref{f5.13}). Si $x$ e $y$ fueran adyacentes, habría un ciclo de longitud $2m+1$, lo cual contradice la hipótesis. 

\begin{figure}[ht]
	\begin{tikzpicture}[scale=1]
	\SetVertexSimple[Shape=circle,MinSize=5 pt,FillColor=white]
	\Vertex[x=0.00, y=0.00]{0}
	\Vertex[x=-2.50, y=-1.3]{1}
	\Vertex[x=-3, y=-2.0]{2}
	\Vertex[x=-2.6 , y=-2.8]{3}
	\Vertex[x=1.00, y=-1.4]{4}
	\Vertex[x=1.00 , y=-2.1]{5}
	\Vertex[x=2, y=-2.8]{6}
	\draw (-2.6,-3.2) node {$x$};
	\draw (2,-3.2) node {$y$};
	\draw (0,0.3) node {$z$};
	\Edges(1,2,3)
	\Edges(4,5,6)
	\begin{scope}   [dashed]  % now dashed is for the lines inside the scope
	\Edge (0)(1)
	\Edge (0)(4)
	\Edge (6)(3)
	\end{scope}
	\end{tikzpicture}
	\caption{Vértices adyacentes en el mismo nivel inducen un ciclo impar} \label{f5.13}
\end{figure}

Se deduce entonces que el algoritmo greedy asigna el color 1 a los vértices en el nivel $0,2,4,\ldots$, y el color 2 a los vértices en los niveles $1,3,5,\ldots$. Por consiguiente $\chi(G_0)=2$. Repitiendo el mismo argumento para cada componente de $G$ obtenemos el resultado deseado.
\end{proof}

\begin{subsection}{Ejercicios}\label{ejercicio5.7} \begin{enumerate}
\item Encontrar órdenes de los vértices del grafo del cubo (Fig. \ref{f5.12}) para los cuales el algoritmo greedy requiera 2, 3 y 4 colores respectivamente.
\item \label{ejercicio5.7.2} Probar que para cualquier grafo $G$ existe un orden de los vértices para el cual el algoritmo greedy requiera $\chi(G)$ colores. [Ayuda: use un coloreado de vértices de $\chi(G)$ colores para definir el orden.]
\item Denotemos $e_i(G)$ el número de vértices del grafo $G$ cuya valencia es estrictamente mayor que $i$. Use el algoritmo greedy para probar que si $e_i(G) \le i+1$ para algún $i$, entonces $\chi(G) \le i+1$.
\item El grafo $M_r$ ($r\ge 2$) se obtiene a partir del grafo cíclico $C_{2r}$ añadiendo aristas extras que unen los vértices opuestos. Probar que
$$
\begin{aligned}
\text{(i)}\quad& \text{$M_r$ es bipartito cuando $r$ es impar,} \\
\text{(ii)}\quad&\text{$\chi(M_r)=3$ cuando $r$ es par y $r\not= 2$,} \\
\text{(iii)}\quad& \text{$\chi(M_2)=4$.}
\end{aligned}
$$
\end{enumerate}
\end{subsection}

\end{section}

\begin{section}{Ejercicios}
\begin{enumerate}
\item ?`Para qué valores de $n$ es verdadero que el grafo completo $K_n$ tiene una caminata euleriana.
\item Usar el principio de inducción para probar que si $G=(V,E)$ es un grafo con $|V|=2m$, y $G$ no tiene 3-ciclos, entonces $|E|\le m^2$.
\item Sea $X=\{1,2,3,4,5\}$ y denotemos $V$ el conjunto de los 2-subconjuntos de $X$. Denotemos $E$ el conjunto de pares de elementos de $V$ que son disjuntos entre si (como subconjuntos de $X$). Probar que este grafo es otra versión del grafo de Petersen (ejercicio \ref{ejercicios5.2} (\ref{ejercicio5.2.2}).
\item Sea $G$ un grafo bipartito con un número impar de vértices. Probar que $G$ tiene un ciclo hamiltoniano.
\item El {\em $k$-cubo} $Q_k$ es el grafo cuyos vértices son las palabras de longitud $k$ en el alfabeto $\{0,1\}$ y cuyas aristas unen palabras que difieren en exactamente una posición. Probar que:
$$
\begin{aligned}
\text{(i)} \quad &\text{$Q_k$ es regular de valencia $k$,}\\
\text{(ii)} \quad &\text{$Q_k$ es bipartito.}
\end{aligned}
$$
\item Probar que el grafo $Q_k$ definido en el ejercicio anterior tiene un ciclo hamiltoniano.
\item Probar que el grafo de Petersen no tiene ciclos hamiltonianos.
\item En el juego del dominó las reglas especifican que las fichas deben ser puestas en una linea de tal forma que en dos fichas adyacentes coinciden los números adyacentes, es decir si $[x|y]$ está al lado de $[x'|y']$ debe ser $y=x'$. Mirando las fichas de dominó $[x|y]$ con $x\not=y$ como las aristas del grafo completo $K_7$ probar que existe un juego de dominó donde se utilizan todas las fichas.
\item Calcular el número de caminatas eulerianas de $K_7$ y el número de juegos de dominó completos. 
\item Probar que si $\alpha: V_1 \to  V_2$ es un isomorfismo de grafos entre $G_1=(V_1,E_1)$ y  $G_2=(V_2,E_2)$ entonces la función $\beta:E_1 \to  E_2$ definida por
$$
\beta\{x,y\} = \{\alpha(x),\alpha(y)\} \qquad (\{x,y\} \in E_1)
$$
es una biyección.
\item Si $G$ es un grafo regular $k$-valente con $n$ vértices, entonces
$$
\chi(G)\ge \frac{n}{n-k}.
$$
\item Construir cinco grafos regulares conexos mutuamente no isomorfos con valencia 3 y ocho vértices.
\item Probar que el grafo completo $K_{2n+1}$ es la unión de $n$ ciclos hamiltonianos sin aristas comunes.
\item ?`Es posible para un caballo visitar todos los casilleros de un tablero de ajedrez exactamente una vez y volver al casillero original? Interprete su respuesta en términos de ciclos hamiltonianos en un cierto grafo.
\item El {\em grafo impar} $O_n$ se define de la siguiente manera  \index{grafo impar} (cuando $k\ge 2$): los vértices son los $(k-1)$-subconjuntos de un $(2k-1)$-conjunto, y las aristas unen los conjuntos disjuntos. (Luego $O_3$ es el grafo de Petersen). Probar que $\chi(O_k)=3$ para $k\ge 2$.
\item Probar que si $G$ es un grafo con $n$ vértices, $m$ aristas y $c$ componentes entonces
$$
n-c \le m \le \frac12(n-c)(n-c+1).
$$
Construir ejemplos mostrando que ambos extremos de las desigualdades pueden ser alcanzados para todos los valores de $n$ y $c$ con $n\ge c$.
\item Una sucesión $d_1,d_2,\ldots,d_n$ es {\em gráfica} si existe un grafo cuyos vértices pueden ser ordenados en la forma $v_1,v_2,\ldots,v_n$ de tal forma que $\delta(v_i)=d_i$ ($1\le i \le n$). Probar que si una sucesión $d_1,d_2,\ldots,d_n$ es gráfica y $d_1 \ge d_2 \ge \cdots \ge d_n$ entonces
$$
d_1 + d_2 + \cdots + d_n \le k(k-1) + \sum_{i=k+1}^n
\operatorname{min}(k,d_i)
$$
para $1 \le k \le n$.
\item Sea $G=(V,E)$ un grafo con al menos tres vértices tal que
$$
\delta(v) \ge \frac12 |V|\qquad (v\in V).
$$
Probar que $G$ tiene un ciclo hamiltoniano.
\item Probar que si $\tilde G$ es el complemento del grafo $G$, entonces $\chi(G)\chi(\tilde G)\le n$, donde $n$ es el número de vértices
de $G$.
\end{enumerate}

\end{section}

\chapter[Árboles]{Árboles (*)}

\begin{section}{Contando las hojas de un árbol con raíz}
\label{6.1}

Recordemos que un {árbol} es un grafo conexo que no contiene ciclos. Los árboles aparecen en contextos diferentes y a menudo un vértice del árbol se distingue de los otros. Por ejemplo en el árbol genealógico que describe la descendencia de un rey, nosotros podemos enfatizar la posición especial del rey poniéndolo en lo más alto del árbol. En general, nosotros llamaremos al vértice notable la {\em raíz} del árbol, y a un árbol con una raíz \index{raíz} específica lo llamaremos {\em árbol con raíz}. (Esta \index{árbol con raíz} terminología, aunque estándar, tiene el defecto que en la representación pictórica la raíz aparece en lo más alto del árbol y el árbol 'crece' hacia abajo.)

Para el estudio de un árbol con raíz es natural ubicar los vértices en niveles, de la misma manera que lo hicimos para los grafos bipartitos en la sección \ref{5.7}. Diremos que el vértice raíz es el {\it nivel 0} y que sus vecinos forman el {\it nivel 1}. Para cada $k\ge 2$, el {\it nivel $k$} está formado por aquellos vértices que son adyacentes a vértices del nivel $k-1$, excepto aquellos que ya pertenecen al nivel $k-2$. El árbol con raíz representado en la Fig. \ref{f6.1} puede ser dibujado nuevamente como se lo muestra a la derecha de manera de visualizar los niveles.  \index{niveles de un árbol}

\begin{figure}[ht]
	\renewcommand{\varx}{1} % variable para cambiar coordenada x
	\renewcommand{\vary}{1} % variable para cambiar coordenada y
	\renewcommand{\varc}{1}
	\begin{tabular}{llllll}
		& 
		\begin{tikzpicture}[scale=1]
		%\SetVertexSimple[Shape=circle,MinSize=5 pt,FillColor=white]
		\Vertex[x=-2.00, y=0.00, L=$a$]{0}
		\Vertex[x=-1.00, y=0.00, L=$b$]{1}
		\Vertex[x=1.0, y=0.00, L=$c$]{2}
		\Vertex[x=0.00, y=-1.00, L=$r$]{9}
		\Vertex[x=0.00, y=-2.00, L=$g$]{6}
		\Vertex[x=-1.00, y=-3.00, L=$h$]{7}
		\Vertex[x=1.00, y=-3.00, L=$i$]{8}
		\Vertex[x=-2.00, y=-3.00, L=$d$]{3}
		\Vertex[x=-2.00, y=-2.00, L=$e$]{4}
		\Vertex[x=-1.00, y=-2.00, L=$f$]{5}
		\Edges(0,1,9,2,9,6,7,5,7,4,7,3)
		\Edges(6,8)
		\end{tikzpicture}
		&
		\qquad\quad
		& 
		\begin{tikzpicture}[scale=1.2]
		%\SetVertexSimple[Shape=circle,MinSize=5 pt,FillColor=white]
		\Vertex[x=0.00, y=0, L=$r$]{9}
		\Vertex[x=-1.00, y=-1.00, L=$b$]{1}
		\Vertex[x=0.0, y=-1.00, L=$c$]{2}
		\Vertex[x=1.00, y=-1.00, L=$g$]{6}
		\Vertex[x=-1.00, y=-2.00, L=$a$]{0}
		\Vertex[x=0.00, y=-2.00, L=$h$]{7}
		\Vertex[x=1.00, y=-2.00, L=$i$]{8}
		\Vertex[x=-1.00 , y=-3.00, L=$f$]{5}
		\Vertex[x=0.00, y=-3.00, L=$e$]{4}
		\Vertex[x=1.00 , y=-3.00, L=$d$]{3}
		\Edges(0,1,9,2,9,6,7,5,7,4,7,3)
		\Edges(6,8)
		\end{tikzpicture}
		&
		\,
		&
		\begin{tikzpicture}[scale=1.2]
		\draw (0,0) node {Nivel 0};
		\draw (0,-1) node {Nivel 1};
		\draw (0,-2) node {Nivel 2};
		\draw (0,-3) node {Nivel 3};
		\end{tikzpicture}
	\end{tabular}
	\caption{Un árbol con raíz y sus niveles} \label{f6.1}
\end{figure}

Un vértice en un árbol con raíz se llama una {\em hoja} si \index{hoja} pertenece al nivel $i$ ($i\ge 0$) y no es adyacente a ningún vértice del nivel $i+1$. Un vértice que no es una hoja es llamado {\em interno}.  \index{vértice interno} La {\em altura} de un árbol con raíz  \index{altura de un árbol con raíz} es el máximo valor de $k$ para el cual el nivel $k$ es no vacío. Luego el árbol de la Fig. \ref{f6.1} tiene seis hojas, cuatro vértices internos y su altura es tres.

\begin{subsection}{Ejercicios}\label{ejercicios6.1}
\begin{enumerate}
\item En la siguiente tabla $n_5(h)$ es el número de árboles con raíz no isomorfos que tienen 5 vértices y altura $h$. (Dos árboles con raíz son isomorfos si hay un isomorfismo de grafos, sin considerar la raíz, que lleva la raíz de uno en la del otro.) Verifique la tabla construyendo los ejemplos para cada caso. $$
\begin{matrix}
h: &1 &2 &3 &4 \\
n_5(h): &1 &4 & 3 &1
\end{matrix}
$$
\item Si consideramos los árboles comunes (sin raíz), ?`cuál es el número de árboles no isomorfos con 5 vértices? Hacer una lista y controlar que la lista del ejercicio anterior sea completa.
\item Construir dos árboles con raíz no isomorfos ambos con 12 vértices, 6 hojas y altura 4.
\end{enumerate}
\end{subsection}

Las dos propiedades que usamos en la sección \ref{5.5} para definir un árbol, ser un grafo conexo y sin ciclos, tienen consecuencias obvias cuando pensamos los vértices por niveles. Puesto que todo árbol es conexo entonces cada vértice pertenece a algún nivel. Más importante aún, puesto que un árbol no tiene ciclos  cada vértice $v$ del nivel $i$ es adyacente a uno y solo uno $u$ del nivel $i-1$. A veces enfatizaremos esto diciendo que $u$ es el {\it {padre}} de $v$ o que $v$ es un {\it {hijo}} de $u$. Cada \index{padre de un vértice}  \index{hijo de un vértice} vértice, excepto el raíz, tiene un único padre, pero un vértice puede tener una cantidad arbitraria de hijos (incluso ninguno). Claramente, un vértice es una hoja si y solo si no tiene hijos.

En muchas aplicaciones ocurre que cada padre (vértice interno) tiene la misma cantidad de hijos. Cuando cada padre tiene $m$ hijos diremos que el árbol es {\em $m$-ario}, en particular cuando $m=2$ diremos que el árbol es {\it binario}  \index{árbol binario}  \index{árbol ternario} y cuando $m=3$ diremos que es {\it ternario}.

\begin{teorema}\label{t6.1} La altura de un árbol con raíz\,\,\, $m$-ario con $l$ hojas es por lo menos $\log_ml$.
\end{teorema}
\begin{proof} Puesto que
$$
h \ge \log_ml \quad \Leftrightarrow \quad m^h \ge l
$$
es suficiente probar la afirmación equivalente: todo árbol con raíz $m$-ario de altura $h$ tiene a lo más $m^h$ hojas. La demostración es por inducción sobre $h$.

Claramente la afirmación es verdadera cuando $h=0$ puesto que en este caso el árbol es solo un vértice (la raíz) que es una hoja. Supongamos que la afirmación es verdadera cuando $0\le h \le h_0$ y sea $T$ un árbol con raíz $m$-ario de altura $h_0+1$. Si eliminamos la raíz y las aristas a las cuales pertenece obtenemos $m$ árboles $T_1,\ldots,T_m$ cuyas raíces son los vértices del nivel 1 de $T$. Cada $T_i$ es un árbol con raíz de altura $h_0$ o menos, luego por hipótesis inductiva tiene a lo más $m^{h_0}$ hojas. Pero las hojas de $T$ son precisamente las hojas de los árboles $T_1,\ldots,T_m$ y por consiguiente el número de hojas es a lo más $m \times m^{h_0}= m^{h_0+1}$.

Por el principio de inducción completa se sigue que la afirmación es verdadera para todo $h\ge 0$.
\end{proof}

Puesto que $\log_ml$ no es generalmente un número entero, el teorema anterior puede ser mejorado un poco. Por ejemplo si $m=3$
y $l=10$ la desigualdad
$$
h \ge \log_ml=2,0959\ldots
$$
implica que $h\ge 3$. En general podemos decir que
$$
h\ge\lceil \log_ml\rceil,
$$
donde $\lceil x \rceil$ denota el menor entero $z$ tal que $z\ge
x$.

Una aplicación frecuente del Teorema \ref{t6.1} es en los  \index{árbol de decisión} {\it árboles de decisión}. Cada vértice interno de un árbol de decisión representa una decisión y los posibles resultados de esa decisión son las aristas que unen ese vértice con los vértices del nivel siguiente. Los posibles resultados finales del procedimiento son las hojas del árbol. Si el resultado de una decisión puede ser solo verdadero o falso entonces tenemos un árbol binario. A continuación daremos un ejemplo con un árbol ternario.

\begin{ejemplo} \label{monedafalsa}(El problema de la moneda falsa) Supongamos que tenemos una moneda genuina con la etiqueta 0 y que tenemos otras $r$ monedas indistinguibles de 0 por la apariencia excepto que tienen las etiquetas $1,2,\ldots,r$. Se sospecha que una moneda podría ser falsa, es decir o más liviana o más pesada. Probar que son necesarias al menos $\lceil \log_3(2r+1)\rceil $ pesadas en una balanza para decidir que moneda (si hay alguna) es falsa y en ese caso ver si es más pesada o liviana. Muestre un procedimiento que use exactamente este número de pesadas cuando $r=4$.
\end{ejemplo}
\begin{proof} Hay $2r+1$ posibles resultados finales u hojas en el árbol de decisión:
$$
B,1P,1L,\ldots,rP,rL;
$$
donde $B$ significa que todas las monedas son buenas, $iL$ significa que la moneda $i$ es más liviana y $iP$ que es más pesada. El árbol de decisión es ternario, puesto que hay tres posibles resultados de cada decisión (es decir de cada pesada entre un grupo de monedas y otro). Estos son:

\begin{alignat*}3
&<\quad & &:\quad& &\text{el grupo de la izquierda es más liviano}\\
&=\quad & &:\quad& &\text{los dos grupos pesan igual}\\
&>\quad & &:\quad& &\text{el grupo de la izquierda es más pesado.}
\end{alignat*}
Por consiguiente la altura del árbol de decisión es al menos $\lceil \log_3(2r+1)\rceil$.

Cuando $r=4$ entonces $\lceil \log_3(2r+1)\rceil=2$, y la solución con dos pesadas se gráfica en la Fig. \ref{f6.2}

\begin{figure}[ht]
	\begin{tikzpicture}[scale=1.2]
	{\renewcommand{\VertexShape}{rectangle}
		\Vertex[x=0.00, y=0, L ={\;$0,1 | 2,3$\;}]{0}
		\Vertex[x=-3, y=-1.5, L={$2|3$}]{1}
		\Vertex[x=0, y=-1.5, L={$0|4$}]{2}
		\Vertex[x=3, y=-1.5, L={\,$2|3$\,}]{3}
		\Vertex[x=-3.8, y=-3]{$3H$}
		\Vertex[x=-3, y=-3]{$1L$}
		\Vertex[x=-2.2, y=-3]{$2H$}
		\Vertex[x=-0.8, y=-3]{$4H$}
		\Vertex[x=0, y=-3]{$G$}
		\Vertex[x=0.8, y=-3]{$4L$}
		\Vertex[x=2.2, y=-3]{$2L$}
		\Vertex[x=3, y=-3]{$1H$}
		\Vertex[x=3.8, y=-3]{$3L$}
	}
	\SetVertexSimple[Shape=rectangle,FillColor=white,MinSize=8 pt]
	%\draw (0, 0) node {$\boxed{\text{Primera casa}}$};
	\Edges(0,1,0,2,0,3)
	\Edges(1,$3H$,1,$1L$,1,$2H$)
	\Edges(2,$4H$,2,$G$,2,$4L$)
	\Edges(3,$2L$,3,$1H$,3,$3L$)
	\end{tikzpicture}
	\caption{Solución del problema de la moneda falsa cuando $r=4$}
	\label{f6.2}
\end{figure}

\end{proof}

\begin{subsection}{Ejercicios} (continuación)
\begin{enumerate}
\item Suponga que se organiza un campeonato de fútbol-5 donde participan 20 equipos. El campeonato es por eliminación simple y no hay empates. Cons\-truir un esquema para el torneo basado en un árbol con raíz y pruebe que son necesarias al menos 5 rondas.
\item ?`Cuál es la cota inferior en el número de pesadas necesarias en el problema de la moneda falsa (ver el ejemplo \ref{monedafalsa}) cuando son seis monedas? Desarrolle un esquema que logre este número de pesadas.
\item Considere la siguiente variante del problema de la moneda falsa. Hay ocho monedas y sabemos que hay exactamente una que es más liviana. Todas las demás son genuinas pero no hay ninguna moneda con la etiqueta 0. Encontrar una cota inferior teórica del número de pesadas necesarias para detectar la moneda falsa y probar que este número puede ser alcanzado.
\end{enumerate}
\end{subsection}

\end{section}

\begin{section}{Árboles expandidos y el problema MST} \label{6.2}
Supongamos que $G=(V,E)$ es un grafo conexo y que $T$ es un subconjunto de $E$ tal que
\begin{enumerate}
\item[(i)] cada vértice de $G$ pertenece a una arista en $T$;
\item[(ii)] las aristas de $T$ forman un árbol.
\end{enumerate}
En esta caso decimos que $T$ es un {\em árbol expandido} para \index{árbol expandido} $G$. Por ejemplo, un árbol expandido para el grafo de la Fig. \ref{f6.3} se indica con las líneas más gruesas.

\begin{figure}[ht]
	\begin{tikzpicture}[scale=1.2]
	\SetVertexSimple[Shape=circle,MinSize=5 pt,FillColor=white]
	\Vertex[x=0.00, y=0]{0}
	\Vertex[x=0.00, y=2.00]{1}
	\Vertex[x=1.90, y=0.62]{2}
	\Vertex[x=1.18, y=-1.62]{3}
	\Vertex[x=-1.18, y=-1.62]{4}
	\Vertex[x=-1.90, y=0.62]{5}
	\Edges(1,2,3,4,5,1)
	\Edges(5,0,c,2)
	\Edges(0,e,3)
	\Edges(0,c)
	\Vertex[x=0.59, y=0.19]{c}
	\Vertex[x=0.36, y=-0.49]{e}
	\tikzstyle{edge} = [draw,line width=2.5pt]
	\draw[edge] (5) -- (1) -- (2);
	\draw[edge] (1) -- (0) -- (e) -- (c) -- (e) -- (3) -- (4);
	\tikzstyle{edge} = [draw,line width=1pt]
	\draw[edge] (0) -- (c);
	\draw (0,2.3) node {$a$};
	\draw (-2.2, 0.62) node {$b$};
	\draw (2.2, 0.62) node {$e$};
	\draw (-0.2, -0.2) node {$c$};
	\draw (0.59, 0.5) node {$d$};
	\draw (0.66, -0.49) node {$f$};
	\draw (-1.48, -1.62) node {$g$};
	\draw (1.48, -1.62) node {$h$};
	\end{tikzpicture}
	\caption{Un grafo y uno de sus árboles expandidos} \label{f6.3}
\end{figure}

Es fácil hacer ``crecer'' un árbol expandido: tome un vértice arbitrario $v$ del ``árbol parcial'' inicial y agregue aristas con un extremo en $v$ y el otro extremo que no pertenezca al árbol parcial inicial. El árbol expandido de la Fig. \ref{f6.3} puede construirse haciéndolo crecer desde el vértice $a$ y conectando los otros vértices en el orden $b$, $c$, $e$, $f$, $d$, $h$, $g$, usando las aristas $ab$, $ac$, $ae$, $cf$, $fd$, $fh$, $hg$. En general, si hay $n$ vértices nosotros deberemos hacer $n-1$ pasos, después de los cuales tendremos $1+(n-1)=n$ vértices y $n-1$ aristas (el cual es el número correcto de acuerdo al Teorema \ref{teo-arboles}).

Verifiquemos que el método siempre funciona: sea $S$ el conjunto de vértices del árbol parcial que se ha logrado en un paso intermedio, es decir que $S$ no es ni vacío ni todo $V$. Si no existe una arista que tenga un extremo en $S$ y el otro en el complemento $\overline{S}$, entonces no existe un camino entre $S$ y $\overline{S}$ y por lo tanto $G$ es disconexo, lo cual contradice las hipótesis. Por consiguiente siempre existe una arista disponible en cada etapa de la construcción.

\begin{subsection}{Ejercicios}
\begin{enumerate}
\item Encontrar árboles expandidos para el cubo (Fig. \ref{f5.12}) y para el grafo de Petersen.
\item Muestre esquemáticamente todos los árboles expandidos del grafo completo $K_4$ (hay 16).
\end{enumerate}
\end{subsection}

Los árboles expandidos son útiles en muchos contextos. Por ejemplo, su\-pon\-ga\-mos que cierta cantidad de ciudades deben ser unidas de a pares por gasoductos de tal forma que quede una red de gasoductos conexa. Algunos pares de ciudades puede ser imposible unirlas por razones geográficas y cada posible conexión tiene asociada un costo de construcción. Formalmente, tenemos un grafo $G=(V,E)$ cuyos vértices son ciudades y sus aristas son las posibles conexiones. Además te\-ne\-mos una función $w$ de $E$ a $\mathbb N$ de tal forma que $w(e)$ representa el costo de cons\-truc\-ción de la arista $e$. Diremos que $G$ y $w$ es un {\em grafo con pesos} y que $w$ es la {\em función de pesos}. \index{grafo con pesos}    \index{función de pesos} 

En el problema del gasoducto lo que se pretende es construir una red conexa al mínimo costo. Un red de ese tipo corresponde a un árbol expandido $T$ para $G$ cuyo peso total
$$
w(T) = \sum_{e \in T} w(e)
$$
es lo mas pequeño posible. Nos referiremos a este problema como el {\em problema MST }(del inglés MST = minimum spanning tree = \index{MST}  \index{minimum spanning tree} árbol expandido mínimo) para el grafo con pesos $G$.  \index{árbol expandido mínimo}

Dado que los valores de $w$ son enteros positivos, claramente el problema MST debe tener solución, puesto que hay solo un número finito de árboles expandidos $T$ para $G$ y cada uno de ellos da un valor entero positivo $w(T)$. En otras palabras, existe un árbol expandido mínimo $T_0$ tal que
$$
w(T_0) \le w(T)
$$
para todos los árboles expandidos $T$ de $G$. Sin embargo puede haber varios con la misma propiedad.

Un algoritmo simple para el problema MST se basa en aplicar la estrategia greedy al método explicado anteriormente. Específicamente: en cada paso se agrega la arista ``más barata'' que une un nuevo vértice al árbol parcial. (Si hay varias aristas con la misma propiedad se selecciona una de ellas.) Por ejemplo, si en la Fig. \ref{f6.4} comenzamos con $u$, luego debemos agregar aristas en el orden $uv$, $ux$, $uy$, $yz$. Por otro lado, si comenzáramos por $y$, entonces agregamos las aristas en el orden $yz$, $yu$, $uv$, $ux$.

\begin{figure}[ht]
	\begin{tikzpicture}[scale=1.2]
	\SetVertexSimple[Shape=circle,MinSize=5 pt,FillColor=white]
	\Vertex[x=0.00, y=2.00]{1}
	\Vertex[x=1.90, y=0.62]{2}
	\Vertex[x=1.18, y=-1.62]{3}
	\Vertex[x=-1.18, y=-1.62]{4}
	\Vertex[x=-1.90, y=0.62]{5}
	\Edges(1,2,3,4,5,1)
	\Edges(1,3,5,2,4,1)
	\tikzstyle{edge} = [draw,line width=2.5pt]
	\draw[edge] (1) -- (2) -- (1) -- (3) -- (1) -- (4) -- (5);
	\draw (0,2.3) node {$u$};
	\draw (-2.2, 0.62) node {$z$};
	\draw (2.2, 0.62) node {$v$};
	\draw (-1.48, -1.62) node {$y$};
	\draw (1.48, -1.62) node {$x$};
	\GraphInit[vstyle=Classic]
	\Edge[label=2](1)(2)
	\Edge[label=6](1)(5)
	\Edge[label=5](1)(4)
	\Edge[label=4](1)(3)
	\Edge[label=8](2)(5)
	\Edge[label=7](2)(4)
	\Edge[label=5](2)(3)
	\Edge[label=6](3)(5)
	\Edge[label=7](3)(4)
	\Edge[label=2](4)(5)
	\end{tikzpicture}
	\caption{Un árbol expandido mínimo} \label{f6.4}
\end{figure}

El  algoritmo puede ser descripto informalmente en los siguientes pasos:
\begin{itemize}
\item Inicializar un árbol con un sólo vértice (elegido arbitrariamente).
\item Agrandar el árbol agregando una arista: entre las aristas que conectan al árbol con vértices que aún no están en el árbol, elegir una de peso mínimo.
\item Repetir el paso anterior hasta que todos los vértices estén en el árbol
\end{itemize}

Un primera impresión nos dice que sería bastante sorprendente que el algoritmo greedy funcione para el problema MST, especialmente cuando recordamos que el algoritmo greedy para el problema de coloración de vértices no siempre produce una coloración con el menor número posible de colores. Pero en el caso del problema MST se tiene más suerte.

\begin{teorema}\label{t6.2} Sea $G=(V,E)$ grafo conexo con función de pesos $w: E \to  \mathbb N$, y supongamos que $T$ es el árbol expandido para $G$ construido por el algoritmo greedy. Entonces
$$
w(T) \le w(U) \nopagebreak
$$
para todo árbol expandido $U$ de $G$.
\end{teorema}
\begin{proof} Denotemos $e_1,e_2,\ldots,e_{n-1}$ las aristas de $T$ en el orden en que aplicamos el algoritmo greedy. Si $U=T$ el resultado es obviamente verdadero. Si $U\not=T$ entonces hay aristas de $T$ que no están en $U$ y supongamos que la primera es $e_k$. Denotemos $S$ el conjunto de vértices en el árbol parcial que se construye por el greedy justo antes de agregar $e_k$ y sea $e_k=xy$ donde $x$ está en $S$ e $y$ no está en $S$. Puesto que $U$ es un árbol expandido existe un camino de $x$ a $y$ y si uno viaja a través de este camino encontrará una arista $e^*$ con un vértice en $S$ y el otro no. Ahora bien, cuando $e_k$ es seleccionada para $T$ en el algoritmo greedy, $e^*$ es también candidata a ser seleccionada, pero no lo es. Por consiguiente debemos tener que $w(e^*) \ge w(e)$. Si $e^*$ aparece en $T$, entonces por el razonamiento anterior es una arista que viene después (en el orden dado) de $e_k$.

El resultado de remover $e^*$ de $U$ y reemplazarla por $e_1$ es un árbol expandido $U_1$, para el cual
$$
w(U_1) = w(U) -w(e^*)+w(e_k) \le w(U).
$$
Más aún, la primera arista de $T$ que no está en $U_1$ aparece después de $e_k$ en el orden dado. En consecuencia podemos repetir el procedimiento obteniendo una sucesión de árboles expandidos $U_1,U_2,\ldots,$ con la propiedad que cada uno tiene una secuencia inicial de aristas en común con las aristas de $T$ más larga que el anterior y además $w(U_i) \ge w(U_{i+1})$. El proceso termina cuando obtenemos un árbol expandido $U_r$ igual a $T$ y tenemos
$$
w(T)=w(U_r) \le w(U_{r-1}) \le \cdots \le w(U_1) \le w(U),
$$
como queríamos demostrar.
\end{proof}

En forma más detallada, el algoritmo puede ser implementado con el siguiente pseudocódigo:

\vskip .5cm

\begin{minipage}{400pt}
\noindent {\sc Algoritmo de Prim}
\vskip .2cm
\begin{small}
\begin{verbatim}
# pre: G grafo con vértices 1,...,n y pesos w(i,j)
#      (w(i,j) = infinito si ij no es arista de G)
# post: devuelve F un MST de G
S = [1] # lista de vértices utilizados en el MST (comienza con 1)
Q = [2,...,n]  # lista de vértices aún no utilizados en el MST
L = [[2,1,p2],...,[n,1,pn]]]  # con pi = w(i,1)
# se irá modificando L de tal forma que si Q = [u1,...,ur], 
# L  = [[u1,v1,p1],...,[ur,vr,pr]] donde 
# vi = vértice en S adyacente a ui tal que w(ui,vi) es mínimo, 
# pi = w(ui,vi)
F = grafo con vértices 1,...,n y sin aristas.
while Q != []:
    uk = vértice en Q tal que pk = w(uk,vk) es mínimo
    F.append(arista uk,vk)
    Q = Q - {uk}
    S.append(uk)
    L = L - {[uk,vk,pk]}
    for [ui,vi,pi] in L:
        if w(ui,uk) < pi:
            vi = uk
            pi = w(ui,uk)
    # el for modifica L
\end{verbatim}
\end{small}
\end{minipage}

 \vskip .5cm

Notemos que terminamos el proceso no cuando $S$ contiene todos los vértices, si no cuando, de manera equivalente, su complemento $Q$ queda vacío.   Existe una forma de ir viendo el progreso del algoritmo por medio de una tabla de tres columnas.

\vskip .3cm

$$
\begin{matrix}
\text{I} &\text{II} & \text{III} \\
x & y & w(xy)\\
 .& .&. \\
. & .&. \\
. &. & .
\end{matrix}
$$
\vskip .3cm

La Columna I lista los vértices que no están en $S$, que es el conjunto de vértices ya conectados al árbol parcial. Para cada $x$ en la Columna I la correspondiente entrada $y$ en la Columna II es un vértice en $S$ tal que la arista $xy$ es una de las aristas más baratas que unen el vértice $x$ con alguno de $S$. La Columna III contiene el valor $w(xy)$.

En el $i$-ésimo paso de la construcción tenemos que $|S|=i$ y hay $n-i$ vértices en la Columna I. Tenemos entonces que seleccionar una de las entradas más pequeñas de la Columna III, digamos $w(x_0y_0)$, y esto conlleva $n-i-1$ comparaciones. Ahora debemos actualizar la tabla debido a que agregamos $x_0$ a $S$ por medio de la arista $x_0y_0$. Primero debemos borrar la fila cuya primera posición tiene a $x_0$. Después en cada fila debemos verificar si la entrada correspondiente a la Columna II puede ser reemplazada por $x_0$ o no. Es decir para la fila $"x\quad y \quad w(xy)"$ debemos verificar si $xx_0$ es arista y si lo fuera y además $w(xx_0) < w(xy)$, entonces debemos reemplazar $y$ por $x_0$. Esto agrega otras $n-i-1$ comparaciones. El número total de comparaciones requeridas es
$$
\sum_{i=1}^{n-1} 2(n-i-1) = (n-1)(n-2).
$$
Esto nos dice que para encontrar un MST de un grafo deben hacerse alrededor de $n^2$ operaciones.

\begin{subsection}{Ejercicios} (continuación)
\begin{enumerate}
\item Usar el algoritmo greedy para encontrar un MST del grafo representado en la Fig. \ref{f6.5}. ?`Es en este caso el MST único?
\end{enumerate}

\begin{figure}[ht]
	\begin{tikzpicture}[scale=1.8]
	%\SetVertexSimple[Shape=circle,MinSize=5 pt,FillColor=white]
	\Vertex[x=0.00, y=0.00, L = $a$]{1}
	\Vertex[x=1.5, y=1, L =$b$]{2}
	\Vertex[x=1.5, y=-0, L=$c$]{3}
	\Vertex[x=1.5, y=-1, L=$d$]{4}
	\Vertex[x=3, y=1, L =$e$]{5}
	\Vertex[x=3, y=-0, L=$f$]{6}
	\Vertex[x=3, y=-1, L=$g$]{7}
	\Vertex[x=4.5, y=1, L =$h$]{8}
	\Vertex[x=4.5, y=-0, L=$i$]{9}
	\Vertex[x=4.5, y=-1, L=$j$]{10}
	\Vertex[x=6, y=-0, L=$k$]{11}
	\Edge[label=2](1)(2)
	\Edge[label=8](1)(3)
	\Edge[label=1](1)(4)
	\Edge[label=6](2)(3)
	\Edge[label=7](3)(4)
	\Edge[label=1](2)(5)
	\Edge[label=1](3)(6)
	\Edge[label=9](4)(7)
	\Edge[label=2](5)(8)
	\Edge[label=6](6)(9)
	\Edge[label=1](7)(10)
	\Edge[label=9](8)(11)
	\Edge[label=2](9)(11)
	\Edge[label=4](10)(11)
	\Edge[label=5](3)(5)
	\Edge[label=1](3)(6)
	\Edge[label=2](3)(7)
	\Edge[label=3](5)(6)
	\Edge[label=4](6)(7)
	\Edge[label=9](5)(9)
	\Edge[label=3](7)(9)
	\Edge[label=7](8)(9)
	\Edge[label=1](9)(10)
	\end{tikzpicture}
	\caption{Encontrar el MST} \label{f6.5}
\end{figure}

\begin{enumerate}
\item Sea $G$ un grafo con pesos cuyos vértices son $x,a,b,c,d,e,f$ y cuyas aristas y pesos vienen dados por la siguiente tabla:

\begin{align*}
xa &&xb &&xc &&xd &&xe &&xf &&ab &&bc &&cd &&de &&ef &&fa \\
6  &&3  &&2  &&4  &&3  &&7  &&6  &&2  &&3  &&1  &&8  &&6.
\end{align*}

Encontrar todos los árboles expandidos mínimos para $G$. \item Suponga que $T$ es un árbol expandido mínimo en un grafo con pesos $K$ y sea $e^*$ una arista de $K$ que no es de $T$. Sea $e$ una arista de $T$ perteneciente al único camino en $T$ que une los vértices de $e^*$. Probar que $w(e) \le w(e^*)$.
\item Escribir un ``programa'' para el algoritmo greedy basado en el método ta\-bu\-lar mostrado más arriba.

\end{enumerate}
\end{subsection}
\end{section}

\begin{section}{Ejercicios}
\begin{enumerate}
\item Sea $T$ un árbol con raíz $m$-ario con $n$ vértices, $l$ hojas e $i$ vértices internos. Probar que
$$
n=mi+1
$$
y encontrar la ecuación de $l$ en términos de $m$ y $n$.
\item Para cada uno de los seis diferentes árboles (sin raíz) con seis vértices (ver el ejercicio \ref{ejercicios5.5}) (\ref{ejercicio5.5.1}), encontrar el número de elecciones esencialmente diferentes que podemos hacer de un vértice para designarlo como raíz. En base a esto calcular el número de árboles con raíz diferentes que tengan seis vértices.
\item Probar que el número de árboles expandidos mínimos diferentes en $K_5$ es 125 (no trate de listarlos). \item Denotemos los vértices del grafo completo $K_n$ con $1, 2, \ldots,n$, y para cada árbol expandido $T$ de $K_n$ definamos el {\em símbolo de Prüfer} $(p_1,p_2,\ldots,p_{n-2})$ de la \index{símbolo de Prüfer} siguiente manera: el símbolo de Prüfer de un árbol con dos vértices es $0$, el símbolo de Prüfer de un árbol $T$ con $n$ vértices es $(j,q_1,\ldots,q_{n-3})$ donde
\begin{enumerate}
\item[(a)]  Si $i$ es el primer vértice de $T$ (en el orden dado) que tiene valencia uno, entonces $j$ es el único vértice adyacente a $i$.
\item[(b)] $(q_1,\ldots,q_{n-3})$ es el símbolo de Prüfer del árbol que se obtiene a partir de $T$ eliminando la arista $ij$.
\end{enumerate}
Probar que la construcción del símbolo de Prüfer define una biyección desde el conjunto de árboles expandidos de $K_n$ al conjunto de $(n-2)$-uplas ordenadas del conjunto $\{1,2,\ldots,n\}$. Deducir de lo anterior que $K_n$ tiene $n^{n-2}$ árboles expandidos.
\item Supongamos que tenemos cuatro monedas y sabemos que una de ellas puede ser falsa (más liviana o más pesada) . Probar que encontrar la moneda falsa (si existe) requiere teóricamente al menos dos pesadas, pero que este número no es posible lograrlo. (En este problema no se da un moneda verdadera.)
\end{enumerate}
\end{section}

\appendix
\setcounter{chapter}{0}
\renewcommand{\thechapter}{\Alph{chapter}}
\chapter[Permutaciones]{Permutaciones}

\begin{section}{Permutaciones}
Recordemos  que una { {\em permutación}} de un conjunto finito no vacío $X$ es una biyección de $X$ en $X$. (Frecuentemente tomamos $X$ como $ [[1,n]]=\{1,2,\ldots,n\}$.) Por ejemplo una permutación típica de $[[1,5]]$ es la función $\alpha$ definida por las ecuación
$$
\alpha(1)=2,\quad \alpha(2)=4,\quad \alpha(3)=5,\quad
\alpha(4)=1,\quad \alpha(5)=3.
$$

Denotemos el conjunto de todas las permutaciones de $[[1,n]]$ con $S_n$. Por ejemplo, $S_3$ contiene las $3!=6$ permutaciones siguientes:
$$
\begin{matrix} 1&2&3 \\ \downarrow&\downarrow&\downarrow\\1 &2 &
3\end{matrix}\qquad
\begin{matrix} 1&2&3 \\ \downarrow&\downarrow&\downarrow\\1 &3 &2 \end{matrix}\qquad
\begin{matrix} 1&2&3 \\ \downarrow&\downarrow&\downarrow\\2 & 1&3 \end{matrix}
\qquad
\begin{matrix} 1&2&3 \\ \downarrow&\downarrow&\downarrow\\2 & 3&1
\end{matrix}\qquad
\begin{matrix} 1&2&3 \\
\downarrow&\downarrow&\downarrow\\3 &1 &2 \end{matrix}\qquad
\begin{matrix} 1&2&3 \\ \downarrow&\downarrow&\downarrow\\3 &2 &1
\end{matrix}
$$

En la práctica, usualmente asignamos alguna interpretación concreta a un elemento de $S_n$. Como vimos en la sección \ref{permutaciones}, podemos usar la interpretación ``selecciones ordenadas sin repetición''  \index{selección ordenada sin repetición} donde, en este caso seleccionamos los elementos de $\{1,2,3,\ldots,n\}$ en algún orden hasta que no queda ninguno. Una interpretación relacionada es que una permutación efectúa un {\it reacomodamiento} de $\{1,2,3,\ldots,n\}$; por ejemplo, la permutación $\alpha$ vista más arriba efectúa el reacomodamiento de 12345, en 24513, así:
$$
\begin{matrix} 1&2&3&4&5 \\
\downarrow&\downarrow&\downarrow&\downarrow&\downarrow\\2 &4 &5 &1
& 3
\end{matrix}
$$
En algunas circunstancias es conveniente mirar una permutación y el correspondiente reacomodamiento como la misma cosa, pero esto puede traer dificultades si debemos considerar sucesivos reacomodamientos. Es importante tener en cuenta que 
$$
\text{\it una permutación es una función con ciertas características.}
$$

Cuando las permutaciones son tratadas como funciones es claro como deben combinarse. Consideremos $\alpha$ la permutación de $[[1,5]]$ antes mencionada, y sea $\beta$ la permutación de $[[1,5]]$ dada por 
$$
\beta(1)=3,\quad \beta(2)=5,\quad \beta(3)=1,\quad
\beta(4)=4,\quad \beta(5)=2.
$$
La función compuesta $\beta\alpha$ es la permutación definida por $(\beta\alpha)(i)= \beta(\alpha(i))$ ($1\le i\le 5$), esto es 
$$
\beta\alpha(1)=5,\quad \beta\alpha(2)=4,\quad
\beta\alpha(3)=2,\quad \beta(4)\alpha=3,\quad \beta\alpha(5)=1.
$$
(Recordemos que, como siempre, $\beta\alpha$ significa ``primero $\alpha$, entonces $\beta$''.) En términos de reacomodamientos tenemos
$$\begin{aligned}
\alpha\quad&\quad\begin{matrix} 1&2&3&4&5 \\
\downarrow&\downarrow&\downarrow&\downarrow&\downarrow\\2 &4 &5 &1
& 3
\end{matrix} \\
\beta \quad&\quad \begin{matrix} 1&2&3&4&5 \\
\downarrow&\downarrow&\downarrow&\downarrow&\downarrow\\5 &4 &2 &3
& 1
\end{matrix}.
 \end{aligned}
$$

Existen cuatro características de la composición de permutaciones de gran importancia, y están listadas en el próximo teorema.

\begin{teorema}\label{tA3} Las siguientes propiedades valen en el conjunto $S_n$ de todas las permutaciones de $\{1,2,3,...,n\}$.
\begin{enumerate}
\item[(i)] Si $\pi$ y $\sigma$ pertenecen a $S_n$, entonces $\pi\sigma$ también.
\item[(ii)] Para cualesquiera permutaciones $\pi$, $\sigma$, $\tau$ en $S_n$,
$$
(\pi\sigma)\tau=\pi(\sigma\tau).$$
\item[(iii)] La función identidad, denotada por $\operatorname{id}$ y definida por $\operatorname{id}(r) =r$ para todo $r$ en $\mathbb N_n$, es una permutación y para cualquier $\sigma$ en $S_n$,
tenemos
$$
\operatorname{id}\sigma=\sigma\operatorname{id}=\sigma.$$
\item[(iv)] Para toda permutación $\pi$ en $S_n$ hay una permutación inversa $\pi^{-1}$ en $S_n$ tal que
$$
\pi\pi^{-1} = \pi^{-1}\pi = \operatorname{id}.
$$
\end{enumerate}
\end{teorema}
\begin{proof} Todas las afirmaciones se deducen de propiedades conocidas de funciones en general y funciones biyectivas en particular. Por otro lado, es fácil convencerse de la validez de las mismas mirando las permutaciones como reacomodamiento de elementos. 
\end{proof}

Es conveniente tener una notación más compacta para las permutaciones. Consideremos otra vez la permutación $\alpha$ de $\{1,2,3,4,5\}$, y notemos en particular que
$$
\alpha(1)=2,\qquad \alpha(2)=4,\qquad \alpha(4)=1.
$$
Así $\alpha$ lleva 1 a 2, 2 a 4 y 4 a 1, y por esta razón decimos que los símbolos 1,2,4 forma un {\it ciclo } (de longitud 3). Del mismo modo, los símbolos 3 y 5 forman un ciclo de longitud 2, y
escribimos:
$$
\alpha=(1\,2\,4)(3\,5).
$$
Esta es la {\it notación cíclica} para $\alpha$. Cualquier \index{notación cíclica} permutación $\pi$ puede ser escrita cíclicamente de la siguiente manera:
\begin{itemize}
\item comencemos con algún símbolo (digamos el 1) y veamos el efecto de $\pi$ sobre él y sus sucesores hasta que alcancemos el 1
nuevamente;
\item elijamos un símbolo que todavía no haya aparecido y construyamos el ciclo que se deriva de él; 
\item repitamos el procedimiento hasta que se terminen los símbolos.
\end{itemize}
Por ejemplo, la permutación $\beta$ definida antes tiene la notación cíclica
$$
\beta=(1\,3)(2\,5)(4),
$$
donde observamos que el símbolo 4 forma un ciclo ``degenerado'' por sí solo, puesto que $\beta(4)=4$. En algunas ocasiones podemos omitir estos ciclos de longitud 1 cuando escribimos una permutación en notación cíclica, puesto que corresponden a símbolos que no son afectados por la permutación. Sin embargo, usualmente es útil {\it no} adoptar esta convención hasta que uno se familiariza con la notación.

Aunque la representación de una permutación en notación cíclica es esencialmente única, hay dos manera obvias en las que podemos cambiar la notación sin alterar la permutación. Primero, cada ciclo puede empezar en cualquiera de sus símbolos; por ejemplo $(7\,8\,2\,1\,3)$ y $(1\,2\,7\,8\,2)$ describen el mismo ciclo. Segundo, el orden de los ciclos no es importante; por ejemplo $(1\,2\,4) (3\,5)$ y $(3\,5) (1\,2\,4)$ denotan la misma permutación. Pero las características importantes son el número de ciclos, la longitud del ciclo, y la disposición de los símbolos dentro de los ciclos, y éstas están determinadas de manera única. Por eso, la rotación cíclica nos dice bastantes cosas útiles sobre una permutación.

\begin{ejemplo}\label{cartas} Cartas numeradas del 1 al 12 son distribuidas en una mesa en la manera en que se muestra en la parte izquierda de la tabla que sigue. Luego las cartas son levantadas  por filas (de izquierda a derecha y de arriba hacia abajo) y se redistribuyen con el mismo arreglo, pero por columnas, no por filas (de arriba hacia abajo y de izquierda a derecha), apareciendo como se muestra en la parte derecha de la tabla.
$$
\begin{matrix} 1& 2& 3\\
4 &5 &6 \\
7 &8 & 9\\
10 &11 & 12 \end{matrix}\qquad \qquad\qquad
\begin{matrix}1 &5 &9 \\
2 &6 &10 \\
3& 7& 11\\
4&8 & 12 \end{matrix}
$$
?`Cuántas veces debe repetirse este procedimiento hasta que las cortas aparezcan dispuestas como estaban inicialmente?
\end{ejemplo}
\begin{proof}[Solución] Sea $\pi$ la permutación que efectúa el reordenamiento; esto es $\pi(i) =j$ si la carta $j$ aparece en la posición previamente ocupada por la carta $i$. Trabajando con la notación cíclica para $\pi$ encontramos que
$$
\pi=(1)(2\,\,5\,\,6\,\,10\,\,4)(3\,\,9\,\,11\,\,8\,\,7)(12).
$$
Los ciclos degenerados $(1)$ y $(12)$ indican que las cartas 1 y 12 nunca cambian de posición. Las otros ciclos tienen longitud 5, así que cuando el proceso se haya realizado 5 veces las cartas
reaparecerán en sus posiciones originales. Otra forma de expresar el resultado es decir que $\pi^5= \operatorname{id}$, donde $\pi^5$ representa las cinco repeticiones de la permutación $\pi$.
\end{proof}
\end{section}

\begin{section}{Ejercicios}
\begin{enumerate}
\item Escribir en notación cíclica la permutación que realiza el siguiente reordenamiento:
$$
\begin{matrix} 1&2&3&4&5&6&7&8&9 \\
\downarrow&\downarrow&\downarrow&\downarrow&
\downarrow&\downarrow&\downarrow&\downarrow&\downarrow
\\ 3&5 &7 &8 &4 &6 &1 &2 &9
\end{matrix}.
$$
\item Sean $\sigma$ y $\tau$ las permutaciones de $[[1,8]]$ cuyas representaciones en la notación cíclica son
$$
\sigma= (1\,2\,3)(4\,5\,6)(7\,8),\qquad
\tau=(1\,3\,5\,7)(2\,6)(4)(8).
$$
Escribir en notación cíclica $\sigma\tau$, $\tau\sigma$, $\sigma^2$, $\sigma^{-1}$, $\tau^{-1}$. 
\item Resolver el problema presentado en el ejemplo \ref{cartas} cuando hay 20 cartas acomodadas en 5 filas de 4.
\item Probar que hay exactamente tres elementos de $S_4$ que tienen dos ciclos de longitud 2, escritos en la notación cíclica. 
\item Sea $K$ el subconjunto de $S_4$ que contiene la identidad y las tres permutaciones descritas en el ejercicio previo. Escribir la ``tabla de multiplicación'' para $K$, interpretando la multiplicación como la composición de permutaciones.
\item Calcular en número total de permutaciones $\sigma$ de $\mathbb [[1,6]]$ que satisfacen $\sigma^2=\text{id}$ y $\sigma\not=\text{id}$.
\item Sean $\alpha$ y $\beta$ permutaciones de $[[1,9]]$ cuyas representaciones en la notación cíclica son:
$$
\alpha= (1237)(49)(58)(6),\qquad \beta=(135)(246)(789).
$$
Escribir en notación cíclica $\alpha\beta$, $\beta\alpha$, $\alpha^2$, $\beta^2$, $\alpha^{-1}$, $\beta^{-1}$.
\item Sea $X_1=\{0,1\}$, y para $i\ge 2$ definamos $X_i$ como el conjunto de subconjuntos de $X_{i-1}$. Encontrar el valor más pequeño para el cual $|X_i|>10^{100}$.
\item Por cada entero $i$ en el rango $1 \le i \le n-1$ definimos $\tau_i$ como la permutación de $[[1,n]]$ que intercambia $i$ e $i+1$ y no afecta los otros elementos. Explícitamente 
$$
\tau_i = (1)(2)\cdots(i-1)(i\ i+1)(i+2)\cdots(n).
$$
Probar que toda permutación de $[[1,n]]$ puede ser expresada en términos de $\tau_1,\tau_2,\ldots,\tau_{n-1}$. 
\item Una permutación de $[[1,n]]$ que tenga solo un ciclo (necesariamente de longitud $n$) es llamada {\it cíclica}  \index{permutación cíclica}. Probar que hay $(n-1)!$ permutaciones cíclicas de $[[1,n]]$.
\item Un mazo de 52 cartas es dividido en dos partes iguales y luego se alternan las cartas de una y otra parte. Es decir si la numeración original era $1,2,3,\ldots,54$, el nuevo orden es $1,27,2,28,\ldots$?`Cuántas veces se debe repetir este procedimiento para obtener de nuevo el mazo original? 
\end{enumerate}
\end{section}

\appendix
\setcounter{chapter}{1}
\renewcommand{\thechapter}{\Alph{chapter}}
\chapter[El principio del tamiz]{El principio del tamiz} \label{principiodeltamiz}

\begin{section}{El principio del tamiz}\label{Ap1.2}
El principio más básico del conteo (proposición \ref{principiodeadicion}) dice que $|A \cup B|$ es la suma de $|A|$ y $|B|$, cuando $A$ y $B$ son conjuntos
disjuntos. Si $A$ y $B$ no son disjuntos, cuando sumamos $|A|$ y $|B|$ estamos contando $A \cap B$ dos veces. Entonces, para obtener la respuesta correcta debemos restar $|A \cap B|$:
$$
|A \cup B| = |A|+|B| - |A \cap B|.
$$

Un método similar puede aplicarse a tres conjuntos. Cuando sumamos $|A|$, $|B|$ y $|C|$, los elementos de $A \cap B$, $B \cap C$, y $C \cap A$ son contados dos veces (si no están en los tres
conjuntos). Para corregir esto, restamos $|A \cap B|$, $|B \cap C|$ y $|C \cap A|$. Pero ahora los elementos de $A \cap B \cap C$, contados originalmente tres veces, han sido descontados tres
veces. Luego, para conseguir la respuesta correcta, debemos sumar $|A \cap B \cap C|$. Así
$$
|A \cup B\cup C|= \alpha_1-\alpha_2+\alpha_3,
$$ 
donde
$$\gathered
\alpha_1=|A|+|B|+|C|,\qquad \alpha_2= |A \cap B|+|B \cap C|+|C \cap A|, \\
\alpha_3 = |A \cap B \cap C|.
\endgathered
$$

Este resultado es un caso simple de lo que suele ser llamado, por razones obvias, el principio de inclusión y exclusión. También
\index{principio de inclusion y exclusion} se lo llama el { \it principio del tamiz}.  \index{principio del tamiz}

\begin{teorema}\label{tA1.2} Si $A_1,A_2,\ldots,A_n$ son conjuntos finitos, entonces
$$ |A_1 \cup A_2 \cup \ldots \cup A_n|= \alpha_1-\alpha_2+\alpha_3 + \cdots +(-1)^n\alpha_n, $$ donde $\alpha_i$ es la suma de los 
cardinales de las intersecciones de los conjuntos tomados de a $i$ por vez ($1 \le i \le n$).
\end{teorema}
\begin{proof} Debemos demostrar que cada elemento $x$ de la unión hace una contribución neta de 1 al miembro de la derecha.
Supongamos que $x$ pertenece a $k$ de los conjuntos $A_1, A_z,\ldots,A_n$. Entonces $x$ contribuye con $k$ en la suma $\alpha_1=|A_1|+\cdots+|A_n|$. En la suma $\alpha_2$, $x$ contribuye 1 en $|A_i \cap A_j|$ cuando $A_i$ y $A_j$ están entre los $k$ conjuntos que contienen a $x$. Existen $\binom{k}{2}$ de esos pares, por lo tanto $\binom{k}{2}$ es la contribución de $x$ a $\alpha_2$. En general la contribución de $x$ a $\alpha_i$ ($1 \le i \le n$) es $\binom{k}{i}$. Por lo tanto el total con que contribuye $x$ al lado derecho de la igualdad es 
$$
\binom{k}{1} -\binom{k}{2} + \cdots + (-1)^{k-1} \binom{k}{k},
$$
porque los términos con $i > k$ dan cero.

Por el teorema del binomio aplicado a $(1-1)^k=0$, se deduce que la expresión de arriba  es igual a
$\binom{k}{0}$, que vale 1.
\end{proof}

Un corolario simple del Teorema \ref{tA1.2} es a menudo más útil en la práctica. Supongamos que $X$ es un conjuntos finito y $A_1,A_2,\ldots,A_n$ son subconjuntos de $X$ (cuya unión no necesariamente es igual a $X$). Si $|X| = N$, entonces el número de elementos de $X$ que no están en ninguno de esos subconjuntos es
$$\begin{aligned}
|X-(A_1 \cup A_2 \cup \ldots \cup A_n)|&=
|X|-|A_1 \cup A_2 \cup \ldots \cup A_n| \\
&= N- \alpha_1 + \alpha_2 - \cdots + (-1)^n\alpha_n.
\end{aligned}
$$

\begin{ejemplo} Hay 73 estudiantes en el primer año de la Escuela de Artes de la universidad. De ellos, 52 saben tocar el piano, 25 el violín y 20 la flauta; 17 pueden tocar tanto el piano como el violín, 12 el piano y la flauta; pero solo Juan Rictero puede tocar los tres instrumentos ?`Cuántos alumnos no saben tocar ninguno de esos instrumentos?
\end{ejemplo}
\begin{proof}[Solución] Con $V$, $P$ y $F$ denotaremos los conjuntos de estudiantes que saben tocar el violín, el piano y la flauta
respectivamente. Usando la información dada tenemos que $$
\begin{aligned}
\alpha_1&= |P| + |V| + |F|= 52+25+20=97 \\
\alpha_2&= |P\cap V| + |V\cap F| + |P\cap F|=17+7+12=36 \\
\alpha_3&= |P\cap V\cap F|= 1.
\end{aligned}
$$
Por consiguiente, el número de estudiantes que o pertenecen a ninguno de los tres conjuntos $P$, $V$ y $F$ es
$$
73-97+36-1=11.
$$
\end{proof}

\begin{ejemplo} Ejemplo Una secretaria ineficiente tiene $n$ cartas distintas y $n$ sobres con direcciones ?`De cuántas maneras puede
ella arreglárselas para meter cada carta en un sobre equivocado? (Esto es comúnmente llamado el {\it problema del desarreglo} del
cual hay varias formulaciones pintorescas.) 
\end{ejemplo}
\begin{proof}[Solución] Podemos considerar cada carta y su correspondiente sobre  como si estuvieran etiquetadas con un entero $i$ en el rango $1 \le i \le n$. El acto de poner las cartas en los sobres puede describirse como una permutación $\pi$  de $\mathbb N_n$: $\pi(i)=j$ si la carta $i$ va en el sobre $j$. Necesitamos saber  el número de {\em desarreglos}, esto es, las permutaciones $\pi$ tales que
$\pi(i)\not=i$ para todo $i$ en $\mathbb N_n$.

Denotemos $A_i$ ($1 \le i \le n$) el subconjunto de $S_n$ (el conjunto de permutaciones de $\mathbb N_n$) que contiene aquellos $\pi$ tales que $\pi(i)=i$. Diremos que los elementos de $A_i$
{\it fijan} $i$. Por el principio del tamiz, el número de desarreglos es 
$$
d_n= n! -\alpha_1+\alpha_2 - \cdots +(-1)^n\alpha_n,
$$
donde $\alpha_r$ es la suma de los cardinales de las intersecciones de los $A_i$ tomando r por vez. En otras palabras, $\alpha_r$ es el número de permutaciones que fijan $r$ símbolos
dados, tomando todas las maneras de elegir los $r$ símbolos. Ahora hay $\binom{n}{r} $ maneras de elegir $r$ símbolos, y el número de permutaciones que los fijan es solo el número de permutaciones de
los restantes $n-r$ símbolos, que es $(n-r)!$.  Por lo tanto
$$
\alpha_r = \binom{n}{r} \dot (n-r)! = \frac{n!}{r!},\qquad d_n=
n!\left(1-\frac{1}{1!} + \frac{1}{2!}-\cdots
+(-1)^n\frac{1}{n!}\right). \nopagebreak$$
\end{proof}

\begin{subsection}{Ejercicios}\label{ejerciciosA1.2}
\begin{enumerate}
\item En una clase de 67 estudiantes de matemática, 47 leen francés, 35
leen alemán y 23 leen ambos lenguajes ?`Cuántos estudiantes no lee
ninguno de los dos lenguajes? Si además 20 leen ruso, de los
cuales 12 también leen francés, 11 leen alemán y 5 leen los tres
lenguajes, ?`cuántos estudiantes no leen ninguno de los tres
lenguajes?
\item Encuentre el número de formas de ordenar las letras A,E,M,O,U,Y en
una secuencia de tal forma que las palabras ME e YOU no aparezcan.
\item
Calcule el número $d_4$ de desarreglos de $\{1,2,3,4\}$ y escriba,
en la notación cíclica, las permutaciones relevantes.
\item
Use el principio de inducción para probar que la fórmula para
$d_n$ satisface la recursión
$$
d_1=0, \quad d_2=1,\quad d_n= (n-1)(d_{n-1}+d_{n-2}) \ (n\ge 3).
$$
\item
Probar que el número de desarreglos de $\{1,2,\ldots,n\}$ en el
cual un objeto dado (digamos el 1) está en un 2-ciclo es
$(n-1)d_{n-2}$. Utilizando esto dar una prueba directa de la
fórmula recursiva del ejercicio anterior.
\end{enumerate}
\end{subsection}

\end{section}

\appendix
\setcounter{chapter}{2}
\renewcommand{\thechapter}{\Alph{chapter}}
 \chapter[La función de Euler]{La función de Euler}

 \begin{section}{La función de Euler} \label{A2.1 }

 En esta sección probaremos un útil e importante teorema, usando sólo
 los conceptos de conteo más básicos.

 El teorema se refiere a las
 propiedades de divisibilidad de los enteros. Recordemos que dos enteros $x$ e
 $y$ son {\it coprimos} si el $\mcd(x,y) = 1$. Por cada $n \ge 1$ sea
 $\phi(n)$ el número de
 enteros $x$ en el rango $1 \le x \le n$ tal que $x$ y $n$ son coprimos.
 Podemos
 calcular los primeros valores de $\phi(n)$ haciendo una tabla (Tabla
 \ref{tablaA2.1.1}).

 \begin{table}[h]
 %Tabla A2.1.1
 \begin{alignat*}3
 &n& \text{\quad Coprimos a $n$\quad }& &\phi(n)&\\
&&&&&\\
 &1&1&&1& \\
 &2&1&&1& \\
 &3&1,2&&2& \\
 &4&1,3&&2& \\
 &5&1,2,3,4&&4& \\
 &6&1,5&&2& \\
 &7&1,2,3,4,5,6&&6& \\
 &8&1,3,5,7&&4&
 \end{alignat*}
\caption{} \label{tablaA2.1.1}
\end{table}

La función es llamada {\em función de Euler}, debido a Leonhard
Euler  \index{Euler, Leonhard} (1707-1783). Cuando $n$ es primo,
digamos $n=p$, cada uno de los enteros $1,2,\ldots,p-1$ es coprimo
con $p$, entonces tenemos
$$
\phi(p)=p-1\qquad\text{ ($p$ primo).}
$$
Nuestra tarea ahora es probar un resultado respecto a la suma de
los valores $\phi(d)$, donde los $d$ son todos los divisores de un
número positivo $n$ dado. Por ejemplo, cuando $n=12$, lo divisores
$d$ son $1$, $2$, $3$, $4$, $5$, $6$ y $12$, podemos ver que
\begin{align*}
&\quad\phi(1)+\phi(2)+\phi(3)+\phi(4)+\phi(6)+\phi(12)\\
&= 1 +1+2+2+2+4 \\
&=12.
\end{align*}

Debemos demostrar que la suma es siempre igual al entero $n$ dado.

\begin{teorema}\label{tA2.1b} Para cualquier $n$ entero positivo,
$$
\sum_{d|n} \phi(d)=n.
$$
\end{teorema}
\begin{proof} Sea $S$ el conjunto de pares de enteros $(d,f)$ que
satisfacen
$$
d|n, \qquad 1\le f \le d, \qquad \mcd(f,d)=1.
$$

\begin{table}[h]
%Tabla A2.1.2
\begin{align*}
&  &f &&1  &&2 &&3 &&4 &&5  &&6 &&7 &&8 &&9 &&10 &&11 &&12 && &\phi(d)\\
&d &  &&   &&  &&  &&  &&   &&  &&  &&  &&  &&   &&   &&   && &   \\
&1 &  &&12 &&  &&  &&  &&   &&  &&  &&  &&  &&   &&   &&   && &1  \\
&2 &  &&6  &&  &&  &&  &&   &&  &&  &&  &&  &&   &&   &&   && &1  \\
&3 &  &&4  &&8 &&  &&  &&   &&  &&  &&  &&  &&   &&   &&   && &2  \\
&4 &  &&3  &&  &&9 &&  &&   &&  &&  &&  &&  &&   &&   &&   && &2  \\
&6 &  &&2  &&  &&  &&  &&10 &&  &&  &&  &&  &&   &&   &&   && &2  \\
&12&  &&1  &&  &&  &&  &&5  &&  &&7 &&  &&  &&   &&11 &&   && &4  \\
&  &  &&   &&  &&  &&  &&   &&  &&  &&  &&  &&   &&   &&   && &12
\end{align*}
\caption{} \label{tablaA2.1.2}
\end{table}

La Tabla \ref{tablaA2.1.2} muestra $S$ cuando $n=12$; la ``marca''
que indica que $(d,f)$ pertenece a $S$ es un número cuya
importancia explicaremos en seguida. Por lo general, el número de
``marcas'' en la fila $d$ es el número de $f$'s en el rango $1\le
f\le d$ que satisfacen que el $\mcd(d,f)=1$; esto es $\phi(d)$.
Por lo tanto, contando $S$ por el método de las filas obtenemos
$$
|S| = \sum_{d|n} \phi(d).
$$
Para demostrar que $|S|=n$ debemos construir una biyección $\beta$
de $S$ en $\mathbb N_n$. Dado un par $(d,f)$ en $S$, definimos
$$
\beta(d,f) = f n/d.
$$
En la tabla, $\beta(d,f)$ es la ``marca'' en la fila $d$ y la
columna $f$. Como $d| n$, el valor de $\beta$, es un entero y como
$1\le f\le d$, entonces $\beta(d,f)$ pertenece a $\mathbb N_n$.

Para probar que $\beta$ es una inyección observemos que
$$
\beta(d,f) = \beta(d',f') \quad \Rightarrow \quad fn/d = f'n/d'
\quad \Rightarrow \quad fd'=f'd.
$$
Pero $f$ y $d$ son coprimos, así como también lo son $f'$ y $d'$,
así que podemos concluir que $d=d'$ y $f=f'$.

Para demostrar que $\beta$ es una suryección, supongamos que nos
dan un $x$ que pertenece a $\mathbb N_n$. Sea $g_x$ el mcd de $x$
y $n$, y sea
$$
d_x = n/g_x, \qquad f_x = x /g_x.
$$
Puesto que $g_x$ es un divisor de $x$ y $n$, entonces $d_x$ y
$f_x$ son enteros, y como $g_x$ es el mcd, $d_x$ y $f_x$ son
coprimos. Ahora
$$
\beta(d_x,f_x) = f_x n/d_x = x,
$$
y por lo tanto $\beta$ es suryectiva.

Luego $\beta$ es biyectiva y $|S|=n$, como queríamos demostrar.
\end{proof}

\begin{subsection}{Ejercicios}
\begin{enumerate}
\item Encontrar los valores de $\phi(19), \phi(20), \phi(21)$.
\item Probar que si $x$ y $n$ son coprimos, entonces lo son $n-x$ y
$n$. Deducir que $\phi(n)$ es par para todo $n \ge 3$.
\item Probar que, si $p$ es un primo y $m$ es un entero positivo,
entonces un entero $x$ en el rango $1 \le x \le p^m$ {\it no} es
coprimo a $p^m$ si y solo si es un múltiplo de $p$. Deducir que
$\phi(p^m) = p^m - p^{m-1}$.
\item Encontrar un contraejemplo que confirme que es falsa la conjetura
$\phi(ab)= \phi(a)\phi(b)$, para enteros cualesquiera $a$ y $b$.
Trate de modificar la conjetura de tal forma que no pueda
encontrar un contraejemplo.
\item Probar que para cualesquiera enteros positivos $n$ y $m$ se cumple:
$$
\phi(n^m) =n^{m-1}\phi(n).
$$
\item Calcular $\phi(1000)$ y $\phi(1001)$.
\end{enumerate}
\end{subsection}

\end{section}

\begin{section}{Una aplicación aritmética del principio del
tamiz}\label{Ap2.2} Por cientos de años los matemáticos han
estudiado problemas sobre números primos y la factorización de los
enteros. Nuestra breve discusión sobre estos temas en los primeros
capítulos debería haber convencido al lector de que estos
problemas son difíciles, porque los primos mismos se encuentran
irregularmente distribuidos, y porque no hay una forma directa de
encontrar la factorización en primos de un entero dado. De todos
modos, si se nos da la factorización en primos de un entero, es
relativamente fácil responder ciertas preguntas sobre sus
propiedades aritméticas. Supongamos, por ejemplo que queremos
listar todos los divisores de un entero $n$ y sabemos que la
factorización de $n$ es
$$
n=p_1^{e_1}p_2^{e_2}\cdots p_r^{e_r}.
$$
Entonces un entero $d$ es divisor de $n$ si y solo si no tiene
divisores primos distintos de los de $n$, y ningún primo divide
más veces a $d$ que a $n$. Visto así, los divisores son
precisamente los enteros que pueden escribirse de la forma
$$
d=p_1^{f_1}p_2^{f_2}\cdots p_r^{f_r},
$$
donde cada $f_i$ ($1\le i \le r$) satisface $0\le f_i \le e_i$.
Por ejemplo dado que $60= 2^2 \times 3 \times 5$ podemos listar
rápidamente todos los divisores de 60.

Un problema similar es encontrar el número de enteros $x$ en el
rango $1 \le x \le n$ que son coprimos con $n$. En la sección
\ref{A2.1 } denotamos este número con $\phi(n)$, el valor de la
función $\phi$ de Euler en $n$. Ahora demostraremos que si la
factorización en primos de $n$ es conocida, entonces $\phi(n)$
puede ser calculado por el principio del tamiz.

\begin{ejemplo}?`Cuál es el valor de $\phi(60)$? En otras
palabras, ?`cuántos enteros $x$ en el rango $1 \le x \le 60$
satisfacen $\operatorname{mcd}(x,60)=1$?
\end{ejemplo}
\begin{proof}[Solución] Sabemos que $60 =2^2 \times 3 \times 5$, así que
podemos contar el números de enteros $x$ en el rango $1 \le x \le
60$ que no son divisibles por 2, 3 o 5. Con $A(2)$ denotemos el
subconjunto de $\mathbb N_{60}$ que contiene los enteros que \it
son \rm divisibles por 2, con $A(2,3)$ aquellos que \it son \rm
divisibles por 2 y 3, y así sucesivamente, entonces tenemos
$$\begin{aligned}
\phi(60)&=60-|A(2) \cup A(3) \cup A(5)| \\
&= 60-|A(2) + A(3) + A(5)| \\
&\qquad+(|A(2,3) + |A(2,5)| + |A(3,5)|)-|A(2,3,5)|,
\end{aligned}
$$
por el principio del tamiz. Ahora $|A(2)|$ es el número de
múltiplos de 2 en $\mathbb N_{60}$ que es $60 /2 = 30$. Del mismo
modo $|A(2,3)|$ es el número de múltiplos de $2 \times 3$, que es
$60 /(2\times 3) = 10$, y así siguiendo, por lo tanto
$$
\phi(60) = 60 -(30+20+10)+(10+6+4)-2=16.
$$
\end{proof}

El mismo método puede ser usado para dar una fórmula explícita
para $\phi(n)$ en el caso general.

\begin{teorema}\label{tA2.2} Sea $n \ge 2$ un entero cuya factorización
es $n=p_1^{e_1}p_2^{e_2}\ldots p_r^{e_r}$. Entonces
$$
\phi(n)=n\left(1-\frac{1}{p_1}\right)\left(1-\frac{1}{p_2}\right)\cdots\left(1-\frac{1}{p_r}\right).
$$
\end{teorema}
\begin{proof} Denotemos $A_j$ el subconjunto de $\mathbb N_n$
que contiene los múltiplos de $p_j$ ($1\le j \le r$). Entonces
$$
\begin{aligned}
\phi(n) &= n- |A_1 \cup A_2 \cup \cdots \cup A_r| \\
       &= n -\alpha_1+ \alpha_2- \cdots +(-1)^r\alpha_r
\end{aligned}
$$
donde $\alpha_i$ es la suma de los cardinales de las
intersecciones de los conjuntos tomados de a $i$. Una intersección
típica como
$$
A_{j_1}\cup A_{j_2}\cup \cdots \cup A_{j_i}
$$
contiene los múltiplos de $P= p_{j_1}\times p_{j_2}\times \cdots
\times p_{j_i}$ en $\mathbb N_n$, y estos son los números
$$
P,2P,3P,\ldots,\left(\frac{n}{p}\right)P.
$$
Luego la cardinalidad de una intersección típica es $n/P$, y
$\alpha_i$ es la suma de términos como
$$
\frac{n}{P}=
n\left(\frac{1}{p_{j_1}}\right)\left(\frac{1}{p_{j_2}}\right)\cdots
\left(\frac{1}{p_{j_i}}\right).
$$
Se sigue que
$$
\begin{aligned} \phi(n) = n - n\left(\frac{1}{p_1} + \frac{1}{p_2} +
\cdots +\frac{1}{p_r}\right) +n\left(\frac{1}{p_1p_2}
+ \frac{1}{p_1p_3}+\cdots\right) &+ \cdots \\
\cdots &+ (-1)^r \left(\frac{1}{p_1p_2 \cdots p_r}\right) \\
=n\left(1-\frac{1}{p_1}\right)&\left(1-\frac{1}{p_2}\right)\cdots\left(1-\frac{1}{p_r}\right).
\end{aligned}
$$
\end{proof}

\end{section}

\appendix
\setcounter{chapter}{3}
\renewcommand{\thechapter}{\Alph{chapter}}
\chapter{Grafos planares}

\begin{section}{Grafos Planares}\label{Ap4.1}
Usualmente el diagrama de un grafo se realiza en el plano por la
comodidad que esto representa. Esto no significa que todo grafos
sea lo que se denomina un {\em grafo planar}. ?`Qué es un grafo
\index{grafo planar} planar? Es un grafo tal que {\it existe} un
diagrama del grafo en el plano tal que no hay ningún cruce de
aristas. Por ejemplo, el grafo $K_3$ es claramente planar (Fig.
\ref{fA4.1} (a)). Claro que podríamos dibujar a $K_3$ como en la
Fig. \ref{fA4.1} (b) y no parecería planar.

\begin{figure}[ht]
	\begin{tabular}{cccc}
		&
		\begin{tikzpicture}[scale=1]
		\SetVertexSimple[Shape=circle,FillColor=white,MinSize=8 pt]
		\Vertex[x=0.00, y=0]{a}
		\Vertex[x=-0.5, y=1]{b}
		\Vertex[x=2., y=0]{c}
		\Edges(a,b,c,a)
		\end{tikzpicture}
		&
		\qquad
		& 
		\begin{tikzpicture}[scale=1]
		\draw[-,line width=1pt] (0,0) -- (1.1,0.9) -- (2,0);
		\SetVertexSimple[Shape=circle,FillColor=white,MinSize=8 pt]
		\Vertex[x=0.00, y=0]{a}
		\Vertex[x=-0.5, y=1]{b}
		\Vertex[x=2., y=0]{c}
		\Edges(a,b,c)
		\tikzstyle{vertex}=[circle,minimum size=5pt]
		\node[vertex] (v0) at (1.1,0.9) {};
		\Edges(a,v0,c)

		%\node[vertex] (v1) at (-0.5,1) {b};
		%\node[vertex] (v2) at (2,0) {c};
		\end{tikzpicture} 
		\\
		&(a)&&(b)
	\end{tabular}
	\caption{Dibujos de $K_3$} \label{fA4.1}
\end{figure}

Pero la definición es que un grafo es planar si se {\em puede}
dibujar en el plano sin cruces de aristas, no si {\em todo} dibujo
no tiene cruces. (Si la definición fuera así, ningún grafo seria
planar, pues siempre se puede dibujar cualquier grafo con cruces.)
Otro ejemplo, ya visto, $K_4$ puede ser dibujado como en la Fig
\ref{fA4.2} (a) y no parece planar, pero dibujado como en la Fig.
\ref{fA4.2} (b) muestra que $K_4$ es planar.

\begin{figure}[ht]
	\begin{tabular}{cccc}
		&
		\begin{tikzpicture}[scale=1]
		\SetVertexSimple[Shape=circle,FillColor=white,MinSize=8 pt]
		\Vertex[x=0.00, y=0]{a}
		\Vertex[x=2, y=0]{b}
		\Vertex[x=2, y=2]{c}
		\Vertex[x=0, y=2]{d}
		\Edges(a,b,c,d,a)
		\Edges(a,c)
		\Edges(b,d)
		\end{tikzpicture}
		&
		\qquad
		& 
		\begin{tikzpicture}[scale=1]
				\SetVertexSimple[Shape=circle,FillColor=white,MinSize=8 pt]
		\Vertex[x=0.00, y=0]{a}
		\Vertex[x=1.15, y=2]{b}
		\Vertex[x=2.31, y=0]{c}
		\Vertex[x=1.15, y=0.8]{d}
		\Edges(a,b,c,d,a)
		\Edges(a,c)
		\Edges(b,d)
		\end{tikzpicture} 
		\\
		&(a)&&(b)
	\end{tabular}
	\caption{Dibujos de $K_4$} \label{fA4.2}
\end{figure}

En vista de estos ejemplos, una pregunta es ?`existen grafos no
planares? Por ejemplo si dibujáramos $K_{16}$ parecería imposible
que fuera planar, dada la gran cantidad de cortes, pero ?`cómo
podemos estar seguros?

Observemos primero que si $G$ es planar y $H$ es subgrafo de $G$,
entonces $H$ es planar, pues, si podemos dibujar a $G$ en el plano
sin cortes de aristas, entonces $H$ que esta ``metido'' en $G$,
también puede ser así dibujado. Así, como ya vimos que $K_4$ es
planar, sabemos que todo subgrafo de él es planar; es decir, todo
grafo con cuatro o menos vértices es planar. Esta observación
tiene consecuencias en la otra dirección también: si encontramos
un grafo $H$ que {\em no} sea planar, entonces todo grafo $G$ que
lo tenga como subgrafo deberá necesariamente ser no planar, pues
si $G$ fuera planar, $H$ también lo seria. Así, si queremos probar
que $K_{16}$ no es planar, bastará con encontrar algún subgrafo
mas sencillo de el que no lo sea. De hecho, probaremos que $K_5$
no es planar, con lo cual todos los grafos $K_n$, con $n\ge 5$ son
no planares.

En lo que sigue veremos un arma poderosa para probar que un grafo
es no planar: la llamada ``fórmula de Euler''.
 \index{fórmula de Euler}
Supongamos que un grafo {\em sí} es planar. Escojamos un diagrama
de él en el plano (puede haber muchos, escojamos uno). Este
diagrama divide al plano en varias regiones. Por ejemplo, si $G$
esta representado por el dibujo de la Fig. \ref{fA4.3}, entonces
se obtienen regiones que numeraremos como en la Fig. \ref{fA4.4}
(1 es la región ``exterior'' a todo el grafo).

\begin{figure}[h]
\begin{tikzpicture}[scale=0.7]
\SetVertexSimple[Shape=circle,FillColor=white,MinSize=8 pt]
\Vertex[x=0.00, y=0]{0}
\Vertex[x=0.5, y=2]{1}
\Vertex[x=0.8, y=-2]{2}
\Vertex[x=1.5, y=-0.2]{3}
\Vertex[x=1.8, y=-2.2]{4}
\Vertex[x=3.5, y=1.9]{5}
\Vertex[x=3, y=0.1]{6}
\Vertex[x=3.7, y=-1.7]{7}
\Vertex[x=4, y=-0.1]{8}
\Vertex[x=6, y=0.3]{9}
\Vertex[x=5.5, y=1.8]{10}
\Vertex[x=6, y=-2.2]{11}
\Vertex[x=6.8, y=2]{12}
\Vertex[x=7, y=-1.8]{13}
\Vertex[x=7.5, y=0.8]{14}
\Vertex[x=8.5, y=1.8]{15}
\Vertex[x=9, y=-1]{16}
\Edges(0,1,3,4,2,0)
\Edges(1,5,6,7,4,7,8,5,9,8,9,7,11,9,5,10,12,14,15,16,13,11)
\end{tikzpicture} 
\caption{Un grafo planar} \label{fA4.3}
\end{figure}

En realidad, también podríamos considerar a la región formada por
las regiones 3 y 4 juntas, o 2, 5 y 6 juntas, etc. Pero nuestra
preocupación estará centrada en una de estas regiones ``simples'',
a las cuales llamaremos {\em caras}.
 \index{caras de un grafo planar}

\begin{figure}[h]
	\begin{tikzpicture}[scale=0.7]
	\SetVertexSimple[Shape=circle,FillColor=white,MinSize=8 pt]
	\Vertex[x=0.00, y=0]{0}
	\Vertex[x=0.5, y=2]{1}
	\Vertex[x=0.8, y=-2]{2}
	\Vertex[x=1.5, y=-0.2]{3}
	\Vertex[x=1.8, y=-2.2]{4}
	\Vertex[x=3.5, y=1.9]{5}
	\Vertex[x=3, y=0.1]{6}
	\Vertex[x=3.7, y=-1.7]{7}
	\Vertex[x=4, y=-0.1]{8}
	\Vertex[x=6, y=0.3]{9}
	\Vertex[x=5.5, y=1.8]{10}
	\Vertex[x=6, y=-2.2]{11}
	\Vertex[x=6.8, y=2]{12}
	\Vertex[x=7, y=-1.8]{13}
	\Vertex[x=7.5, y=0.8]{14}
	\Vertex[x=8.5, y=1.8]{15}
	\Vertex[x=9, y=-1]{16}
	\Edges(0,1,3,4,2,0)
	\Edges(1,5,6,7,4,7,8,5,9,8,9,7,11,9,5,10,12,14,15,16,13,11)
	\draw (-2, 1) node {1};
	\draw (3.5, 0) node {2};
	\draw (4.3, 0.8) node {3};
	\draw (4.3, -0.7) node {4};
	\draw (2.2, 0) node {5};
	\draw (0.8, 0) node {6};
	\draw (5.4, -1.2) node {7};
	\draw (7.2, -0.5) node {8};
	\end{tikzpicture} 
	\caption{Regiones de un grafo planar} \label{fA4.4}
\end{figure}

Observemos que no podemos hablar propiamente de las caras del
grafo (aunque a veces lo haremos así) pues ellas son en realidad
dependientes del diagrama, no del grafo. Sin embargo, algo puede
decirse acerca de ellas:

\begin{teorema}\label{tA4.1} (Fórmula de Euler) Sea $G$ un grafo
conexo, con $v$ vértices, y $e$ aristas. Supongamos que en algún
diagrama planar de $G$, existen $f$ caras. Entonces, $v-e+f=2$.
\end{teorema}

Antes de ver la prueba, observemos que, puesto que $v$ y $e$
dependen de $G$ y no del diagrama, la fórmula de Euler dice que no
importa como dibujemos a $G$ en el plano (siempre y cuando esto
sea posible), entonces siempre obtendremos $e-v+2$ caras. Por lo
tanto, el {\it número} de caras es algo independiente del
diagrama, y podemos hablar del ``número de caras de un grafo
planar''. Otra observación es que en el número de caras estamos
contando la cara infinita, es decir, la exterior a todo el grafo.
Finalmente, observemos que se pide que $G$ sea conexo. La fórmula
debe ser alterada en caso contrario.

\begin{proof}[Demostración del Teorema \ref{tA4.1}]
Supongamos que la fórmula de Euler no sea cierta. Es decir,
supongamos que existen grafos planares para los cuales la fórmula
no es válida. Tomemos, de todos estos contraejemplos, alguno con
$e$ tan chico como sea posible, y llamemos $G$ a ese grafo.
Observemos que $G$ debe tener por lo menos un ciclo, pues si fuera
acíclico, como es conexo, sería un árbol. Ahora bien, en un árbol,
$e=v-1$. Además, por ser acíclico, no hay caras, salvo la cara
infinita, es decir, $f$ seria 1. Pero entonces
$v-e+f=v-(v-1)+1=v-v+1+1=2$ y $G$ no sería un contraejemplo. Así
pues, $G$ tiene al menos un ciclo. Sea $xy$ alguna arista
perteneciente a algún ciclo, y consideremos $H=G-xy$. Como $xy$
pertenece a algún ciclo, es una arista que separa dos caras en
$G$. Esas dos caras ahora son una sola en $H$. (Ver Fig.
\ref{fA4.5}).

\begin{figure}[ht]
	\begin{tabular}{cccc}
		&
		\begin{tikzpicture}[scale=0.7]
		\SetVertexSimple[Shape=circle,FillColor=white,MinSize=8 pt]
		\Vertex[x=0.00, y=0]{0}
		\Vertex[x=0.5, y=-1]{1}
		\Vertex[x=-0.5, y=-2]{2}
		\Vertex[x=0, y=-3]{3}
		\Vertex[x=2, y=-4]{4}
		\Vertex[x=3.5, y=-3]{5}
		\Vertex[x=3.5, y=-2]{6}
		\draw (3., -2.1) node {$y$};
		\Vertex[x=2, y=-1]{7}
		\draw (2, -1.5) node {$x$};
		\Vertex[x=1.8, y=0.2]{8}
		\Vertex[x=3, y=0]{9}
		\Vertex[x=5, y=-0.2]{10}
		\Vertex[x=4.5, y=-2.7]{11}
		\Edges(0,1,2,3,4,5,6)
		\Edges(6,7)
		\Edges(7,8,0)
		\Edges(1,7)
		\Edges(8,9,10,11,5)
		\draw (1.5, -2.5) node {$A$};
		\draw (3.5, -1) node {$B$};
		\end{tikzpicture}
		&
		\qquad
		& 
		\begin{tikzpicture}[scale=0.7]
		\SetVertexSimple[Shape=circle,FillColor=white,MinSize=8 pt]
		\Vertex[x=0.00, y=0]{0}
		\Vertex[x=0.5, y=-1]{1}
		\Vertex[x=-0.5, y=-2]{2}
		\Vertex[x=0, y=-3]{3}
		\Vertex[x=2, y=-4]{4}
		\Vertex[x=3.5, y=-3]{5}
		\Vertex[x=3.5, y=-2]{6}
		\draw (3., -2.1) node {$y$};
		\Vertex[x=2, y=-1]{7}
		\draw (2, -1.5) node {$x$};
		\Vertex[x=1.8, y=0.2]{8}
		\Vertex[x=3, y=0]{9}
		\Vertex[x=5, y=-0.2]{10}
		\Vertex[x=4.5, y=-2.7]{11}
		\Edges(0,1,2,3,4,5,6)
		\Edges(7,8,0)
		\Edges(1,7)
		\Edges(8,9,10,11,5)
		\end{tikzpicture} 
	\end{tabular}
	\caption{Eliminar una arista} \label{fA4.5}
\end{figure}
Así, si $f_H,e_H$ y $v_H$ denotan el número de caras, aristas y
vértices de $H$ respectivamente, tenemos que $f_H=f-1$. Además,
como borramos una arista, $e_H=e-1$, y como el número de vértices
no cambia, $v_H=v$.

Pero, $e_H=e-1$ es menor que $e$, y $G$ era un contraejemplo con
un número tan chico como fuera posible de aristas, por lo tanto,
$H$ no es un contraejemplo, es decir, $v_H-e_H+f_H=2$.
Reemplazando, obtenemos:
$$
v-e+f=v_H-(e_H+1)+f_H+1=v_H-e_H-1+f_H+1=v_H-e_H+f_H=2,
$$
lo cual dice que $G$ no es un contraejemplo, absurdo.
\end{proof}

La fórmula de Euler es una herramienta muy poderosa en la teoría
de grafos planares. Para empezar, permite probar que un grafo
planar no puede tener muchas aristas, en relación a sus vértices

\begin{corolario}\label{cA4.1} Sea $G$ un grafo planar con al menos 3
vértices. Entonces, $e\le 3v-6$, donde $e$ es el número de aristas
y $v$ el número de vértices de $G$.
\end{corolario}
\begin{proof} Consideremos las caras de $G$. Si es una cara
distinta de la cara infinita, es porque viene de un ciclo. Ahora
bien, todo ciclo debe tener por lo menos 3 aristas, así que
podemos concluir que hay por lo menos 3 aristas en el borde de esa
cara. Si, en cambio, es la cara infinita y el grafo tiene más de
tres aristas entonces ``toca'' 3 o más aristas. Si el grafo tiene
menos de 3 aristas (y ningún ciclo), es uno de los de la Fig.
\ref{fA4.6}. Como estamos suponiendo que hay al menos 3 vértices,
en realidad solo hay que considerar el último caso, y ese tiene
$e=2$, $v=3$, y $2\le 3 \times 3-6$.

\begin{figure}[ht]
	\begin{tabular}{cccc}
		&
		\begin{tikzpicture}[scale=0.7]
		\SetVertexSimple[Shape=circle,FillColor=white,MinSize=8 pt]
		\Vertex[x=0.00, y=0]{0}
		\Vertex[x=2, y=0]{1}
		\Edges(0,1)
		\end{tikzpicture}
		&
		\qquad\qquad
		& 
		\begin{tikzpicture}[scale=0.7]
		\SetVertexSimple[Shape=circle,FillColor=white,MinSize=8 pt]
		\Vertex[x=0.00, y=0]{0}
		\Vertex[x=2, y=0]{1}
		\Vertex[x=4, y=0]{2}
		\Edges(0,1,2)
		\end{tikzpicture} 
	\end{tabular}
	\caption{Grafos acíclicos con menos de 3 aristas} \label{fA4.6}
\end{figure}

Así pues, podemos suponer que en nuestro grafo, todas las caras
tienen al menos 3 aristas en su borde. Es decir:
$$
\begin{aligned}
3\le &\,\text{\rm Número de aristas en el borde de cara }1 \\
3\le &\,\text{\rm Número de aristas en el borde de cara }2\\
&\vdots \\
3\le &\,\text{\rm Número de aristas en el borde de cara }f.
\end{aligned}
$$
Si sumamos estas desigualdades, del lado izquierdo obtendremos
$3f$. En el lado derecho, cada arista puede, o bordear dos caras,
o bordear una. Pero ciertamente, no puede haber aristas que sean
borde de 3 caras. Así, si sumamos en el lado izquierdo, la suma
nos dará menor o igual a $2e$. Por lo tanto, $3f\le 2e$. Tomando
la fórmula de Euler y multiplicándola por 3, obtenemos:
$3v-3e+3f=6$. Usando ahora $3f\le 2e$, tenemos
$$
6=3v-3e+3f\le 3v-3e+2e=3v-e,$$ es decir, $e\le 3v-6$.
\end{proof}

Este corolario nos permite probar inmediatamente la no planaridad
de un número significativo de grafos. Por ejemplo, recordemos que
queríamos ver que $K_5$ era no planar. Esto lo obtenemos en forma
directa, pues $K_5$ tiene 5 vértices y 10 aristas, por lo tanto,
si fuera planar debiéramos tener que $10\le 3\times5-6=15-6=9$, lo cual
no es cierto.

\end{section}

\begin{section}{El problema del agua-luz-gas}\label{Ap4.2}
Este es un conocido problema de escuela primaria: existen tres
casas, y tres centrales: la del agua, la de la luz y la del gas.
Trazar las cañerías desde las centrales a las casas sin que se
crucen. Una solución (pero haciendo trampa) es mandar las tres
cañerías a una casa, y de ella sacarlas las tres a la otra, y de
ella las tres a la otra:

\begin{figure}[ht]
	\begin{tikzpicture}[scale=1.2]
	\draw[-,line width=0.8pt] (0,0.2) -- (3,0.2) -- (6,0.2);
	\draw[-,line width=0.8pt] (0,-0.2) -- (3,-0.2) -- (6,-0.2);
	{\renewcommand{\VertexShape}{rectangle}
	\Vertex[x=0.00, y=0, L=Primera casa]{0}
	\Vertex[x=3, y=0, L=Segunda casa]{1}
	\Vertex[x=6, y=0, L=Tercera casa]{2}
	\Vertex[x=0.00, y=-2, L=$A$]{3}
	\Vertex[x=3, y=-2, L=$L$]{4}
	\Vertex[x=6, y=-2, L=$G$]{5}
	}
	\SetVertexSimple[Shape=rectangle,FillColor=white,MinSize=8 pt]
	%\draw (0, 0) node {$\boxed{\text{Primera casa}}$};
	\Edges(0,1,2)
	\Edges(3,0)
	\Edges(4,0)
	\Edges(5,0)

	\end{tikzpicture}
	\caption{Una solución tramposa} \label{fA4.7}
\end{figure}

En realidad, no permitiremos el uso de intermediarios, es decir el
problema será llevar directamente la cañería desde cada central a
cada casa. En el lenguaje de la teoría de grafos, consiste en
representar, en el plano, al grafo $K_{3,3}$ (Fig. \ref{fA4.8}).

\begin{figure}[ht]
	\begin{tikzpicture}[scale=1.2]
	{\renewcommand{\VertexShape}{rectangle}
		\Vertex[x=0.00, y=0, L=Primera casa]{0}
		\Vertex[x=3, y=0, L=Segunda casa]{1}
		\Vertex[x=6, y=0, L=Tercera casa]{2}
		\Vertex[x=0.00, y=-2, L=$A$]{3}
		\Vertex[x=3, y=-2, L=$L$]{4}
		\Vertex[x=6, y=-2, L=$G$]{5}
	}
	\Edges(0,4)
	\Edges(0,5)
	\Edges(1,3)\Edges(1,4)
	\Edges(1,5)
	\Edges(2,3)\Edges(2,4)
	\Edges(2,5)
	\Edges(3,0)
	\Edges(4,0)
	\Edges(5,0)

	\end{tikzpicture}
	\caption{Luz-agua-gas es $K_{3,3}$} \label{fA4.8}
\end{figure}

La pregunta es entonces si $K_{3,3}$ es planar o no. Veamos si
podemos usar esta fórmula que probamos recién: $K_{3,3}$ tiene 9
aristas, y 6 vértices. Desafortunadamente, $3 \times 6-6=18-6=12$ es
ciertamente mayor que 9, así que solo sabemos que quizás es
planar. Pero, observemos que $K_{3,3}$, por ser bipartito, no
tiene ningún triángulo como subgrafo. Así pues, deduciremos la no
planaridad de $K_{3,3}$ del si\-guien\-te

\begin{corolario}\label{cA4.2} Si $G$ es un grafo planar con por lo menos 3
vértices y que no tiene ningún triángulo como subgrafo, entonces
$e\le 2v-4$.
\end{corolario}
\begin{proof} Es similar a la demostración del Corolario \ref{cA4.1}, pero como no hay
triángulos, todo ciclo tiene por lo menos 4 aristas, es decir,
cada cara esta bordeada por al menos 4 aristas. Las únicas
excepciones con al menos 3 vértices son:

\begin{figure}[ht]
	\begin{tabular}{cccccc}
	&
	\begin{tikzpicture}[scale=0.5]
	\SetVertexSimple[Shape=circle,FillColor=white,MinSize=8 pt]
	\Vertex[x=0.00, y=0]{0}
	\Vertex[x=2, y=0]{1}
	\Vertex[x=4, y=0]{2}
	\Edges(0,1,2)
	\end{tikzpicture}
	&
	\qquad
	& 
	\begin{tikzpicture}[scale=0.5]
	\SetVertexSimple[Shape=circle,FillColor=white,MinSize=8 pt]
	\Vertex[x=0.00, y=0]{0}
	\Vertex[x=2, y=0]{1}
	\Vertex[x=4, y=0]{2}
	\Vertex[x=6, y=0]{3}
	\Edges(0,1,2,3)
	\end{tikzpicture} 
	&
	\qquad
	&
	\begin{tikzpicture}[scale=0.5]
	\SetVertexSimple[Shape=circle,FillColor=white,MinSize=8 pt]
	\Vertex[x=0.00, y=0]{0}
	\Vertex[x=0, y=1.3]{1}
	\Vertex[x=-1, y=-1]{2}
	\Vertex[x=1, y=-1]{3}
	\Edges(0,1,0,2,0,3)
	\end{tikzpicture} 
	\end{tabular}
	\caption{Grafos acíclicos con menos de 4 aristas y al menos 3
	vértices} \label{fA4.9}
\end{figure}

En el primer caso, $e=2$, $v=3$ y $2 \times 3-4=6-4=2$. En el segundo y
terceros, $e=3$, $v=4$ y $2 \times 4-4=8-4=4\ge 3$. Así pues, podemos
suponer que cada cara esta bordeada por al menos 4 aristas.
Sumando cara a cara, como antes, obtenemos ahora $4f\le 2e$, es
decir, $2f\le e$. Multiplicando la fórmula de Euler por 2,
tenemos: $4=2v-2e+2f\le 2v-2e+e=2v-e$, es decir, $e\le 2v-4$.
\end{proof}

Retornando a $K_{3,3}$, como no tiene triángulos, podemos aplicar
este corolario, y si fuera planar, debería cumplirlo. Pero
habíamos dicho que $K_{3,3}$ tiene 9 aristas y 6 vértices, y
$2\times6-4=12-4=8$. Por lo tanto, $K_{3,3}$ no es planar.

Una ultima observación acerca de grafos planares: existe un
teorema muy interesante, de difícil demostración (la prueba tiene
31 casos y subcasos para considerar) debido a Kuratowski, que
 \index{Teorema de Kuratowski}
 dice que $K_5$ y $K_{3,3}$ son los dos grafos
``básicos'' no planares, en el siguiente sentido: un grafo $G$ es
no planar si y solo si existe un subgrafo de $G$, digamos $H$, tal
que $H$ se ``ve'' como $K_5$ o como $K_{3,3}$, es decir, $H$ es
uno de ellos, excepto que tal vez, ``agregue'' en alguna o algunas
aristas, vértices en el medio. Por ejemplo, $H$ puede lucir como

\begin{figure}[ht]
	\begin{tikzpicture}[scale=1]
	\SetVertexSimple[Shape=circle,MinSize=5 pt,FillColor=white]
	\Vertex[x=0.00, y=2.00]{1}
	\Vertex[x=1.90, y=0.62]{2}
	\Vertex[x=1.18, y=-1.62]{3}
	\Vertex[x=-1.18, y=-1.62]{4}
	\Vertex[x=-1.90, y=0.62]{5}
	\Edges(1,2,3,4,5,1)
	\Edges(1,3,5,2,4,1)
	\Vertex[x=0.95, y=1.31]{a}
	\Vertex[x=0.3, y=1.08]{b}
	\Vertex[x=1.54, y=-0.5]{c}
	\Vertex[x=-0.36, y=-0.5]{d}
	\Vertex[x=-0.95, y=1.31]{e}
	\Vertex[x=-0.39, y=-1.62]{f}
	\Vertex[x=0.39, y=-1.62]{g}
	\end{tikzpicture}
	\caption{}\label{fA4.10}
\end{figure}

\end{section}

\begin{section}{El teorema de los cuatro colores} \label{Ap4.3}

Juntaremos ahora lo que hemos visto en esta sección con lo que
vimos en la anterior, para tratar uno de los problemas mas famosos
y recalcitrantes de la teoría de grafos, a saber: ?`cuántos
colores se necesitan para colorear un grafo planar? En otras
palabras, si quiero estar seguro de poder colorear propiamente los
vértices de cualquier grafo planar, ?`cuántos colores necesito
tener? De hecho, una pregunta más básica sería si existe una
cantidad finita de colores que me permitan colorear cualquier tipo
de grafo planar, por grande que sea. (Es claro que la respuesta
para grafos en general es negativa, pues $K_n$ requiere $n$
colores.) Como $K_4$ es planar, sabemos que necesitamos por lo
menos 4 colores. No podemos decir que necesitamos necesariamente
5, pues hemos visto que $K_5$ no es planar. Pero, podría haber
otro grafo, complicado pero planar, que requiera 5, o más,
colores. A mediados del siglo pasado la conjetura de que bastan 4
colores fue hecha, y en 1879 A. Kempe publicó una prueba de este
hecho, que paso a llamarse el teorema de los cuatro colores.
Desafortunadamente para Kempe, en 1889 (diez años después) otro
matemático, P. Heawood, probó que la prueba de Kempe contenía un
error. Heawood no fue completamente destructivo: mostró que
adaptando la prueba de Kempe, podía probarse que con 5 colores
bastaba para colorear cualquier grafo planar (el teorema de los
cinco colores). Así pues, quedo planteado el problema de saber si
el teorema de los cuatro colores era cierto, o bien si existía
algún grafo planar para el cual 5 colores fueran necesarios.
(Pero, al menos, gracias a Heawood, no era necesario buscar alguno
que necesitara 6, o 7 u 8 colores, pues no existen, gracias al
teorema de los cinco colores.) De hecho, en ese mismo artículo,
Heawood probó más cosas: existe algo llamado género de un grafo,
que es un número entero no negativo. Heawood demostró que existía
una \index{género de un grafo} fórmula (expresión aritmética) que
para cada género $g\ge 1$ da la cantidad de colores que permite
colorear todos los grafos de género $g$. Los grafos planares tiene
género igual a 0 y aplicando la fórmula para $g=0$ obtenemos el
número 4. Sin embargo, Heawood pudo probar que la fórmula es
válida si $g$ es mayor o igual a 1. El hecho de que esta fórmula
existiera ``convenció'' a mucha gente de que el teorema de los
cuatro colores debía ser cierto y que una prueba no tardaría en
hallarse. Sin embargo, pese al esfuerzo de muchos matemáticos y
pese al desarrollo de la teoría, el teorema de los cuatro colores
no pudo probarse hasta 1975, cuando dos matemáticos \index{Teorema
de los cuatro colores} norteamericanos, K. Appel y W. Haken, lo
probaron. Más aún, no pudieron probarlos solos, sino que debieron
usar la ``ayuda'' de un poderoso (para esa época) computador. Así,
aún cuando el teorema fue probado, un gran sentimiento de
desconfianza se generó, sobretodo en una época en la cual el
acceso fácil a tiempo de computador no era común. Veinte años han
pasado y la prueba ahora ha sido controlada numerosas veces y no
genera tanta resistencia como antes. En 2005, Benjamin Werner y Georges Gonthier formalizaron una prueba del teorema utilizando el asistente de demostraciones Coq. Esto eliminó la necesidad de confiar en los diversos programas de computadora utilizados para verificar casos particulares y ahora  solo es necesario confiar en Coq, mas precisamente en el kernel de Coq. Este resultado se encuentra en el trabajo \textit{A computer-checked proof of the four colour theorem} de Georges Gonthier (2005).  Aún así, si alguien pudiese
publicar una prueba que fuese ``leíble por humanos'' sería muy
bien bienvenido.

Obviamente por lo dicho arriba, no daremos una prueba del teorema
de los cuatro colores. Sí daremos una del teorema de los cinco
colores\index{Teorema de los cinco colores}, mencionando donde se encuentra la dificultad de la demostración del teorema de los  cuatro colores, y dando una idea de
que es lo que Appel y Haken (y el computador) hicieron.

\begin{lema} \label{lA4.3.1} Sea $G$ un grafo planar. Entonces, existe
un vértice de $G$ de valencia 5 o menos.
\end{lema}
\begin{proof} Si el orden de $G$ es menor o igual a 2, esto
es obvio, pues la valencia de cualquier vértice no superará 2.
Así, podemos suponer que hay al menos 3 vértices, y por lo tanto,
sabemos que $e\le 3v-6$, donde $e$ es el numero de aristas y $v$
el de vértices.

Supongamos ahora que la valencia de todos los vértices sea al
menos 6. Entonces, si sumamos las valencias de todos los vértices,
la suma sera mayor o igual a $6v$. Pero la suma de todos las
valencias es igual a $2e$ (lema del apretón de manos). Así,
tenemos que $2e\ge 6v$. Por otro lado, como $e\le 3v-6$, tenemos
que $2e\le 6v-12$, es decir, obtenemos $6v-12\ge 2e\ge 6v$, o
$-12\ge 0$, lo cual es un absurdo.
\end{proof}

\begin{teorema}\label{tA4.3.1}(Teorema de los cinco colores) Si $G$ es
planar, $\chi (G)\le 5$.
\end{teorema}
\begin{proof}
 Supongamos que no sea cierto. De todos los
contraejemplos al teorema, escojamos uno con la menor cantidad de
vértices posible y llamémosle $G$. Por el lema anterior, existe un
vértice $x$ de $G$ con valencia menor o igual a 5. Consideremos
$H=G-x$, que es un grafo con menos vértices que $G$ y por lo tanto
no puede ser un contraejemplo; es decir, $\chi (H)\le 5$. Así,
podemos colorear $H$ con 5 colores. Si la valencia de $x$ en $G$
es $0,1,2,3$ o 4, los vértices adyacentes a $x$ ``usan'' a lo sumo
4 de los 5 colores, así que podemos colorear a $x$ con el quinto
color, y tendríamos que $\chi (G)=5$, lo cual no es posible pues
$G$ es un contraejemplo. Así pues, podemos suponer que la valencia
de $x$ es 5. Ahora bien, si los cinco vértices adyacentes a $x$ no
usan cinco colores, estamos como antes, y podemos colorear a $x$
con el color faltante. Así, no solo podemos suponer que hay cinco
vértices adyacentes a $x$, sino también que cada uno esta
coloreado con un color distinto. Llamemos a estos vértices
$y,z,u,w,t$, y supongamos que $y$ de color 1, $z$ de color dos,
etc.

\begin{figure}[h]
\begin{tikzpicture}[scale=0.5]
\SetVertexSimple[Shape=circle,FillColor=white,MinSize=8 pt]
\Vertex[x=0.00, y=0]{0}
\Vertex[x=0.5, y=2]{1}
\Vertex[x=4, y=1.5]{2}
\Vertex[x=3, y=-1.4]{3}
\Vertex[x=0.5, y=-2]{4}
\Vertex[x=-3, y=0]{5}
\draw (-0.5, 0.4) node {$x$};
\draw (-0.1, 2) node {$y$};
\draw (1.4, 2) node {$(1)$};
\draw (3.4, 1.7) node {$z$};
\draw (4.8, 1.7) node {$(2)$};
\draw (2.4, -1.6) node {$u$};
\draw (3.8, -1.6) node {$(3)$};
\draw (-0.2, -2) node {$w$};
\draw (1.3, -2) node {$(4)$};
\draw (-3.4,0.5) node {$t$};
\draw (-2.3, 0.5) node {$(5)$};
\Edges(0,1,0,2,0,3,0,4,0,5)
\end{tikzpicture} 
\caption{}\label{fA4.11}
\end{figure}

Supongamos primero que no haya, entre $y$ y $u$, ningún camino tal
que el color de todos sus vértices sea 1 o 3. Entonces, podemos
cambiarle el color a $y$, de color 1 a color 3. Además, a los
vértices adyacentes a $y$ que tengan color 3, les cambiamos el
color de 3 a 1. A los vértices adyacentes a estos, que tengan
color 1, los cambiamos a 3, y así sucesivamente. Después de
realizar todos estos cambios, todavía tenemos un coloreo propio.
Ahora bien, como estamos suponiendo que no hay ningún camino de
color 1 y 3 exclusivamente entre $y$ y $u$, resulta que $u$ no
cambia de color, es decir, retiene el color 3. Pero $y$ ahora
tiene también el color 3, y ningún otro vértice adyacente a $x$
tiene el color 1. Pero, entonces, podemos colorear a $x$ con el
color 1 sin problemas, absurdo pues $\chi(G)\ge 6$.

Así pues, existe un camino con todos los vértices de color 1 y 3
entre $y$ y $u$. Igualmente, si no hubiera ningún camino con todos
los vértices de color 2 y 4 entre $z$ y $w$, le podemos cambiar el
color a $z$ de 2 a 4 sin problemas, y colorear a $x$ con el color
2. Así, también podemos suponer que existe un camino con todos los
vértices de color 2 y 4 entre $z$ y $w$.

\begin{figure}[h]
	\begin{tikzpicture}[scale=0.5]
	\draw[-,line width=1pt,dashed] (0.5,2) -- (5,2.5) -- (5.5,1)-- (5.5,0.2) -- (3, -1.4) ;
	\draw[-,line width=1pt,dashed] (4,1.5) -- (6,0.5) -- (6.5,-0.5)-- (5.5,-2.3) -- (0.5, -2) ;
	\SetVertexSimple[Shape=circle,FillColor=white,MinSize=8 pt]
	\Vertex[x=0.00, y=0]{0}
	\Vertex[x=0.5, y=2]{1}
	\Vertex[x=4, y=1.5]{2}
	\Vertex[x=3, y=-1.4]{3}
	\Vertex[x=0.5, y=-2]{4}
	\Vertex[x=-3, y=0]{5}
	\draw (-0.5, 0.4) node {$x$};
	\draw (-0.1, 2) node {$y$};
	\draw (3.4, 1.7) node {$z$};
	\draw (2.4, -1.6) node {$u$};
	\draw (-0.2, -2) node {$w$};
	\draw (-3.4,0.5) node {$t$};
	\Edges(0,1,0,2,0,3,0,4,0,5)
	\end{tikzpicture} 
	\caption{Caminos de $y$ a $u$ y de $z$ a $w$} \label{fA4.12}
\end{figure}

Por la Fig. \ref{fA4.12} es claro que tenemos un problema: ?`por
donde se cruzan los caminos $A$ y $ B$? Más precisamente, el
camino $A$, junto con las aristas $xy$ y $xu$, forma un ciclo $C$.
Este ciclo tiene un interior y un exterior. El ciclo $ D$ formado
por $B$ y las aristas $xz$, $xw$ cruza al ciclo $C$ en el punto
$x$, pues la arista $xz$ esta en el interior y la arista $xw$ en
el exterior de $C$. Por lo tanto, $D$ debe cruzar a $C$ en algún
otro punto. Pero no puede hacerlo, pues en el resto, $C$ esta
coloreado con los colores 1 y 3, y $D$ con los colores 2 y 4.
Hemos llegado a una contradicción.
\end{proof}

Analicemos un poco la prueba: hemos probado dos cosas 1) todo
grafo planar debe tener una de las siguientes ``configuraciones'',
es decir, parte de él debe lucir como alguno de los grafos de la
Fig. \ref{fA4.13}.

\begin{figure}[h]
	\begin{tikzpicture}[scale=0.5]
	\SetVertexSimple[Shape=circle,FillColor=white,MinSize=8 pt]
	\Vertex[x=0.00, y=0]{0}
	\Vertex[x=3, y=0]{1}
	\Vertex[x=5, y=0]{2}
	\Vertex[x=7, y=1]{3}
	\Vertex[x=9, y=-1]{4}
	\Vertex[x=11, y=1]{5}
	\Vertex[x=13, y=1]{6}
	\Vertex[x=15, y=-1]{7}
	\Vertex[x=17, y=1]{8}
	\Vertex[x=15, y=-3]{9}
	\Edges(1,2)
	\Edges(3,4,5)
	\Edges(6,7,8,7,9)
	\Vertex[x=0, y=-3]{a}
	\Vertex[x=4, y=-3]{b}
	\Vertex[x=0, y=-7]{c}
	\Vertex[x=4, y=-7]{d}
	\Edges(a,d)
	\Edges(b,c)
	\Vertex[x=2, y=-5]{e}
	\Vertex[x=7, y=-4]{f}
	\Vertex[x=11, y=-4]{g}
	\Vertex[x=9, y=-5.5]{h}
	\Vertex[x=9, y=-3]{i}
	\Vertex[x=7, y=-7]{j}
	\Vertex[x=11, y=-7]{k}
	\Edges(f,h,g,h,i,h,j,h,k)
	\end{tikzpicture} 
\caption{Posibles configuraciones} \label{fA4.13}
\end{figure}

Esto lo probamos con el Lema \ref{lA4.3.1}. Es decir, probamos que
ese conjunto de configuraciones es lo que se llama {\it
inevitable}.

Además, probamos que si un grafo planar tiene una de esas
configuraciones, puede ser coloreado con 5 colores (esto es lo que
hicimos en el teorema). Es decir, probamos que ese conjunto de
configuraciones es lo que se llama {\it irreducible} (para 5
colores). Kempe creyó que había sido capaz de probar que todo
grafo planar que tuviera esas configuraciones podía ser coloreado
con 4 colores, y muchos autores después de Heawood trataron de
probar lo mismo. Pero luego se descubrieron nuevas técnicas, tanto
para probar que un conjunto de configuraciones es irreducible,
como para probar que es inevitable. Lo que no se podía hacer era
encontrar un conjunto que fuera al mismo tiempo irreducible (para
4 colores) e inevitable. Finalmente, Appel y Haken encontraron un
conjunto que satisfacía esas propiedades. Solo que en vez detener
6 elementos, como en el caso del teorema de 5 colores, el conjunto
de Appel y Haken tiene 1480 elementos, y ningún ser humano es
capaz de probarlo, sino que es necesario un computador para
comprobar la inevitabilidad e irreducibilidad.

\end{section}

\printindex

\end{document}
